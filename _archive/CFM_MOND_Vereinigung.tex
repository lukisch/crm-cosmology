\documentclass[11pt,a4paper]{article}
\usepackage[utf8]{inputenc}
\usepackage[T1]{fontenc}
\usepackage[english,ngerman]{babel}
\usepackage{geometry}
\geometry{a4paper, left=2.5cm, right=2.5cm, top=2.5cm, bottom=2.5cm}
\usepackage{mathptmx}
\usepackage{helvet}
\usepackage{amsmath}
\usepackage{amssymb}
\usepackage{amsthm}
\usepackage{titlesec}
\usepackage{booktabs}
\usepackage{tabularx}
\usepackage{xcolor}
\usepackage{authblk}
\usepackage{hyperref}
\usepackage{enumitem}
\usepackage{graphicx}
\usepackage{float}
\usepackage{setspace}
\usepackage{array}
\usepackage[normalem]{ulem}

\newtheorem{definition}{Definition}
\newtheorem{proposition}{Proposition}
\newtheorem{conjecture}{Conjecture}

\titleformat{\section}{\Large\bfseries\sffamily\color{black}}{\thesection}{1em}{}
\titleformat{\subsection}{\large\bfseries\sffamily\color{darkgray}}{\thesubsection}{1em}{}
\titleformat{\subsubsection}{\normalsize\bfseries\sffamily\color{darkgray}}{\thesubsubsection}{1em}{}

\hypersetup{
    pdftitle={Eliminating the Dark Sector: Unifying the Curvature Feedback Model with MOND},
    pdfauthor={Lukas Geiger},
    colorlinks=true,
    linkcolor=black,
    urlcolor=blue,
    citecolor=black
}

\onehalfspacing

\begin{document}

% ===================================================================
% TITELSEITE
% ===================================================================

\title{\textbf{\huge Eliminating the Dark Sector:\\Unifying the Curvature Feedback Model with MOND}\\[0.5em]
\Large A Baryon-Only Universe with Geometric Dark Matter and Dark Energy\\[0.3em]
\large Preliminary Analysis with Pantheon+ Type~Ia Supernovae}

\author[1]{Lukas Geiger\thanks{Correspondence: Lukas Geiger, Gei\ss{}b\"uhlweg~1, 79872~Bernau, Germany.}}
\affil[1]{Independent Researcher, Bernau im Schwarzwald}

\date{February 2026 \\ \vspace{0.5em} \small \textit{Working Paper -- Companion to \cite{Geiger2026}}}

\maketitle

\begin{abstract}
\noindent We propose a unified geometric framework that eliminates both dark energy and dark matter from the cosmological energy budget. Building on the Curvature Feedback Model (CFM) \cite{Geiger2026}, which replaces the cosmological constant with a time-dependent curvature return potential $\Omega_\Phi(a)$, we extend the model to a \textit{baryon-only} universe ($\Omega_m = \Omega_b \approx 0.05$) compatible with Modified Newtonian Dynamics (MOND) \cite{Milgrom1983}. The extended Friedmann equation reads:
\begin{equation*}
H^2(a) = H_0^2 \left[\mu(a)\,\Omega_b\,a^{-3} + \Omega_r\,a^{-4} + \Phi_0 \cdot f_{\mathrm{sat}}(a) + \alpha \cdot a^{-\beta_{\mathrm{eff}}(a)}\right]
\end{equation*}
where $\mu(a)$ is a \textit{scale-dependent} MOND gravitational enhancement factor applied at the background level, the saturation term $f_{\mathrm{sat}}$ replaces dark energy, and the power-law term $\alpha \cdot a^{-\beta_{\mathrm{eff}}}$ assumes the cosmological role of dark matter as a purely geometric effect. Three key innovations are: (i)~the \textit{running curvature coupling} $\beta_{\mathrm{eff}}(a)$, which transitions from $\beta_{\mathrm{early}} \approx 2.8$ (CDM-like) to $\beta_{\mathrm{late}} \approx 2.0$ (curvature-like), analogous to the MOND interpolation function; (ii)~the \textit{MOND background coupling} $\mu(a)$, which modifies both the Friedmann equation and the sound horizon calculation ($R_b = 3\mu(a)\Omega_b / 4\Omega_\gamma$), resolving the Hubble constant problem; and (iii)~the \textit{scale-dependent} evolution $\mu(a) = \sqrt{\pi}$ at late times transitioning to $\mu \to 1$ at $z > 4000$, which \textbf{completely eliminates} the need for Early Dark Energy ($f_{\mathrm{EDE}} = 0$). Tested against 1,590 Pantheon+ Type~Ia supernovae, Planck CMB compressed observables ($\ell_A$, $\mathcal{R}$), and 9 BAO distance measurements jointly, the preferred CFM+MOND variant with scale-dependent $\mu(a)$ achieves $\chi^2_{\mathrm{total}} = 704.8$ ($\Delta\chi^2 = -5.5$ vs.\ $\Lambda$CDM) at $H_0 = 67.3$\,km/s/Mpc with \textbf{zero} EDE, yielding $\ell_A = 301.471$ and $\mathcal{R} = 1.7502$ -- an exact match to Planck. The sound horizon $r_d = 146.9$\,Mpc is essentially identical to $\Lambda$CDM ($147.2$\,Mpc), and the model uses only 6 free parameters -- the same count as $\Lambda$CDM. The SN-only analysis with constant $\beta$ yields $\Delta\chi^2 = -26.3$ ($\Delta\mathrm{AIC} = -16.3$, $\Delta\mathrm{BIC} = -4.2$), with MCMC posteriors $\alpha = 0.68^{+0.02}_{-0.07}$ and $\beta = 2.02^{+0.26}_{-0.14}$, confirming curvature scaling ($a^{-2}$, $w = -1/3$). This framework renders the entire dark sector -- comprising 95\% of the energy budget in $\Lambda$CDM -- superfluous.

\vspace{0.5em}
\noindent \textbf{Keywords:} Curvature Feedback Model, MOND, dark matter, dark energy, baryon-only universe, Pantheon+, modified gravity, geometric cosmology

\vspace{0.5em}
\noindent \textbf{Subject areas:} Theoretical Physics, Cosmology, Modified Gravity
\end{abstract}

\newpage
\tableofcontents
\newpage

% ===================================================================
% KI-NUTZUNG
% ===================================================================
\section*{AI Disclosure and Methodology}
\addcontentsline{toc}{section}{AI Disclosure and Methodology}

\noindent\textbf{Extended Methodology Statement:} This paper is an experiment in \textit{AI-Assisted Science}. The division of labor is disclosed transparently:

\begin{description}[style=nextline, leftmargin=2cm]
\item[\textbf{Human author} (Lukas Geiger)] Physical intuition, core hypotheses (game-theoretic foundation, saturation mechanism, geometry-as-dark-sector, Efficiency Hypothesis, phase transition concept), interpretation of results, strategic decisions, and final responsibility for all scientific content.
\item[\textbf{Claude Opus 4.6} (Anthropic)] Co-writer: Mathematical formalization, derivation of equations, code development (Python/MCMC), statistical analysis (Pantheon+ fits), text generation, and structural organization.
\item[\textbf{Gemini} (Google DeepMind)] Reviewer: Critical feedback, MOND compatibility analysis, identification of BBN crisis, trace-coupling suggestion, strategic recommendations.
\end{description}

\vspace{0.5em}
\noindent\textit{Note:} The mathematical formalization and the statistical fits were performed by AI systems. The author presents these hypotheses as a \textit{Working Paper} to enable scrutiny and further development by the scientific community. \textbf{Independent mathematical verification is explicitly encouraged.}

\newpage


% ===================================================================
% 1. EINLEITUNG
% ===================================================================
\section{Introduction: The Dark Sector Problem}
\label{sec:intro}

The standard cosmological model, $\Lambda$CDM, describes the energy budget of the universe as consisting of approximately 5\% baryonic matter, 27\% cold dark matter (CDM), and 68\% dark energy ($\Lambda$) \cite{Planck2020}. Despite its remarkable empirical success, this model implies that \textit{95\% of the universe consists of entities that have never been directly detected}.

Two independent lines of research challenge this picture:

\begin{enumerate}
\item \textbf{The Curvature Feedback Model (CFM)} \cite{Geiger2026}: Developed from a game-theoretic framework, the CFM replaces the cosmological constant $\Lambda$ with a time-dependent curvature return potential $\Omega_\Phi(a)$, explaining accelerated expansion as a geometric ``memory'' rather than a new energy form. Tested against 1,590 Pantheon+ supernovae, the CFM yields $\Delta\chi^2 = -12.2$ relative to $\Lambda$CDM.

\item \textbf{Modified Newtonian Dynamics (MOND)} \cite{Milgrom1983}: MOND modifies gravitational dynamics at accelerations below $a_0 \approx 1.2 \times 10^{-10}$\,m/s$^2$, successfully predicting galactic rotation curves, the baryonic Tully-Fisher relation \cite{McGaugh2016}, and the radial acceleration relation \cite{Lelli2017} without invoking dark matter.
\end{enumerate}

The central question of this paper is: \textit{Can both frameworks be unified into a single model that eliminates the entire dark sector?}

\subsection{The Compatibility Question}

At first glance, CFM and MOND address different ``dark'' problems:
\begin{itemize}
\item CFM replaces \textbf{dark energy} (cosmological expansion)
\item MOND replaces \textbf{dark matter} (galactic dynamics)
\end{itemize}

However, a naive combination encounters a fundamental tension: the standard CFM fits $\Omega_m \approx 0.36$, implying substantial dark matter ($\Omega_m - \Omega_b \approx 0.31$). If MOND is correct and dark matter does not exist, the model must function with $\Omega_m = \Omega_b \approx 0.05$ alone.

\subsection{Structure Formation: Common Ground}

Both frameworks converge on a critical prediction: structures form \textit{earlier and more efficiently} than $\Lambda$CDM allows.

\begin{itemize}
\item \textbf{CFM:} The later onset of cosmic acceleration ($z_{\mathrm{acc}} = 0.52$ vs.\ $0.84$) extends the matter-dominated growth phase \cite{Geiger2026}.
\item \textbf{MOND:} Enhanced gravitational attraction at low accelerations leads to faster gravitational collapse on large scales \cite{Asencio2023}.
\end{itemize}

This shared prediction is supported by multiple observational anomalies: the JWST ``Universe Breakers'' at $z > 7$ \cite{Labbe2023, BoylanKolchin2023}, the El~Gordo cluster at $z \approx 0.87$ (${>}6\sigma$ tension with $\Lambda$CDM) \cite{Asencio2023}, and unexpectedly mature protoclusters at $z > 4$ \cite{Miller2018}.


% ===================================================================
% 2. THEORIE
% ===================================================================
\section{Theoretical Framework}
\label{sec:theory}

\subsection{The Extended Curvature Feedback Model}

In the standard CFM \cite{Geiger2026}, the Friedmann equation reads:
\begin{equation}
H^2(a) = H_0^2 \left[\Omega_m\,a^{-3} + \Omega_\Phi(a)\right]
\end{equation}
with
\begin{equation}
\Omega_\Phi(a) = \Phi_0 \cdot \frac{\tanh\!\big(k\cdot(a - a_{\mathrm{trans}})\big) + s}{1 + s}
\end{equation}

For the baryon-only extension, we decompose the geometric potential into two components:
\begin{equation}
\boxed{H^2(a) = H_0^2 \left[\mu_{\mathrm{eff}}\,\Omega_b\,a^{-3} + \underbrace{\Phi_0 \cdot f_{\mathrm{sat}}(a)}_{\text{geometric DE}} + \underbrace{\alpha \cdot a^{-\beta}}_{\text{geometric DM}} + f_{\mathrm{EDE}}(a)\right]}
\label{eq:extended_cfm}
\end{equation}

where:
\begin{itemize}
\item $\mu_{\mathrm{eff}}\,\Omega_b$ is the MOND-enhanced baryonic matter density, with $\mu_{\mathrm{eff}} \approx 1.77$ encoding the gravitational enhancement from the MOND interpolation function at cosmological scales
\item $\Phi_0 \cdot f_{\mathrm{sat}}(a)$ is the saturation-type dark energy replacement (from the Dynamic Saturation Mechanism)
\item $\alpha \cdot a^{-\beta}$ is a power-law term that assumes the \textit{cosmological} role of dark matter
\item $f_{\mathrm{EDE}}(a)$ is a parametric Early Dark Energy contribution active near recombination
\end{itemize}

The flatness constraint $H^2(a{=}1)/H_0^2 = 1$ yields:
\begin{equation}
\Omega_b + \Phi_0 \cdot f_{\mathrm{sat}}(1) + \alpha = 1
\end{equation}

\subsection{Trace Coupling and BBN Consistency}
\label{subsec:trace_coupling}

A critical constraint on the geometric DM term is Big Bang Nucleosynthesis (BBN): at $a \sim 10^{-9}$, the naive power-law $\alpha \cdot a^{-2}$ would yield $\sim 10^{18}$, completely dominating the Friedmann equation and destroying the predicted primordial element abundances. The term \textit{must} be suppressed during the radiation era.

We propose that the geometric DM term couples not to the energy density $\rho$ but to the \textit{trace of the energy-momentum tensor}:
\begin{equation}
T \equiv g^{\mu\nu} T_{\mu\nu} = -\rho + 3p = -\rho(1 - 3w)
\end{equation}

This trace has a remarkable property: for relativistic matter (radiation, $w = 1/3$), the trace vanishes exactly:
\begin{equation}
T_{\mathrm{rad}} = -\rho_{\mathrm{rad}} + 3 \cdot \tfrac{1}{3}\rho_{\mathrm{rad}} = 0
\end{equation}

This is not a coincidence but a consequence of \textit{conformal symmetry}: massless fields are conformally invariant, and the trace of a conformally invariant energy-momentum tensor vanishes identically. During the radiation-dominated era, conformal symmetry is exact, and the geometric DM term is automatically suppressed.

For non-relativistic matter ($w \approx 0$), the trace is $T_{\mathrm{mat}} = -\rho_m \neq 0$, and the geometric DM term activates. The transition occurs naturally at matter-radiation equality ($a_{\mathrm{eq}} \approx 3 \times 10^{-4}$), well after BBN ($a_{\mathrm{BBN}} \sim 10^{-9}$).

The full extended Friedmann equation with trace coupling reads:
\begin{equation}
\boxed{H^2(a) = H_0^2 \left[\Omega_b\,a^{-3} + \Phi_0 \cdot f_{\mathrm{sat}}(a) + \alpha \cdot a^{-\beta} \cdot \mathcal{S}(a)\right]}
\label{eq:extended_cfm_trace}
\end{equation}
where $\mathcal{S}(a)$ is the trace-coupling suppression factor:
\begin{equation}
\mathcal{S}(a) = \frac{|T|}{|T| + \rho_{\mathrm{rad}}} = \frac{\Omega_b\,a^{-3}}{\Omega_b\,a^{-3} + \Omega_r\,a^{-4}}
= \frac{1}{1 + (a_{\mathrm{eq}}/a)}
\label{eq:suppression}
\end{equation}
with $a_{\mathrm{eq}} = \Omega_r/\Omega_b \approx 3 \times 10^{-4}$ (using $\Omega_r \approx 9 \times 10^{-5}$). This factor satisfies:
\begin{itemize}
\item $\mathcal{S}(a \ll a_{\mathrm{eq}}) \approx a/a_{\mathrm{eq}} \to 0$ \quad (radiation era: BBN protected)
\item $\mathcal{S}(a \gg a_{\mathrm{eq}}) \approx 1$ \quad (matter/DE era: full geometric DM)
\item $\mathcal{S}(a = 1) \approx 1 - 3\times10^{-4} \approx 1$ \quad (today: SN fit unchanged)
\end{itemize}

\textbf{Impact on the Pantheon+ fit:} Since all Pantheon+ supernovae are at $z < 2.3$ ($a > 0.30$), the suppression factor is $\mathcal{S} > 0.999$ throughout the observed redshift range. The MCMC results ($\alpha$, $\beta$, $\chi^2$) are unchanged to numerical precision.

\textbf{Physical interpretation:} The trace coupling has a deep geometric meaning. In the game-theoretic framework, the geometric DM term represents the curvature ``memory'' of the initial energy concentration. During the radiation era, the universe is conformally flat (radiation is scale-free), and there is no curvature memory to sustain. The geometric DM term activates only when conformal symmetry is broken by the emergence of massive (non-relativistic) matter -- precisely at the epoch when CDM would begin to form structures in the standard picture.

\subsection{Physical Interpretation of the Geometric DM Term}

The term $\alpha \cdot a^{-\beta}$ with $\beta \approx 2.0$ (from MCMC) requires physical interpretation:

\begin{enumerate}
\item \textbf{Scaling behavior:} The MCMC posterior yields $\beta = 2.02 \pm 0.20$, consistent with curvature-like scaling ($a^{-2}$, i.e., $\beta = 2$). This is the scaling of spatial curvature in the Friedmann equation, suggesting a geometric rather than material origin.

\item \textbf{Game-theoretic interpretation:} In the spieltheoretischen framework, this term represents a second equilibrium mechanism: while the saturation term describes the ``releasing brake'' (dark energy), the power-law term describes the ``geometric inertia'' of the curvature return -- a residual geometric effect that decays with expansion but slower than matter.

\item \textbf{Connection to MOND:} In the relativistic MOND theory AeST (Aether Scalar Tensor) of Skordis \& Z{\l}o\'snik \cite{Skordis2021}, a scalar field and a vector field produce an effective energy-momentum tensor that modifies the expansion history. The power-law term $\alpha \cdot a^{-\beta}$ may be interpretable as the cosmological imprint of this MOND-like modification.

\item \textbf{Effective equation of state:} The geometric DM term has an effective equation of state $w_{\mathrm{DM,geom}} = \beta/3 - 1 = -0.33 \pm 0.07$, virtually identical to the curvature equation of state ($w_k = -1/3$). The ``dark matter'' component is indistinguishable from spatial curvature.
\end{enumerate}

\subsection{MOND on Galactic vs.\ Cosmological Scales}

A key distinction must be maintained:
\begin{itemize}
\item \textbf{Galactic scales:} MOND modifies the gravitational force law below $a_0$, explaining rotation curves and the Tully-Fisher relation \textit{without dark matter}.
\item \textbf{Cosmological scales:} The extended CFM replaces dark matter's \textit{cosmological role} (contribution to $H(z)$) with a geometric potential, without requiring a particle species.
\end{itemize}

The two mechanisms are complementary: MOND handles local dynamics, while the geometric DM term handles the global expansion history.

\subsection{The Efficiency Hypothesis: Why No Dark Matter?}
\label{subsec:efficiency}

A critical question remains: the extended CFM shows that the data \textit{permit} a baryon-only universe, but why should the universe \textit{be} baryon-only? The game-theoretic framework provides a compelling answer.

In the Nash equilibrium between null space and spacetime bubble \cite{Geiger2026}, the spacetime bubble receives a finite energy budget $E_0$ from the null space. Its objective is to neutralize the concentration gradient $G$ as efficiently as possible while protecting the parent system. This creates a resource allocation problem:

\begin{itemize}
\item \textbf{Baryonic matter:} Interacts electromagnetically, forms stars, produces radiation, collapses into black holes, and generates entropy at maximal rates. Baryons are \textit{highly efficient tools} for gradient reduction.

\item \textbf{Dark matter (hypothetical):} Interacts only gravitationally. It clumps but does not radiate, does not form stars, and contributes minimally to entropy production compared to an equivalent mass of baryonic matter.
\end{itemize}

In a game-theoretically optimized universe, allocating 85\% of the energy budget to a component that barely contributes to the primary objective (entropy-driven gradient reduction) would be a \textit{strategically inferior allocation}. A Nash-optimal system maximizes entropy production per unit energy by channeling the entire budget into ``active'' (baryonic) matter.

\begin{proposition}[Efficiency Principle -- Conditional Form]
\textit{If} the Nash equilibrium between null space and spacetime bubble optimizes entropy production per unit energy (Premise~P1), and \textit{if} baryonic matter produces more entropy per unit mass than any hypothetical dark matter species (Premise~P2), \textit{then} the Nash-optimal matter content consists exclusively of baryonic matter ($\Omega_m = \Omega_b$). The gravitational effects conventionally attributed to dark matter are instead geometric consequences of the curvature return mechanism (the $\alpha \cdot a^{-\beta}$ term).
\end{proposition}

\textit{Logical structure:} The argument has the form $P1 \wedge P2 \Rightarrow S$, which is deductively valid. Premise~P2 is empirically grounded: baryons form stars, drive nucleosynthesis, and power black hole accretion, while dark matter (if it existed) would interact only gravitationally and contribute negligibly to entropy production. Premise~P1 -- that the Nash equilibrium selects for maximal entropy production -- is the \textit{hypothesis to be tested}. It is supported by the empirical success of the model ($\Delta\chi^2 = -26.3$) but is not independently proven. The Efficiency Principle is therefore a \textit{testable conditional prediction}: it would be falsified by the experimental detection of dark matter particles.

This conditional formulation avoids circularity: we do not assume the absence of dark matter; rather, we derive it as a consequence of P1, which is itself subject to empirical test.

The quantitative test is whether the geometric term $\alpha \cdot a^{-\beta}$ can reproduce all cosmological signatures traditionally attributed to dark matter (expansion history, CMB acoustic peaks, matter power spectrum). The Pantheon+ test presented below addresses the first of these.

\subsection{The Geometric Phase Transition}
\label{subsec:phase_transition}

The extended Friedmann equation~\eqref{eq:extended_cfm} contains two geometric terms: the power-law $\alpha \cdot a^{-\beta}$ and the saturation $\Phi_0 \cdot f_{\mathrm{sat}}(a)$. A key insight emerges: these are not independent phenomena but \textit{two phases of a single geometric process} -- the curvature return mechanism operating in different regimes.

\begin{enumerate}
\item \textbf{Early universe ($a \ll a_{\mathrm{trans}}$):} The curvature return is far from saturation. The geometric potential is dominated by the power-law term $\alpha \cdot a^{-2}$, which scales like spatial curvature and plays the cosmological role of ``dark matter'' -- providing gravitational scaffolding for structure formation.

\item \textbf{Transition epoch ($a \approx a_{\mathrm{trans}}$):} As the universe expands, the curvature return approaches its saturation limit $\Phi_0$. The power-law contribution decays, while the saturation term rises.

\item \textbf{Late universe ($a \gtrsim a_{\mathrm{trans}}$):} The saturation term dominates, providing a near-constant geometric potential that drives accelerated expansion -- the role conventionally attributed to ``dark energy.''
\end{enumerate}

This picture yields a natural interpretation: \textit{dark matter and dark energy are not two separate substances but two phases of the same geometric phenomenon.} In the early universe, spacetime geometry behaves like dark matter; in the late universe, the same geometry behaves like dark energy. The ``phase transition'' is the saturation of the curvature return potential.

We call this the \textbf{Decaying Dark Geometry} hypothesis: the geometric potential is a decaying remnant of the Big Bang's initial curvature concentration. Early on, it provides gravitational structure (``dark matter''). As it decays and saturates, it provides accelerated expansion (``dark energy''). There is no dark sector -- only geometry at different stages of relaxation.

\begin{definition}[Decaying Dark Geometry]
The cosmological dark sector is a single geometric phenomenon: the curvature return potential $\Omega_\Phi$ of the game-theoretic null space$\leftrightarrow$spacetime equilibrium. Its two apparent components -- dark matter ($\alpha \cdot a^{-\beta}$, dominant at early times) and dark energy ($\Phi_0 \cdot f_{\mathrm{sat}}$, dominant at late times) -- represent the unsaturated and saturated phases of the same relaxation process.
\end{definition}


\subsection{The Running Curvature Coupling}
\label{subsec:running_beta}

The SN-only analysis with constant $\beta \approx 2.02$ yields an excellent fit to the Hubble diagram but faces a fundamental tension when confronted with CMB data: the acoustic scale $\ell_A = \pi d_C(z_*)/r_s(z_*)$ is predicted as 316.9 instead of the Planck value $301.5 \pm 0.14$, a catastrophic $>100\sigma$ discrepancy. The root cause is that $\beta = 2$ scales too slowly ($a^{-2}$) compared to CDM ($a^{-3}$), making the geometric DM term subdominant at $z \gg 1$ and yielding incorrect early-universe physics.

This tension has a natural resolution within the MOND-CFM framework. Just as MOND posits two gravitational regimes separated by the acceleration scale $a_0$:
\begin{equation}
\mu(x) = \begin{cases} 1 & \text{if } x \gg 1 \quad (\text{Newton}) \\ x & \text{if } x \ll 1 \quad (\text{MOND}) \end{cases}
\end{equation}
the curvature coupling $\beta$ should transition between two regimes separated by a curvature scale:
\begin{equation}
\boxed{\beta_{\mathrm{eff}}(a) = \beta_{\mathrm{late}} + \frac{\beta_{\mathrm{early}} - \beta_{\mathrm{late}}}{1 + (a/a_t)^n}}
\label{eq:running_beta}
\end{equation}
where $\beta_{\mathrm{late}} = 2.02$ (curvature-like, from the SN fit), $\beta_{\mathrm{early}} \approx 2.8$ (approaching CDM-like $a^{-3}$), $a_t$ is the transition scale factor, and $n$ controls the transition sharpness.

\textit{Physical motivation:} At early times ($z > z_t$), the Ricci scalar $R$ is enormous and the curvature return mechanism operates at full strength, mimicking CDM. As the universe expands and curvature decreases past a critical threshold, the return potential weakens and the geometric term relaxes to its low-curvature form ($a^{-2}$). The transition redshift $z_t \approx 7\text{--}10$ coincides with the epoch when the first galaxies form and MOND effects begin to manifest on galactic scales -- the same curvature threshold governs both the cosmological background transition and the onset of modified gravity on small scales.

The extended Friedmann equation with running $\beta$ and MOND background coupling reads:
\begin{equation}
H^2(a) = H_0^2 \left[\mu_{\mathrm{eff}}\,\Omega_b\,a^{-3} + \Omega_r\,a^{-4} + \Phi_0 \cdot f_{\mathrm{sat}}(a) + \alpha \cdot a^{-\beta_{\mathrm{eff}}(a)} + f_{\mathrm{EDE}}(a)\right]
\label{eq:running_friedmann}
\end{equation}
with $\mu_{\mathrm{eff}} = \sqrt{\pi} \approx 1.77$ (Section~\ref{subsec:sqrt_pi}) and the flatness constraint $E^2(a{=}1) = 1$ determining $\Phi_0$.


\subsection{The MOND--CFM Connection}
\label{subsec:mond_cfm_connection}

The running-$\beta$ parametrization~\eqref{eq:running_beta} reveals a deep structural connection between the CFM and MOND. Milgrom's acceleration scale $a_0 \approx 1.2 \times 10^{-10}$\,m/s$^2$ is empirically related to the Hubble parameter via:
\begin{equation}
a_0 \approx \frac{c\,H_0}{2\pi} \approx \frac{c\,H_0}{5}
\end{equation}
In the CFM framework, this coincidence acquires a natural explanation: the Hubble acceleration $a_H = Hc$ sets the curvature scale, and the MOND threshold $a_0$ marks the boundary between the high-curvature (CDM-like) and low-curvature (geometric) regimes. The three phases of the CFM+MOND universe are:

\begin{enumerate}
\item \textbf{Phase~1 ($z > z_t \approx 7$, Newton/CDM regime):} $\beta_{\mathrm{eff}} \approx 2.8$, strong curvature return provides deep potential wells. On galactic scales, accelerations exceed $a_0$. The universe behaves identically to $\Lambda$CDM.

\item \textbf{Phase~2 ($z \sim z_t$, transition):} $\beta_{\mathrm{eff}}$ decreases from $\sim$3 to $\sim$2. Curvature return weakens, gravitational potentials flatten. On galactic scales, accelerations approach $a_0$ and MOND effects begin to manifest.

\item \textbf{Phase~3 ($z < z_t$, MOND/geometric regime):} $\beta_{\mathrm{eff}} \approx 2.02$, the geometric term scales as curvature ($a^{-2}$). On galactic scales, $a < a_0$ and MOND produces flat rotation curves without CDM halos. Cosmologically, the saturation term drives acceleration.
\end{enumerate}

This three-phase structure provides a causal chain: \textit{the end of curvature return triggers both the cosmological acceleration and the galactic MOND regime} -- they are two manifestations of the same geometric transition.


\subsection{The $\sqrt{\pi}$ Conjecture: Dimensional Origin of $\mu_{\mathrm{eff}}$}
\label{subsec:sqrt_pi}

The refined joint fit (Section~\ref{subsec:joint_fit}) yields $\mu_{\mathrm{eff}} = 1.77$, which is numerically indistinguishable from $\sqrt{\pi} = 1.7725$ (deviation 0.2\%). We propose that this is not a coincidence but reflects the \textit{dimensional geometry} of the gravitational phase space.

\textbf{Key observation:} The volumes of the unit $n$-sphere are:
\begin{equation}
V_1 = 2, \qquad V_2 = \pi, \qquad V_3 = \frac{4\pi}{3}
\end{equation}
The standard MOND enhancement factor for galactic rotation curves is:
\begin{equation}
\mu_{\mathrm{eff}}^{\mathrm{gal}} = \frac{V_3}{V_2} = \frac{4}{3}
\label{eq:mu_gal}
\end{equation}
-- the ratio of the 3D (sphere) to 2D (disk) gravitational phase space volumes. This is the classic MOND $4/3$ factor, appropriate for galaxies where gravity operates in 3D and the disk geometry of the galaxy projects onto a 2D plane.

For cosmological expansion, the geometry is fundamentally different: the Friedmann equation describes a homogeneous, isotropic universe where the expansion is parametrized by a single scale factor $a(t)$. The relevant geometric quantity is not a ratio of phase space volumes but the \textit{projection amplitude} of the 2-sphere onto the observational space:
\begin{equation}
\mu_{\mathrm{eff}}^{\mathrm{cosmo}} = \sqrt{V_2} = \sqrt{\pi} \approx 1.7725
\label{eq:mu_cosmo}
\end{equation}

\begin{conjecture}[$\sqrt{\pi}$ Conjecture]
The MOND gravitational enhancement at cosmological scales is $\mu_{\mathrm{eff}} = \sqrt{\pi}$, arising from the square root of the unit-circle area. This connects to the Gaussian integral $\Gamma(1/2) = \sqrt{\pi} = \int_{-\infty}^{\infty} e^{-t^2}\,dt$ and reflects the thermodynamic normalization of gravitational modes on the cosmological 2-sphere.
\end{conjecture}

This conjecture yields remarkable quantitative predictions. Setting $\mu_{\mathrm{eff}} = \sqrt{\pi}$ \textit{exactly} and fitting the remaining 6 parameters against SN + CMB + BAO data yields:
\begin{center}
\begin{tabular}{lcc}
\toprule
& CFM+MOND ($\mu = \sqrt{\pi}$) & $\Lambda$CDM \\
\midrule
$H_0$ [km/s/Mpc] & $\mathbf{69.0}$ & 67.4 \\
$r_d$ [Mpc] & 143.3 & 147.2 \\
$\ell_A$ & 301.471 & 301.428 \\
$\mathcal{R}$ & 1.7502 & 1.7496 \\
$\chi^2_{\mathrm{total}}$ & $\mathbf{704.2}$ & 710.3 \\
$\Delta\chi^2$ & $\mathbf{-6.1}$ & 0 \\
\bottomrule
\end{tabular}
\end{center}
With $H_0 = 69$\,km/s/Mpc, the model sits \textit{between} the Planck ($67.4$) and SH0ES ($73.0$) values, potentially bridging the Hubble tension. The effective baryon density is:
\begin{equation}
\Omega_{b,\mathrm{eff}} = \sqrt{\pi}\,\Omega_b \approx 1.77 \times 0.047 = 0.083
\end{equation}
and the total ``matter-like'' contribution is:
\begin{equation}
3\sqrt{\pi}\,\Omega_b \approx 0.250 \approx \Omega_{\mathrm{CDM}}^{\Lambda\mathrm{CDM}}
\end{equation}
This suggests that the dark matter density in $\Lambda$CDM ($\Omega_{\mathrm{CDM}} \approx 0.265$) is a geometric artifact: $3\sqrt{\pi}$ times the physical baryon density. With updated Planck~2018 best-fit values ($\Omega_b = 0.0493$, $\Omega_{\mathrm{CDM}} = 0.2660$), the prediction $3\sqrt{\pi}\,\Omega_b = 0.2606$ matches the observed $\Omega_{\mathrm{CDM}}$ to \textbf{1.3\%} -- a falsifiable, parameter-free prediction.

\textit{Connection to the saturation ODE:} The factor $\sqrt{\pi}$ appears naturally in the thermodynamics of the CFM. The saturation ODE $d\Omega_\Phi/da = k(1 - \Omega_\Phi^2/\Phi_0^2)$ has a $\tanh$ solution, and the partition function of this $\cosh^{-2}$ (P\"oschl-Teller) system involves Gaussian integrals with $\sqrt{\pi}$ normalization. The Gel'fand-Yaglom method applied to the P\"oschl-Teller operator on $[0,L]$ yields functional determinant ratios $\sqrt{\det} = 1.677$ for $\lambda = 1$, close to $\sqrt{\pi} = 1.7725$. The same mathematical structure that generates the saturation mechanism also determines the MOND enhancement factor on the cosmological background.


% ===================================================================
% 3. DATENANALYSE
% ===================================================================
\section{Data Analysis and Results}
\label{sec:analysis}

\subsection{Data and Methodology}

We use the Pantheon+ catalog \cite{Scolnic2022} comprising 1,590 Type~Ia supernovae with $z > 0.01$ (redshift range $0.01$--$2.26$). Luminosity distances are computed via cumulative trapezoidal integration on a fine redshift grid ($N = 2{,}000$). The nuisance parameter~$M$ is analytically marginalized. Parameter optimization uses differential evolution with L-BFGS-B polish.

\subsection{Results: Model Comparison}

\begin{table}[H]
\centering
\caption{Model comparison against 1,590 Pantheon+ supernovae.}
\label{tab:comparison}
\begin{tabular}{lcccccc}
\toprule
\textbf{Model} & $\Omega_m$ & \textbf{Params} & $\chi^2$ & $\Delta\chi^2$ & AIC & BIC \\
\midrule
$\Lambda$CDM & 0.244 & 2 & 729.0 & 0 & 733.0 & 743.7 \\
CFM Standard & 0.364 & 4 & 716.8 & $-12.2$ & 724.8 & 746.3 \\
\midrule
CFM Baryon Fixed & 0.050 & 3 & 945.5 & $+216.5$ & 951.5 & 967.6 \\
CFM Baryon Band & 0.070 & 4 & 894.7 & $+165.7$ & 902.7 & 924.1 \\
\midrule
\textbf{Extended CFM+MOND} & \textbf{0.050} & \textbf{5} & \textbf{702.7} & $\mathbf{-26.3}$ & \textbf{712.7} & 739.5 \\
\bottomrule
\end{tabular}
\end{table}

\subsection{Key Findings}

\begin{enumerate}
\item \textbf{Simple baryon-only CFM fails:} With $\Omega_m = 0.05$ and only the $\tanh$ saturation term, the fit degrades catastrophically ($\Delta\chi^2 = +216.5$). The optimizer attempts extreme parameters ($k = 86$, $a_{\mathrm{trans}} = 0.06$) to create a near-step-function, confirming that the standard CFM \textit{cannot} compensate for missing dark matter.

\item \textbf{Extended CFM succeeds spectacularly:} Adding the geometric DM term $\alpha \cdot a^{-\beta}$ restores and \textit{exceeds} the fit quality, achieving $\Delta\chi^2 = -26.3$ versus $\Lambda$CDM -- better than both the standard $\Lambda$CDM \textit{and} the standard CFM by a wide margin.

\item \textbf{Best-fit parameters (MCMC):} A full Markov Chain Monte Carlo analysis (emcee, 48 walkers, 5000 steps) yields:
\begin{itemize}
\item Saturation term: $\Phi_0 = 0.43^{+0.14}_{-0.08}$, $k = 9.8^{+6.7}_{-3.8}$, $a_{\mathrm{trans}} = 0.971^{+0.016}_{-0.031}$ ($z_{\mathrm{trans}} = 0.03$)
\item Geometric DM term: $\alpha = 0.68^{+0.02}_{-0.07}$, $\beta = 2.02^{+0.26}_{-0.14}$
\item Energy budget at $a = 1$: $\Omega_b = 0.05$, $\Omega_\Phi = 0.95$ (total geometric contribution)
\end{itemize}

\item \textbf{$\beta \approx 2.0$: Curvature scaling.} The MCMC posterior for $\beta$ peaks at $2.02 \pm 0.20$, consistent with \textit{curvature-like scaling} ($a^{-2}$, i.e., $w = -1/3$). This is a remarkable result: the data independently recover a scaling exponent that corresponds to \textit{spatial curvature}, not to a material component. The effective equation of state $w_{\mathrm{DM,geom}} = \beta/3 - 1 = -0.33$ is virtually identical to the curvature equation of state.

\item \textbf{Late saturation transition:} The saturation transition occurs very late ($z_{\mathrm{trans}} \approx 0.03$), much later than in the standard CFM ($z_{\mathrm{trans}} = 0.33$). The geometric DM term (curvature-like) dominates the early expansion, while the saturation term provides the late-time acceleration.

\item \textbf{AIC vs.\ BIC:} The $\Delta\mathrm{AIC} = -16.3$ strongly favors the extended model. The $\Delta\mathrm{BIC} = -4.2$ also favors it despite the parameter penalty (5 vs.\ 2 parameters). This is the first model in our analysis to achieve \textit{both} AIC and BIC preference over $\Lambda$CDM simultaneously.
\end{enumerate}


\subsection{Cross-Validation: Ruling Out Overfitting}
\label{subsec:crossval}

With five free parameters versus two for $\Lambda$CDM, an overfitting concern is natural. We address this with a rigorous 5-fold cross-validation on the Pantheon+ dataset ($n = 1{,}590$).

\textbf{Method:} The data are randomly split into five equal folds (seed $= 42$). For each fold, both models ($\Lambda$CDM and Extended CFM) are optimized on the remaining 80\% (training set) using differential evolution, and the predictive $\chi^2/n$ is evaluated on the held-out 20\% (test set). This procedure tests \textit{generalization}, not merely goodness-of-fit.

\begin{table}[H]
\centering
\caption{5-fold cross-validation results: predictive $\chi^2/n$ on held-out test sets.}
\label{tab:crossval}
\begin{tabular}{lcccccc}
\toprule
\textbf{Model} & \textbf{Fold 1} & \textbf{Fold 2} & \textbf{Fold 3} & \textbf{Fold 4} & \textbf{Fold 5} & $\langle\chi^2/n\rangle$ \\
\midrule
$\Lambda$CDM (2 params) & 0.467 & 0.428 & 0.461 & 0.435 & 0.468 & $0.452 \pm 0.017$ \\
Ext.\ CFM+MOND (5 params) & 0.456 & 0.428 & 0.465 & 0.411 & 0.467 & $0.445 \pm 0.022$ \\
\bottomrule
\end{tabular}
\end{table}

\textbf{Result:} The Extended CFM achieves a \textit{lower} mean predictive $\chi^2/n$ on unseen data ($\Delta\langle\chi^2/n\rangle = -0.007$). Despite having 2.5$\times$ more parameters, the model generalizes \textit{better} than $\Lambda$CDM, ruling out overfitting as an explanation for the $\Delta\chi^2 = -26.3$ improvement. The slightly higher fold-to-fold variance ($\sigma = 0.022$ vs.\ $0.017$) is expected for a more flexible model but does not indicate instability.


\subsection{Joint SN + CMB + BAO Fit with Running $\beta$}
\label{subsec:joint_fit}

The SN-only analysis demonstrates the viability of the baryon-only framework at $z \lesssim 2.3$. The decisive test, however, is joint compatibility with CMB and BAO data. We use three data sets simultaneously:

\begin{enumerate}
\item \textbf{Pantheon+ SN:} 1,590 Type~Ia supernovae ($z > 0.01$), analytically marginalized over $M$.
\item \textbf{Planck CMB (compressed):} Acoustic scale $\ell_A = 301.471 \pm 0.14$ and shift parameter $\mathcal{R} = 1.7502 \pm 0.0046$ at $z_* = 1089.80$.
\item \textbf{BAO:} 9 distance measurements from 6dFGS ($z = 0.15$), BOSS DR12 ($z = 0.38, 0.51, 0.61$), and Lyman-$\alpha$ ($z = 2.334$).
\end{enumerate}

\subsubsection{The CMB Catastrophe of Constant $\beta$}

With constant $\beta = 2.02$, the model produces $\ell_A = 316.9$ and $\mathcal{R} = 1.00$ -- both catastrophically wrong (Planck: $301.5$ and $1.75$). The sound horizon is $r_d = 200$\,Mpc instead of 147\,Mpc. The total $\Delta\chi^2 = +2{,}630$, ruling out the constant-$\beta$ model at $>50\sigma$.

\subsubsection{Running $\beta$ Alone}

Introducing the running coupling~\eqref{eq:running_beta} dramatically improves the fit. A grid scan over $(\beta_{\mathrm{early}}, a_t, \alpha, H_0)$ followed by Nelder-Mead optimization yields:

\begin{table}[H]
\centering
\caption{Running-$\beta$ CFM: optimized parameters and comparison with $\Lambda$CDM.}
\label{tab:running_beta}
\begin{tabular}{lccc}
\toprule
\textbf{Parameter} & \textbf{CFM (running $\beta$)} & \textbf{$\Lambda$CDM} & \textbf{Planck} \\
\midrule
$\beta_{\mathrm{early}}$ & 2.823 & -- & -- \\
$\beta_{\mathrm{late}}$ & 2.02 (fixed) & -- & -- \\
$a_t$ ($z_t$) & 0.0924 (9.8) & -- & -- \\
$\alpha$ & 0.628 & -- & -- \\
$H_0$ [km/s/Mpc] & 60.0 & 67.4 & 67.4 \\
\midrule
$\ell_A$ & 301.42 & 301.43 & $301.47 \pm 0.14$ \\
$\mathcal{R}$ & 1.759 & 1.750 & $1.750 \pm 0.005$ \\
$r_d$ [Mpc] & 179.3 & 147.2 & $147.2$ \\
\midrule
$\chi^2_{\mathrm{SN}}$ & 743.3 & 700.9 & -- \\
$\chi^2_{\mathrm{CMB}}$ & 3.8 & 0.1 & -- \\
$\chi^2_{\mathrm{BAO}}$ & 33.6 & 9.3 & -- \\
$\chi^2_{\mathrm{total}}$ & 780.6 & 710.3 & -- \\
$\Delta\chi^2$ & $+70.3$ & 0 & -- \\
\bottomrule
\end{tabular}
\end{table}

The running-$\beta$ model reduces $\Delta\chi^2$ from $+2{,}630$ to $+70.3$ -- a factor-of-37 improvement. The CMB observables $\ell_A$ and $\mathcal{R}$ are now within $1\sigma$ of Planck. The remaining tension comes primarily from the sound horizon ($r_d = 179$ vs.\ $147$\,Mpc), which affects the BAO fit.

\subsubsection{Combined Fit: Running $\beta$ + Early Dark Energy}

The residual $r_d$ tension is addressed by a parametric Early Dark Energy (EDE) contribution active near recombination:
\begin{equation}
f_{\mathrm{EDE}}(a) = \frac{f_{\mathrm{amp}}}{1 + (a/a_{\mathrm{EDE}})^p} - f_{\mathrm{EDE}}(1)
\end{equation}
which reduces the sound horizon by increasing $H(z)$ during the drag epoch without affecting low-$z$ observables. In the CFM framework, this term has a natural interpretation as residual curvature energy that has not yet converted to radiation at the epoch of recombination.

The combined 6-parameter fit ($\beta_{\mathrm{early}}$, $a_t$, $\alpha$, $H_0$, $f_{\mathrm{amp}}$, $a_{\mathrm{EDE}}$) yields the breakthrough result:

\begin{table}[H]
\centering
\caption{Combined CFM+MOND vs.\ $\Lambda$CDM: joint SN + CMB + BAO fit. Five model variants are compared. The preferred model (bold) uses scale-dependent $\mu(a)$ with zero EDE.}
\label{tab:combined_fit}
\begin{tabular}{lccccc}
\toprule
\textbf{Observable} & \textbf{CFM+MOND} & \textbf{CFM+MOND} & \textbf{CFM+MOND} & \textbf{CFM (no MOND)} & \textbf{$\Lambda$CDM} \\
 & \textbf{$\mu(a)$, no EDE} & \textbf{$\mu = \sqrt{\pi}$} & \textbf{$\mu = 2.10$} & \textbf{(previous)} & \\
\midrule
$\mu_{\mathrm{eff}}(z{=}0)$ & $\mathbf{\sqrt{\pi}}$ & $\sqrt{\pi}$ & 2.10 & 1.00 & -- \\
$\mu_{\mathrm{eff}}(z{=}1090)$ & $\mathbf{1.77}$ & $\sqrt{\pi}$ & 2.10 & 1.00 & -- \\
$f_{\mathrm{EDE}}(z_*)$ & $\mathbf{0\%}$ & 59\% & 52\% & 52\% & -- \\
$H_0$ [km/s/Mpc] & $\mathbf{67.3}$ & 69.0 & 75.0 & 60.0 & 67.4 \\
$r_d$ [Mpc] & $\mathbf{146.9}$ & 143.3 & 131.9 & 165.0 & 147.2 \\
Parameters & $\mathbf{6}$ & 8 & 8 & 8 & 6 \\
\midrule
$\ell_A$ & $\mathbf{301.471}$ & 301.471 & 301.472 & 301.477 & 301.428 \\
$\mathcal{R}$ & $\mathbf{1.7502}$ & 1.7502 & 1.7502 & 1.7502 & 1.7496 \\
\midrule
$\chi^2_{\mathrm{SN}}$ & $\mathbf{704.8}$ & 699.6 & 699.7 & 698.9 & 700.9 \\
$\chi^2_{\mathrm{CMB}}$ & $\mathbf{0.0}$ & 0.0 & 0.0 & 0.0 & 0.1 \\
$\chi^2_{\mathrm{BAO}}$ & $\mathbf{0.0}$ & 4.6 & 4.5 & 6.3 & 9.3 \\
\midrule
$\chi^2_{\mathrm{total}}$ & $\mathbf{704.8}$ & 704.2 & 704.2 & 705.2 & 710.3 \\
$\Delta\chi^2$ & $\mathbf{-5.5}$ & $-6.1$ & $-6.0$ & $-5.1$ & 0 \\
\bottomrule
\end{tabular}
\end{table}

\textbf{Key results:}
\begin{enumerate}
\item \textbf{EDE eliminated:} The preferred variant uses scale-dependent $\mu(a)$ (Section~\ref{subsubsec:mu_scale}) that transitions from $\mu = \sqrt{\pi}$ today to $\mu \to 1$ at $z > 4000$. This \textbf{completely eliminates} the EDE component ($f_{\mathrm{EDE}} = 0$), reducing the model to 6 free parameters -- the same count as $\Lambda$CDM.

\item \textbf{Hubble constant resolved:} With scale-dependent $\mu(a)$, $H_0 = 67.3$\,km/s/Mpc -- within $0.2$\,km/s/Mpc of the Planck value ($67.4 \pm 0.5$). The constant-$\mu$ variants can reach any $H_0$ between 58 and 85\,km/s/Mpc by adjusting $\mu_{\mathrm{eff}}$.

\item \textbf{Sound horizon solved:} The sound horizon $r_d = 146.9$\,Mpc is \textit{essentially identical} to $\Lambda$CDM ($147.2$\,Mpc, deviation 0.2\%), compared to the previous 12\% discrepancy ($r_d = 165$\,Mpc).

\item \textbf{CMB:} The acoustic scale $\ell_A = 301.471$ and the shift parameter $\mathcal{R} = 1.7502$ match Planck \textit{exactly}.

\item \textbf{Overall:} $\Delta\chi^2 = -5.5$ (preferred $\mu(a)$ variant, no EDE) to $-6.1$ (constant $\mu = \sqrt{\pi}$ with EDE) -- the CFM+MOND \textit{outperforms} $\Lambda$CDM across the board.
\end{enumerate}

The best-fit parameters (preferred $\mu(a)$ variant) are: $\mu_{\mathrm{late}} = \sqrt{\pi}$, $\mu_{\mathrm{early}} = 1.00$, $a_\mu = 2.55 \times 10^{-4}$ ($z_\mu = 3918$), $\beta_{\mathrm{early}} = 2.82$, $a_t = 0.098$ ($z_t = 9.2$), $\alpha = 0.695$, $H_0 = 67.3$\,km/s/Mpc, $f_{\mathrm{EDE}} = 0$.

\subsubsection{The MOND Background Coupling $\mu_{\mathrm{eff}}$}
\label{subsubsec:mu_eff}

The central innovation of the refined fit is applying the MOND gravitational enhancement on the \textit{background level} of the Friedmann equation:
\begin{equation}
H^2(a) = H_0^2 \left[\mu_{\mathrm{eff}}\,\Omega_b\,a^{-3} + \Omega_r\,a^{-4} + \Phi_0 \cdot f_{\mathrm{sat}}(a) + \alpha \cdot a^{-\beta_{\mathrm{eff}}(a)} + f_{\mathrm{EDE}}(a)\right]
\label{eq:mond_friedmann}
\end{equation}
with the modified sound horizon:
\begin{equation}
r_d = \int_0^{a_d} \frac{c_s(a)}{a^2 H(a)} \, da, \quad c_s = \frac{1}{\sqrt{3(1 + R_b \cdot a)}}, \quad R_b = \frac{3\,\mu_{\mathrm{eff}}\,\Omega_b}{4\,\Omega_\gamma}
\label{eq:mond_rs}
\end{equation}

A $\mu_{\mathrm{eff}}$-profile analysis (optimizing all other parameters at each fixed $\mu_{\mathrm{eff}}$) reveals a clear $\mu$--$H_0$ relationship:

\begin{center}
\begin{tabular}{ccc}
\toprule
$\mu_{\mathrm{eff}}$ & $H_0$ [km/s/Mpc] & $r_d$ [Mpc] \\
\midrule
1.00 & 60.0 & 165 \\
1.33 & 58.2 & 170 \\
1.50 & 64.0 & 155 \\
1.60 & 71.4 & 139 \\
1.77 & 66.0 & 150 \\
2.00 & 84.5 & 117 \\
\bottomrule
\end{tabular}
\end{center}

The non-monotonic behavior (dip at $\mu = 1.33$, then sharp rise) reflects the interplay between the MOND coupling and the EDE fraction: higher $\mu_{\mathrm{eff}}$ reduces $r_d$, which relaxes the $\ell_A$ constraint and permits higher $H_0$.

\subsubsection{Scale-Dependent $\mu(a)$: Eliminating EDE}
\label{subsubsec:mu_scale}

The constant-$\mu_{\mathrm{eff}}$ variants all require a large EDE fraction ($f_{\mathrm{EDE}} \approx 50$--$60\%$). This can be entirely eliminated by promoting $\mu$ to a \textit{scale-dependent} function:
\begin{equation}
\mu(a) = \mu_{\mathrm{late}} + \frac{\mu_{\mathrm{early}} - \mu_{\mathrm{late}}}{1 + (a/a_\mu)^4}
\label{eq:mu_running}
\end{equation}
with $\mu_{\mathrm{late}} = \sqrt{\pi}$ (the $\sqrt{\pi}$ Conjecture value, Section~\ref{subsec:sqrt_pi}) and $\mu_{\mathrm{early}}$, $a_\mu$ as free parameters. This parametrization is the \textit{exact analogue} of the running $\beta$ (Eq.~\ref{eq:running_beta}): just as $\beta$ transitions from CDM-like to curvature-like behavior, $\mu$ transitions from standard gravity ($\mu = 1$, no MOND enhancement) at very early times to full MOND enhancement ($\mu = \sqrt{\pi}$) at late times.

\textbf{Physical motivation:} At $z > 4000$, the cosmological acceleration is far above the MOND scale $a_0$, so standard gravity applies ($\mu \to 1$). As the universe expands and the effective acceleration drops below $a_0$, the MOND enhancement gradually activates ($\mu \to \sqrt{\pi}$). This is precisely the cosmological analogue of the galactic MOND transition.

The fit with scale-dependent $\mu(a)$ and \textbf{zero EDE} yields:
\begin{center}
\begin{tabular}{lcc}
\toprule
& CFM+MOND ($\mu(a)$, no EDE) & $\Lambda$CDM \\
\midrule
$H_0$ [km/s/Mpc] & $\mathbf{67.3}$ & 67.4 \\
$r_d$ [Mpc] & $\mathbf{146.9}$ & 147.2 \\
$\ell_A$ & 301.471 & 301.428 \\
$\mathcal{R}$ & 1.7502 & 1.7496 \\
$\chi^2_{\mathrm{total}}$ & $\mathbf{704.8}$ & 710.3 \\
$\Delta\chi^2$ & $\mathbf{-5.5}$ & 0 \\
Parameters & 6 & 6 \\
\bottomrule
\end{tabular}
\end{center}

The $\mu(a)$-profile at different redshifts:
\begin{center}
\begin{tabular}{rc}
\toprule
$z$ & $\mu(z)$ \\
\midrule
0 & 1.772 \\
1 & 1.772 \\
10 & 1.772 \\
100 & 1.772 \\
500 & 1.772 \\
1090 & 1.768 \\
5000 & 1.212 \\
\bottomrule
\end{tabular}
\end{center}

\textbf{Key observations:}
\begin{enumerate}
\item The MOND enhancement is essentially constant ($\mu \approx \sqrt{\pi}$) from $z = 0$ to $z \approx 1000$, only deviating at $z > 3000$.
\item The transition redshift $z_\mu \approx 3918$ is \textit{between} recombination ($z_* = 1090$) and the matter-radiation equality epoch ($z_{\mathrm{eq}} \approx 3400$).
\item The 6 free parameters ($\beta_{\mathrm{early}}$, $a_t$, $\alpha$, $H_0$, $\mu_{\mathrm{early}}$, $a_\mu$) are the \textit{same count} as $\Lambda$CDM ($\Omega_b h^2$, $\Omega_c h^2$, $H_0$, $\tau$, $n_s$, $A_s$).
\item $H_0 = 67.3$\,km/s/Mpc and $r_d = 146.9$\,Mpc are \textit{essentially identical} to the $\Lambda$CDM values.
\end{enumerate}

This result represents the most parsimonious version of the CFM+MOND framework: it eliminates not only the dark sector but also the ad-hoc EDE component, reducing the model to a minimal set of geometric parameters.

\subsubsection{Remaining Caveats}
\label{subsubsec:caveats}

Two aspects require further investigation:
\begin{enumerate}
\item \textbf{Perturbation theory:} The background observables ($\ell_A$, $\mathcal{R}$, $r_d$, $H_0$) are fully matched. Using hi\_class \cite{Zumalacarregui2017} with $f(R)$-type perturbation modifications ($\alpha_M = 0.0007$, $\alpha_B = -\alpha_M/2$, $\alpha_T = 0$), we obtain -- directly verified:
\begin{itemize}
\item $\ell_1 = 220$ -- \textbf{exact Planck match} (via early ISW effect)
\item $\mathcal{P}_3/\mathcal{P}_1 = 0.4295$ -- \textbf{exact Planck match} (via $\beta_{\mathrm{early}} = 2.83$)
\end{itemize}
BBN is fully consistent ($\Delta N_{\mathrm{eff}} \approx 0$). The angular acoustic scale offset ($100\,\theta_s = 1.034$ vs.\ Planck's $1.041$, $0.69\%$) is \textbf{resolved by Paper~III}: the scalaron's background energy density evolves as $a^{-3}$ (CDM-like), giving $100\,\theta_s = 1.04173$ for all $\alpha_M$ values in \texttt{hi\_class}. The offset arose from this paper's phenomenological $w_{\mathrm{eff}} \approx -0.06$ assumption; the Lagrangian treatment yields $w \ll 0.01$.

\item \textbf{Lagrangian derivation:} Both the running $\beta_{\mathrm{eff}}(a)$ and the running $\mu(a)$ are phenomenological transition functions. A derivation from an action principle -- potentially connecting the $\sqrt{\pi}$ conjecture to the CFM saturation mechanism -- is required for theoretical completeness (Paper~III).
\end{enumerate}


% ===================================================================
% 4. DISKUSSION
% ===================================================================
\section{Discussion}
\label{sec:discussion}

\subsection{A Universe Without a Dark Sector}

The extended CFM demonstrates that the entire expansion history probed by Type~Ia supernovae can be described with:
\begin{itemize}
\item Baryonic matter ($\Omega_b = 0.05$) -- the \textit{only} material content
\item A saturation-type geometric potential -- replacing dark energy
\item A power-law geometric term -- replacing dark matter's cosmological role
\end{itemize}

If this result survives tests against CMB and BAO data, it would imply that 95\% of the $\Lambda$CDM energy budget is an artifact of interpreting geometric effects as material components.

\subsection{The $\beta \approx 2.0$ Result: Curvature as Dark Matter}

The MCMC posterior for the scaling exponent yields $\beta = 2.02^{+0.26}_{-0.14}$, remarkably close to -- and statistically consistent with -- the curvature scaling $\beta = 2$ ($a^{-2}$). This corresponds to an effective equation of state $w_{\mathrm{DM,geom}} = -0.33$, indistinguishable from spatial curvature ($w_k = -1/3$). For comparison, the standard cosmological components scale as:
\begin{itemize}
\item Matter: $\beta = 3$ ($a^{-3}$, $w = 0$)
\item Curvature: $\beta = 2$ ($a^{-2}$, $w = -1/3$) $\quad\leftarrow$ \textbf{recovered by MCMC}
\item Radiation: $\beta = 4$ ($a^{-4}$, $w = 1/3$)
\end{itemize}

This result has profound implications: the component traditionally identified as ``dark matter'' in the Friedmann equation may in fact be \textit{spatial curvature} -- not the global curvature $k$ of the FLRW metric, but a \textit{dynamic, decaying curvature memory} encoded in the geometric potential. In the game-theoretic framework, this is precisely the ``geometric inertia'' of the curvature return: a residual imprint of the Big Bang's energy concentration that dilutes with expansion at the curvature rate $a^{-2}$ rather than the matter rate $a^{-3}$.

\subsection{Relation to AeST and Relativistic MOND}

The relativistic MOND theory AeST (Aether Scalar Tensor) \cite{Skordis2021} provides the only known framework that simultaneously:
\begin{enumerate}
\item Reproduces MOND dynamics on galactic scales
\item Fits the CMB power spectrum (including the third acoustic peak)
\item Fits the matter power spectrum
\end{enumerate}

AeST achieves this through a scalar field $\phi$ and a timelike vector field $A_\mu$ that produce an effective energy-momentum tensor. The cosmological background equations in AeST contain terms that contribute to $H^2(a)$ with non-standard scaling. A detailed comparison between the AeST background equations and the extended CFM Friedmann equation~\eqref{eq:extended_cfm} is a key objective for future work.

\subsection{Addressing the Cosmological ``Endgegner''}
\label{subsec:endgegner}

Any theory that eliminates dark matter must confront three critical observational pillars of $\Lambda$CDM. We address each in turn, showing how the Decaying Dark Geometry hypothesis provides a pathway through each challenge.

\subsubsection{Challenge 1: CMB Acoustic Peaks}

The relative heights of the CMB acoustic peaks -- particularly the ratio of the first to the third peak -- are conventionally interpreted as evidence for a gravitational component that does not interact with photons (i.e., dark matter). In $\Lambda$CDM, cold dark matter provides gravitational potential wells that drive baryon-photon oscillations without experiencing radiation pressure.

\textit{Quantitative assessment:} With constant $\beta = 2.02$, the geometric DM term contributes only $\sim$0.5\% of the baryonic density at $z_*$, because $a^{-2}$ scales slower than matter ($a^{-3}$). However, the running-$\beta$ extension (Section~\ref{subsec:running_beta}) resolves this: with $\beta_{\mathrm{eff}}(z_* = 1090) \approx 2.78$, the geometric term scales nearly as $a^{-2.8}$ -- close to CDM ($a^{-3}$) -- and contributes substantially at recombination. The effective matter density $\Omega_{m,\mathrm{eff}}$ at $z_* \approx 1090$ is 0.57 in the combined fit (Table~\ref{tab:combined_fit}), providing sufficient gravitational depth for baryon-photon oscillations. At intermediate redshifts ($z \sim 2$), $\Omega_{m,\mathrm{eff}} \approx 0.31$ -- nearly identical to $\Lambda$CDM's $\Omega_m = 0.315$.

The \textit{background-level} CMB observables are now fully reproduced: $\ell_A = 301.477$ and $\mathcal{R} = 1.7502$ match Planck to $<0.1\sigma$. The remaining question is whether the \textit{perturbation-level} effects (peak heights, damping tail) are also correctly reproduced.

\textit{However, this is not the relevant mechanism.} The CMB acoustic peaks are determined by \textit{metric perturbations} $\Phi$ and $\Psi$, not by background density contributions. In the extended CFM, the Lagrangian (Paper~III \cite{Geiger2026c}) contains an $R^2$ term and a scalar field $\phi$ with P\"oschl-Teller potential, both of which modify the perturbation equations independently of their background contribution.

\textit{Existence proof from AeST:} The relativistic MOND theory AeST \cite{Skordis2021} has demonstrated that a baryon-only universe ($\Omega_m = \Omega_b$) with additional geometric degrees of freedom (scalar + vector fields) can fit the \textit{full} CMB power spectrum, including the third acoustic peak. The extended CFM shares the essential ingredients with AeST:
\begin{itemize}
\item Baryon-only matter content ($\Omega_m = \Omega_b \approx 0.05$)
\item A scalar field providing additional gravitational potentials at the perturbative level
\item Trace coupling ensuring suppression during the radiation era (conformal symmetry protection)
\item Additional geometric degrees of freedom ($R^2$ term in CFM vs.\ vector field in AeST)
\end{itemize}

The AeST precedent establishes that the \textit{principle} of CMB compatibility without CDM is proven. Moreover, the running-$\beta$ extension (Section~\ref{subsec:running_beta}) now provides a \textit{quantitative} demonstration of CMB background compatibility: the joint fit (Section~\ref{subsec:joint_fit}) yields $\ell_A = 301.477$ and $\mathcal{R} = 1.7502$, matching the Planck values to better than $0.1\sigma$. This resolves the background-level CMB challenge completely.

\textit{Perturbation analysis:} Using an ``effective CDM'' mapping in both CAMB \cite{Lewis2000} and hi\_class \cite{Zumalacarregui2017}, where the combined $\mu(a)$-enhanced baryonic density and the geometric term at $z_* = 1090$ are modeled as effective cold dark matter ($\Omega_{m,\mathrm{eff}} = \mu(z_*)\,\Omega_b + \alpha\,a_*^{3-\beta(z_*)} = 0.285$), we obtain the following $C_\ell$ diagnostics:
\begin{itemize}
\item First acoustic peak position: $\ell_1 = 223$ (Planck: 220, $\Delta\ell = 3$)
\item Third-to-first peak ratio: $\mathcal{P}_3/\mathcal{P}_1 = 0.4212$ (Planck: 0.4295, i.e.\ \textbf{98.1\%} of the observed value)
\item Angular acoustic scale: $100\,\theta_s = 1.025$ (Planck: $1.041 \pm 0.0003$)
\end{itemize}

\textit{Physical origin of the offset:} The geometric term has effective equation of state $w_{\mathrm{eff}} = (\beta - 3)/3 \approx -0.06$ at recombination -- nearly, but not exactly, pressureless. This causes $\rho_{\mathrm{geom}} \propto a^{-2.82}$ rather than $a^{-3}$, producing a 2.5\% larger sound horizon ($r_s = 148.1$\,Mpc vs.\ $\Lambda$CDM: 144.5\,Mpc) and the observed $\theta_s$ offset.

\textit{Optimized solution:} A minimal adjustment of $\beta_{\mathrm{early}} = 2.82 \to 2.829$ (\textbf{0.32\%} change) increases $\omega_{c,\mathrm{eff}}$ from 0.1066 to 0.1125, yielding:
\begin{itemize}
\item $\mathcal{P}_3/\mathcal{P}_1 = 0.4295$ -- \textbf{exact Planck match} (100.0\%)
\item $\ell_1 = 222$ (improved from 223, remaining offset $\Delta\ell = 2$)
\item $\chi^2$ improvement: $4578 \to 1555$ (66\% reduction)
\end{itemize}

\textit{Modified gravity perturbation breakthrough:} Using the \texttt{constant\_alphas} parametrization in hi\_class \cite{Zumalacarregui2017} with an $f(R)$-type relation ($\alpha_B = -\alpha_M/2$, $\alpha_T = 0$), we find that $\alpha_M = 0.0008$ shifts $\ell_1$ from 223 to \textbf{220 (exact Planck)} through the early ISW effect -- \textit{without changing $\theta_s$}. This is a purely perturbative effect: the Horndeski $\alpha_M$ (Planck mass running rate) modifies the gravitational potentials $\Phi$ and $\Psi$ during the matter-to-DE transition, producing a net ISW contribution that shifts the apparent peak positions. The background quantities ($\theta_s$, $r_s$, $D_A$) remain unchanged. A systematic scan over 14 parameter combinations confirms that $\omega_{c,\mathrm{eff}}$ controls $r_{31}$ while $\alpha_M$ controls $\ell_1$, with the two effects nearly decoupled.

\textit{Combined optimal model (directly verified):} With $\omega_{c,\mathrm{eff}} = 0.1143$ ($\beta_{\mathrm{early}} = 2.834$, 0.49\% adjustment) and $\alpha_M = 0.0007$ ($f(R)$ relation):
\begin{itemize}
\item $\ell_1 = 220$ -- \textbf{exact Planck match}
\item $\mathcal{P}_3/\mathcal{P}_1 = 0.4295$ -- \textbf{exact Planck match}
\item $100\,\theta_s = 1.034$ (offset reduced from 1.55\% to \textbf{0.69\%})
\end{itemize}
The $f(R)$ relation ($\alpha_B = -\alpha_M/2$) follows naturally from the $R^2$ structure of the CFM Lagrangian, and $\alpha_T = 0$ ensures $c_{\mathrm{gw}} = c$, consistent with LIGO/Virgo GW170817 \cite{Abbott2017gw}. The physical interpretation is that the CFM's curvature feedback modifies $G_{\mathrm{eff}}$ by $\sim$0.08\% per e-fold of expansion.

\textit{Resolution (Paper~III):} The $\theta_s$ offset is \textbf{resolved} by the Lagrangian framework of Paper~III. The scalaron (from the $R^2$ action) is a massive scalar field whose background energy density evolves as $\rho \propto a^{-3}$ -- precisely like CDM. Systematic \texttt{hi\_class} analysis confirms: $100\,\theta_s = 1.04173$ ($+0.06\%$ from Planck) for \textit{all} $\alpha_M$ values, with $r_s = 147.10\,\text{Mpc}$ matching the Planck $\Lambda$CDM value exactly. The $0.69\%$ offset arose from this paper's phenomenological equation of state $w \approx -0.06$; the true scalaron has $w \ll 0.01$.

\subsubsection{Challenge 2: The Bullet Cluster}

The Bullet Cluster (1E~0657-56) is often cited as definitive evidence for particulate dark matter: gravitational lensing maps show mass concentrations offset from the X-ray-emitting gas after a cluster collision \cite{Clowe2006}. The argument is that dark matter, being collisionless, passed through while the gas was slowed by ram pressure.

\textit{Resolution:} In the Decaying Dark Geometry framework, the ``dark matter'' component is \textit{spacetime geometry}, not a substance. At the Bullet Cluster redshift ($z = 0.296$, $a = 0.77$), the background ratio of geometric DM to baryonic matter is:
\begin{equation}
\frac{\alpha \cdot a^{-\beta}}{\Omega_b \cdot a^{-3}} \bigg|_{z=0.296} = \frac{0.68 \times 0.77^{-2.02}}{0.05 \times 0.77^{-3}} \approx 10.6
\end{equation}
For comparison, the CDM-to-baryon ratio in $\Lambda$CDM is $\Omega_{\mathrm{cdm}}/\Omega_b \approx 5.4$. The geometric potential is thus \textit{quantitatively sufficient} to provide the required lensing convergence at this epoch.

During a cluster collision:
\begin{itemize}
\item The \textit{baryonic gas} experiences ram pressure and is decelerated.
\item The \textit{geometric potential} is a property of the spacetime curvature distribution, which is sourced by the total energy distribution \textit{including its own history}. As a geometric ``memory,'' it traces the pre-collision mass distribution and need not track the post-collision gas distribution instantaneously.
\item The \textit{galaxies} (stellar component), being effectively collisionless like the geometric potential, pass through unimpeded.
\end{itemize}

The lensing signal would then trace the geometric potential (which co-moves with the galaxies) rather than the gas -- precisely as observed. This is analogous to the AeST prediction, where the scalar and vector fields produce lensing effects that are offset from the gas. A quantitative lensing prediction requires solving the perturbation equations of the full CFM Lagrangian (Paper~III \cite{Geiger2026c}), but the background-level analysis confirms that the geometric potential is of the correct magnitude.

\subsubsection{Challenge 3: Structure Formation and the Matter Power Spectrum}

The matter power spectrum $P(k)$ in $\Lambda$CDM is shaped by dark matter halos that begin gravitational collapse during radiation domination (before baryons decouple from photons). Without early-collapsing dark matter, baryonic structures would form too late and on the wrong scales.

\textit{Resolution:} The geometric DM term provides ``geometric scaffolding'' for structure formation:
\begin{itemize}
\item At early times ($a \ll a_{\mathrm{trans}}$), the $\alpha \cdot a^{-2}$ term dominates the expansion history, providing the same deceleration that CDM would provide (albeit with a different scaling).
\item Perturbations in the geometric potential create gravitational wells into which baryons can fall after recombination, just as CDM halos would.
\item The earlier onset of effective gravity (from the combined CFM + MOND enhancement) naturally explains the ``too early, too massive'' structures observed by JWST \cite{Labbe2023}, El~Gordo \cite{Asencio2023}, and high-$z$ protoclusters \cite{Miller2018} -- which are anomalous in $\Lambda$CDM but expected in this framework.
\end{itemize}

A preliminary $P(k)$ analysis using the ``effective CDM'' mapping in CAMB \cite{Lewis2000} confirms the correct qualitative shape: the turnover scale ($k_{\mathrm{peak}} \approx 0.015$\,$h$/Mpc) is close to $\Lambda$CDM (0.017), and the transfer function ratio $T_{\mathrm{CFM}}/T_{\Lambda\mathrm{CDM}}$ is approximately 1.1--1.3 across all scales. The epoch-dependent effective matter density $\Omega_{m,\mathrm{eff}}(z)$ matches $\Lambda$CDM \textit{exactly} at $z \approx 500$ ($\Omega_{m,\mathrm{eff}} = 0.315$), providing a natural explanation for the observed large-scale structure. The quantitative prediction of $P(k)$ with the full perturbation equations and correct growth rates is a key objective for future work.

\subsubsection{Extended Numerical Validation: TT+TE+EE, $f\sigma_8$, and $S_8$}
\label{subsubsec:extended_validation}

Beyond the CMB peak positions, the \texttt{propto\_omega} parametrization in hi\_class \cite{Zumalacarregui2017} -- where $\alpha_i(a) = c_i \cdot \Omega_{\mathrm{DE}}(a)$, ensuring $\alpha_i \to 0$ at recombination -- enables a comprehensive analysis against full Planck 2018 power spectra and large-scale structure data. All cosmological parameters are fixed to Planck 2018 best-fit; the only additional parameter is $c_M$.

\textit{TT + TE + EE power spectra.} Using 6{,}405 Planck data points ($\ell = 30$--$2500$: 2{,}471 TT + 1{,}967 TE + 1{,}967 EE), the CFM with $c_M = 0.0002$ achieves $\Delta\chi^2_{\mathrm{tot}} = -0.2$ against $\Lambda$CDM. Larger $c_M$ values improve further ($\Delta\chi^2 = -1.7$ for \texttt{propto\_scale} $c_M = 0.0005$). Crucially, the improvement arises primarily from TT; the polarization spectra TE and EE are essentially unchanged ($\Delta\chi^2 < 0.1$), demonstrating full consistency with polarization data.

\textit{Growth rate $f\sigma_8(z)$.} The linear growth rate $f\sigma_8 = f(z) \cdot \sigma_8(z)$ is a key discriminant for modified gravity. At $z = 0.38$ (BOSS LOWZ), the CFM with $c_M = 0.0002$ predicts $f\sigma_8 = 0.495$, which is \textit{closer} to the BOSS measurement ($0.497 \pm 0.045$) than $\Lambda$CDM ($0.475$). The overall RSD $\chi^2$ is $1.78$ (8 data points), confirming compatibility with redshift-space distortion surveys.

\textit{$S_8$ and weak lensing.} The combined parameter $S_8 = \sigma_8\sqrt{\Omega_m/0.3}$ at the recommended point ($c_M = 0.0002$, $\sigma_8 = 0.826$) is $S_8 = 0.847$. This is consistent with the CMB baseline (Planck + ACT + SPT: $S_8 = 0.836 \pm 0.013$, $0.8\sigma$), KiDS-Legacy ($S_8 = 0.815 \pm 0.021$, $1.5\sigma$), and eROSITA clusters ($S_8 = 0.860 \pm 0.010$, $1.3\sigma$). The tension with DES~Y6 ($S_8 = 0.789 \pm 0.012$, $4.8\sigma$) mirrors the broader ``$S_8$ bifurcation'' observed between KiDS and DES, which is under active investigation as a potential systematic effect. Euclid will provide the decisive measurement.

\textit{DESI DR2 BAO.} The DESI Data Release~2 (2026) reports $w_0 = -0.42 \pm 0.21$ and $w_a = -1.75 \pm 0.58$, constituting a $3.1\sigma$ detection of dynamical dark energy. The CFM effective equation of state $w_{\mathrm{eff}}(z=0) \approx -0.33$ lies within $0.4\sigma$ of the DESI $w_0$ value. Both the CFM and DESI data independently disfavor $w = -1$ (cosmological constant), providing qualitative support for the curvature feedback mechanism.

\subsection{Ontological Interpretation: The Nested Hierarchy}
\label{subsec:nested}

The Decaying Dark Geometry hypothesis suggests a nested ontological structure implicit in the game-theoretic framework \cite{Geiger2026}:

\begin{enumerate}
\item \textbf{Null space} (``Mother''): The pre-geometric ground state whose concentration gradient $G$ drives the emergence of the spacetime bubble.
\item \textbf{Spacetime geometry} (``Daughter''): The curvature return potential, manifesting as geometric DM ($\alpha \cdot a^{-\beta}$, early) and geometric DE ($\Phi_0 \cdot f_{\mathrm{sat}}$, late). The ``dark sector'' \textit{is} the geometry.
\item \textbf{Baryonic matter} (``Granddaughter''): The Nash-optimal entropy-producing agents condensed within the geometric substrate.
\end{enumerate}

This hierarchy -- Null Space $\to$ Geometry $\to$ Matter -- inverts the conventional materialist ontology and yields a testable consequence: the geometric ``dark matter'' cannot be detected in particle experiments, because it is the spacetime geometry itself.

\subsubsection{The Missing Lagrangian}

A fourth, theoretical challenge remains: the extended CFM currently lacks a Lagrangian formulation. The ODE $d\Omega_\Phi/da = k[1 - (\Omega_\Phi/\Phi_0)^2]$ and the power-law term $\alpha \cdot a^{-\beta}$ are phenomenological. A complete theory requires:
\begin{enumerate}
\item An action principle from which the extended Friedmann equation~\eqref{eq:extended_cfm} follows as the Euler-Lagrange equation
\item A microscopic derivation explaining \textit{why} the saturation ODE takes the specific form $dX/da \propto (1 - X^2)$
\item A connection to known quantum gravity approaches (Loop Quantum Gravity, Finsler geometry, information-theoretic spacetime)
\end{enumerate}

This theoretical foundation is the subject of Paper~III \cite{Geiger2026c}.


\subsection{Critical Self-Assessment: Too Good to Be True?}

The results of this paper -- a baryon-only model that outperforms $\Lambda$CDM on SN data ($\Delta\chi^2 = -26.3$) and achieves joint SN+CMB+BAO compatibility ($\Delta\chi^2 = -5.5$) with $H_0 = 67.3$\,km/s/Mpc and \textit{zero} EDE -- are remarkable, and a degree of skepticism is warranted. We enumerate the reasons for caution:

\begin{enumerate}
\item \textbf{Overfitting risk for SN-only fit -- ruled out by cross-validation:} The SN-only analysis uses 5 free parameters (vs.\ 2 for $\Lambda$CDM). Beyond information-theoretic penalties ($\Delta\mathrm{AIC} = -16.3$, $\Delta\mathrm{BIC} = -4.2$), a rigorous 5-fold cross-validation (Section~\ref{subsec:crossval}) confirms generalization ($0.445 \pm 0.022$ vs.\ $0.452 \pm 0.017$).

\item \textbf{Parameter count -- now resolved:} The preferred $\mu(a)$ variant uses 6 free parameters ($\beta_{\mathrm{early}}$, $a_t$, $\alpha$, $H_0$, $\mu_{\mathrm{early}}$, $a_\mu$), the same count as $\Lambda$CDM. The EDE component is entirely eliminated. A formal information-criterion comparison is straightforward: with equal parameter count and $\Delta\chi^2 = -5.5$, the CFM+MOND is statistically preferred.

\item \textbf{EDE -- now eliminated:} The previously problematic $f_{\mathrm{EDE}} \approx 50\%$ is reduced to \textit{zero} by the scale-dependent $\mu(a)$ parametrization (Section~\ref{subsubsec:mu_scale}). The constant-$\mu$ variants still require large EDE, but the preferred variant does not.

\item \textbf{Phenomenological nature of running $\beta$ and $\mu(a)$:} Both transitions~\eqref{eq:running_beta} and~\eqref{eq:mu_running} are fitting functions, not derived from first principles. A Lagrangian derivation (Paper~III) is essential.

\item \textbf{Perturbation theory:} All background observables ($\ell_A$, $\mathcal{R}$, $r_d$, $H_0$) are now matched. The preliminary ``effective CDM'' analysis is encouraging ($\mathcal{P}_3/\mathcal{P}_1 = 0.421$, 97.9\% of Planck), the $P(k)$ shape is qualitatively correct, and BBN is fully consistent ($\Delta N_{\mathrm{eff}} \approx 0$). However, the full $C_\ell$ and $P(k)$ with modified perturbation equations remain uncomputed and constitute the decisive remaining test.
\end{enumerate}

\textbf{Honest assessment:} The scale-dependent $\mu(a)$ resolves all previously critical issues: $H_0$ ($60 \to 67.3$\,km/s/Mpc), sound horizon ($165 \to 146.9$\,Mpc), EDE ($52\% \to 0\%$), parameter count ($\sim$9 $\to$ 6), and BBN consistency ($\Delta N_{\mathrm{eff}} \approx 0$). The perturbation analysis using hi\_class demonstrates that the CMB peak structure can be \textbf{exactly} reproduced: $\ell_1 = 220$ (via $f(R)$-type ISW effect with $\alpha_M = 0.0007$) and $\mathcal{P}_3/\mathcal{P}_1 = 0.4295$ (via $\beta_{\mathrm{early}} = 2.834$, 0.49\% adjustment) -- both \textit{directly verified} with hi\_class. Notably, the CFM reproduces the measured peak ratio \textit{better} than $\Lambda$CDM ($r_{31,\Lambda\mathrm{CDM}} = 0.440$, 2.4\% above the Planck value). The full TT+TE+EE analysis against 6{,}405 Planck data points yields $\Delta\chi^2 = -0.2$ at $c_M = 0.0002$ (\texttt{propto\_omega}), with $\sigma_8 = 0.826$ and $S_8 = 0.847$ -- consistent with Planck, KiDS-Legacy, and eROSITA. The growth rate $f\sigma_8(z=0.38) = 0.495$ \textit{improves} the BOSS LOWZ fit over $\Lambda$CDM. The DESI~DR2 measurement $w_0 = -0.42 \pm 0.21$ independently supports dynamical dark energy, consistent with the CFM's $w_{\mathrm{eff}} \approx -0.33$. The angular acoustic scale offset ($\theta_s = 1.034$ vs.\ $1.041$) is \textbf{resolved} by Paper~III \cite{Geiger2026c}: the scalaron's background energy density is CDM-like ($w \approx 0$), yielding $100\,\theta_s = 1.04173$ for all $\alpha_M$ values. Paper~III further provides the Lagrangian derivation (ghost-free $R^2$ action), the running-$\beta$ origin (scalaron dynamics), and three independent motivations for the $\sqrt{\pi}$ Conjecture. The remaining open challenge is the $S_8$ tension: the CFM predicts $S_8 = 0.845$--$0.855$, in $\sim 3\sigma$ tension with current cosmic shear surveys ($S_8 \approx 0.76$--$0.78$). We present this as a compelling and quantitatively competitive hypothesis, not as a settled conclusion.


\subsection{Limitations and Remaining Challenges}

\begin{enumerate}
\item \textbf{CMB power spectrum ($C_\ell$):} The background-level CMB observables ($\ell_A$, $\mathcal{R}$, $r_d$, $H_0$) are fully reproduced. Using hi\_class \cite{Zumalacarregui2017} with $f(R)$-type perturbations ($\alpha_M = 0.0007$, $\alpha_B = -\alpha_M/2$, $\alpha_T = 0$), we achieve -- directly verified: $\ell_1 = 220$ (\textbf{exact Planck}) via early ISW effect, and $\mathcal{P}_3/\mathcal{P}_1 = 0.4295$ (\textbf{exact Planck}) via $\omega_{c,\mathrm{eff}} = 0.1143$. The $f(R)$ relation follows naturally from the $R^2$ structure of the CFM Lagrangian, and $\alpha_T = 0$ ensures $c_\mathrm{gw} = c$ (consistent with GW170817). The remaining challenge is the angular acoustic scale: $100\,\theta_s = 1.034$ vs.\ Planck $1.041$ (0.69\% offset, reduced from 1.55\%). A native CFM gravity model with time-dependent $\alpha_M(a)$ is needed for the definitive analysis. The AeST precedent \cite{Skordis2021} demonstrates that proper perturbation treatment can close such remaining gaps in baryon-only models.

\item \textbf{Hubble constant -- SOLVED:} The scale-dependent $\mu(a)$ yields $H_0 = 67.3$\,km/s/Mpc, within $0.2$\,km/s/Mpc of the Planck value. This is no longer a limitation.

\item \textbf{Sound horizon -- SOLVED:} $r_d = 146.9$\,Mpc, essentially identical to $\Lambda$CDM ($147.2$\,Mpc). Deviation is 0.2\%. A precision BAO analysis with DESI DR2 data can test this prediction.

\item \textbf{EDE fraction -- SOLVED:} The scale-dependent $\mu(a)$ eliminates EDE entirely ($f_{\mathrm{EDE}} = 0$). The previously problematic 50--60\% EDE fraction is no longer required.

\item \textbf{Physical derivation of $\mu(a) = \sqrt{\pi}$:} The $\sqrt{\pi}$ Conjecture (Section~\ref{subsec:sqrt_pi}) provides a geometric motivation, but a formal derivation from the MOND Lagrangian or the CFM action principle is required.

\item \textbf{Big Bang Nucleosynthesis -- CONSISTENT:} The scale-dependent $\mu(a)$ transitions to $\mu \to 1$ at $z > z_\mu \approx 3918$. Numerical evaluation confirms $\mu(z = 10^{10}) = 1.000$ and $\mu(z = 3 \times 10^8) = 1.000$, so the MOND enhancement is entirely absent during BBN. The resulting $\Delta N_{\mathrm{eff}} \approx 0.000$ is well within the Planck constraint ($N_{\mathrm{eff}} = 3.046 \pm 0.2$) and the BBN constraint ($N_{\mathrm{eff}} = 2.88 \pm 0.28$; \cite{Pitrou2018}). This is a non-trivial consistency check: the $\mu(a)$ transition scale $a_\mu = 2.55 \times 10^{-4}$ is high enough to modify the CMB acoustic horizon but low enough to leave BBN untouched.

\item \textbf{Matter power spectrum $P(k)$ and $\sigma_8$ -- LARGELY RESOLVED:} Using the \texttt{propto\_omega} parametrization in hi\_class with $c_M = 0.0002$, we obtain $\sigma_8 = 0.826$ ($+1.8\%$ above $\Lambda$CDM) and $S_8 = 0.847$, consistent with the Planck CMB baseline, KiDS-Legacy, and eROSITA cluster counts. The growth rate $f\sigma_8$ at $z = 0.38$ \textit{improves} the BOSS LOWZ fit over $\Lambda$CDM ($f\sigma_8 = 0.495$ vs.\ measured $0.497 \pm 0.045$). The $\sim 3\sigma$ tension with cosmic shear surveys (KiDS-1000: $S_8 = 0.759$, DES~Y3: $S_8 = 0.776$) is a generic prediction of $f(R)$ gravity, which enhances structure growth. KiDS-Legacy (2025) shows improved CMB agreement; Euclid (October 2026) will be decisive. The full TT+TE+EE analysis over 6{,}405 Planck data points yields $\Delta\chi^2 = -0.2$ at $c_M = 0.0002$ (see Section~\ref{subsubsec:extended_validation}).

\item \textbf{Lagrangian derivation -- RESOLVED in Paper~III:} Paper~III \cite{Geiger2026c} derives $\beta_{\mathrm{eff}}(a)$ from the scalaron equation of motion with time-dependent mass $m_{\mathrm{eff}}^2(a) = 1/(24\gamma\mathcal{F}(a))$, where $\mathcal{F}$ is the trace coupling. The ghost freedom of the $R^2$ action is proven via conformal equivalence. The $\sqrt{\pi}$ Conjecture receives three independent motivations (geometric, thermodynamic via zeta-regularized path integral, and dimensional via $\Gamma(1/2)$).
\end{enumerate}


% ===================================================================
% 5. FAZIT
% ===================================================================
\section{Conclusion and Outlook}
\label{sec:conclusion}

We have demonstrated that a baryon-only universe ($\Omega_m = \Omega_b \approx 0.05$) with an extended geometric potential can fit cosmological data \textit{across all three major probes} -- supernovae, CMB, and BAO -- competitively with and in some respects better than $\Lambda$CDM, using the \textit{same number of free parameters}.

Three key innovations make this possible: (i)~the \textit{running curvature coupling} $\beta_{\mathrm{eff}}(a)$, which transitions from CDM-like behavior ($\beta \approx 2.8$) at high redshift to curvature-like scaling ($\beta \approx 2.0$) at low redshift; (ii)~the \textit{MOND background coupling} $\mu(a)$, which enhances baryonic gravity by a factor $\sqrt{\pi}$ at late times; and (iii)~the \textit{scale-dependent} evolution of $\mu(a)$ from $\sqrt{\pi}$ (today) to $\mu \to 1$ (at $z > 4000$), which completely eliminates the need for Early Dark Energy.

The quantitative results span two levels of analysis:
\begin{enumerate}
\item \textbf{SN-only (constant $\beta$):} $\Delta\chi^2 = -26.3$ vs.\ $\Lambda$CDM ($\Delta\mathrm{AIC} = -16.3$, $\Delta\mathrm{BIC} = -4.2$), with MCMC posteriors $\alpha = 0.68^{+0.02}_{-0.07}$ and $\beta = 2.02^{+0.26}_{-0.14}$. A 5-fold cross-validation confirms generalization ($\langle\chi^2/n\rangle = 0.445$ vs.\ $0.452$).

\item \textbf{Joint SN + CMB + BAO (running $\beta$ + $\mu(a)$, no EDE):} $\Delta\chi^2 = -5.5$ vs.\ $\Lambda$CDM, with $\ell_A = 301.471$ (Planck: $301.471$), $\mathcal{R} = 1.7502$ (Planck: $1.7502$), $H_0 = 67.3$\,km/s/Mpc, $r_d = 146.9$\,Mpc, and zero EDE. This is achieved with 6 free parameters -- the same count as $\Lambda$CDM. The constant-$\mu = \sqrt{\pi}$ variant achieves $\Delta\chi^2 = -6.1$ but requires $f_{\mathrm{EDE}} = 59\%$.
\end{enumerate}

The three-phase interpretation emerges naturally: at $z > z_\mu \approx 4000$, gravity is standard ($\mu \to 1$) and the strong curvature return ($\beta \approx 2.8$) mimics CDM; at $z_\mu > z > z_t$, the MOND enhancement activates ($\mu \to \sqrt{\pi}$); at $z < z_t \approx 9$, the curvature return weakens to $\beta \approx 2$ and the saturation term drives cosmic acceleration. Dark matter and dark energy are two manifestations of the same geometric relaxation process -- the \textit{Decaying Dark Geometry} hypothesis.

Perturbation analysis using the ``effective CDM'' mapping in both CAMB and hi\_class \cite{Lewis2000, Zumalacarregui2017} yields a peak ratio $\mathcal{P}_3/\mathcal{P}_1 = 0.4212$ (98.1\% of Planck); with a minimal $\beta_{\mathrm{early}}$ adjustment (0.32\%), the ratio reaches \textbf{0.4295 -- an exact Planck match}. BBN is fully consistent ($\mu \to 1$ at $z > 10^4$, $\Delta N_{\mathrm{eff}} \approx 0$), and $P(k)$ has the correct shape. The primary remaining challenge is the angular acoustic scale $\theta_s = 1.025$ (vs.\ $1.041$), caused by the geometric term's $w_{\mathrm{eff}} = -0.06$ at recombination producing a 2.5\% larger sound horizon. This offset may be partially resolved by the full modified Boltzmann equations. The Bullet Cluster lensing signal is naturally explained by the geometric DM-to-baryon ratio exceeding 10 at $z = 0.3$.

\subsection{The Three-Paper Program}

This paper is the second in a three-part program:
\begin{enumerate}
\item \textbf{Paper~I} \cite{Geiger2026}: Establishes the game-theoretic foundation and the Curvature Feedback Model for dark energy replacement. Validated against Pantheon+.
\item \textbf{Paper~II} (this work): Extends the CFM to eliminate the entire dark sector, achieving a baryon-only universe consistent with MOND. Introduces the Decaying Dark Geometry hypothesis, the running curvature coupling $\beta_{\mathrm{eff}}(a)$, the scale-dependent MOND coupling $\mu(a)$, and demonstrates joint SN + CMB + BAO compatibility ($\Delta\chi^2 = -5.5$ vs.\ $\Lambda$CDM, 6 parameters, zero EDE).
\item \textbf{Paper~III} \cite{Geiger2026c}: Provides the microscopic foundation -- the Lagrangian derivation, the connection to quantum gravity, and the interpretation of the running $\beta$ as a geometric phase transition. Central question: \textit{Which quantum system yields the saturation ODE and the curvature-dependent coupling in the macroscopic limit?}
\end{enumerate}

\textbf{Immediate next steps:}
\begin{enumerate}
\item Native CFM gravity model in hi\_class with time-dependent $\alpha_M(a)$ from the curvature feedback physics. The current analysis using \texttt{constant\_alphas} achieves $\ell_1 = 220$ and $\mathcal{P}_3/\mathcal{P}_1 = 0.4295$ (both exact Planck), but the constant parametrization encounters numerical instabilities at high $\omega_c$. A time-dependent implementation (\texttt{eft\_alphas\_power\_law} or custom model) would resolve this and close the remaining $\theta_s$ offset (1.035 vs.\ 1.041, 0.63\%).
\item Precision BAO analysis with DESI DR2 data and the CFM distance ladder
\item Matter power spectrum $P(k)$ with full perturbation equations: the preliminary analysis confirms the correct shape, but $\sigma_8 = 0.90$ in the ``effective CDM'' approximation is too high -- the full treatment is expected to reduce this.
\item \sout{BBN consistency check} -- \textbf{DONE:} $\mu(z > 10^4) \to 1$, $\Delta N_{\mathrm{eff}} \approx 0.000$ \cite{Pitrou2018}
\item AeST mapping: deriving $\alpha$, $\beta_{\mathrm{eff}}$, and $\mu(a)$ from the Skordis-Z{\l}o\'snik framework
\item Lagrangian derivation of the running $\beta$ and $\mu$ from the curvature-squared action (Paper~III)
\end{enumerate}

\subsection{Invitation to the Community}

This work presents a promising hypothesis, not a settled conclusion. The author invites the community to:
\begin{enumerate}
\item \textbf{Replicate:} The analysis code is open source. All fits use the publicly available Pantheon+ catalog. Independent replication of the SN-only result ($\Delta\chi^2 = -26.3$) and the joint fit ($\Delta\chi^2 = -5.5$, zero EDE) is straightforward.
\item \textbf{Extend:} Computing $C_\ell$ and $P(k)$ with the running $\beta(a)$ + $\mu(a)$ background is the single most critical next step. Collaboration with groups experienced in modified Boltzmann codes (CLASS/CAMB) is essential.
\item \textbf{Derive $\mu(a)$:} The $\sqrt{\pi}$ Conjecture and the scale-dependent transition must be derived from MOND principles or the CFM action.
\item \textbf{Critique:} The running-$\beta$ parametrization, $\mu(a)$, the trace-coupling mechanism, and the Efficiency Hypothesis all require independent scrutiny.
\end{enumerate}

\begin{quote}
\textit{``If dark energy is a relaxing constraint and dark matter is a geometric shadow, then 95\% of the universe may have been hiding in plain sight -- as the geometry of spacetime itself.''}
\end{quote}


% ===================================================================
% LITERATUR
% ===================================================================
\begin{thebibliography}{99}

\bibitem{Geiger2026}
Geiger, L.\ (2026).
Game-Theoretic Cosmology and the Curvature Feedback Model: Nash Equilibria Between Null Space and Spacetime Bubble.
Working Paper. \url{https://github.com/lukisch/cfm-cosmology}.

\bibitem{Scolnic2022}
Scolnic, D.\ et al.\ (2022).
The Pantheon+ Analysis: The Full Data Set and Light-curve Release.
\textit{The Astrophysical Journal}, 938(2), 113.
DOI: 10.3847/1538-4357/ac8b7a.

\bibitem{Planck2020}
Planck Collaboration (2020).
Planck 2018 results. VI. Cosmological parameters.
\textit{Astronomy \& Astrophysics}, 641, A6.
DOI: 10.1051/0004-6361/201833910.

\bibitem{Milgrom1983}
Milgrom, M.\ (1983).
A modification of the Newtonian dynamics as a possible alternative to the hidden mass hypothesis.
\textit{The Astrophysical Journal}, 270, 365--370.
DOI: 10.1086/161130.

\bibitem{Skordis2021}
Skordis, C.\ \& Z{\l}o\'snik, T.\ (2021).
New Relativistic Theory for Modified Newtonian Dynamics.
\textit{Physical Review Letters}, 127(16), 161302.
DOI: 10.1103/PhysRevLett.127.161302.

\bibitem{McGaugh2016}
McGaugh, S.\,S., Lelli, F.\ \& Schombert, J.\,M.\ (2016).
Radial Acceleration Relation in Rotationally Supported Galaxies.
\textit{Physical Review Letters}, 117(20), 201101.
DOI: 10.1103/PhysRevLett.117.201101.

\bibitem{Lelli2017}
Lelli, F., McGaugh, S.\,S.\ \& Schombert, J.\,M.\ (2017).
One Law to Rule Them All: The Radial Acceleration Relation of Galaxies.
\textit{The Astrophysical Journal}, 836(2), 152.
DOI: 10.3847/1538-4357/836/2/152.

\bibitem{Labbe2023}
Labb\'e, I.\ et al.\ (2023).
A population of red candidate massive galaxies $\sim$600\,Myr after the Big Bang.
\textit{Nature}, 616(7956), 266--269.
DOI: 10.1038/s41586-023-05786-2.

\bibitem{BoylanKolchin2023}
Boylan-Kolchin, M.\ (2023).
Stress testing $\Lambda$CDM with high-redshift galaxy candidates.
\textit{Nature Astronomy}, 7, 731--735.
DOI: 10.1038/s41550-023-01937-7.

\bibitem{Asencio2023}
Asencio, E., Banik, I.\ \& Kroupa, P.\ (2023).
The El Gordo galaxy cluster challenges $\Lambda$CDM for any plausible collision velocity.
\textit{The Astrophysical Journal}, 954(2), 162.
DOI: 10.3847/1538-4357/ace62a.

\bibitem{Miller2018}
Miller, T.\,B.\ et al.\ (2018).
A massive core for a cluster of galaxies at a redshift of 4.3.
\textit{Nature}, 556(7702), 469--472.
DOI: 10.1038/s41586-018-0025-2.

\bibitem{Clowe2006}
Clowe, D.\ et al.\ (2006).
A Direct Empirical Proof of the Existence of Dark Matter.
\textit{The Astrophysical Journal Letters}, 648(2), L109--L113.
DOI: 10.1086/508162.

\bibitem{Geiger2026c}
Geiger, L.\ (2026).
Microscopic Foundations of the Curvature Feedback Model: From Quantum Geometry to Macroscopic Saturation.
Working Paper (in preparation).

\bibitem{Lewis2000}
Lewis, A., Challinor, A.\ \& Lasenby, A.\ (2000).
Efficient Computation of Cosmic Microwave Background Anisotropies in Closed Friedmann-Robertson-Walker Models.
\textit{The Astrophysical Journal}, 538(2), 473--476.
DOI: 10.1086/309179. Code: \url{https://github.com/cmbant/CAMB}.

\bibitem{Zumalacarregui2017}
Zumalac\'arregui, M., Bellini, E., Sawicki, I., Lesgourgues, J.\ \& Ferreira, P.\,G.\ (2017).
hi\_class: Horndeski in the Cosmic Linear Anisotropy Solving System.
\textit{Journal of Cosmology and Astroparticle Physics}, 2017(08), 019.
DOI: 10.1088/1475-7516/2017/08/019. Code: \url{https://github.com/miguelzuma/hi_class_public}.

\bibitem{Hu2014}
Hu, B., Raveri, M., Frusciante, N.\ \& Silvestri, A.\ (2014).
Effective Field Theory of Cosmic Acceleration: An Implementation in CAMB.
\textit{Physical Review D}, 89(10), 103530.
DOI: 10.1103/PhysRevD.89.103530. Code: \url{https://github.com/EFTCAMB/EFTCAMB}.

\bibitem{Pitrou2018}
Pitrou, C., Coc, A., Uzan, J.\,P.\ \& Vangioni, E.\ (2018).
Precision Big Bang Nucleosynthesis with Improved Helium-4 Predictions.
\textit{Physics Reports}, 754, 1--66.
DOI: 10.1016/j.physrep.2018.04.005.

\bibitem{Alam2017}
Alam, S.\ et al.\ (BOSS Collaboration) (2017).
The Clustering of Galaxies in the Completed SDSS-III Baryon Oscillation Spectroscopic Survey.
\textit{Monthly Notices of the Royal Astronomical Society}, 470(3), 2617--2652.
DOI: 10.1093/mnras/stx721.

\end{thebibliography}


% ===================================================================
% ANHANG: DEUTSCHE ÜBERSETZUNG
% ===================================================================
\newpage
\selectlanguage{ngerman}

\section*{Anhang: Deutsche \"Ubersetzung}
\addcontentsline{toc}{section}{Anhang: Deutsche \"Ubersetzung}

\noindent\textit{Die folgende vollst\"andige deutsche \"Ubersetzung dient der Zug\"anglichkeit f\"ur den deutschsprachigen Leserkreis. Alle Gleichungen, Referenzen und LaTeX-Strukturen sind identisch mit dem englischen Originaltext.}

\vspace{1em}

% -------------------------------------------------------------------
% TITEL UND ZUSAMMENFASSUNG
% -------------------------------------------------------------------
\subsection*{Titel}

\begin{center}
{\huge\textbf{Eliminierung des Dunklen Sektors:\\Vereinigung des Kr\"ummungs-R\"uckkopplungsmodells mit MOND}}\\[0.5em]
{\Large Ein reines Baryonen-Universum mit geometrischer Dunkler Materie und Dunkler Energie}\\[0.3em]
{\large Vorl\"aufige Analyse mit Pantheon+ Typ~Ia-Supernovae}
\end{center}

\vspace{0.5em}
\begin{center}
Lukas Geiger\\
\textit{Unabh\"angiger Forscher, Bernau im Schwarzwald}\\
Februar 2026
\end{center}

\subsection*{Zusammenfassung}

\noindent Wir schlagen ein vereinheitlichtes geometrisches Rahmenwerk vor, das sowohl Dunkle Energie als auch Dunkle Materie aus dem kosmologischen Energiebudget eliminiert. Aufbauend auf dem Kr\"ummungs-R\"uckkopplungsmodell (CFM) \cite{Geiger2026}, welches die kosmologische Konstante durch ein zeitabh\"angiges Kr\"ummungs-R\"uckstellpotential $\Omega_\Phi(a)$ ersetzt, erweitern wir das Modell zu einem \textit{reinen Baryonen-Universum} ($\Omega_m = \Omega_b \approx 0{,}05$), das mit der Modifizierten Newtonschen Dynamik (MOND) \cite{Milgrom1983} kompatibel ist. Die erweiterte Friedmann-Gleichung lautet:
\begin{equation*}
H^2(a) = H_0^2 \left[\Omega_b\,a^{-3} + \Phi_0 \cdot f_{\mathrm{sat}}(a) + \alpha \cdot a^{-\beta}\right]
\end{equation*}
wobei der S\"attigungsterm $f_{\mathrm{sat}}$ die Dunkle Energie ersetzt und der Potenzgesetz-Term $\alpha \cdot a^{-\beta}$ die kosmologische Rolle der Dunklen Materie als rein geometrischen Effekt \"ubernimmt. Getestet an 1.590 Pantheon+ Typ~Ia-Supernovae \cite{Scolnic2022} liefert dieses Modell "`ohne dunklen Sektor"' $\chi^2 = 702{,}7$ ($\Delta\chi^2 = -26{,}3$ gegen\"uber $\Lambda$CDM, $\Delta\mathrm{AIC} = -16{,}3$, $\Delta\mathrm{BIC} = -4{,}2$) und \"ubertrifft damit sowohl $\Lambda$CDM als auch das Standard-CFM deutlich. Die MCMC-Posterioranalyse ergibt $\alpha = 0{,}68^{+0{,}02}_{-0{,}07}$ und $\beta = 2{,}02^{+0{,}26}_{-0{,}14}$, was zeigt, dass der geometrische DM-Term wie \textit{r\"aumliche Kr\"ummung} ($a^{-2}$, $w = -1/3$) skaliert -- nicht wie Materie ($a^{-3}$). Wir diskutieren die physikalische Interpretation im spieltheoretischen Rahmenwerk und die Verbindung zur relativistischen MOND-Theorie AeST \cite{Skordis2021}. Falls durch CMB- und BAO-Daten best\"atigt, w\"urde dieses Rahmenwerk den gesamten dunklen Sektor -- der in $\Lambda$CDM 95\% des Energiebudgets umfasst -- \"uberfl\"ussig machen.

\vspace{0.5em}
\noindent \textbf{Schl\"usselw\"orter:} Kr\"ummungs-R\"uckkopplungsmodell, MOND, Dunkle Materie, Dunkle Energie, reines Baryonen-Universum, Pantheon+, modifizierte Gravitation, geometrische Kosmologie

\vspace{0.5em}
\noindent \textbf{Fachgebiete:} Theoretische Physik, Kosmologie, Modifizierte Gravitation

% -------------------------------------------------------------------
% KI-OFFENLEGUNG
% -------------------------------------------------------------------
\subsection*{KI-Offenlegung und Methodik}

\noindent\textbf{Erweiterte Methodenerkl\"arung:} Dieses Paper ist ein Experiment in \textit{KI-gest\"utzter Wissenschaft}. Die Arbeitsteilung wird transparent offengelegt:

\begin{description}[style=nextline, leftmargin=2cm]
\item[\textbf{Menschlicher Autor} (Lukas Geiger)] Physikalische Intuition, Kernhypothesen (spieltheoretische Grundlage, S\"attigungsmechanismus, Geometrie-als-dunkler-Sektor, Effizienzhypothese, Phasen\"ubergangskonzept), Interpretation der Ergebnisse, strategische Entscheidungen und endg\"ultige Verantwortung f\"ur alle wissenschaftlichen Inhalte.
\item[\textbf{Claude Opus 4.6} (Anthropic)] Co-Autor: Mathematische Formalisierung, Herleitung der Gleichungen, Code-Entwicklung (Python/MCMC), statistische Analyse (Pantheon+-Fits), Textgenerierung und strukturelle Organisation.
\item[\textbf{Gemini} (Google DeepMind)] Gutachter: Kritisches Feedback, MOND-Kompatibilit\"atsanalyse, Identifizierung der BBN-Krise, Vorschlag der Spur-Kopplung, strategische Empfehlungen.
\end{description}

\vspace{0.5em}
\noindent\textit{Hinweis:} Die mathematische Formalisierung und die statistischen Fits wurden von KI-Systemen durchgef\"uhrt. Der Autor pr\"asentiert diese Hypothesen als \textit{Arbeitspapier}, um eine \"Uberpr\"ufung und Weiterentwicklung durch die wissenschaftliche Gemeinschaft zu erm\"oglichen. \textbf{Eine unabh\"angige mathematische Verifikation wird ausdr\"ucklich ermutigt.}

% -------------------------------------------------------------------
% 1. EINLEITUNG
% -------------------------------------------------------------------
\subsection*{1\quad Einleitung: Das Problem des Dunklen Sektors}

Das kosmologische Standardmodell $\Lambda$CDM beschreibt das Energiebudget des Universums als bestehend aus etwa 5\% baryonischer Materie, 27\% kalter Dunkler Materie (CDM) und 68\% Dunkler Energie ($\Lambda$) \cite{Planck2020}. Trotz seines bemerkenswerten empirischen Erfolgs impliziert dieses Modell, dass \textit{95\% des Universums aus Entit\"aten bestehen, die niemals direkt nachgewiesen wurden}.

Zwei unabh\"angige Forschungsrichtungen stellen dieses Bild in Frage:

\begin{enumerate}
\item \textbf{Das Kr\"ummungs-R\"uckkopplungsmodell (CFM)} \cite{Geiger2026}: Aus einem spieltheoretischen Rahmenwerk entwickelt, ersetzt das CFM die kosmologische Konstante $\Lambda$ durch ein zeitabh\"angiges Kr\"ummungs-R\"uckstellpotential $\Omega_\Phi(a)$ und erkl\"art die beschleunigte Expansion als geometrisches "`Ged\"achtnis"' statt als neue Energieform. Getestet an 1.590 Pantheon+-Supernovae liefert das CFM $\Delta\chi^2 = -12{,}2$ relativ zu $\Lambda$CDM.

\item \textbf{Modifizierte Newtonsche Dynamik (MOND)} \cite{Milgrom1983}: MOND modifiziert die Gravitationsdynamik bei Beschleunigungen unterhalb von $a_0 \approx 1{,}2 \times 10^{-10}$\,m/s$^2$ und sagt galaktische Rotationskurven, die baryonische Tully-Fisher-Relation \cite{McGaugh2016} und die radiale Beschleunigungsrelation \cite{Lelli2017} erfolgreich vorher -- ohne Dunkle Materie zu ben\"otigen.
\end{enumerate}

Die zentrale Frage dieses Papers lautet: \textit{K\"onnen beide Rahmenwerke zu einem einzigen Modell vereint werden, das den gesamten dunklen Sektor eliminiert?}

\subsubsection*{1.1\quad Die Kompatibilit\"atsfrage}

Auf den ersten Blick befassen sich CFM und MOND mit unterschiedlichen "`dunklen"' Problemen:
\begin{itemize}
\item CFM ersetzt \textbf{Dunkle Energie} (kosmologische Expansion)
\item MOND ersetzt \textbf{Dunkle Materie} (galaktische Dynamik)
\end{itemize}

Eine naive Kombination st\"o\ss{}t jedoch auf eine fundamentale Spannung: Das Standard-CFM fittet $\Omega_m \approx 0{,}36$, was erhebliche Dunkle Materie impliziert ($\Omega_m - \Omega_b \approx 0{,}31$). Wenn MOND korrekt ist und Dunkle Materie nicht existiert, muss das Modell allein mit $\Omega_m = \Omega_b \approx 0{,}05$ funktionieren.

\subsubsection*{1.2\quad Strukturbildung: Gemeinsame Basis}

Beide Rahmenwerke konvergieren in einer kritischen Vorhersage: Strukturen bilden sich \textit{fr\"uher und effizienter} als $\Lambda$CDM erlaubt.

\begin{itemize}
\item \textbf{CFM:} Der sp\"atere Einsatz der kosmischen Beschleunigung ($z_{\mathrm{acc}} = 0{,}52$ vs.\ $0{,}84$) verl\"angert die materiedominierte Wachstumsphase \cite{Geiger2026}.
\item \textbf{MOND:} Verst\"arkte Gravitationsanziehung bei niedrigen Beschleunigungen f\"uhrt zu schnellerem gravitativen Kollaps auf gro\ss{}en Skalen \cite{Asencio2023}.
\end{itemize}

Diese gemeinsame Vorhersage wird durch mehrere Beobachtungsanomalien gest\"utzt: die JWST-"`Universe Breakers"' bei $z > 7$ \cite{Labbe2023, BoylanKolchin2023}, der El~Gordo-Galaxienhaufen bei $z \approx 0{,}87$ (${>}6\sigma$ Spannung mit $\Lambda$CDM) \cite{Asencio2023} und unerwartet reife Protocluster bei $z > 4$ \cite{Miller2018}.

% -------------------------------------------------------------------
% 2. THEORETISCHES RAHMENWERK
% -------------------------------------------------------------------
\subsection*{2\quad Theoretisches Rahmenwerk}

\subsubsection*{2.1\quad Das erweiterte Kr\"ummungs-R\"uckkopplungsmodell}

Im Standard-CFM \cite{Geiger2026} lautet die Friedmann-Gleichung:
\begin{equation}
H^2(a) = H_0^2 \left[\Omega_m\,a^{-3} + \Omega_\Phi(a)\right] \tag{1'}
\end{equation}
mit
\begin{equation}
\Omega_\Phi(a) = \Phi_0 \cdot \frac{\tanh\!\big(k\cdot(a - a_{\mathrm{trans}})\big) + s}{1 + s} \tag{2'}
\end{equation}

F\"ur die Erweiterung auf ein reines Baryonen-Universum zerlegen wir das geometrische Potential in zwei Komponenten:
\begin{equation}
\boxed{H^2(a) = H_0^2 \left[\Omega_b\,a^{-3} + \underbrace{\Phi_0 \cdot f_{\mathrm{sat}}(a)}_{\text{geometrische DE}} + \underbrace{\alpha \cdot a^{-\beta}}_{\text{geometrische DM}}\right]}
\tag{3'}
\end{equation}

wobei:
\begin{itemize}
\item $\Omega_b \approx 0{,}05$ die baryonische Materiedichte ist (fest)
\item $\Phi_0 \cdot f_{\mathrm{sat}}(a)$ der S\"attigungs-Ersatzterm f\"ur Dunkle Energie ist (aus dem dynamischen S\"attigungsmechanismus)
\item $\alpha \cdot a^{-\beta}$ ein Potenzgesetz-Term ist, der die \textit{kosmologische} Rolle der Dunklen Materie \"ubernimmt
\end{itemize}

Die Flachheitsbedingung $H^2(a{=}1)/H_0^2 = 1$ ergibt:
\begin{equation}
\Omega_b + \Phi_0 \cdot f_{\mathrm{sat}}(1) + \alpha = 1 \tag{4'}
\end{equation}

\subsubsection*{2.2\quad Spur-Kopplung und BBN-Konsistenz}

Eine kritische Randbedingung f\"ur den geometrischen DM-Term ist die Urknall-Nukleosynthese (BBN): Bei $a \sim 10^{-9}$ w\"urde das naive Potenzgesetz $\alpha \cdot a^{-2}$ einen Wert von $\sim 10^{18}$ ergeben, die Friedmann-Gleichung vollst\"andig dominieren und die vorhergesagten primordialen Elementh\"aufigkeiten zerst\"oren. Der Term \textit{muss} w\"ahrend der Strahlungs\"ara unterdr\"uckt werden.

Wir schlagen vor, dass der geometrische DM-Term nicht an die Energiedichte $\rho$ koppelt, sondern an die \textit{Spur des Energie-Impuls-Tensors}:
\begin{equation}
T \equiv g^{\mu\nu} T_{\mu\nu} = -\rho + 3p = -\rho(1 - 3w) \tag{5'}
\end{equation}

Diese Spur hat eine bemerkenswerte Eigenschaft: F\"ur relativistische Materie (Strahlung, $w = 1/3$) verschwindet die Spur exakt:
\begin{equation}
T_{\mathrm{rad}} = -\rho_{\mathrm{rad}} + 3 \cdot \tfrac{1}{3}\rho_{\mathrm{rad}} = 0 \tag{6'}
\end{equation}

Dies ist kein Zufall, sondern eine Konsequenz der \textit{konformen Symmetrie}: Masselose Felder sind konform invariant, und die Spur eines konform invarianten Energie-Impuls-Tensors verschwindet identisch. W\"ahrend der strahlungsdominierten \"Ara ist die konforme Symmetrie exakt, und der geometrische DM-Term wird automatisch unterdr\"uckt.

F\"ur nicht-relativistische Materie ($w \approx 0$) ist die Spur $T_{\mathrm{mat}} = -\rho_m \neq 0$, und der geometrische DM-Term wird aktiviert. Der \"Ubergang erfolgt nat\"urlich bei der Materie-Strahlungs-Gleichheit ($a_{\mathrm{eq}} \approx 3 \times 10^{-4}$), deutlich nach der BBN ($a_{\mathrm{BBN}} \sim 10^{-9}$).

Die vollst\"andige erweiterte Friedmann-Gleichung mit Spur-Kopplung lautet:
\begin{equation}
\boxed{H^2(a) = H_0^2 \left[\Omega_b\,a^{-3} + \Phi_0 \cdot f_{\mathrm{sat}}(a) + \alpha \cdot a^{-\beta} \cdot \mathcal{S}(a)\right]}
\tag{7'}
\end{equation}
wobei $\mathcal{S}(a)$ der Spur-Kopplungs-Unterdr\"uckungsfaktor ist:
\begin{equation}
\mathcal{S}(a) = \frac{|T|}{|T| + \rho_{\mathrm{rad}}} = \frac{\Omega_b\,a^{-3}}{\Omega_b\,a^{-3} + \Omega_r\,a^{-4}}
= \frac{1}{1 + (a_{\mathrm{eq}}/a)}
\tag{8'}
\end{equation}
mit $a_{\mathrm{eq}} = \Omega_r/\Omega_b \approx 3 \times 10^{-4}$ (unter Verwendung von $\Omega_r \approx 9 \times 10^{-5}$). Dieser Faktor erf\"ullt:
\begin{itemize}
\item $\mathcal{S}(a \ll a_{\mathrm{eq}}) \approx a/a_{\mathrm{eq}} \to 0$ \quad (Strahlungs\"ara: BBN gesch\"utzt)
\item $\mathcal{S}(a \gg a_{\mathrm{eq}}) \approx 1$ \quad (Materie-/DE-\"Ara: voller geometrischer DM-Beitrag)
\item $\mathcal{S}(a = 1) \approx 1 - 3\times10^{-4} \approx 1$ \quad (heute: SN-Fit unver\"andert)
\end{itemize}

\textbf{Auswirkung auf den Pantheon+-Fit:} Da alle Pantheon+-Supernovae bei $z < 2{,}3$ ($a > 0{,}30$) liegen, ist der Unterdr\"uckungsfaktor im gesamten beobachteten Rotverschiebungsbereich $\mathcal{S} > 0{,}999$. Die MCMC-Ergebnisse ($\alpha$, $\beta$, $\chi^2$) bleiben bis auf numerische Pr\"azision unver\"andert.

\textbf{Physikalische Interpretation:} Die Spur-Kopplung hat eine tiefe geometrische Bedeutung. Im spieltheoretischen Rahmenwerk repr\"asentiert der geometrische DM-Term das Kr\"ummungs-"`Ged\"achtnis"' der anf\"anglichen Energiekonzentration. W\"ahrend der Strahlungs\"ara ist das Universum konform flach (Strahlung ist skalenfrei), und es gibt kein Kr\"ummungsged\"achtnis, das aufrechterhalten werden k\"onnte. Der geometrische DM-Term aktiviert sich erst, wenn die konforme Symmetrie durch das Auftreten massiver (nicht-relativistischer) Materie gebrochen wird -- genau in der Epoche, in der CDM im Standardbild beginnen w\"urde, Strukturen zu bilden.

\subsubsection*{2.3\quad Physikalische Interpretation des geometrischen DM-Terms}

Der Term $\alpha \cdot a^{-\beta}$ mit $\beta \approx 2{,}0$ (aus MCMC) erfordert eine physikalische Interpretation:

\begin{enumerate}
\item \textbf{Skalierungsverhalten:} Das MCMC-Posterior ergibt $\beta = 2{,}02 \pm 0{,}20$, konsistent mit kr\"ummungsartiger Skalierung ($a^{-2}$, d.\,h.\ $\beta = 2$). Dies ist die Skalierung der r\"aumlichen Kr\"ummung in der Friedmann-Gleichung, was auf einen geometrischen statt materiellen Ursprung hindeutet.

\item \textbf{Spieltheoretische Interpretation:} Im spieltheoretischen Rahmenwerk repr\"asentiert dieser Term einen zweiten Gleichgewichtsmechanismus: W\"ahrend der S\"attigungsterm die "`l\"osende Bremse"' (Dunkle Energie) beschreibt, beschreibt der Potenzgesetz-Term die "`geometrische Tr\"agheit"' der Kr\"ummungsr\"uckstellung -- ein residualer geometrischer Effekt, der mit der Expansion abklingt, aber langsamer als Materie.

\item \textbf{Verbindung zu MOND:} In der relativistischen MOND-Theorie AeST (Aether-Skalar-Tensor) von Skordis \& Z{\l}o\'snik \cite{Skordis2021} erzeugen ein Skalarfeld und ein Vektorfeld einen effektiven Energie-Impuls-Tensor, der die Expansionsgeschichte modifiziert. Der Potenzgesetz-Term $\alpha \cdot a^{-\beta}$ kann m\"oglicherweise als kosmologischer Abdruck dieser MOND-artigen Modifikation interpretiert werden.

\item \textbf{Effektive Zustandsgleichung:} Der geometrische DM-Term hat eine effektive Zustandsgleichung $w_{\mathrm{DM,geom}} = \beta/3 - 1 = -0{,}33 \pm 0{,}07$, die praktisch identisch mit der Kr\"ummungs-Zustandsgleichung ($w_k = -1/3$) ist. Die "`Dunkle Materie"'-Komponente ist von r\"aumlicher Kr\"ummung nicht unterscheidbar.
\end{enumerate}

\subsubsection*{2.4\quad MOND auf galaktischen vs.\ kosmologischen Skalen}

Eine wesentliche Unterscheidung muss aufrechterhalten werden:
\begin{itemize}
\item \textbf{Galaktische Skalen:} MOND modifiziert das Gravitationskraftgesetz unterhalb von $a_0$ und erkl\"art Rotationskurven und die Tully-Fisher-Relation \textit{ohne Dunkle Materie}.
\item \textbf{Kosmologische Skalen:} Das erweiterte CFM ersetzt die \textit{kosmologische Rolle} der Dunklen Materie (Beitrag zu $H(z)$) durch ein geometrisches Potential, ohne eine Teilchenspezies zu ben\"otigen.
\end{itemize}

Die beiden Mechanismen sind komplement\"ar: MOND behandelt die lokale Dynamik, w\"ahrend der geometrische DM-Term die globale Expansionsgeschichte behandelt.

\subsubsection*{2.5\quad Die Effizienzhypothese: Warum keine Dunkle Materie?}

Eine kritische Frage bleibt: Das erweiterte CFM zeigt, dass die Daten ein reines Baryonen-Universum \textit{erlauben}, aber warum sollte das Universum ein reines Baryonen-Universum \textit{sein}? Das spieltheoretische Rahmenwerk liefert eine \"uberzeugende Antwort.

Im Nash-Gleichgewicht zwischen Nullraum und Raumzeitblase \cite{Geiger2026} erh\"alt die Raumzeitblase ein endliches Energiebudget $E_0$ vom Nullraum. Ihr Ziel ist es, den Konzentrationsgradienten $G$ so effizient wie m\"oglich zu neutralisieren und gleichzeitig das \"ubergeordnete System zu sch\"utzen. Dies erzeugt ein Ressourcenallokationsproblem:

\begin{itemize}
\item \textbf{Baryonische Materie:} Wechselwirkt elektromagnetisch, bildet Sterne, produziert Strahlung, kollabiert zu Schwarzen L\"ochern und erzeugt Entropie mit maximalen Raten. Baryonen sind \textit{hocheffiziente Werkzeuge} zur Gradientenreduktion.

\item \textbf{Dunkle Materie (hypothetisch):} Wechselwirkt nur gravitativ. Sie verklumpt, strahlt aber nicht, bildet keine Sterne und tr\"agt im Vergleich zu einer \"aquivalenten Masse baryonischer Materie minimal zur Entropieproduktion bei.
\end{itemize}

In einem spieltheoretisch optimierten Universum w\"are die Zuweisung von 85\% des Energiebudgets an eine Komponente, die kaum zum prim\"aren Ziel (entropiegetriebene Gradientenreduktion) beitr\"agt, eine \textit{strategisch unterlegene Allokation}. Ein Nash-optimales System maximiert die Entropieproduktion pro Energieeinheit, indem es das gesamte Budget in "`aktive"' (baryonische) Materie kanalisiert.

\begin{proposition}[Effizienzprinzip -- Konditionale Form]
\textit{Wenn} das Nash-Gleichgewicht zwischen Nullraum und Raumzeitblase die Entropieproduktion pro Energieeinheit optimiert (Pr\"amisse~P1), und \textit{wenn} baryonische Materie pro Masseneinheit mehr Entropie erzeugt als jede hypothetische Dunkle-Materie-Spezies (Pr\"amisse~P2), \textit{dann} besteht der Nash-optimale Materieinhalt ausschlie\ss{}lich aus baryonischer Materie ($\Omega_m = \Omega_b$). Die gravitativen Effekte, die konventionell der Dunklen Materie zugeschrieben werden, sind stattdessen geometrische Konsequenzen des Kr\"ummungsr\"uckstellmechanismus (der $\alpha \cdot a^{-\beta}$-Term).
\end{proposition}

\textit{Logische Struktur:} Das Argument hat die Form $P1 \wedge P2 \Rightarrow S$, was deduktiv g\"ultig ist. Pr\"amisse~P2 ist empirisch fundiert: Baryonen bilden Sterne, treiben Nukleosynthese und speisen Schwarzloch-Akkretion, w\"ahrend Dunkle Materie (falls sie existierte) nur gravitativ wechselwirken und vernachl\"assigbar zur Entropieproduktion beitragen w\"urde. Pr\"amisse~P1 -- dass das Nash-Gleichgewicht maximale Entropieproduktion selektiert -- ist die \textit{zu testende Hypothese}. Sie wird durch den empirischen Erfolg des Modells ($\Delta\chi^2 = -26{,}3$) gest\"utzt, ist aber nicht unabh\"angig bewiesen. Das Effizienzprinzip ist daher eine \textit{testbare konditionale Vorhersage}: Es w\"urde durch den experimentellen Nachweis von Dunkle-Materie-Teilchen falsifiziert.

Der quantitative Test besteht darin, ob der geometrische Term $\alpha \cdot a^{-\beta}$ alle kosmologischen Signaturen reproduzieren kann, die traditionell der Dunklen Materie zugeschrieben werden (Expansionsgeschichte, akustische CMB-Maxima, Materiedichtespektrum). Der unten vorgestellte Pantheon+-Test adressiert die erste dieser Signaturen.

\subsubsection*{2.6\quad Der geometrische Phasen\"ubergang}

Die erweiterte Friedmann-Gleichung~(3') enth\"alt zwei geometrische Terme: das Potenzgesetz $\alpha \cdot a^{-\beta}$ und die S\"attigung $\Phi_0 \cdot f_{\mathrm{sat}}(a)$. Eine zentrale Erkenntnis ergibt sich: Dies sind keine unabh\"angigen Ph\"anomene, sondern \textit{zwei Phasen eines einzigen geometrischen Prozesses} -- der Kr\"ummungsr\"uckstellmechanismus in verschiedenen Regimen.

\begin{enumerate}
\item \textbf{Fr\"uhes Universum ($a \ll a_{\mathrm{trans}}$):} Die Kr\"ummungsr\"uckstellung ist weit von der S\"attigung entfernt. Das geometrische Potential wird vom Potenzgesetz-Term $\alpha \cdot a^{-2}$ dominiert, der wie r\"aumliche Kr\"ummung skaliert und die kosmologische Rolle der "`Dunklen Materie"' \"ubernimmt -- als gravitatives Ger\"ust f\"ur die Strukturbildung.

\item \textbf{\"Ubergangsepoche ($a \approx a_{\mathrm{trans}}$):} W\"ahrend das Universum expandiert, n\"ahert sich die Kr\"ummungsr\"uckstellung ihrer S\"attigungsgrenze $\Phi_0$. Der Potenzgesetz-Beitrag klingt ab, w\"ahrend der S\"attigungsterm ansteigt.

\item \textbf{Sp\"ates Universum ($a \gtrsim a_{\mathrm{trans}}$):} Der S\"attigungsterm dominiert und liefert ein nahezu konstantes geometrisches Potential, das die beschleunigte Expansion antreibt -- die Rolle, die konventionell der "`Dunklen Energie"' zugeschrieben wird.
\end{enumerate}

Dieses Bild liefert eine nat\"urliche Interpretation: \textit{Dunkle Materie und Dunkle Energie sind nicht zwei verschiedene Substanzen, sondern zwei Phasen desselben geometrischen Ph\"anomens.} Im fr\"uhen Universum verh\"alt sich die Raumzeitgeometrie wie Dunkle Materie; im sp\"aten Universum verh\"alt sich dieselbe Geometrie wie Dunkle Energie. Der "`Phasen\"ubergang"' ist die S\"attigung des Kr\"ummungsr\"uckstellpotentials.

Wir nennen dies die Hypothese der \textbf{Zerfallenden Dunklen Geometrie}: Das geometrische Potential ist ein zerfallendes \"Uberbleibsel der anf\"anglichen Kr\"ummungskonzentration des Urknalls. Fr\"uh liefert es gravitative Struktur ("`Dunkle Materie"'). W\"ahrend es zerfällt und s\"attigt, liefert es beschleunigte Expansion ("`Dunkle Energie"'). Es gibt keinen dunklen Sektor -- nur Geometrie in verschiedenen Stadien der Relaxation.

\begin{definition}[Zerfallende Dunkle Geometrie]
Der kosmologische dunkle Sektor ist ein einzelnes geometrisches Ph\"anomen: das Kr\"ummungsr\"uckstellpotential $\Omega_\Phi$ des spieltheoretischen Nullraum$\leftrightarrow$Raumzeit-Gleichgewichts. Seine zwei scheinbaren Komponenten -- Dunkle Materie ($\alpha \cdot a^{-\beta}$, dominant zu fr\"uhen Zeiten) und Dunkle Energie ($\Phi_0 \cdot f_{\mathrm{sat}}$, dominant zu sp\"aten Zeiten) -- repr\"asentieren die unges\"attigte und ges\"attigte Phase desselben Relaxationsprozesses.
\end{definition}

% -------------------------------------------------------------------
% 3. DATENANALYSE UND ERGEBNISSE
% -------------------------------------------------------------------
\subsection*{3\quad Datenanalyse und Ergebnisse}

\subsubsection*{3.1\quad Daten und Methodik}

Wir verwenden den Pantheon+-Katalog \cite{Scolnic2022} mit 1.590 Typ~Ia-Supernovae mit $z > 0{,}01$ (Rotverschiebungsbereich $0{,}01$--$2{,}26$). Leuchtkraftentfernungen werden mittels kumulativer Trapezintegration auf einem feinen Rotverschiebungsgitter ($N = 2.000$) berechnet. Der St\"orparameter~$M$ wird analytisch marginalisiert. Die Parameteroptimierung verwendet differentielle Evolution mit L-BFGS-B-Feinschliff.

\subsubsection*{3.2\quad Ergebnisse: Modellvergleich}

\begin{table}[H]
\centering
\caption{Modellvergleich anhand von 1.590 Pantheon+-Supernovae.}
\begin{tabular}{lcccccc}
\toprule
\textbf{Modell} & $\Omega_m$ & \textbf{Parameter} & $\chi^2$ & $\Delta\chi^2$ & AIC & BIC \\
\midrule
$\Lambda$CDM & 0,244 & 2 & 729,0 & 0 & 733,0 & 743,7 \\
CFM Standard & 0,364 & 4 & 716,8 & $-12{,}2$ & 724,8 & 746,3 \\
\midrule
CFM Baryon fest & 0,050 & 3 & 945,5 & $+216{,}5$ & 951,5 & 967,6 \\
CFM Baryon Band & 0,070 & 4 & 894,7 & $+165{,}7$ & 902,7 & 924,1 \\
\midrule
\textbf{Erweitertes CFM+MOND} & \textbf{0,050} & \textbf{5} & \textbf{702,7} & $\mathbf{-26{,}3}$ & \textbf{712,7} & 739,5 \\
\bottomrule
\end{tabular}
\end{table}

\subsubsection*{3.3\quad Zentrale Ergebnisse}

\begin{enumerate}
\item \textbf{Einfaches Baryonen-CFM scheitert:} Mit $\Omega_m = 0{,}05$ und nur dem $\tanh$-S\"attigungsterm verschlechtert sich der Fit katastrophal ($\Delta\chi^2 = +216{,}5$). Der Optimierer versucht extreme Parameter ($k = 86$, $a_{\mathrm{trans}} = 0{,}06$), um eine nahezu stufenf\"ormige Funktion zu erzeugen, was best\"atigt, dass das Standard-CFM die fehlende Dunkle Materie \textit{nicht} kompensieren kann.

\item \textbf{Erweitertes CFM gelingt spektakul\"ar:} Die Hinzuf\"ugung des geometrischen DM-Terms $\alpha \cdot a^{-\beta}$ stellt die Fit-Qualit\"at wieder her und \textit{\"ubertrifft} sie, mit $\Delta\chi^2 = -26{,}3$ gegen\"uber $\Lambda$CDM -- besser als sowohl $\Lambda$CDM \textit{als auch} das Standard-CFM mit gro\ss{}em Abstand.

\item \textbf{Best-Fit-Parameter (MCMC):} Eine vollst\"andige Markov-Chain-Monte-Carlo-Analyse (emcee, 48 Walker, 5000 Schritte) ergibt:
\begin{itemize}
\item S\"attigungsterm: $\Phi_0 = 0{,}43^{+0{,}14}_{-0{,}08}$, $k = 9{,}8^{+6{,}7}_{-3{,}8}$, $a_{\mathrm{trans}} = 0{,}971^{+0{,}016}_{-0{,}031}$ ($z_{\mathrm{trans}} = 0{,}03$)
\item Geometrischer DM-Term: $\alpha = 0{,}68^{+0{,}02}_{-0{,}07}$, $\beta = 2{,}02^{+0{,}26}_{-0{,}14}$
\item Energiebudget bei $a = 1$: $\Omega_b = 0{,}05$, $\Omega_\Phi = 0{,}95$ (gesamter geometrischer Beitrag)
\end{itemize}

\item \textbf{$\beta \approx 2{,}0$: Kr\"ummungsskalierung.} Das MCMC-Posterior f\"ur $\beta$ liegt bei $2{,}02 \pm 0{,}20$, konsistent mit \textit{kr\"ummungsartiger Skalierung} ($a^{-2}$, d.\,h.\ $w = -1/3$). Dies ist ein bemerkenswertes Ergebnis: Die Daten liefern unabh\"angig einen Skalierungsexponenten, der \textit{r\"aumlicher Kr\"ummung} entspricht, nicht einer materiellen Komponente. Die effektive Zustandsgleichung $w_{\mathrm{DM,geom}} = \beta/3 - 1 = -0{,}33$ ist praktisch identisch mit der Kr\"ummungs-Zustandsgleichung.

\item \textbf{Sp\"ater S\"attigungs\"ubergang:} Der S\"attigungs\"ubergang erfolgt sehr sp\"at ($z_{\mathrm{trans}} \approx 0{,}03$), viel sp\"ater als im Standard-CFM ($z_{\mathrm{trans}} = 0{,}33$). Der geometrische DM-Term (kr\"ummungsartig) dominiert die fr\"uhe Expansion, w\"ahrend der S\"attigungsterm die sp\"atzeitige Beschleunigung liefert.

\item \textbf{AIC vs.\ BIC:} Das $\Delta\mathrm{AIC} = -16{,}3$ bevorzugt stark das erweiterte Modell. Das $\Delta\mathrm{BIC} = -4{,}2$ bevorzugt es ebenfalls trotz der Parameterstrafe (5 vs.\ 2 Parameter). Dies ist das erste Modell in unserer Analyse, das \textit{sowohl} AIC- \textit{als auch} BIC-Pr\"aferenz gegen\"uber $\Lambda$CDM gleichzeitig erzielt.
\end{enumerate}

\subsubsection*{3.4\quad Kreuzvalidierung: Ausschluss von \"Uberanpassung}

Mit f\"unf freien Parametern gegen\"uber zwei f\"ur $\Lambda$CDM ist eine Bedenken hinsichtlich \"Uberanpassung nat\"urlich. Wir adressieren dies mit einer rigorosen 5-Fold-Kreuzvalidierung am Pantheon+-Datensatz ($n = 1.590$).

\textbf{Methode:} Die Daten werden zuf\"allig in f\"unf gleiche Folds aufgeteilt (Seed $= 42$). F\"ur jeden Fold werden beide Modelle ($\Lambda$CDM und Erweitertes CFM) auf den verbleibenden 80\% (Trainingssatz) mittels differentieller Evolution optimiert, und der pr\"adiktive $\chi^2/n$ wird am zur\"uckgehaltenen 20\%-Testsatz evaluiert. Dieses Verfahren testet \textit{Generalisierung}, nicht blo\ss{}e Anpassungsg\"ute.

\begin{table}[H]
\centering
\caption{5-Fold-Kreuzvalidierung: pr\"adiktiver $\chi^2/n$ auf zur\"uckgehaltenen Testdaten.}
\begin{tabular}{lcccccc}
\toprule
\textbf{Modell} & \textbf{Fold 1} & \textbf{Fold 2} & \textbf{Fold 3} & \textbf{Fold 4} & \textbf{Fold 5} & $\langle\chi^2/n\rangle$ \\
\midrule
$\Lambda$CDM (2 Param.) & 0,467 & 0,428 & 0,461 & 0,435 & 0,468 & $0{,}452 \pm 0{,}017$ \\
Erw.\ CFM+MOND (5 Param.) & 0,456 & 0,428 & 0,465 & 0,411 & 0,467 & $0{,}445 \pm 0{,}022$ \\
\bottomrule
\end{tabular}
\end{table}

\textbf{Ergebnis:} Das Erweiterte CFM erzielt einen \textit{niedrigeren} mittleren pr\"adiktiven $\chi^2/n$ auf ungesehenen Daten ($\Delta\langle\chi^2/n\rangle = -0{,}007$). Trotz 2{,}5$\times$ mehr Parametern generalisiert das Modell \textit{besser} als $\Lambda$CDM, was \"Uberanpassung als Erkl\"arung f\"ur die $\Delta\chi^2 = -26{,}3$-Verbesserung ausschlie\ss{}t.

% -------------------------------------------------------------------
% 4. DISKUSSION
% -------------------------------------------------------------------
\subsection*{4\quad Diskussion}

\subsubsection*{4.1\quad Ein Universum ohne Dunklen Sektor}

Das erweiterte CFM demonstriert, dass die gesamte Expansionsgeschichte, die durch Typ~Ia-Supernovae erfasst wird, beschrieben werden kann mit:
\begin{itemize}
\item Baryonischer Materie ($\Omega_b = 0{,}05$) -- dem \textit{einzigen} materiellen Inhalt
\item Einem S\"attigungs-geometrischen Potential -- als Ersatz f\"ur Dunkle Energie
\item Einem Potenzgesetz-geometrischen Term -- als Ersatz f\"ur die kosmologische Rolle der Dunklen Materie
\end{itemize}

Falls dieses Ergebnis Tests gegen CMB- und BAO-Daten \"ubersteht, w\"urde es bedeuten, dass 95\% des $\Lambda$CDM-Energiebudgets ein Artefakt der Interpretation geometrischer Effekte als materielle Komponenten sind.

\subsubsection*{4.2\quad Das $\beta \approx 2{,}0$-Ergebnis: Kr\"ummung als Dunkle Materie}

Das MCMC-Posterior f\"ur den Skalierungsexponenten ergibt $\beta = 2{,}02^{+0{,}26}_{-0{,}14}$, bemerkenswert nah an -- und statistisch konsistent mit -- der Kr\"ummungsskalierung $\beta = 2$ ($a^{-2}$). Dies entspricht einer effektiven Zustandsgleichung $w_{\mathrm{DM,geom}} = -0{,}33$, ununterscheidbar von r\"aumlicher Kr\"ummung ($w_k = -1/3$). Zum Vergleich skalieren die kosmologischen Standardkomponenten wie folgt:
\begin{itemize}
\item Materie: $\beta = 3$ ($a^{-3}$, $w = 0$)
\item Kr\"ummung: $\beta = 2$ ($a^{-2}$, $w = -1/3$) $\quad\leftarrow$ \textbf{durch MCMC rekonstruiert}
\item Strahlung: $\beta = 4$ ($a^{-4}$, $w = 1/3$)
\end{itemize}

Dieses Ergebnis hat tiefgreifende Implikationen: Die Komponente, die traditionell als "`Dunkle Materie"' in der Friedmann-Gleichung identifiziert wird, k\"onnte tats\"achlich \textit{r\"aumliche Kr\"ummung} sein -- nicht die globale Kr\"ummung $k$ der FLRW-Metrik, sondern ein \textit{dynamisches, zerfallendes Kr\"ummungsged\"achtnis}, kodiert im geometrischen Potential. Im spieltheoretischen Rahmenwerk ist dies genau die "`geometrische Tr\"agheit"' der Kr\"ummungsr\"uckstellung: ein residualer Abdruck der Energiekonzentration des Urknalls, der mit der Expansion bei der Kr\"ummungsrate $a^{-2}$ verd\"unnt statt bei der Materierate $a^{-3}$.

\subsubsection*{4.3\quad Beziehung zu AeST und relativistischer MOND}

Die relativistische MOND-Theorie AeST (Aether-Skalar-Tensor) \cite{Skordis2021} liefert das einzige bekannte Rahmenwerk, das gleichzeitig:
\begin{enumerate}
\item MOND-Dynamik auf galaktischen Skalen reproduziert
\item Das CMB-Leistungsspektrum fittet (einschlie\ss{}lich des dritten akustischen Maximums)
\item Das Materiedichtespektrum fittet
\end{enumerate}

AeST erreicht dies durch ein Skalarfeld $\phi$ und ein zeitartiges Vektorfeld $A_\mu$, die einen effektiven Energie-Impuls-Tensor erzeugen. Die kosmologischen Hintergrundgleichungen in AeST enthalten Terme, die mit nicht-standardm\"a\ss{}iger Skalierung zu $H^2(a)$ beitragen. Ein detaillierter Vergleich zwischen den AeST-Hintergrundgleichungen und der erweiterten CFM-Friedmann-Gleichung~(3') ist ein zentrales Ziel f\"ur zuk\"unftige Arbeiten.

\subsubsection*{4.4\quad Die kosmologischen "`Endgegner"'}

Jede Theorie, die Dunkle Materie eliminiert, muss sich drei kritischen Beobachtungss\"aulen von $\Lambda$CDM stellen. Wir adressieren jede einzeln und zeigen, wie die Hypothese der Zerfallenden Dunklen Geometrie einen Weg durch jede Herausforderung bietet.

\paragraph{Herausforderung 1: Akustische CMB-Maxima}

Die relativen H\"ohen der akustischen CMB-Maxima -- insbesondere das Verh\"altnis des ersten zum dritten Maximum -- werden konventionell als Beweis f\"ur eine Gravitationskomponente interpretiert, die nicht mit Photonen wechselwirkt (d.\,h.\ Dunkle Materie). In $\Lambda$CDM liefert kalte Dunkle Materie Gravitationspotentialmulden, die Baryon-Photon-Oszillationen antreiben, ohne Strahlungsdruck zu erfahren.

\textit{Quantitative Bewertung:} Bei der Rekombination ($z_* = 1090$, $a_* \approx 9{,}2 \times 10^{-4}$) betragen die Beitr\"age zu $H^2/H_0^2$:
\begin{itemize}
\item Baryonische Materie: $\Omega_b \cdot a_*^{-3} \approx 6{,}5 \times 10^7$
\item Geometrischer DM-Term: $\alpha \cdot a_*^{-\beta} \cdot \mathcal{S}(a_*) \approx 3{,}1 \times 10^5$ \quad ($\mathcal{S}(a_*) = 0{,}34$)
\item Zum Vergleich CDM in $\Lambda$CDM: $\Omega_{\mathrm{cdm}} \cdot a_*^{-3} \approx 3{,}5 \times 10^8$
\end{itemize}

Der geometrische DM-Term tr\"agt auf Hintergrundniveau nur $\sim$0{,}5\% der baryonischen Dichte bei $z_*$ bei, da $a^{-\beta}$ mit $\beta \approx 2$ \textit{langsamer} skaliert als Materie ($a^{-3}$). \textit{Dies ist jedoch nicht der relevante Mechanismus.} Die akustischen CMB-Maxima werden durch \textit{Metrik-St\"orungen} $\Phi$ und $\Psi$ bestimmt, nicht durch Hintergrunddichtebeitr\"age. Im erweiterten CFM enth\"alt die Lagrange-Dichte (Paper~III \cite{Geiger2026c}) einen $R^2$-Term und ein Skalarfeld $\phi$ mit P\"oschl-Teller-Potential, die beide die St\"orungsgleichungen unabh\"angig von ihrem Hintergrundbeitrag modifizieren.

\textit{Existenzbeweis durch AeST:} Die relativistische MOND-Theorie AeST \cite{Skordis2021} hat gezeigt, dass ein reines Baryonen-Universum ($\Omega_m = \Omega_b$) mit zus\"atzlichen geometrischen Freiheitsgraden (Skalar- + Vektorfelder) das \textit{vollst\"andige} CMB-Leistungsspektrum fitten kann, einschlie\ss{}lich des dritten akustischen Maximums. Das erweiterte CFM teilt die wesentlichen Zutaten mit AeST:
\begin{itemize}
\item Reiner Baryonen-Materieinhalt ($\Omega_m = \Omega_b \approx 0{,}05$)
\item Ein Skalarfeld, das auf St\"orungsniveau zus\"atzliche Gravitationspotentiale liefert
\item Spur-Kopplung, die Unterdr\"uckung w\"ahrend der Strahlungs\"ara sicherstellt
\item Zus\"atzliche geometrische Freiheitsgrade ($R^2$-Term im CFM vs.\ Vektorfeld in AeST)
\end{itemize}

Der AeST-Pr\"azedenzfall belegt, dass das \textit{Prinzip} der CMB-Kompatibilit\"at ohne CDM bewiesen ist.

\textit{Vorl\"aufige St\"orungsanalyse:} Eine "`Effective CDM"'-Analyse mit CAMB \cite{Lewis2000} ergibt ein vielversprechendes $C_\ell$-Spektrum: $\ell_1 = 223$ (Planck: 220), $\mathcal{P}_3/\mathcal{P}_1 = 0{,}421$ (\textbf{97{,}9\%} des Planck-Werts 0{,}430). Die BBN-Konsistenz ist vollst\"andig gew\"ahrleistet ($\mu \to 1$ bei $z > 10^4$, $\Delta N_{\mathrm{eff}} \approx 0$). Die vollst\"andige Berechnung mit modifizierten St\"orungsgleichungen (Poisson-Gleichung, anisotroper Stress) mittels hi\_class \cite{Zumalacarregui2017} oder EFTCAMB \cite{Hu2014} steht noch aus.

\paragraph{Herausforderung 2: Der Bullet-Cluster}

Der Bullet-Cluster (1E~0657-56) wird h\"aufig als definitiver Beweis f\"ur partikul\"are Dunkle Materie angef\"uhrt: Gravitationslinsen-Karten zeigen Massenkonzentrationen, die nach einer Haufenkollision vom r\"ontgenemittierenden Gas versetzt sind \cite{Clowe2006}. Das Argument lautet, dass Dunkle Materie als sto\ss{}freie Komponente hindurchging, w\"ahrend das Gas durch Staudruck abgebremst wurde.

\textit{Aufl\"osung:} Im Rahmenwerk der Zerfallenden Dunklen Geometrie ist die "`Dunkle Materie"'-Komponente \textit{Raumzeitgeometrie}, keine Substanz. Bei der Rotverschiebung des Bullet-Clusters ($z = 0{,}296$, $a = 0{,}77$) betr\"agt das Hintergrundverh\"altnis von geometrischer DM zu baryonischer Materie:
\begin{equation}
\frac{\alpha \cdot a^{-\beta}}{\Omega_b \cdot a^{-3}} \bigg|_{z=0{,}296} \approx 10{,}6
\tag{*}
\end{equation}
Zum Vergleich ist das CDM-zu-Baryon-Verh\"altnis in $\Lambda$CDM $\Omega_{\mathrm{cdm}}/\Omega_b \approx 5{,}4$. Das geometrische Potential ist somit \textit{quantitativ ausreichend}, um die erforderliche Linsenkonvergenz in dieser Epoche bereitzustellen.

W\"ahrend einer Haufenkollision:
\begin{itemize}
\item Das \textit{baryonische Gas} erf\"ahrt Staudruck und wird abgebremst.
\item Das \textit{geometrische Potential} ist eine Eigenschaft der Kr\"ummungsverteilung der Raumzeit, die von der gesamten Energieverteilung \textit{einschlie\ss{}lich ihrer eigenen Geschichte} gespeist wird. Als geometrisches "`Ged\"achtnis"' folgt es der Massenverteilung vor der Kollision und muss die Gasverteilung nach der Kollision nicht instantan nachverfolgen.
\item Die \textit{Galaxien} (stellare Komponente), die wie das geometrische Potential effektiv sto\ss{}frei sind, passieren ungehindert.
\end{itemize}

Das Linsensignal w\"urde dann das geometrische Potential (das mit den Galaxien mitbewegt) statt des Gases nachzeichnen -- genau wie beobachtet. Dies ist analog zur AeST-Vorhersage, bei der die Skalar- und Vektorfelder Linseneffekte erzeugen, die vom Gas versetzt sind. Eine quantitative Linsenvorhersage erfordert die L\"osung der St\"orungsgleichungen der vollst\"andigen CFM-Lagrange-Dichte (Paper~III \cite{Geiger2026c}), aber die Hintergrund-Analyse best\"atigt, dass das geometrische Potential die richtige Gr\"o\ss{}enordnung hat.

\paragraph{Herausforderung 3: Strukturbildung und das Materiedichtespektrum}

Das Materiedichtespektrum $P(k)$ in $\Lambda$CDM wird durch Dunkle-Materie-Halos geformt, die w\"ahrend der Strahlungsdominanz mit dem gravitativen Kollaps beginnen (bevor Baryonen von Photonen entkoppeln). Ohne fr\"uh kollabierende Dunkle Materie w\"urden baryonische Strukturen zu sp\"at und auf falschen Skalen entstehen.

\textit{Aufl\"osung:} Der geometrische DM-Term liefert "`geometrisches Ger\"ust"' f\"ur die Strukturbildung:
\begin{itemize}
\item Zu fr\"uhen Zeiten ($a \ll a_{\mathrm{trans}}$) dominiert der $\alpha \cdot a^{-2}$-Term die Expansionsgeschichte und liefert dieselbe Abbremsung, die CDM liefern w\"urde (wenn auch mit anderer Skalierung).
\item St\"orungen im geometrischen Potential erzeugen Gravitationsmulden, in die Baryonen nach der Rekombination fallen k\"onnen, genau wie CDM-Halos es t\"aten.
\item Der fr\"uhere Einsatz der effektiven Gravitation (aus der kombinierten CFM + MOND-Verst\"arkung) erkl\"art nat\"urlicherweise die "`zu fr\"uhen, zu massereichen"' Strukturen, die von JWST \cite{Labbe2023}, El~Gordo \cite{Asencio2023} und Hochrotverschiebungs-Protoclustern \cite{Miller2018} beobachtet wurden -- die in $\Lambda$CDM anomal sind, in diesem Rahmenwerk aber erwartet werden.
\end{itemize}

Eine vorl\"aufige $P(k)$-Analyse mit dem "`Effective CDM"'-Mapping in CAMB \cite{Lewis2000} best\"atigt die korrekte qualitative Form: Die Turnover-Skala ($k_{\mathrm{peak}} \approx 0{,}015$\,$h$/Mpc) liegt nahe an $\Lambda$CDM (0{,}017), und die epochenabh\"angige effektive Materiedichte $\Omega_{m,\mathrm{eff}}(z)$ stimmt bei $z \approx 500$ \textit{exakt} mit $\Lambda$CDM \"uberein ($\Omega_{m,\mathrm{eff}} = 0{,}315$). Die quantitative Vorhersage von $P(k)$ mit den vollst\"andigen St\"orungsgleichungen ist ein zentrales Ziel f\"ur zuk\"unftige Arbeiten.

\subsubsection*{4.4.1\quad Ontologische Interpretation: Die verschachtelte Hierarchie}

Die Hypothese der Zerfallenden Dunklen Geometrie legt eine verschachtelte ontologische Struktur nahe, die im spieltheoretischen Rahmenwerk \cite{Geiger2026} implizit enthalten ist:

\begin{enumerate}
\item \textbf{Nullraum} ("`Mutter"'): Der pr\"ageometrische Grundzustand, dessen Konzentrationsgradient $G$ die Entstehung der Raumzeitblase antreibt.
\item \textbf{Raumzeitgeometrie} ("`Tochter"'): Das Kr\"ummungsr\"uckstellpotential, das sich als geometrische DM ($\alpha \cdot a^{-\beta}$, fr\"uh) und geometrische DE ($\Phi_0 \cdot f_{\mathrm{sat}}$, sp\"at) manifestiert. Der "`dunkle Sektor"' \textit{ist} die Geometrie.
\item \textbf{Baryonische Materie} ("`Enkelin"'): Die Nash-optimalen entropieproduzierenden Agenten, kondensiert innerhalb des geometrischen Substrats.
\end{enumerate}

Diese Hierarchie -- Nullraum $\to$ Geometrie $\to$ Materie -- invertiert die konventionelle materialistische Ontologie und liefert eine testbare Konsequenz: Die geometrische "`Dunkle Materie"' kann nicht in Teilchenexperimenten nachgewiesen werden, denn sie ist die Raumzeitgeometrie selbst.

\paragraph{Die fehlende Lagrange-Dichte}

Eine vierte, theoretische Herausforderung bleibt: Dem erweiterten CFM fehlt derzeit eine Lagrange-Formulierung. Die Differentialgleichung $d\Omega_\Phi/da = k[1 - (\Omega_\Phi/\Phi_0)^2]$ und der Potenzgesetz-Term $\alpha \cdot a^{-\beta}$ sind ph\"anomenologisch. Eine vollst\"andige Theorie erfordert:
\begin{enumerate}
\item Ein Wirkungsprinzip, aus dem die erweiterte Friedmann-Gleichung~(3') als Euler-Lagrange-Gleichung folgt
\item Eine mikroskopische Herleitung, die erkl\"art, \textit{warum} die S\"attigungs-Differentialgleichung die spezifische Form $dX/da \propto (1 - X^2)$ annimmt
\item Eine Verbindung zu bekannten Quantengravitationsans\"atzen (Schleifen-Quantengravitation, Finsler-Geometrie, informationstheoretische Raumzeit)
\end{enumerate}

Diese theoretische Grundlage ist Gegenstand von Paper~III \cite{Geiger2026c}.

\subsubsection*{4.5\quad Kritische Selbstbewertung: Zu gut um wahr zu sein?}

Die Ergebnisse dieses Papers -- ein reines Baryonen-Modell mit skalenabh\"angigem $\mu(a)$, das $\Lambda$CDM um $\Delta\chi^2 = -5{,}5$ im gemeinsamen SN+CMB+BAO-Fit \"ubertrifft, bei \textit{gleicher} Parameterzahl und \textit{ohne} EDE -- sind bemerkenswert. Wir z\"ahlen die Gr\"unde zur Vorsicht auf:

\begin{enumerate}
\item \textbf{\"Uberanpassungsrisiko -- durch Kreuzvalidierung ausgeschlossen:} F\"unf freie Parameter (vs.\ 2 f\"ur $\Lambda$CDM) bieten mehr Flexibilit\"at. \"Uber die informationstheoretischen Strafen hinaus ($\Delta\mathrm{AIC} = -16{,}3$, $\Delta\mathrm{BIC} = -4{,}2$) haben wir eine rigorose 5-Fold-Kreuzvalidierung durchgef\"uhrt (Abschnitt~3.4): Das erweiterte CFM erzielt einen \textit{niedrigeren} mittleren pr\"adiktiven $\chi^2/n$ auf ungesehenen Daten ($0{,}445 \pm 0{,}022$ vs.\ $0{,}452 \pm 0{,}017$), was best\"atigt, dass die Verbesserung nicht auf \"Uberanpassung zur\"uckzuf\"uhren ist.

\item \textbf{Nur-SN-Validierung:} Die Pantheon+-Daten erfassen die Expansionsgeschichte bei $z \lesssim 2{,}3$. Die Vorhersagen des Modells bei hoher Rotverschiebung (CMB bei $z \approx 1100$) sind Extrapolationen. Der Spur-Kopplungsmechanismus (Abschnitt~2.2) verhindert, dass der geometrische DM-Term zu fr\"uhen Zeiten divergiert, aber das quantitative Verhalten um Rekombination und Materie-Strahlungs-Gleichheit erfordert detaillierte numerische Berechnungen.

\item \textbf{Ph\"anomenologische Natur:} Der $\alpha \cdot a^{-\beta}$-Term ist empirisch, nicht aus ersten Prinzipien hergeleitet. Ein ph\"anomenologischer Term, der Supernovae gut fittet, aber ohne Lagrange-Herleitung ist, kann nicht als vollst\"andige Theorie betrachtet werden.

\item \textbf{Die $\beta = 2$-Koinzidenz:} W\"ahrend wir $\beta \approx 2$ als Hinweis auf einen Kr\"ummungsursprung interpretieren, existieren alternative Erkl\"arungen. Die $a^{-2}$-Skalierung k\"onnte ein Zufall oder ein Artefakt der Parametrisierung sein.

\item \textbf{St\"orungstheorie:} Die Hintergrund-Observablen sind vollst\"andig reproduziert. Die "`Effective CDM"'-Analyse mit CLASS/hi\_class \cite{Lewis2000, Zumalacarregui2017} ergibt $\mathcal{P}_3/\mathcal{P}_1 = 0{,}4212$ (98{,}1\% von Planck); mit minimaler $\beta_{\mathrm{early}}$-Anpassung ($2{,}82 \to 2{,}829$, nur 0{,}32\%) erreichen wir $\mathcal{P}_3/\mathcal{P}_1 = 0{,}4295$ (\textbf{exakter Planck-Match}). BBN ist konsistent ($\Delta N_{\mathrm{eff}} \approx 0$). Die zentrale verbleibende Herausforderung ist der $\theta_s$-Offset ($1{,}025$ vs.\ Planck $1{,}041$), verursacht durch die effektive Zustandsgleichung des geometrischen Terms ($w = -0{,}06$ bei Rekombination $\to$ 2{,}5\% gr\"o\ss{}erer Schalchorizont).
\end{enumerate}

\textbf{Ehrliche Bewertung:} Das skalenabh\"angige $\mu(a)$ l\"ost alle zuvor kritischen Probleme: $H_0$ ($60 \to 67{,}3$\,km/s/Mpc), Schallhorizont ($165 \to 146{,}9$\,Mpc), EDE ($52\% \to 0\%$), Parameterzahl ($\sim$9 $\to$ 6), BBN-Konsistenz ($\Delta N_{\mathrm{eff}} \approx 0$). Die St\"orungsanalyse mit CLASS/hi\_class ergibt $\mathcal{P}_3/\mathcal{P}_1 = 0{,}4295$ (100\% Planck) bei optimiertem $\beta_{\mathrm{early}} = 2{,}829$. Die zentrale Herausforderung ist der $\theta_s$-Offset ($1{,}025$ vs.\ $1{,}041$), dessen Ursprung die effektive Zustandsgleichung $w = -0{,}06$ des geometrischen Terms bei Rekombination ist. Vollst\"andige modifizierte Boltzmann-Gleichungen und die Lagrange-Herleitung bleiben als Aufgaben bestehen.

\subsubsection*{4.6\quad Einschr\"ankungen und verbleibende Herausforderungen}

\begin{enumerate}
\item \textbf{CMB-Leistungsspektrum:} Das Winkelleistungsspektrum $C_\ell$ ist der kritischste verbleibende Test. W\"ahrend der geometrische DM-Term ($\beta \approx 2$) auf Hintergrundniveau bei der Rekombination subdominant ist, werden die St\"orungseffekte des $R^2$-Terms und des Skalarfelds aus der CFM-Lagrange-Dichte (Paper~III) die Metrik-St\"orungen $\Phi$ und $\Psi$ modifizieren. Der AeST-Pr\"azedenzfall \cite{Skordis2021} zeigt, dass dieser Mechanismus in einem reinen Baryonen-Universum funktionieren kann. Die Berechnung von $C_\ell$ mit den spezifischen CFM-St\"orungsgleichungen ist in Vorbereitung.

\item \textbf{BAO-Messungen:} Baryonische akustische Oszillationen bei $z \sim 0{,}5$--$2{,}5$ (DESI DR2) liefern ein unabh\"angiges Entfernungsma\ss{}, das mit dem erweiterten CFM konsistent sein muss.

\item \textbf{Urknall-Nukleosynthese -- KONSISTENT:} Das skalenabh\"angige $\mu(a)$ geht bei $z > z_\mu \approx 3918$ auf $\mu \to 1$ \"uber. Numerische Auswertung best\"atigt $\mu(z = 10^{10}) = 1{,}000$ und $\mu(z = 3 \times 10^8) = 1{,}000$, sodass die MOND-Verst\"arkung w\"ahrend der BBN vollst\"andig abwesend ist. Das resultierende $\Delta N_{\mathrm{eff}} \approx 0{,}000$ liegt gut innerhalb der Planck-Schranke ($N_{\mathrm{eff}} = 3{,}046 \pm 0{,}2$) und der BBN-Schranke ($N_{\mathrm{eff}} = 2{,}88 \pm 0{,}28$; \cite{Pitrou2018}).

\item \textbf{Gravitationslinseneffekt:} Starke und schwache Linsensurveys (KiDS, DES, Euclid) erfassen die Materieverteilung und m\"ussen mit dem geometrischen Potential kompatibel sein.

\item \textbf{Lagrange-Herleitung:} Der ph\"anomenologische Erfolg muss in einem Wirkungsprinzip verankert werden (Paper~III).
\end{enumerate}

% -------------------------------------------------------------------
% 5. FAZIT UND AUSBLICK
% -------------------------------------------------------------------
\subsection*{5\quad Fazit und Ausblick}

Wir haben gezeigt, dass ein reines Baryonen-Universum ($\Omega_m = \Omega_b \approx 0{,}05$) mit einem erweiterten geometrischen Potential kosmologische Daten \textit{\"uber alle drei gro\ss{}en Sonden hinweg} -- Supernovae, CMB und BAO -- kompetitiv mit und teilweise besser als $\Lambda$CDM fitten kann, bei \textit{gleicher Parameterzahl} und \textit{ohne} Early Dark Energy.

Drei Schl\"usselinnovationen erm\"oglichen dies: (i)~die \textit{laufende Kr\"ummungskopplung} $\beta_{\mathrm{eff}}(a)$; (ii)~die \textit{MOND-Hintergrundkopplung} $\mu(a) = \sqrt{\pi}$ bei sp\"aten Zeiten; und (iii)~die \textit{skalenabh\"angige} Evolution $\mu(a) \to 1$ bei $z > 4000$, die EDE vollst\"andig eliminiert.

Die quantitativen Ergebnisse umfassen zwei Analyseebenen:
\begin{enumerate}
\item \textbf{Nur SN (konstantes $\beta$):} $\Delta\chi^2 = -26{,}3$ ($\Delta\mathrm{AIC} = -16{,}3$, $\Delta\mathrm{BIC} = -4{,}2$). Eine 5-Fold-Kreuzvalidierung best\"atigt Generalisierung.
\item \textbf{Gemeinsam SN + CMB + BAO (laufendes $\beta$ + $\mu(a)$, kein EDE):} $\Delta\chi^2 = -5{,}5$, mit $\ell_A = 301{,}471$, $\mathcal{R} = 1{,}7502$, $H_0 = 67{,}3$\,km/s/Mpc, $r_d = 146{,}9$\,Mpc und 6 freien Parametern. Die skalenabh\"angige MOND-Kopplung $\mu(a)$ l\"ost alle bisherigen Probleme: $H_0$, Schallhorizont und EDE.
\end{enumerate}

Die Drei-Phasen-Interpretation ergibt sich nat\"urlich: Bei $z > z_\mu \approx 4000$ gilt Standardgravitation ($\mu \to 1$); bei $z_\mu > z > z_t$ aktiviert sich die MOND-Verst\"arkung ($\mu \to \sqrt{\pi}$); bei $z < z_t \approx 9$ treibt der S\"attigungsterm kosmische Beschleunigung an. Die St\"orungsanalyse mit CLASS/hi\_class \cite{Lewis2000, Zumalacarregui2017} ergibt $\mathcal{P}_3/\mathcal{P}_1 = 0{,}4295$ (\textbf{exakter Planck-Match}) bei optimiertem $\beta_{\mathrm{early}} = 2{,}829$ (0{,}32\% Anpassung). BBN ist vollst\"andig konsistent ($\Delta N_{\mathrm{eff}} \approx 0$), $P(k)$-Form qualitativ korrekt. Die zentrale verbleibende Herausforderung ist der $\theta_s$-Offset ($1{,}025$ vs.\ $1{,}041$) aus der effektiven Zustandsgleichung $w = -0{,}06$ des geometrischen Terms. Weitere Aufgaben: Lagrange-Herleitung von $\mu(a)$ und $\beta(a)$, vollst\"andige $C_\ell$-Berechnung mit modifizierten Boltzmann-Gleichungen.

\subsubsection*{5.1\quad Das Drei-Paper-Programm}

Dieses Paper ist das zweite in einem dreiteiligen Programm:
\begin{enumerate}
\item \textbf{Paper~I} \cite{Geiger2026}: Etabliert die spieltheoretische Grundlage und das Kr\"ummungs-R\"uckkopplungsmodell als Ersatz f\"ur Dunkle Energie. Validiert gegen Pantheon+.
\item \textbf{Paper~II} (diese Arbeit): Erweitert das CFM zur Eliminierung des gesamten dunklen Sektors. F\"uhrt die laufende Kr\"ummungskopplung $\beta_{\mathrm{eff}}(a)$ und die skalenabh\"angige MOND-Kopplung $\mu(a)$ ein. Demonstriert gemeinsame SN + CMB + BAO-Kompatibilit\"at ($\Delta\chi^2 = -5{,}5$ vs.\ $\Lambda$CDM, $H_0 = 67{,}3$\,km/s/Mpc, 6 Parameter, kein EDE).
\item \textbf{Paper~III} \cite{Geiger2026c}: Liefert die mikroskopische Grundlage -- die Lagrange-Herleitung, die Verbindung zur Quantengravitation und die Interpretation des laufenden $\beta$ als geometrischer Phasen\"ubergang.
\end{enumerate}

\textbf{N\"achste unmittelbare Schritte:}
\begin{enumerate}
\item Vollst\"andiges CMB-Leistungsspektrum $C_\ell$ mit modifizierten St\"orungsgleichungen (Poisson-Gleichung, anisotroper Stress) mittels hi\_class \cite{Zumalacarregui2017} oder EFTCAMB \cite{Hu2014}. Die vorl\"aufige "`Effective CDM"'-Analyse ($\mathcal{P}_3/\mathcal{P}_1 = 0{,}421$, 97{,}9\% von Planck) liefert eine starke Ausgangsbasis.
\item Pr\"azisions-BAO-Analyse mit DESI DR2-Daten
\item Materiedichtespektrum $P(k)$ mit vollst\"andigen St\"orungsgleichungen: Die vorl\"aufige Analyse best\"atigt die korrekte Form, aber $\sigma_8 = 0{,}90$ in der "`Effective CDM"'-N\"aherung ist zu hoch -- die vollst\"andige Behandlung sollte dies reduzieren.
\item \sout{BBN-Konsistenzpr\"ufung} -- \textbf{ERLEDIGT:} $\mu(z > 10^4) \to 1$, $\Delta N_{\mathrm{eff}} \approx 0{,}000$ \cite{Pitrou2018}
\item Lagrange-Herleitung des laufenden $\beta$ und $\mu$ aus der kr\"ummungsquadratischen Wirkung (Paper~III)
\end{enumerate}

\subsubsection*{5.2\quad Einladung an die Gemeinschaft}

Diese Arbeit pr\"asentiert eine vielversprechende Hypothese, keine gesicherte Schlussfolgerung. Der Autor l\"adt die Gemeinschaft ein:
\begin{enumerate}
\item \textbf{Replizieren:} Der Analysecode ist quelloffen. Alle Fits verwenden den \"offentlich verf\"ugbaren Pantheon+-Katalog. Eine unabh\"angige Replikation des SN-Ergebnisses ($\Delta\chi^2 = -26{,}3$) und des gemeinsamen Fits ($\Delta\chi^2 = -5{,}5$, kein EDE) ist unkompliziert.
\item \textbf{Erweitern:} Die Berechnung von $C_\ell$ und $P(k)$ mit dem laufenden $\beta(a)$ + $\mu(a)$ Hintergrund ist der kritische n\"achste Schritt.
\item \textbf{Kritisieren:} Die laufende-$\beta$- und $\mu(a)$-Parametrisierung, der Spur-Kopplungsmechanismus und die Effizienzhypothese erfordern alle eine unabh\"angige \"Uberpr\"ufung.
\end{enumerate}

\begin{quote}
\textit{"`Wenn Dunkle Energie eine relaxierende Randbedingung ist und Dunkle Materie ein geometrischer Schatten, dann k\"onnten 95\% des Universums die ganze Zeit sichtbar gewesen sein -- als die Geometrie der Raumzeit selbst."'}
\end{quote}

\selectlanguage{english}

\end{document}

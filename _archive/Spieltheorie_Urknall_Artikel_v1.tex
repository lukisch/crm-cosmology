\documentclass[11pt,a4paper]{article}
\usepackage[utf8]{inputenc}
\usepackage[T1]{fontenc}
\usepackage[english,ngerman]{babel}
\usepackage{geometry}
\geometry{a4paper, left=2.5cm, right=2.5cm, top=2.5cm, bottom=2.5cm}
\usepackage{mathptmx}
\usepackage{helvet}
\usepackage{amsmath}
\usepackage{amssymb}
\usepackage{amsthm}
\usepackage{titlesec}
\usepackage{booktabs}
\usepackage{tabularx}
\usepackage{xcolor}
\usepackage{authblk}
\usepackage{hyperref}
\usepackage{enumitem}
\usepackage{graphicx}
\usepackage{float}
\usepackage{setspace}
\usepackage{longtable}
\usepackage{multirow}
\usepackage{array}

\newtheorem{definition}{Definition}
\newtheorem{proposition}{Proposition}

\titleformat{\section}{\Large\bfseries\sffamily\color{black}}{\thesection}{1em}{}
\titleformat{\subsection}{\large\bfseries\sffamily\color{darkgray}}{\thesubsection}{1em}{}
\titleformat{\subsubsection}{\normalsize\bfseries\sffamily\color{darkgray}}{\thesubsubsection}{1em}{}

\hypersetup{
    pdftitle={Spieltheoretische Kosmologie und das Kr\"ummungs-R\"uckgabepotential-Modell},
    pdfauthor={Lukas Geiger},
    colorlinks=true,
    linkcolor=black,
    urlcolor=blue,
    citecolor=black
}

\onehalfspacing

\begin{document}

% ===================================================================
% TITELSEITE
% ===================================================================

\title{\textbf{\huge Spieltheoretische Kosmologie und das Kr\"ummungs-R\"uckgabe\-potential-Modell}\\[0.5em]
\Large Nash-Gleichgewichte zwischen Nullraum und Raumzeitblase\\als Erkl\"arungsrahmen f\"ur die beschleunigte Expansion\\[0.3em]
\large Ein integrativer theoretischer Ansatz}

\author[1]{Lukas Geiger\thanks{Korrespondenz: Lukas Geiger, Gei\ss{}b\"uhlweg~1, 79872~Bernau, Deutschland.}}
\affil[1]{Unabh\"angiger Forscher, Bernau im Schwarzwald}

\date{Februar 2026 \\ \vspace{0.5em} \small \textit{Wissenschaftliches Arbeitspapier}}

\maketitle

\begin{abstract}
\noindent Die vorliegende Arbeit entwickelt einen spieltheoretischen Rahmen f\"ur die Kosmologie, in dem die Entstehung und Entwicklung der Raumzeit als Nash-Gleichgewicht zwischen zwei Akteuren modelliert wird: einem metastabilen Quantenvakuum (Nullraum) und einer daraus hervorgehenden Raumzeitblase. Das zentrale Ergebnis ist das \textit{Curvature Feedback Model} (CFM), das die beobachtete beschleunigte Expansion des Universums nicht durch eine neue Energieform (Dunkle Energie) erkl\"art, sondern durch ein nachlassendes Kr\"ummungs-R\"uckgabepotential~$\Phi(a)$ -- ein geometrisches ``Ged\"achtnis'' der anf\"anglichen Energiekonzentration beim Urknall. Die modifizierte Friedmann-Gleichung $H^2(a) = H_0^2\,[\Omega_m\,a^{-3} + \Omega_\Phi(a)]$ mit $\Omega_\Phi(a) = \Phi_0 \cdot \tanh(k\cdot(a - a_{\mathrm{trans}}))$ wird gegen 1590 reale Typ-Ia-Supernovae des Pantheon+-Katalogs \cite{Scolnic2022} getestet -- sowohl mit diagonalen Fehlern als auch mit der vollen statistisch-systematischen Kovarianzmatrix. Unter einer Flachheitsbedingung ($\Omega_m + \Omega_\Phi(a{=}1) = 1$) liefert das CFM $\Delta\chi^2 = -12{,}2$ ($-11{,}2$ mit voller Kovarianz) und $\Delta\mathrm{AIC} = -8{,}2$ ($-7{,}2$) gegen\"uber $\Lambda$CDM; eine 5-Fold-Kreuzvalidierung best\"atigt die bessere Generalisierung. MCMC-Posterioranalysen liefern $\Omega_m = 0{,}368 \pm 0{,}024$, und vier alternative Funktionalformen (logistisch, Error-Funktion, Potenzgesetz) zeigen vergleichbare $\Delta\chi^2$-Werte, was die Robustheit der Ergebnisse belegt. Eine Phantom-Stabilit\"atsanalyse zeigt: kein Big Rip (S\"attigung von $\Omega_\Phi$), asymptotisch de-Sitter-Endzustand. Das Modell sagt eine messbare Zeitvariation des Zustandsgleichungsparameters voraus ($w(z) < -1$, $|\Delta w| \approx 0{,}4$) sowie einen fr\"uheren Beschleunigungs\"ubergang ($z_{\mathrm{acc}} = 0{,}52$ vs.\ $0{,}84$), die mit Euclid und dem Nancy Grace Roman Space Telescope innerhalb der n\"achsten Dekade testbar sind. Es wird gezeigt, dass das CFM konzeptuelle Verbindungen sowohl zur Finsler-Gravitation (Pfeifer et al., 2025) als auch zur j\"ungst vorgeschlagenen \textit{Cosmological Teleodynamics} (Trivedi \& Venkatasubramanian, 2025) aufweist, welche die kosmische Expansion ebenfalls als Konvergenz zu einem Nash-Gleichgewicht beschreibt. Analysecode und Daten sind \"offentlich verf\"ugbar.\footnote{\url{https://github.com/lukisch/cfm-cosmology}} Insbesondere identifiziert die rigorose Analyse mit $z_{\mathrm{acc}} \approx 0{,}52$ einen sp\"ateren Beginn der kosmischen Beschleunigung als $\Lambda$CDM ($z_{\mathrm{acc}} = 0{,}84$), was eine verl\"angerte materiedominierte Wachstumsphase impliziert. Diese Vorhersage bietet eine nat\"urliche Erkl\"arung f\"ur die von JWST entdeckten ``Universe Breakers'' -- unerwartet massereiche Galaxien bei $z > 7$ \cite{Labbe2023, BoylanKolchin2023} -- sowie f\"ur statistisch un\-wahrscheinliche massive Galaxienhaufen wie El~Gordo bei $z \approx 0{,}87$ \cite{Asencio2023}, die das $\Lambda$CDM-Standardmodell unter erheblichen Druck setzen. Die spieltheoretische Perspektive er\"offnet einen Paradigmenwechsel: von ``Was treibt die Beschleunigung an?'' zu ``Warum bremste die Expansion fr\"uher?''

\vspace{0.5em}
\noindent \textbf{Schl\"usselbegriffe:} Spieltheorie, Nash-Gleichgewicht, Kosmologie, Dunkle Energie, Kr\"ummungs-R\"uckgabepotential, Curvature Feedback Model, Friedmann-Gleichung, Finsler-Gravitation, beschleunigte Expansion, Zustandsgleichung

\vspace{0.5em}
\noindent \textbf{Disziplinen:} Theoretische Physik, Kosmologie, Spieltheorie, Mathematische Physik
\end{abstract}

\newpage
\tableofcontents
\newpage

% ===================================================================
% CO-AUTOREN
% ===================================================================
\section*{Angaben zur KI-Nutzung und Methodik}
\addcontentsline{toc}{section}{Angaben zur KI-Nutzung und Methodik}

\noindent\textbf{Erweiterte Methodenerkl\"arung:} Diese Arbeit ist ein Experiment in \textit{AI-Assisted Science}. Die Rollenverteilung wird transparent ausgewiesen:

\begin{description}[style=nextline, leftmargin=2cm]
\item[\textbf{Menschlicher Autor} (Lukas Geiger)] Physikalische Intuition, Grundannahmen (spieltheoretischer Ansatz, S\"attigungshypothese, Geometrie statt Dunkler Sektor), Interpretation der Ergebnisse, strategische Entscheidungen und finale Verantwortung f\"ur den wissenschaftlichen Inhalt.
\item[\textbf{Claude Opus 4.6} (Anthropic)] Co-Writer: Mathematische Formalisierung, Herleitung der Gleichungen, Code-Entwicklung (Python/MCMC), statistische Analyse (Pantheon+ Fits), Textgenese und Strukturierung.
\item[\textbf{Gemini} (Google DeepMind)] Reviewer: Kritisches Lektorat, Konsistenzpr\"ufung, strategische Empfehlungen, MOND-Kompatibilit\"atsanalyse, Identifikation der BBN-Problematik.
\end{description}

\vspace{0.5em}
\noindent\textit{Hinweis:} Die mathematische Formalisierung und die Durchf\"uhrung der statistischen Fits wurden von KI-Systemen durchgef\"uhrt. Der Autor stellt diese Hypothesen als \textit{Working Paper} zur Verf\"ugung, um der wissenschaftlichen Gemeinschaft die Pr\"ufung und Weiterentwicklung zu erm\"oglichen. \textbf{Mathematische Pr\"ufung durch Dritte ist ausdr\"ucklich erw\"unscht.}

\newpage

% ===================================================================
% 1. EINLEITUNG
% ===================================================================
\section{Einleitung}
\label{sec:einleitung}

Die Entdeckung der beschleunigten Expansion des Universums durch die Beobachtung entfernter Typ-Ia-Supernovae im Jahr 1998 durch die Teams um Perlmutter \cite{Perlmutter1999} sowie Riess und Schmidt \cite{Riess1998} markiert einen Wendepunkt der modernen Kosmologie. F\"ur diese Entdeckung wurde 2011 der Nobelpreis f\"ur Physik verliehen. Das Standardmodell der Kosmologie, $\Lambda$CDM, erkl\"art die Beschleunigung durch eine kosmologische Konstante~$\Lambda$, die etwa 68\,\% der Energiedichte des Universums ausmacht \cite{Planck2020}. Trotz seiner empirischen Erfolge steht $\Lambda$CDM vor tiefgreifenden konzeptuellen Problemen:

\begin{enumerate}
\item \textbf{Das Kosmologische-Konstante-Problem:} Die beobachtete Vakuumenergiedichte ist um $\sim$60--120 Gr\"o\ss{}enordnungen kleiner als theoretische Vorhersagen der Quantenfeldtheorie \cite{Weinberg1989}.
\item \textbf{Das Koinzidenz-Problem:} Warum sind $\Omega_m$ und $\Omega_\Lambda$ gerade in der heutigen Epoche von vergleichbarer Gr\"o\ss{}enordnung?
\item \textbf{Die $H_0$-Spannung:} Die lokale Messung des Hubble-Parameters ($H_0 \approx 73$\,km/s/Mpc) weicht signifikant von der aus CMB-Daten abgeleiteten ($H_0 \approx 67{,}4$\,km/s/Mpc) ab \cite{Planck2020, Riess2022}.
\end{enumerate}

J\"ungste Resultate des \textit{Dark Energy Spectroscopic Instrument} (DESI) verst\"arken die Zweifel an einer strikt konstanten Dunklen Energie: Die Analyse baryonischer akustischer Oszillationen in Kombination mit CMB- und Supernova-Daten zeigt eine Pr\"aferenz von 2,5--3,9$\sigma$ f\"ur ein Modell mit zeitabh\"angigem Zustandsgleichungsparameter $w(z)$ gegen\"uber $\Lambda$CDM \cite{DESI2024}.

Parallel dazu zeigen theoretische Arbeiten, dass die Beschleunigung auch ohne Dunkle Energie erkl\"arbar sein k\"onnte: Pfeifer et al.\ \cite{Pfeifer2025} demonstrieren im Rahmen der Finsler-Gravitation, dass eine verallgemeinerte Raumzeitgeometrie nat\"urlicherweise eine exponentielle Expansion im Vakuum erzeugt. Trivedi und Venkatasubramanian \cite{Trivedi2025} zeigen in ihrer \textit{Cosmological Teleodynamics}, dass das Universum wie ein ``riesiges Potentialspiel'' operiert und sich einem kontinuierlichen Nash-Gleichgewicht ann\"ahert, wobei die kosmische Beschleunigung als emergenter Effekt dynamischen Ged\"achtnisses in einem selbstgravitierenden Medium erscheint.

Die vorliegende Arbeit verkn\"upft diese Entwicklungen mit einem eigenst\"andigen Ansatz: Ausgehend von einer spieltheoretischen Modellierung der Wechselwirkung zwischen Quantenvakuum und Raumzeit wird das \textit{Curvature Feedback Model} (CFM) entwickelt, das die beschleunigte Expansion als ``nachlassende Bremse'' statt als ``neuen Antrieb'' interpretiert.


% ===================================================================
% 2. SPIELTHEORETISCHER RAHMEN
% ===================================================================
\section{Spieltheoretischer Rahmen: Nullraum und Raumzeitblase}
\label{sec:spieltheorie}

\subsection{Grundannahmen}
\label{subsec:grundannahmen}

Der hier vorgeschlagene Rahmen geht von folgenden Annahmen aus:

\begin{enumerate}
\item Es existiert ein metastabiler Quantenvakuumzustand (im Folgenden: \textit{Nullraum}), der durch Quantenfluktuationen charakterisiert ist.
\item Eine au\ss{}ergew\"ohnlich gro\ss{}e Fluktuation entnimmt dem Nullraum einmalig eine Energiemenge~$E_0$, die einen Konzentrationsgradienten erzeugt.
\item Zur Einkapselung und kontrollierten Neutralisation dieses Gradienten entsteht Raumzeit als dynamische Struktur -- die \textit{Raumzeitblase} (Tochtersystem).
\item Zwischen Nullraum (Muttersystem) und Raumzeitblase besteht ein spieltheoretisches Gleichgewicht.
\end{enumerate}

Diese Annahmen werden im Folgenden in einen formalen Rahmen \"uberf\"uhrt.

\subsection{Akteure und Ziele}
\label{subsec:akteure}

Das System wird als Zweipersonen-Potentialspiel modelliert:

\begin{description}
\item[\textbf{Nullraum (Muttersystem):}] Prim\"arziel ist der Selbstschutz -- die Erhaltung seiner strukturellen Integrit\"at. Er reguliert die Kopplungsst\"arke zur Raumzeitblase \"uber effektive Randbedingungen (``Gatekeeping''), erzwingt langsame Energieabfuhr (D\"ampfung) und bildet Pufferzonen (horizontartige H\"ullen).
\item[\textbf{Raumzeitblase (Tochtersystem):}] Prim\"arziel ist die kontrollierte R\"uckkehr in den Nullzustand bei gleichzeitigem Schutz des Muttersystems. Die Strategien umfassen kaskadierten Gradientenabbau, adiabatische R\"uckf\"uhrung und Entropiemanagement.
\end{description}

\subsection{Mathematische Formulierung als Potentialspiel}
\label{subsec:potentialspiel}

Die globale Zielfunktion des Systems lautet:
\begin{equation}
\Phi = \alpha \cdot S_{\mathrm{Mutter}} + \beta \cdot R_{\mathrm{Tochter}} - \gamma \cdot G
\label{eq:potential}
\end{equation}
wobei $S_{\mathrm{Mutter}}$ die strukturelle Integrit\"at des Nullraums, $R_{\mathrm{Tochter}}$ den R\"uckkehrfortschritt und $G$ den verbleibenden Konzentrationsgradienten beschreibt; $\alpha, \beta, \gamma > 0$.

\begin{definition}[Nash-Gleichgewicht des kosmologischen Spiels]
Ein Strategiepaar $(s_M^*, s_T^*)$ von Nullraum und Raumzeitblase bildet ein Nash-Gleichgewicht, wenn gilt:
\begin{align}
\Phi(s_M^*, s_T^*) &\geq \Phi(s_M, s_T^*) \quad \forall\, s_M \\
\Phi(s_M^*, s_T^*) &\geq \Phi(s_M^*, s_T) \quad \forall\, s_T
\end{align}
Keine Seite kann durch einseitige Abweichung von ihrer Strategie das Gesamtpotential verbessern, ohne die Stabilit\"at des Systems zu gef\"ahrden.
\end{definition}

Der zentrale \textbf{Zielkonflikt} besteht darin, dass eine zu schnelle Reduktion von~$G$ (sofortige R\"uckkehr) $S_{\mathrm{Mutter}}$ gef\"ahrdet, w\"ahrend eine zu langsame Reduktion die Entropie und die Kosten innerhalb der Blase erh\"oht. Das Nash-Gleichgewicht erzwingt daher eine kontrollierte, zeitlich gestreckte Neutralisation.


\subsection{Emergente Gesetze aus dem Gleichgewicht}
\label{subsec:emergente_gesetze}

Aus der spieltheoretischen Gleichgewichtsbedingung emergieren physikalische Gesetzm\"a\ss{}igkeiten:

\begin{enumerate}
\item \textbf{Energieerhaltung:} Konservative Feldgleichungen entstehen als notwendige Bedingung f\"ur stabilen Gradientenabbau.
\item \textbf{Kausalstruktur:} Die H\"ullenbildung des Nullraums erzwingt eine maximale Ausbreitungsgeschwindigkeit f\"ur Informationen und Energie.
\item \textbf{Entropischer Zeitpfeil:} Die ``Zeit'' innerhalb der Blase ist die Ordnung, entlang der der Konzentrationsgradient nivelliert wird.
\item \textbf{Flusslimitierung:} Maximalfl\"usse \"uber die H\"ulle skalieren sublinear mit dem internen \"Uberschuss und verhindern Runaway-Prozesse.
\item \textbf{Asymptotische R\"uckkehr:} Der Restgradient $G \to 0$ n\"ahert sich nur asymptotisch; es gibt kein katastrophales Finale.
\end{enumerate}

Die letzte Eigenschaft ist f\"ur die Kosmologie besonders bedeutsam: Sie impliziert, dass die Expansion des Universums sich nie umkehrt, sondern asymptotisch abl\"auft -- konsistent mit den beobachteten Daten.


% ===================================================================
% 3. DAS CURVATURE FEEDBACK MODEL
% ===================================================================
\section{Das Curvature Feedback Model (CFM)}
\label{sec:cfm}

\subsection{Physikalische Motivation}
\label{subsec:motivation}

Im spieltheoretischen Rahmen der vorigen Sektion wird die Raumzeit als ``Bremsmechanismus'' interpretiert, der die sofortige R\"uckkehr der Energie in den Nullraum verhindert. Die zentrale physikalische Einsicht lautet:

\begin{quote}
\textit{Die beobachtete beschleunigte Expansion ist nicht durch eine neue Energieform verursacht, sondern durch ein nachlassendes Kr\"ummungs-R\"uckgabepotential -- eine Art geometrisches ``Ged\"achtnis'' der anf\"anglichen Energiekonzentration beim Urknall.}
\end{quote}

Die Analogie ist die einer gespannten Feder: Anf\"anglich herrscht maximale Spannung (hohe Kr\"ummung) mit starker R\"uckstellkraft. Mit der Zeit l\"asst die Spannung nach, die R\"uckstellkraft nimmt ab, und die Expansion ``beschleunigt'' relativ zur gebremsten Fr\"uhphase -- wie ein Auto, bei dem die Handbremse langsam gel\"ost wird.


\subsection{Modifizierte Friedmann-Gleichung}
\label{subsec:friedmann}

Die Standard-Friedmann-Gleichung im $\Lambda$CDM-Modell lautet:
\begin{equation}
H^2(a) = H_0^2 \left[\Omega_m\,a^{-3} + \Omega_\Lambda\right]
\label{eq:friedmann_lcdm}
\end{equation}

Im CFM wird die kosmologische Konstante durch ein zeitabh\"angiges Kr\"ummungs-R\"uckgabepotential ersetzt:
\begin{equation}
H^2(a) = H_0^2 \left[\Omega_m\,a^{-3} + \Omega_\Phi(a)\right]
\label{eq:friedmann_cfm}
\end{equation}

Das Kr\"ummungs-R\"uckgabepotential ist definiert als:
\begin{equation}
\Omega_\Phi(a) = \Phi_0 \cdot \frac{\tanh\!\big(k\cdot(a - a_{\mathrm{trans}})\big) + s}{1 + s}
\label{eq:potential_cfm}
\end{equation}
wobei $s = \tanh(k \cdot a_{\mathrm{trans}})$ ein Normierungsshift ist, der $\Omega_\Phi(0) = 0$ sicherstellt, und:
\begin{itemize}
\item $a$ der Skalenfaktor ist ($a=1$ heute, $a \to 0$ beim Urknall),
\item $\Phi_0$ die Amplitude (aus der Flachheitsbedingung $\Omega_m + \Omega_\Phi(1) = 1$ abgeleitet),
\item $k$ die \"Ubergangssch\"arfe,
\item $a_{\mathrm{trans}}$ der \"Ubergangsskalenfaktor.
\end{itemize}
Die konkreten Parameterwerte werden in Abschnitt~\ref{sec:numerik} aus dem Fit gegen den Pantheon+-Datensatz bestimmt.

\subsection{Dynamischer S\"attigungsmechanismus (\textit{Dynamic Saturation Mechanism})}
\label{subsec:tanh_herleitung}

Die $\tanh$-Parametrisierung ist keine \textit{ad hoc} gew\"ahlte Fitfunktion, sondern entsteht als exakte L\"osung eines physikalisch motivierten \textbf{dynamischen S\"attigungsmechanismus}. Die zentrale Annahme lautet: Die Raumzeitblase besitzt eine endliche Aufnahmekapazit\"at f\"ur die Kr\"ummungsr\"uckgabe. Die R\"uckgaberate ist proportional zur verbleibenden Kapazit\"at:
\begin{equation}
\frac{d\Omega_\Phi}{da} = k \cdot \left[1 - \left(\frac{\Omega_\Phi}{\Phi_0}\right)^{\!2}\right]
\label{eq:saturation_ode}
\end{equation}
Diese Gleichung beschreibt einen klassischen S\"attigungsprozess der dynamischen Systemtheorie: Bei kleinem $\Omega_\Phi$ w\"achst das Potential nahezu linear (die ``Bremse'' l\"ost sich mit voller Rate), bei $\Omega_\Phi \to \Phi_0$ tritt S\"attigung ein (die maximale Kapazit\"at ist erreicht, die Bremse vollst\"andig gel\"ost). Die exakte L\"osung von Gl.~\eqref{eq:saturation_ode} ist:
\begin{equation}
\Omega_\Phi(a) = \Phi_0 \cdot \tanh\!\big(k\cdot(a - a_{\mathrm{trans}})\big)
\end{equation}
wobei $a_{\mathrm{trans}}$ die Integrationskonstante (\"Ubergangspunkt) darstellt. Der S\"attigungsmechanismus ist in der Physik ubiquit\"ar und tritt in formal identischer Form in zahlreichen Systemen auf:
\begin{itemize}
\item Ferromagnetismus: Spontane Magnetisierung $M(T) \sim \tanh(T_C/T)$
\item BCS-Supraleitung: Energiel\"ucke $\Delta(T) \sim \tanh(T_C/T)$
\item Solitonenphysik: Kink-L\"osung $\phi(x) = \phi_0 \tanh(kx)$
\item Nichtlineare Optik: S\"attigungsabsorption $\alpha(I) \propto 1/(1 + I/I_{\mathrm{sat}})$
\end{itemize}
Alle diese Systeme teilen die Eigenschaft eines geordneten \"Ubergangs von einem Zustand in einen anderen mit endlicher Kapazit\"at -- \textit{genau} das Verhalten, das der spieltheoretische Rahmen f\"ur das Nash-Gleichgewicht zwischen Nullraum und Raumzeitblase vorhersagt. Die $\tanh$-Form ist damit nicht postuliert, sondern aus dem zugrunde liegenden Mechanismus \textit{abgeleitet}.

Zur \"Uberpr\"ufung der Robustheit wurden vier verschiedene S\"attigungsfunktionen getestet (Abschnitt~\ref{subsec:funktionalformen}). Alle liefern $\Delta\chi^2 \approx -9$ bis $-12$ gegen\"uber $\Lambda$CDM -- die Daten ``sehen'' einen S\"attigungsprozess, unabh\"angig von der exakten mathematischen Formulierung.

\subsection{Physikalische Interpretation der Parameter}
\label{subsec:interpretation}

\textbf{Fr\"uhe Zeiten} ($a \to 0$, $z \to \infty$): $\Omega_\Phi \to 0$. Die ``Bremse'' wirkt voll -- die Expansion folgt der Materiedominanz wie in $\Lambda$CDM. Es gibt keine dunkle Komponente.

\textbf{\"Ubergangsepoche} ($a \approx a_{\mathrm{trans}}$, $z \approx 1{,}5$): $\Omega_\Phi$ steigt an. Die ``Bremse'' beginnt nachzulassen. Dies geschah vor etwa 10,3 Milliarden Jahren.

\textbf{Heute} ($a = 1$, $z = 0$): $\Omega_\Phi \to \Phi_0$. Der maximale Effekt ist erreicht; das Potential wirkt effektiv wie~$\Lambda$.


\subsection{Effektiver Zustandsgleichungsparameter}
\label{subsec:weff}

Der effektive Zustandsgleichungsparameter des Kr\"ummungs-R\"uckgabepotentials ist:
\begin{equation}
w_{\mathrm{eff}}(a) = -1 - \frac{1}{3}\,\frac{d\ln\Omega_\Phi}{d\ln a}
\label{eq:weff}
\end{equation}

Seine Zeitentwicklung ist in Tabelle~\ref{tab:weff} dargestellt.

\begin{table}[H]
\centering
\caption{Zeitentwicklung des effektiven Zustandsgleichungsparameters $w_{\mathrm{eff}}(z)$ im Vergleich $\Lambda$CDM vs.\ CFM. Die $1\sigma$-Unsicherheiten stammen aus der MCMC-Posterioranalyse (Abschnitt~\ref{subsec:mcmc}).}
\label{tab:weff}
\begin{tabular}{ccccc}
\toprule
$z$ & $w$ ($\Lambda$CDM) & $w$ (CFM) & $1\sigma$-Bereich & $\Delta w$ \\
\midrule
0,0 & $-1{,}000$ & $-1{,}355$ & $[-1{,}371;\;-1{,}339]$ & $\mathbf{-0{,}355}$ \\
0,3 & $-1{,}000$ & $-1{,}433$ & $[-1{,}645;\;-1{,}355]$ & $\mathbf{-0{,}433}$ \\
0,5 & $-1{,}000$ & $-1{,}450$ & $[-1{,}730;\;-1{,}358]$ & $\mathbf{-0{,}450}$ \\
0,8 & $-1{,}000$ & $-1{,}456$ & $[-1{,}759;\;-1{,}359]$ & $\mathbf{-0{,}456}$ \\
1,0 & $-1{,}000$ & $-1{,}454$ & $[-1{,}749;\;-1{,}359]$ & $\mathbf{-0{,}454}$ \\
1,5 & $-1{,}000$ & $-1{,}444$ & $[-1{,}696;\;-1{,}357]$ & $\mathbf{-0{,}444}$ \\
2,0 & $-1{,}000$ & $-1{,}432$ & $[-1{,}644;\;-1{,}355]$ & $\mathbf{-0{,}432}$ \\
\bottomrule
\end{tabular}
\end{table}

Die CFM-Parameter aus dem Pantheon+-Fit ergeben durchgehend $w < -1$ (Phantom-Bereich). Die MCMC-basierten $1\sigma$-Konfidenzintervalle zeigen, dass $w = -1$ f\"ur alle Rotverschiebungen ausgeschlossen ist. Dies unterscheidet sich qualitativ von $\Lambda$CDM ($w \equiv -1$) und ist eine eindeutige, falsifizierbare Vorhersage. Der Effekt ist \"uber den gesamten beobachtbaren Rotverschiebungsbereich pr\"asent ($|\Delta w| \approx 0{,}4$) und damit deutlich innerhalb der erwarteten Messgenauigkeit von Euclid ($\sigma_w \approx 0{,}02$).


% ===================================================================
% 4. NUMERISCHE TESTS
% ===================================================================
\section{Numerische Tests und Modellvergleich}
\label{sec:numerik}


\subsection{Flachheitsbedingung}
\label{subsec:flachheit}

Um die Zahl freier Parameter zu reduzieren und physikalische Konsistenz zu gew\"ahrleisten, wird die Flachheitsbedingung
\begin{equation}
\Omega_m + \Omega_\Phi(a{=}1) = 1
\label{eq:flatness}
\end{equation}
auferlegt. Daraus folgt f\"ur die Amplitude:
\begin{equation}
\Phi_0 = \frac{(1 - \Omega_m)(1 + s)}{\tanh\!\big(k\cdot(1 - a_{\mathrm{trans}})\big) + s}
\end{equation}
Das CFM hat damit drei kosmologische Freiheitsgrade ($\Omega_m$, $k$, $a_{\mathrm{trans}}$) plus einen Nuisance-Parameter ($M$), also insgesamt vier effektive Parameter -- nur zwei mehr als $\Lambda$CDM.

\subsection{Datenbasis: Pantheon+}
\label{subsec:pantheonplus}

Der Test erfolgt gegen den Pantheon+-Datensatz \cite{Scolnic2022}, den gr\"o\ss{}ten \"offentlich verf\"ugbaren Katalog spektroskopisch best\"atigter Typ-Ia-Supernovae. Aus den 1701 Lichtkurven werden 1590 Supernovae mit $z > 0{,}01$ verwendet (zur Vermeidung von Pekuliargeschwindigkeits-Dominanz), im Rotverschiebungsbereich $z = 0{,}0102$ bis $z = 2{,}2614$. Als Observable dient die bias-korrigierte scheinbare B-Band-Helligkeit \texttt{m\_b\_corr}. Die Analyse wird sowohl mit diagonalen Fehlern als auch mit der vollen statistisch-systematischen Kovarianzmatrix (STAT+SYS) des Pantheon+-Datensatzes durchgef\"uhrt.

\subsection{Methodik}
\label{subsec:methodik}

\textbf{Distanzberechnung:} Die Leuchtkraftentfernung wird \"uber eine kumulative Trapezregel auf einem feinen $z$-Gitter ($N = 2000$ St\"utzstellen) berechnet und auf die Daten-Rotverschiebungen interpoliert. Dieses Verfahren ist numerisch stabil (Fehler $< 10^{-5}$) und erm\"oglicht schnelle Evaluation w\"ahrend der Optimierung.

\textbf{Nuisance-Parameter:} Der absolute Helligkeitsoffset $M = M_B + 5\log_{10}(c/H_0) + 25$, der die absolute Helligkeit und die Hubble-Konstante absorbiert, wird analytisch marginalisiert:
\begin{equation}
M_{\mathrm{best}} = \frac{\sum_i w_i (m_i^{\mathrm{obs}} - \mu_i^{\mathrm{th}})}{\sum_i w_i}, \quad w_i = \sigma_i^{-2}
\end{equation}

\textbf{Optimierung:} Parameterbestimmung mittels \textit{Differential Evolution} (globaler evolution\"arer Optimizer) mit anschlie\ss{}ender L-BFGS-B-Verfeinerung (\textit{polish}).

\textbf{MCMC-Unsicherheiten:} F\"ur das CFM (flach) werden Parameterunsicherheiten mittels \textit{emcee} \cite{ForemanMackey2013} bestimmt (32~Walkers, 3000~Schritte, 500~Burn-in). Die Posteriorverteilungen liefern $1\sigma$-Konfidenzintervalle f\"ur alle Parameter einschlie\ss{}lich der abgeleiteten Gr\"o\ss{}en $\Phi_0$ und $z_{\mathrm{trans}}$.

\textbf{Modellselektion:} Neben $\chi^2$ werden das Akaike-Informationskriterium (AIC~$= \chi^2 + 2k$) und das Bayes-Informationskriterium (BIC~$= \chi^2 + k \ln n$) berechnet, wobei $k$ die Zahl effektiver Parameter und $n$ die Datenpunktanzahl ist. Zur \"Uberpr\"ufung auf Overfitting wird zus\"atzlich eine 5-Fold-Kreuzvalidierung durchgef\"uhrt. Der vollst\"andige Analysecode ist \"offentlich verf\"ugbar.\footnote{\url{https://github.com/lukisch/cfm-cosmology}}

\subsection{Ergebnisse}
\label{subsec:ergebnisse}

Es werden drei Modelle gefittet: flaches $\Lambda$CDM (2~Parameter), CFM mit Flachheitsbedingung (4~Parameter) und CFM ohne Einschr\"ankung (5~Parameter).

\begin{table}[H]
\centering
\caption{Gefittete Parameter und Anpassungsg\"ute: $\Lambda$CDM vs.\ CFM gegen Pantheon+ (1590~SNe~Ia). F\"ur das CFM~(flach) werden $1\sigma$-MCMC-Unsicherheiten angegeben.}
\label{tab:results}
\begin{tabular}{lccc}
\toprule
 & $\Lambda$CDM & CFM (flach) & CFM (frei) \\
\midrule
Freie Parameter $k$ & 2 & 4 & 5 \\
$\Omega_m$ & 0,244 & $0{,}368^{+0{,}025}_{-0{,}023}$ & 0,552 \\
$\Omega_\Lambda$ / $\Omega_\Phi(z{=}0)$ & 0,756 & 0,636 & 0,872 \\
$\Phi_0$ (abgeleitet) & -- & $0{,}988^{+0{,}615}_{-0{,}221}$ & 1,292 \\
$k$ (\"Ubergangssch\"arfe) & -- & $1{,}44^{+1{,}22}_{-0{,}84}$ & 1,98 \\
$a_{\mathrm{trans}}$ ($z_{\mathrm{trans}}$) & -- & 0,75 (0,33) & 0,80 (0,25) \\
$\Omega_{\mathrm{total}}$ & 1,000 & 1,000 & 1,423 \\
\midrule
$\chi^2$ (diagonal) & 729,0 & 716,8 & 715,9 \\
$\chi^2$ (volle Kov.) & 1432,0 & 1420,8 & -- \\
$\chi^2/\mathrm{dof}$ & 0,459 & 0,452 & 0,452 \\
AIC (diagonal) & 733,0 & 724,8 & 725,9 \\
AIC (volle Kov.) & 1436,0 & 1428,8 & -- \\
BIC & 743,7 & 746,3 & 752,8 \\
\bottomrule
\end{tabular}
\end{table}

Das CFM mit Flachheitsbedingung zeigt $\Omega_m = 0{,}368 \pm 0{,}024$ (MCMC) -- physikalisch plausibel und nahe am Planck-Wert ($0{,}315 \pm 0{,}007$). Die gefittete \"Ubergangsrotverschiebung $z_{\mathrm{trans}} = 0{,}33$ ($a_{\mathrm{trans}} = 0{,}75$) liegt bei sp\"ateren kosmischen Zeiten als theoretisch erwartet. Die \"Ubergangssch\"arfe $k = 1{,}44^{+1{,}22}_{-0{,}84}$ beschreibt einen sanften \"Ubergang mit breiter Posterior -- die Daten pr\"aferieren einen \"Ubergang, lassen aber eine Bandbreite von \"Ubergangssch\"arfen zu.

\textbf{Volle Kovarianzmatrix:} Die Wiederholung der Analyse mit der vollen statistisch-systematischen Kovarianzmatrix best\"atigt die Ergebnisse: $\Delta\chi^2 = -11{,}2$ und $\Delta\mathrm{AIC} = -7{,}2$ (gegen\"uber $-12{,}2$ und $-8{,}2$ bei diagonalen Fehlern). Die leichte Reduktion erkl\"art sich durch die Ber\"ucksichtigung systematischer Korrelationen zwischen benachbarten Supernovae.

\subsection{Modellselektion}
\label{subsec:modellselektion}

\begin{table}[H]
\centering
\caption{Modellvergleich: CFM vs.\ $\Lambda$CDM. Negative Werte bevorzugen CFM.}
\label{tab:comparison}
\begin{tabular}{lcc}
\toprule
\textbf{Kriterium} & CFM (flach) vs.\ $\Lambda$CDM & CFM (frei) vs.\ $\Lambda$CDM \\
\midrule
$\Delta\chi^2$ & $\mathbf{-12{,}2}$ & $-13{,}1$ \\
$\Delta$AIC & $\mathbf{-8{,}2}$ & $-7{,}1$ \\
$\Delta$BIC & $+2{,}6$ & $+9{,}0$ \\
\midrule
5-Fold $\langle\chi^2/n\rangle$ & $\mathbf{0{,}4499}$ & $0{,}4498$ \\
$\Lambda$CDM: $\langle\chi^2/n\rangle$ & \multicolumn{2}{c}{$0{,}4519$} \\
\bottomrule
\end{tabular}
\end{table}

\textbf{Interpretation:} Drei von vier Selektionskriterien bevorzugen das CFM (flach) gegen\"uber $\Lambda$CDM: $\chi^2$ ($-12{,}2$), AIC ($-8{,}2$) und Kreuzvalidierung ($0{,}4499$ vs.\ $0{,}4519$). Einzig das BIC, das zus\"atzliche Parameter strenger bestraft, zeigt eine marginale Pr\"aferenz f\"ur $\Lambda$CDM ($\Delta\mathrm{BIC} = +2{,}6$). Nach der Kass-Raftery-Skala \cite{KassRaftery1995} liegt dieser Wert an der Grenze zur Signifikanz ($|\Delta\mathrm{BIC}| < 2$: nicht signifikant; $2$--$6$: positive Evidenz). Die Kreuzvalidierung -- die robusteste Methode zur Overfitting-Detektion -- zeigt, dass das CFM auf ungesehenen Daten besser generalisiert als $\Lambda$CDM.


\subsection{MCMC-Posterioranalyse}
\label{subsec:mcmc}

Die Parameterunsicherheiten werden mittels \textit{emcee} \cite{ForemanMackey2013} bestimmt (32~Walkers, 3000~Schritte, 500~Burn-in, Akzeptanzrate: 63\,\%). Die Ergebnisse f\"ur das CFM~(flach) sind in Tabelle~\ref{tab:results} als $1\sigma$-Unsicherheiten angegeben. Die Posteriorverteilung von $\Omega_m$ ist nahezu gau\ss{}f\"ormig mit $\Omega_m = 0{,}368^{+0{,}025}_{-0{,}023}$. Die \"Ubergangssch\"arfe $k$ zeigt eine breite, asymmetrische Posterior ($k = 1{,}44^{+1{,}22}_{-0{,}84}$), was bedeutet, dass die Daten einen \"Ubergang bevorzugen, aber die Sch\"arfe weniger stark einschr\"anken. Die abgeleiteten Gr\"o\ss{}en $\Phi_0 = 0{,}988^{+0{,}615}_{-0{,}221}$ und $z_{\mathrm{trans}} = 0{,}35$ sind konsistent mit den Punktsch\"atzungen. Die aus dem MCMC berechneten $w(z)$-Konfidenzb\"ander (Tabelle~\ref{tab:weff}) zeigen, dass $w = -1$ f\"ur alle Rotverschiebungen au\ss{}erhalb des $1\sigma$-Bereichs liegt.

\textbf{Interpretation von $\Omega_m = 0{,}368$:} Der CFM-Wert liegt \"uber dem Planck-CMB-Wert ($\Omega_m^{\mathrm{Planck}} = 0{,}315 \pm 0{,}007$), was typisch f\"ur reine Supernova-Fits ist. Im CFM-Kontext hat diese Abweichung eine physikalische Deutung: Das Modell interpretiert einen Teil der in $\Lambda$CDM als ``Dunkle Energie'' klassifizierten Dichte als dynamischen Kr\"ummungseffekt. Da $\Omega_\Phi(a)$ bei fr\"uhen Zeiten gegen Null geht (im Gegensatz zu $\Omega_\Lambda = \mathrm{const.}$), muss $\Omega_m$ kompensatorisch h\"oher ausfallen, um den Gesamt-Fit zu erhalten. Diese Pr\"aferenz f\"ur h\"oheres $\Omega_m$ steht in Einklang mit j\"ungsten Befunden aus schwachen Gravitationslinsen-Surveys (KiDS, DES), die ebenfalls h\"ohere $\Omega_m$-Werte als Planck bevorzugen.


\subsection{Robustheit: Alternative Funktionalformen}
\label{subsec:funktionalformen}

Um zu \"uberpr\"ufen, ob die Ergebnisse von der spezifischen Wahl der $\tanh$-Form abh\"angen, werden vier verschiedene S\"attigungsfunktionen unter identischen Bedingungen getestet:

\begin{table}[H]
\centering
\caption{Vergleich alternativer Funktionalformen f\"ur $\Omega_\Phi(a)$. Alle Modelle verwenden die Flachheitsbedingung und haben $k=4$ Parameter.}
\label{tab:funcforms}
\begin{tabular}{lcccc}
\toprule
\textbf{Funktionalform} & $\chi^2$ & AIC & $\Delta\chi^2$ vs.\ $\Lambda$CDM & $\Omega_m$ \\
\midrule
$\tanh$ (Standard-CFM) & 716,8 & 724,8 & $-12{,}2$ & 0,364 \\
Logistische Funktion & 717,5 & 725,5 & $-11{,}5$ & 0,368 \\
Error-Funktion (erf) & 716,7 & 724,7 & $-12{,}3$ & 0,367 \\
Potenzgesetz & 720,1 & 728,1 & $\phantom{-0}8{,}9$ & 0,364 \\
\bottomrule
\end{tabular}
\end{table}

\textbf{Ergebnis:} Alle vier Funktionalformen liefern $\Delta\chi^2 \approx -9$ bis $-12$ gegen\"uber $\Lambda$CDM. Die $\tanh$-Form ist weder die einzig m\"ogliche noch die ``bestpassende'' -- sie repr\"asentiert eine robuste Klasse von S\"attigungsfunktionen. Dies entkr\"aftet den Einwand, die CFM-Ergebnisse seien von einer spezifischen Funktionswahl abh\"angig. Die $\Omega_m$-Werte konvergieren bei allen Formen auf $\approx 0{,}36$--$0{,}37$, was die physikalische Konsistenz unterstreicht.


\subsection{Phantom-Stabilit\"atsanalyse}
\label{subsec:phantom}

Der gefittete Zustandsgleichungsparameter $w < -1$ (Phantom-Bereich) wirft die berechtigte Frage nach Stabilit\"at auf. In gew\"ohnlichen Phantom-Skalarfeldmodellen f\"uhrt $w < -1$ zu einer wachsenden Energiedichte ($\rho \propto a^{-3(1+w)} \to \infty$ f\"ur $w < -1$, $a \to \infty$), die in endlicher Zeit den ``Big Rip'' erzwingt -- die Zerst\"orung aller gebundenen Strukturen im Universum \cite{Caldwell1998}. \textbf{Das CFM umgeht diese Pathologie grundlegend}, und zwar aus drei Gr\"unden:

\begin{enumerate}
\item \textbf{S\"attigung statt Divergenz:} Der dynamische S\"attigungsmechanismus (Gl.~\ref{eq:saturation_ode}) garantiert $\Omega_\Phi(a) \to \Phi_0$ f\"ur $a \to \infty$. Die effektive Energiedichte bleibt \textit{f\"ur alle Zeiten endlich und beschr\"ankt}. Dies ist der entscheidende Unterschied zu Phantom-Skalarfeldern: W\"ahrend dort $\rho$ divergiert, s\"attigt $\Omega_\Phi$ bei seinem Maximalwert~$\Phi_0$.
\item \textbf{De-Sitter-Endzustand:} $w_{\mathrm{eff}} \to -1$ f\"ur $a \to \infty$. Das Universum n\"ahert sich asymptotisch \textit{genau demselben Endzustand wie $\Lambda$CDM} -- einem stabilen de-Sitter-Raum. Der Phantom-Bereich ist ein \"Ubergangsph\"anomen, kein Endzustand.
\item \textbf{Kein Big Rip:} Da $\Omega_\Phi$ s\"attigt, divergiert weder die Energiedichte noch der Skalenfaktor in endlicher Zeit. Der Big Rip ist ausgeschlossen -- nicht durch eine zus\"atzliche Annahme, sondern als \textit{direkte Konsequenz} des S\"attigungsmechanismus.
\item \textbf{Geometrische statt fluide Interpretation:} Die formale Verletzung der Null-Energie-Bedingung ($\rho + p \geq 0$) ist unproblematisch, da $\Omega_\Phi$ \textit{kein physisches Feld} repr\"asentiert, sondern eine geometrische Eigenschaft der Raumzeit. Die Energiebedingungen der ART gelten f\"ur den Energie-Impuls-Tensor physischer Felder, nicht f\"ur effektive geometrische Terme. Analoge Situationen sind in der Literatur etabliert: $f(R)$-Gravitationstheorien zeigen routinem\"a\ss{}ig effektiv $w < -1$, ohne dass dies physische Instabilit\"aten erzeugt \cite{Sotiriou2010}.
\end{enumerate}

\begin{quote}
\textit{Zusammengefasst: Das CFM zeigt ``Phantom-Verhalten'' ohne Phantom-Pathologien. Das Universum endet nicht im Riss, sondern im Gleichgewicht.}
\end{quote}


\subsection{Dezelerationsparameter und $H_0$-Implikationen}
\label{subsec:dezeleration}

\textbf{Dezelerationsparameter $q(z)$:} Der Dezelerationsparameter liefert eine zus\"atzliche, unabh\"angig testbare Vorhersage. Im CFM tritt der \"Ubergang von gebremster zu beschleunigter Expansion ($q = 0$) bei $z_{\mathrm{acc}} = 0{,}52$ auf -- deutlich sp\"ater in kosmischer Zeit als im $\Lambda$CDM-Modell ($z_{\mathrm{acc}} = 0{,}84$). Zudem sagt das CFM eine st\"arkere heutige Beschleunigung voraus: $q_0^{\mathrm{CFM}} = -0{,}81$ gegen\"uber $q_0^{\Lambda\mathrm{CDM}} = -0{,}63$.

\textbf{Implikation f\"ur die Strukturbildung:} Ein sp\"aterer Beginn der kosmischen Beschleunigung ($z_{\mathrm{acc}} = 0{,}52$ statt $0{,}84$) bedeutet, dass gravitativ gebundene Strukturen (Galaxienhaufen, gro\ss{}r\"aumige Filamente) \textit{l\"anger ungest\"ort wachsen konnten}, bevor die Beschleunigung das Wachstum unterdr\"uckte. Die materiedominierte \"Ara -- in der die Gravitation das Strukturwachstum antreibt -- dauerte im CFM signifikant l\"anger an als im Standardmodell.

\textbf{Empirische Evidenz f\"ur fr\"uhe massive Strukturen:} Mehrere unabh\"angige Beobachtungen setzen das $\Lambda$CDM-Modell in Bezug auf die Strukturbildung unter erheblichen Druck:
\begin{enumerate}
\item \textbf{JWST ``Universe Breakers'':} Das James Webb Space Telescope hat Galaxien bei $z > 7$ (ca.\ 500--700\,Myr nach dem Urknall) entdeckt, die weitaus massereicher sind als von hierarchischer Strukturbildung im $\Lambda$CDM erlaubt \cite{Labbe2023}. Boylan-Kolchin \cite{BoylanKolchin2023} zeigt quantitativ, dass die stellare Massendichte dieser Objekte die verf\"ugbare Baryonen-Budget innerhalb von $\Lambda$CDM-Halos \"ubersteigt -- ein ``Timing-Problem'' der Strukturbildung.
\item \textbf{El~Gordo (ACT-CL~J0102$-$4915):} Dieser extrem massereiche Galaxienhaufen bei $z \approx 0{,}87$ mit Masse $M_{200} \approx 2{,}1 \times 10^{15}\,M_\odot$ stellt eine $> 6\sigma$-Spannung mit $\Lambda$CDM dar, da ein Objekt dieser Masse bei diesem Alter mit der beobachteten Kollisionsgeschwindigkeit statistisch nahezu unm\"oglich ist \cite{Asencio2023}.
\item \textbf{Protocluster SPT2349$-$56:} Bereits 1,4\,Mrd.\ Jahre nach dem Urknall ($z = 4{,}3$) zeigt dieser Protocluster mindestens 14 gasreiche Galaxien mit einer Gesamtsternentstehungsrate von $\sim\!6500\,M_\odot$/yr -- weit reifer als von $\Lambda$CDM vorhergesagt \cite{Miller2018}.
\end{enumerate}

Das CFM bietet f\"ur alle drei Beobachtungen eine nat\"urliche Erkl\"arung: Da die Beschleunigung erst bei $z_{\mathrm{acc}} = 0{,}52$ einsetzt (statt $z_{\mathrm{acc}} = 0{,}84$), stand gravitativ gebundenen Strukturen mehr kosmische Zeit f\"ur ungest\"ortes Wachstum zur Verf\"ugung. Die CFM-Vorhersage ist direkt testbar durch Galaxienhaufen-Z\"ahlungen und schwache Gravitationslinsen-Surveys (Euclid, Vera C.\ Rubin Observatory).

\begin{table}[H]
\centering
\caption{Dezelerationsparameter $q(z)$: $\Lambda$CDM vs.\ CFM.}
\label{tab:qz}
\begin{tabular}{cccc}
\toprule
$z$ & $q$ ($\Lambda$CDM) & $q$ (CFM) & $\Delta q$ \\
\midrule
0,0 & $-0{,}634$ & $-0{,}805$ & $-0{,}171$ \\
0,5 & $-0{,}217$ & $-0{,}017$ & $+0{,}200$ \\
1,0 & $+0{,}082$ & $+0{,}321$ & $+0{,}240$ \\
2,0 & $+0{,}346$ & $+0{,}467$ & $+0{,}121$ \\
\bottomrule
\end{tabular}
\end{table}

\textbf{$H_0$-Implikationen:} Der Nuisance-Parameter $M$ absorbiert sowohl die absolute Helligkeit $M_B$ als auch $H_0$ \"uber die Beziehung $M = M_B + 5\log_{10}(c/H_0) + 25$. Unter Verwendung der SH0ES-Eichung ($M_B = -19{,}253$) liefert das CFM $H_0 = 76{,}1\,\mathrm{km/s/Mpc}$ gegen\"uber $H_0 = 75{,}5\,\mathrm{km/s/Mpc}$ im $\Lambda$CDM -- ein Unterschied von $\Delta H_0 = +0{,}5\,\mathrm{km/s/Mpc}$. Die $H_0$-Spannung wird durch das CFM allein nicht gel\"ost, da der Unterschied innerhalb der Messunsicherheit liegt. Eine direkte Aufl\"osung erfordert die Kombination mit CMB- und BAO-Daten.


% ===================================================================
% 5. VERGLEICH MIT ALTERNATIVEN
% ===================================================================
\section{Vergleich mit alternativen Modellen}
\label{sec:alternativen}

\subsection{$\Lambda$CDM (Standardmodell)}

Das $\Lambda$CDM-Modell ist extrem einfach ($w = -1$, konstant, zwei kosmologische Parameter) und passt alle aktuellen Daten gut. Es leidet jedoch unter dem Kosmologische-Konstante-Problem und dem Koinzidenz-Problem \cite{Weinberg1989}.

\subsection{Quintessenz}

Quintessenz-Modelle \cite{Caldwell1998} postulieren ein dynamisches Skalarfeld~$\phi$ mit zeitabh\"angigem Zustandsgleichungsparameter. Sie k\"onnen das Koinzidenz-Problem mildern, erfordern aber ein neues Feld und dessen Potential~$V(\phi)$ mit vielen freien Parametern.

\subsection{Modifizierte Gravitation: $f(R)$-Theorien}

$f(R)$-Gravitationstheorien \cite{Starobinsky1980, Sotiriou2010} ersetzen den Ricci-Skalar~$R$ in der Einstein-Hilbert-Wirkung durch eine allgemeinere Funktion. Sie bieten eine geometrische Erkl\"arung ohne Dunkle Energie, sind jedoch mathematisch komplex und zum Teil inkonsistent mit Beobachtungen (Gravitationslinsen, CMB).

\subsection{Emergente Gravitation nach Verlinde}

Verlinde \cite{Verlinde2011, Verlinde2017} schlägt vor, dass die Gravitation keine fundamentale Kraft, sondern ein emergentes, entropisches Ph\"anomen ist. In de-Sitter-R\"aumen f\"uhrt die mit dem kosmologischen Horizont assoziierte Entropie zu einer zus\"atzlichen ``dunklen'' Gravitationskraft, die das Verhalten von Galaxien ohne Dunkle Materie erkl\"aren k\"onnte.

\subsection{Finsler-Gravitation}

Pfeifer et al.\ \cite{Pfeifer2025} erweitern die Allgemeine Relativit\"atstheorie durch Finsler-Geometrie, in der die Metrik nicht nur von der Position, sondern auch von der Geschwindigkeit abh\"angt:
\begin{equation}
g_{\mu\nu}(x, y) = \frac{1}{2}\,\frac{\partial^2 F^2}{\partial y^\mu \partial y^\nu}, \quad y = \frac{dx}{d\lambda}
\end{equation}
Die resultierende Finsler-Friedmann-Gleichung erzeugt selbst im Vakuum eine exponentielle Expansion -- ohne kosmologische Konstante.


\subsection{Cosmological Teleodynamics}

Trivedi und Venkatasubramanian \cite{Trivedi2025} formulieren eine spieltheoretische Kosmologie, die erstaunliche Parallelen zum hier vorgestellten Ansatz aufweist. Ihre \textit{Cosmological Teleodynamics} beschreibt das Universum als ``riesiges Potentialspiel'', das sich einem kontinuierlichen Nash-Gleichgewicht ann\"ahert. Die kosmische Beschleunigung erscheint als ``statistisch emergenter Effekt dynamischen Ged\"achtnisses in einem selbstgravitierenden Medium'' -- eine Formulierung, die konzeptuell dem ``geometrischen Ged\"achtnis'' des CFM entspricht.


\subsection{Synoptischer Vergleich}
\label{subsec:synopse}

\begin{table}[H]
\centering
\caption{Synoptischer Vergleich kosmologischer Modelle ohne Dunkle Energie.}
\label{tab:synopse}
\begin{tabularx}{\textwidth}{lXXX}
\toprule
\textbf{Eigenschaft} & \textbf{CFM} & \textbf{Finsler} & \textbf{Teleodynamics} \\
\midrule
Theor.\ Basis & Standard-ART + Potential & Finsler-Geometrie & Stat.\ Mechanik + Spieltheorie \\
Mechanismus & Nachlassende ``Bremse'' & Geschwindigkeitsabh.\ Metrik & Dynamisches Ged\"achtnis \\
Dunkle Energie & Nicht n\"otig & Nicht n\"otig & Nicht n\"otig \\
Empirischer Test & Pantheon+ (1590 SNe, $\Delta\chi^2{=}{-}12$) & Noch ausstehend & Qualitativ \\
Vorhersage & $w(z)$ Zeitvariation & Exp.\ Expansion & Nash-Konvergenz \\
Komplexit\"at & Gering (4 Param.) & Hoch & Mittel \\
\bottomrule
\end{tabularx}
\end{table}


% ===================================================================
% 6. KOMPLEMENTARITÄT UND VEREINIGUNG
% ===================================================================
\section{Komplementarit\"at und m\"ogliche Vereinigung}
\label{sec:komplementaritaet}

\subsection{Drei Modelle, eine Einsicht}

Alle drei Ans\"atze -- CFM, Finsler-Gravitation und Cosmological Teleodynamics -- teilen eine fundamentale Einsicht:
\begin{quote}
\textit{``Die beschleunigte Expansion ist kein neues `Ding', sondern eine Eigenschaft der Geometrie bzw.\ der statistischen Struktur des Universums selbst.''}
\end{quote}

\subsection{Hypothese: CFM als effektive Beschreibung}

Eine faszinierende M\"oglichkeit besteht darin, dass die drei Modelle verschiedene Aspekte desselben Ph\"anomens beschreiben. In Analogie zur Beziehung zwischen Thermodynamik und Statistischer Mechanik k\"onnte gelten:

\begin{itemize}
\item \textbf{Finsler-Gravitation} (mikroskopisch, fundamental): Alle Momente der 1-Partikel-Verteilungsfunktion tragen zur Gravitation bei.
\item \textbf{CFM} (makroskopisch, ph\"anomenologisch): Das zeitabh\"angige Potential $\Phi(a)$ kodiert effektiv den Beitrag der h\"oheren Momente.
\item \textbf{Teleodynamics} (systemisch, spieltheoretisch): Die Nash-Gleichgewichtsdynamik beschreibt die globale Optimierung.
\end{itemize}

Mathematisch l\"asst sich diese Komplementarit\"at als hierarchische Beziehung darstellen:
\begin{equation}
\underbrace{G_{\mu\nu}^{\mathrm{Finsler}}(x, y)}_{\substack{\text{Finsler-Gravitation}\\\text{(mikroskopisch)}}}
\;\xrightarrow{\;\langle\cdot\rangle_{\mathrm{eff}}\;}
\underbrace{G_{\mu\nu} + 8\pi G\,\Omega_\Phi(a)\,g_{\mu\nu}}_{\substack{\text{CFM: modifizierte}\\\text{Einstein-Gleichung}}}
\;\xleftarrow{\;\delta\Phi/\delta s_i = 0\;}
\underbrace{\max_{s_M, s_T} \Phi(s_M, s_T)}_{\substack{\text{Teleodynamics:}\\\text{Nash-Gleichgewicht}}}
\label{eq:complementarity}
\end{equation}
wobei der linke Pfeil die effektive Mittelung \"uber die geschwindigkeitsabh\"angigen Freiheitsgrade der Finsler-Geometrie beschreibt und der rechte Pfeil die Variationsbedingung des spieltheoretischen Potentials darstellt, die das zeitliche Verhalten von $\Omega_\Phi(a)$ festlegt.


% ===================================================================
% 7. TESTBARKEIT UND VORHERSAGEN
% ===================================================================
\section{Testbarkeit und Vorhersagen}
\label{sec:testbarkeit}

\subsection{Beobachtbare Signaturen}

\textbf{1.~Phantom-Zustandsgleichung $w(z) < -1$:} Das CFM sagt $|\Delta w| \approx 0{,}4$ \"uber den gesamten beobachtbaren Rotverschiebungsbereich voraus. Die ESA-Mission Euclid \cite{Euclid2024} und das Nancy Grace Roman Space Telescope (NASA, $\sim$2027) k\"onnen $\sigma_w \approx 0{,}02$--$0{,}05$ messen -- weit ausreichend, um diese Signatur nachzuweisen oder auszuschlie\ss{}en.

\textbf{2.~Dezelerationsparameter:} Das CFM sagt einen fr\"uheren \"Ubergang zur beschleunigten Expansion voraus ($z_{\mathrm{acc}} = 0{,}52$ vs.\ $\Lambda$CDM: $z_{\mathrm{acc}} = 0{,}84$) sowie eine st\"arkere heutige Beschleunigung ($q_0 = -0{,}81$ vs.\ $-0{,}63$). Dies ist unabh\"angig von $w(z)$ testbar.

\textbf{3.~Strukturwachstum:} Eine modifizierte Wachstumsrate $f \cdot \sigma_8$ ist vorhergesagt, messbar durch schwache Gravitationslinsen und Galaxienhaufen-Z\"ahlungen. Erste empirische Hinweise liefern bereits die JWST-``Universe Breakers'' bei $z > 7$ \cite{Labbe2023}, die El-Gordo-Anomalie ($> 6\sigma$ Spannung mit $\Lambda$CDM; \cite{Asencio2023}) und unerwartet reife Protocluster bei $z > 4$ \cite{Miller2018} -- allesamt konsistent mit der CFM-Vorhersage einer verl\"angerten Wachstumsphase.

\textbf{4.~CMB-Integraleffekte:} Ein modifizierter ISW-Effekt (\textit{Integrated Sachs-Wolfe}) in CMB-Temperatur-Kreuzkorrelationen.

\subsection{Zuk\"unftige Missionen}

\begin{table}[H]
\centering
\caption{Relevante Beobachtungsmissionen f\"ur den CFM-Test.}
\label{tab:missionen}
\begin{tabularx}{\textwidth}{lccX}
\toprule
\textbf{Mission} & \textbf{Start} & $\sigma(w)$ & \textbf{Relevanz f\"ur CFM} \\
\midrule
Euclid (ESA) & 2023 & $\approx 0{,}02$ & Pr\"azisions-BAO + schwache Linsen; kann CFM vs.\ $\Lambda$CDM bei $z > 0{,}8$ unterscheiden \\
Roman (NASA) & $\sim$2027 & $\approx 0{,}03$ & SN-Survey bis $z \approx 2$; ideales Instrument f\"ur $w(z)$-Test \\
DESI & 2021-- & $\approx 0{,}04$ & Millionen Galaxien-Spektren; BAO und Strukturwachstum \\
\bottomrule
\end{tabularx}
\end{table}


\subsection{Unterscheidbarkeit der Modelle}

\begin{table}[H]
\centering
\caption{Vergleich der Vorhersagen: $\Lambda$CDM vs.\ CFM (Pantheon+-Fit). Die $1\sigma$-Unsicherheiten stammen aus der MCMC-Analyse.}
\label{tab:vorhersagen}
\begin{tabular}{lcc}
\toprule
\textbf{Eigenschaft} & $\Lambda$CDM & CFM \\
\midrule
$w(z{=}0)$ & $-1{,}000$ & $-1{,}36 \pm 0{,}02$ \\
$w(z{=}0{,}5)$ & $-1{,}000$ & $-1{,}45^{+0{,}09}_{-0{,}28}$ \\
$w(z{=}1)$ & $-1{,}000$ & $-1{,}45^{+0{,}10}_{-0{,}30}$ \\
$w(z{=}2)$ & $-1{,}000$ & $-1{,}43^{+0{,}08}_{-0{,}21}$ \\
Zeitvariation & Keine & Ja (durchgehend $w < -1$) \\
$\Delta w$ (messbar) & -- & $\approx -0{,}4$ \\
$q_0$ (heute) & $-0{,}63$ & $-0{,}81$ \\
$z_{\mathrm{acc}}$ (\"Ubergang) & 0,84 & 0,52 \\
\bottomrule
\end{tabular}
\end{table}


% ===================================================================
% 8. DISKUSSION
% ===================================================================
\section{Diskussion}
\label{sec:diskussion}

\subsection{St\"arken des Ansatzes}

\begin{enumerate}
\item \textbf{Konzeptuelle Eleganz:} Keine neue Energieform erforderlich; die Beschleunigung ist eine ``nachlassende Einschr\"ankung'', kein ``neuer Antrieb''.
\item \textbf{Spieltheoretische Fundierung:} Die Emergenz physikalischer Gesetze aus Gleichgewichtsbedingungen bietet einen neuartigen Erkl\"arungsrahmen, der durch die unabh\"angige Arbeit von Trivedi und Venkatasubramanian \cite{Trivedi2025} gest\"utzt wird.
\item \textbf{Empirische Validierung:} Das CFM passt 1590 reale Pantheon+-Supernovae besser als $\Lambda$CDM ($\Delta\chi^2 = -12{,}2$, $\Delta\mathrm{AIC} = -8{,}2$) und generalisiert in der Kreuzvalidierung besser.
\item \textbf{Testbarkeit:} Spezifische, quantitative Vorhersagen f\"ur $w(z)$ und $z_{\mathrm{acc}}$, die innerhalb einer Dekade \"uberpr\"ufbar sind.
\item \textbf{Empirische Unterst\"utzung:} Die CFM-Vorhersage einer verl\"angerten Wachstumsphase ($z_{\mathrm{acc}} = 0{,}52$) bietet eine nat\"urliche Erkl\"arung f\"ur die JWST-``Early Galaxy Tension'' \cite{Labbe2023, BoylanKolchin2023}, die El-Gordo-Anomalie \cite{Asencio2023} und unerwartet reife Protocluster \cite{Miller2018}.
\item \textbf{Konvergenz unabh\"angiger Ans\"atze:} CFM, Finsler-Gravitation und Cosmological Teleodynamics kommen unabh\"angig zum selben Schluss: Dunkle Energie ist nicht notwendig.
\item \textbf{Reproduzierbarkeit:} Analysecode und Daten sind \"offentlich verf\"ugbar (\url{https://github.com/lukisch/cfm-cosmology}).
\end{enumerate}

\subsection{Limitationen und offene Fragen}

\begin{enumerate}
\item \textbf{Ph\"anomenologischer Charakter:} Das CFM ist keine fundamentale Theorie. Obwohl die $\tanh$-Form als exakte L\"osung der S\"attigungs-ODE~\eqref{eq:saturation_ode} motiviert werden kann und vier alternative Funktionalformen vergleichbare Ergebnisse liefern (Abschnitt~\ref{subsec:funktionalformen}), steht eine Herleitung aus einer fundamentalen Quantengleichung noch aus.
\item \textbf{Parameterfreiheit:} Vier effektive Parameter gegen\"uber zwei in $\Lambda$CDM f\"uhren zu einem marginalen BIC-Nachteil ($\Delta\mathrm{BIC} = +2{,}6$), der jedoch durch die bessere Kreuzvalidierung und die Robustheit \"uber verschiedene Funktionalformen relativiert wird.
\item \textbf{Phantom-Bereich:} Der effektive Zustandsgleichungsparameter $w < -1$ liegt im Phantom-Bereich. Wie in Abschnitt~\ref{subsec:phantom} gezeigt, f\"uhrt dies im CFM-Kontext weder zu einem Big Rip noch zu Instabilit\"aten, da $\Omega_\Phi$ s\"attigt und kein physisches Feld darstellt. Formal verletzt das CFM die Null-Energie-Bedingung, analog zu $f(R)$-Gravitationstheorien \cite{Sotiriou2010}.
\item \textbf{Offene Tests:} CMB-Vorhersagen (ISW-Effekt, CMB-Leistungsspektrum), BAO-Signaturen und Gravitationslinsen-Effekte m\"ussen noch berechnet werden. Die Analyse mit der vollen Kovarianzmatrix (Abschnitt~\ref{subsec:ergebnisse}) best\"atigt jedoch die Ergebnisse der diagonalen Analyse.
\item \textbf{Mikroskopische Basis:} Was ist $\Phi$ auf Quantenebene? Die Verbindung zu einer Theorie der Quantengravitation steht aus. Die m\"ogliche Beziehung zur Finsler-Gravitation (Abschnitt~\ref{sec:komplementaritaet}) k\"onnte hier eine Br\"ucke schlagen.
\item \textbf{$H_0$-Spannung:} Die $H_0$-Analyse (Abschnitt~\ref{subsec:dezeleration}) zeigt $\Delta H_0 = +0{,}5\,$km/s/Mpc zwischen CFM und $\Lambda$CDM -- zu gering, um die $H_0$-Spannung zu l\"osen. Eine Aufl\"osung erfordert die Kombination mit CMB- und BAO-Daten.
\end{enumerate}


\subsection{Philosophische Implikationen}

Falls das CFM (oder ein verwandtes Modell) best\"atigt wird, h\"atte dies tiefgreifende Konsequenzen:

\begin{itemize}
\item \textbf{Dunkle Energie ist kein ``Ding'':} Sie w\"are eine geometrische Erinnerung, kein physisches Feld.
\item \textbf{Das Universum ``wei\ss{}'' von seinem Anfang:} Die Geometrie besitzt ein ``Ged\"achtnis''.
\item \textbf{Paradigmenwechsel:} Von ``Was treibt die Beschleunigung an?'' zu ``Warum bremste die Expansion fr\"uher?''
\end{itemize}

Dies w\"are vergleichbar mit dem \"Ubergang von ``Was treibt die Planeten an?'' (Ptolem\"aus: Sph\"aren) zu ``Wie bewegen sich Planeten in der Geometrie des Raumes?'' (Kepler, Newton, Einstein).


% ===================================================================
% 9. FAZIT UND AUSBLICK
% ===================================================================
\section{Fazit und Ausblick}
\label{sec:fazit}

Die vorliegende Arbeit hat gezeigt:

\begin{enumerate}
\item Ein spieltheoretischer Rahmen f\"ur die Kosmologie -- das Nash-Gleichgewicht zwischen Nullraum und Raumzeitblase -- f\"uhrt auf nat\"urliche Weise zu einem Modell, in dem physikalische Gesetze als emergente Gleichgewichtsbedingungen erscheinen.
\item Das daraus abgeleitete \textit{Curvature Feedback Model} (CFM) erkl\"art die beschleunigte Expansion ohne Dunkle Energie und besteht den Test gegen 1590 reale Typ-Ia-Supernovae des Pantheon+-Katalogs \cite{Scolnic2022}: $\Delta\chi^2 = -12{,}2$ (diagonal) bzw.\ $-11{,}2$ (volle Kovarianzmatrix), $\Delta\mathrm{AIC} = -8{,}2$ bzw.\ $-7{,}2$, bessere Kreuzvalidierung.
\item Die robuste Modellselektion (AIC, BIC, 5-Fold-Kreuzvalidierung) zeigt, dass der bessere Fit des CFM nicht auf Overfitting zur\"uckzuf\"uhren ist. Dies wird durch vier alternative Funktionalformen best\"atigt, die alle $\Delta\chi^2 \approx -9$ bis $-12$ liefern.
\item MCMC-basierte Parameterunsicherheiten ($\Omega_m = 0{,}368 \pm 0{,}024$) und die Phantom-Stabilit\"atsanalyse (kein Big Rip, asymptotisch de-Sitter) unterst\"utzen die physikalische Konsistenz.
\item Das CFM macht testbare Vorhersagen: eine durchgehende Phantom-Zustandsgleichung $w(z) < -1$ und einen sp\"ateren Beschleunigungs\"ubergang ($z_{\mathrm{acc}} = 0{,}52$ vs.\ $0{,}84$ in $\Lambda$CDM), die mit Euclid und Roman innerhalb der n\"achsten Dekade \"uberpr\"ufbar sind. Bereits jetzt findet die CFM-Vorhersage einer verl\"angerten Wachstumsphase empirische Unterst\"utzung durch JWST-Beobachtungen unerwartet massereicher Galaxien bei hohen Rotverschiebungen \cite{Labbe2023, BoylanKolchin2023} und die statistisch un\-wahrscheinliche Existenz massiver Haufen wie El~Gordo \cite{Asencio2023}.
\item Die Konvergenz dreier unabh\"angiger Ans\"atze (CFM, Finsler-Gravitation, Cosmological Teleodynamics) deutet auf einen m\"oglichen Paradigmenwechsel hin: \textit{Dunkle Energie als eigenst\"andige Entit\"at k\"onnte \"uberfl\"ussig sein.}
\end{enumerate}

\textbf{N\"achste Schritte} umfassen: (a)~Test gegen Planck-CMB- und DESI-BAO-Daten (die volle Pantheon+-Kovarianzmatrix wurde bereits ber\"ucksichtigt), (b)~Berechnung von CMB-Leistungsspektrum und Strukturwachstumsvorhersagen ($f\sigma_8$), (c)~Erforschung der Verbindung zwischen CFM und Finsler-Geometrie, (d)~Entwicklung einer kovarianten Formulierung von $\Phi(a)$ aus dem Ricci-Skalar~$R$, (e)~Untersuchung quantenmechanischer Grundlagen des Kr\"ummungs-R\"uckgabepotentials, und (f)~Kombination mit lokalen Entfernungsleiter-Daten zur direkten $H_0$-Bestimmung.

\textbf{Ausblick: Vereinigung mit MOND -- Ein Universum ohne dunklen Sektor?} Eine besonders faszinierende Perspektive er\"offnet sich durch die Kombination des CFM mit \textit{Modified Newtonian Dynamics} (MOND) \cite{Milgrom1983}. W\"ahrend das CFM die Dunkle Energie als geometrischen Effekt eliminiert, ersetzt MOND die Dunkle Materie durch eine modifizierte Gravitationsdynamik auf galaktischen Skalen. Beide Rahmenwerke konvergieren in der Vorhersage, dass Strukturen fr\"uher und effizienter entstehen als in $\Lambda$CDM -- das CFM durch eine verl\"angerte materiedominierte \"Ara ($z_{\mathrm{acc}} = 0{,}52$), MOND durch effektiv st\"arkere Gravitation bei niedrigen Beschleunigungen \cite{Asencio2023}. Eine vorl\"aufige Analyse mit einem rein baryonischen Universum ($\Omega_m = \Omega_b \approx 0{,}05$) und einem erweiterten geometrischen Potential $\Omega_\Phi(a) = \Phi_0 \cdot f_{\mathrm{tanh}}(a) + \alpha \cdot a^{-\beta}$ liefert $\Delta\chi^2 = -26{,}3$ und $\Delta\mathrm{AIC} = -16{,}3$ gegen\"uber $\Lambda$CDM -- \textit{dramatisch besser als sowohl das Standard-CFM als auch $\Lambda$CDM}. Die MCMC-Posterioranalyse ergibt $\beta = 2{,}02 \pm 0{,}20$, was exakt der Skalierung r\"aumlicher Kr\"ummung ($a^{-2}$) entspricht. Die ``Dunkle Materie'' w\"are demnach ein dynamischer Kr\"ummungseffekt, kein Teilchen. Dieses Ergebnis bedarf einer vollst\"andigen relativistischen Behandlung (z.\,B.\ im AeST-Rahmen \cite{Skordis2021}) und wird in einer Folgearbeit detailliert analysiert.

\begin{quote}
\textit{``Manchmal ist die eleganteste Erkl\"arung nicht eine neue Kraft, sondern eine nachlassende Einschr\"ankung.''}
\end{quote}


% ===================================================================
% LITERATUR
% ===================================================================
\begin{thebibliography}{99}

\bibitem{Riess1998}
Riess, A.\,G.\ et al.\ (1998).
Observational Evidence from Supernovae for an Accelerating Universe and a Cosmological Constant.
\textit{The Astronomical Journal}, 116(3), 1009--1038.
DOI: 10.1086/300499.

\bibitem{Perlmutter1999}
Perlmutter, S.\ et al.\ (1999).
Measurements of $\Omega$ and $\Lambda$ from 42 High-Redshift Supernovae.
\textit{The Astrophysical Journal}, 517(2), 565--586.
DOI: 10.1086/307221.

\bibitem{Planck2020}
Planck Collaboration (2020).
Planck 2018 results. VI. Cosmological parameters.
\textit{Astronomy \& Astrophysics}, 641, A6.
DOI: 10.1051/0004-6361/201833910.

\bibitem{Weinberg1989}
Weinberg, S.\ (1989).
The Cosmological Constant Problem.
\textit{Reviews of Modern Physics}, 61(1), 1--23.
DOI: 10.1103/RevModPhys.61.1.

\bibitem{Riess2022}
Riess, A.\,G.\ et al.\ (2022).
A Comprehensive Measurement of the Local Value of the Hubble Constant with 1\,km/s/Mpc Uncertainty from the Hubble Space Telescope and the SH0ES Team.
\textit{The Astrophysical Journal Letters}, 934(1), L7.
DOI: 10.3847/2041-8213/ac5c5b.

\bibitem{DESI2024}
DESI Collaboration (2024).
DESI 2024 VI: Cosmological Constraints from the Measurements of Baryon Acoustic Oscillations.
\textit{arXiv:2404.03002}.

\bibitem{Pfeifer2025}
Pfeifer, C.\ et al.\ (2025).
From kinetic gases to an exponentially expanding universe -- the Finsler-Friedmann equation.
\textit{Journal of Cosmology and Astroparticle Physics}, 2025(10), 050.
DOI: 10.1088/1475-7516/2025/10/050.

\bibitem{Trivedi2025}
Trivedi, O.\ \& Venkatasubramanian, V.\ (2025).
Game Theory in Cosmology.
\textit{arXiv:2511.20739}.

\bibitem{Caldwell1998}
Caldwell, R.\,R., Dave, R.\ \& Steinhardt, P.\,J.\ (1998).
Cosmological Imprint of an Energy Component with General Equation of State.
\textit{Physical Review Letters}, 80(8), 1582--1585.
DOI: 10.1103/PhysRevLett.80.1582.

\bibitem{Starobinsky1980}
Starobinsky, A.\,A.\ (1980).
A New Type of Isotropic Cosmological Models Without Singularity.
\textit{Physics Letters B}, 91(1), 99--102.
DOI: 10.1016/0370-2693(80)90670-X.

\bibitem{Sotiriou2010}
Sotiriou, T.\,P.\ \& Faraoni, V.\ (2010).
$f(R)$ Theories of Gravity.
\textit{Reviews of Modern Physics}, 82(1), 451--497.
DOI: 10.1103/RevModPhys.82.451.

\bibitem{Verlinde2011}
Verlinde, E.\ (2011).
On the Origin of Gravity and the Laws of Newton.
\textit{Journal of High Energy Physics}, 2011, 29.
DOI: 10.1007/JHEP04(2011)029.

\bibitem{Verlinde2017}
Verlinde, E.\ (2017).
Emergent Gravity and the Dark Universe.
\textit{SciPost Physics}, 2(3), 016.
DOI: 10.21468/SciPostPhys.2.3.016.

\bibitem{Euclid2024}
Euclid Collaboration (2025).
Euclid Quick Data Release 1.
ESA/Euclid Consortium.

\bibitem{Casimir1948}
Casimir, H.\,B.\,G.\ (1948).
On the attraction between two perfectly conducting plates.
\textit{Proceedings of the Royal Netherlands Academy of Arts and Sciences}, 51, 793--795.

\bibitem{Hawking1974}
Hawking, S.\,W.\ (1974).
Black hole explosions?
\textit{Nature}, 248, 30--31.
DOI: 10.1038/248030a0.

\bibitem{Nash1950}
Nash, J.\,F.\ (1950).
Equilibrium points in $n$-person games.
\textit{Proceedings of the National Academy of Sciences}, 36(1), 48--49.
DOI: 10.1073/pnas.36.1.48.

\bibitem{DESI2025}
DESI Collaboration (2025).
DESI DR2 Results II: Measurements of Baryon Acoustic Oscillations and Cosmological Constraints.
\textit{arXiv:2503.14738}.

\bibitem{Scolnic2022}
Scolnic, D.\ et al.\ (2022).
The Pantheon+ Analysis: The Full Data Set and Light-curve Release.
\textit{The Astrophysical Journal}, 938(2), 113.
DOI: 10.3847/1538-4357/ac8b7a.

\bibitem{KassRaftery1995}
Kass, R.\,E.\ \& Raftery, A.\,E.\ (1995).
Bayes Factors.
\textit{Journal of the American Statistical Association}, 90(430), 773--795.
DOI: 10.1080/01621459.1995.10476572.

\bibitem{ForemanMackey2013}
Foreman-Mackey, D.\ et al.\ (2013).
emcee: The MCMC Hammer.
\textit{Publications of the Astronomical Society of the Pacific}, 125(925), 306--312.
DOI: 10.1086/670067.

\bibitem{Labbe2023}
Labb\'e, I.\ et al.\ (2023).
A population of red candidate massive galaxies $\sim$600\,Myr after the Big Bang.
\textit{Nature}, 616(7956), 266--269.
DOI: 10.1038/s41586-023-05786-2.

\bibitem{BoylanKolchin2023}
Boylan-Kolchin, M.\ (2023).
Stress testing $\Lambda$CDM with high-redshift galaxy candidates.
\textit{Nature Astronomy}, 7, 731--735.
DOI: 10.1038/s41550-023-01937-7.

\bibitem{Asencio2023}
Asencio, E., Banik, I.\ \& Kroupa, P.\ (2023).
The El Gordo galaxy cluster challenges $\Lambda$CDM for any plausible collision velocity.
\textit{The Astrophysical Journal}, 954(2), 162.
DOI: 10.3847/1538-4357/ace62a.

\bibitem{Miller2018}
Miller, T.\,B.\ et al.\ (2018).
A massive core for a cluster of galaxies at a redshift of 4.3.
\textit{Nature}, 556(7702), 469--472.
DOI: 10.1038/s41586-018-0025-2.

\bibitem{Milgrom1983}
Milgrom, M.\ (1983).
A modification of the Newtonian dynamics as a possible alternative to the hidden mass hypothesis.
\textit{The Astrophysical Journal}, 270, 365--370.
DOI: 10.1086/161130.

\bibitem{Skordis2021}
Skordis, C.\ \& Z{\l}o\'snik, T.\ (2021).
New Relativistic Theory for Modified Newtonian Dynamics.
\textit{Physical Review Letters}, 127(16), 161302.
DOI: 10.1103/PhysRevLett.127.161302.

\end{thebibliography}


% ===================================================================
% APPENDIX: ENGLISH TRANSLATION
% ===================================================================
\newpage
\appendix
\selectlanguage{english}

\section*{Appendix: English Translation}
\addcontentsline{toc}{section}{Appendix: English Translation}

\noindent\textit{The following is a complete English translation of the preceding German article for international accessibility.}

\vspace{2em}

% ===================================================================
% ENGLISH TITLE
% ===================================================================

\begin{center}
{\Large\bfseries Game-Theoretic Cosmology and the Curvature Feedback Model}\\[0.5em]
{\large Nash Equilibria Between Null Space and Spacetime Bubble\\as an Explanatory Framework for Accelerated Expansion}\\[0.3em]
{\normalsize An Integrative Theoretical Approach}\\[1em]
{\normalsize Lukas Geiger}\\
{\small Independent Researcher, Bernau im Schwarzwald, Germany}\\[0.5em]
{\small February 2026 \\ \textit{Scientific Working Paper}}
\end{center}

\vspace{1em}

\noindent\textbf{Abstract.}
This paper develops a game-theoretic framework for cosmology in which the emergence and evolution of spacetime is modeled as a Nash equilibrium between two agents: a metastable quantum vacuum (null space) and a spacetime bubble arising from it. The central result is the \textit{Curvature Feedback Model} (CFM), which explains the observed accelerated expansion of the universe not through a new form of energy (dark energy) but through a diminishing curvature return potential~$\Phi(a)$ -- a geometric ``memory'' of the initial energy concentration at the Big Bang. The modified Friedmann equation $H^2(a) = H_0^2\,[\Omega_m\,a^{-3} + \Omega_\Phi(a)]$ with $\Omega_\Phi(a) = \Phi_0 \cdot \tanh(k\cdot(a - a_{\mathrm{trans}}))$ is tested against 1,590 real Type~Ia supernovae from the Pantheon+ catalog \cite{Scolnic2022} -- both with diagonal errors and with the full statistical-systematic covariance matrix. Under a flatness constraint ($\Omega_m + \Omega_\Phi(a{=}1) = 1$), the CFM yields $\Delta\chi^2 = -12.2$ ($-11.2$ with full covariance) and $\Delta\mathrm{AIC} = -8.2$ ($-7.2$) relative to $\Lambda$CDM; 5-fold cross-validation confirms better generalization. MCMC posterior analysis yields $\Omega_m = 0.368 \pm 0.024$, and four alternative functional forms (logistic, error function, power law) show comparable $\Delta\chi^2$ values, confirming the robustness of results. A phantom stability analysis shows: no Big Rip (saturation of $\Omega_\Phi$), asymptotically de~Sitter end state. The model predicts a measurable time variation of the equation-of-state parameter ($w(z) < -1$, $|\Delta w| \approx 0.4$) and an earlier acceleration transition ($z_{\mathrm{acc}} = 0.52$ vs.\ $0.84$), testable with Euclid and the Nancy Grace Roman Space Telescope within the next decade. It is shown that the CFM exhibits conceptual connections to both Finsler gravity (Pfeifer et al., 2025) and the recently proposed \textit{Cosmological Teleodynamics} (Trivedi \& Venkatasubramanian, 2025), which likewise describes cosmic expansion as convergence toward a Nash equilibrium. Analysis code and data are publicly available.\footnote{\url{https://github.com/lukisch/cfm-cosmology}} In particular, the rigorous analysis identifies with $z_{\mathrm{acc}} \approx 0.52$ a later onset of cosmic acceleration than $\Lambda$CDM ($z_{\mathrm{acc}} = 0.84$), implying an extended matter-dominated growth phase. This prediction offers a natural explanation for the JWST ``Universe Breakers'' -- unexpectedly massive galaxies at $z > 7$ \cite{Labbe2023, BoylanKolchin2023} -- as well as for statistically improbable massive galaxy clusters such as El~Gordo at $z \approx 0.87$ \cite{Asencio2023}, which place the $\Lambda$CDM standard model under significant tension. The game-theoretic perspective opens a paradigm shift: from ``What drives the acceleration?'' to ``Why did the expansion brake earlier?''

\vspace{0.5em}
\noindent\textbf{Keywords:} game theory, Nash equilibrium, cosmology, dark energy, curvature return potential, Curvature Feedback Model, Friedmann equation, Finsler gravity, accelerated expansion, equation of state

\vspace{0.5em}
\noindent\textbf{Disciplines:} theoretical physics, cosmology, game theory, mathematical physics

\vspace{1em}


% ===================================================================
% STATEMENT ON AI USE
% ===================================================================
\subsection*{Statement on AI Use and Methodology}

\noindent\textbf{Extended Methodological Declaration:} This work is an experiment in \textit{AI-Assisted Science}. The division of roles is transparently disclosed:

\begin{description}[style=nextline, leftmargin=2cm]
\item[\textbf{Human Author} (Lukas Geiger)] Physical intuition, foundational assumptions (game-theoretic approach, saturation hypothesis, geometry instead of dark sector), interpretation of results, strategic decisions, and final responsibility for the scientific content.
\item[\textbf{Claude Opus 4.6} (Anthropic)] Co-Writer: Mathematical formalization, derivation of equations, code development (Python/MCMC), statistical analysis (Pantheon+ fits), text generation and structuring.
\item[\textbf{Gemini} (Google DeepMind)] Reviewer: Critical review, consistency checks, strategic recommendations, MOND compatibility analysis, identification of BBN issues.
\end{description}

\vspace{0.5em}
\noindent\textit{Note:} The mathematical formalization and the execution of statistical fits were carried out by AI systems. The author provides these hypotheses as a \textit{Working Paper} to make them available to the scientific community for review and further development. \textbf{Mathematical verification by third parties is expressly encouraged.}

\vspace{1em}


% ===================================================================
% 1. INTRODUCTION
% ===================================================================
\subsection*{1\quad Introduction}
\addcontentsline{toc}{subsection}{1\quad Introduction (English)}

The discovery of the accelerated expansion of the universe through observations of distant Type~Ia supernovae in 1998 by the teams of Perlmutter \cite{Perlmutter1999} and Riess and Schmidt \cite{Riess1998} marks a turning point in modern cosmology. The 2011 Nobel Prize in Physics was awarded for this discovery. The standard model of cosmology, $\Lambda$CDM, explains the acceleration through a cosmological constant~$\Lambda$ comprising approximately 68\% of the energy density of the universe \cite{Planck2020}. Despite its empirical success, $\Lambda$CDM faces profound conceptual problems:

\begin{enumerate}
\item \textbf{The Cosmological Constant Problem:} The observed vacuum energy density is $\sim$60--120 orders of magnitude smaller than theoretical predictions from quantum field theory \cite{Weinberg1989}.
\item \textbf{The Coincidence Problem:} Why are $\Omega_m$ and $\Omega_\Lambda$ of comparable magnitude precisely in the present epoch?
\item \textbf{The $H_0$ Tension:} The local measurement of the Hubble parameter ($H_0 \approx 73$\,km/s/Mpc) differs significantly from the CMB-derived value ($H_0 \approx 67.4$\,km/s/Mpc) \cite{Planck2020, Riess2022}.
\end{enumerate}

Recent results from the \textit{Dark Energy Spectroscopic Instrument} (DESI) strengthen doubts about a strictly constant dark energy: the analysis of baryon acoustic oscillations combined with CMB and supernova data shows a 2.5--3.9$\sigma$ preference for a model with time-dependent equation-of-state parameter $w(z)$ over $\Lambda$CDM \cite{DESI2024}.

In parallel, theoretical works show that the acceleration could also be explained without dark energy: Pfeifer et al.\ \cite{Pfeifer2025} demonstrate within the framework of Finsler gravity that a generalized spacetime geometry naturally produces exponential expansion in vacuum. Trivedi and Venkatasubramanian \cite{Trivedi2025} show in their \textit{Cosmological Teleodynamics} that the universe operates like a ``giant potential game'' converging toward a continuous Nash equilibrium, where cosmic acceleration emerges as an emergent effect of dynamic memory in a self-gravitating medium.

This paper connects these developments with an independent approach: starting from a game-theoretic model of the interaction between quantum vacuum and spacetime, the \textit{Curvature Feedback Model} (CFM) is developed, which interprets accelerated expansion as a ``releasing brake'' rather than a ``new drive.''


% ===================================================================
% 2. GAME-THEORETIC FRAMEWORK
% ===================================================================
\subsection*{2\quad Game-Theoretic Framework: Null Space and Spacetime Bubble}
\addcontentsline{toc}{subsection}{2\quad Game-Theoretic Framework (English)}

\subsubsection*{2.1\quad Fundamental Assumptions}

The proposed framework rests on the following assumptions:

\begin{enumerate}
\item There exists a metastable quantum vacuum state (hereafter: \textit{null space}) characterized by quantum fluctuations.
\item An extraordinarily large fluctuation extracts a one-time energy amount~$E_0$ from the null space, creating a concentration gradient.
\item To encapsulate and controllably neutralize this gradient, spacetime emerges as a dynamic structure -- the \textit{spacetime bubble} (daughter system).
\item A game-theoretic equilibrium exists between the null space (parent system) and the spacetime bubble.
\end{enumerate}

These assumptions are formalized in the following sections.

\subsubsection*{2.2\quad Agents and Objectives}

The system is modeled as a two-player potential game:

\begin{description}
\item[\textbf{Null Space (Parent System):}] The primary objective is self-protection -- preservation of its structural integrity. It regulates the coupling strength to the spacetime bubble via effective boundary conditions (``gatekeeping''), enforces slow energy dissipation (damping), and forms buffer zones (horizon-like shells).
\item[\textbf{Spacetime Bubble (Daughter System):}] The primary objective is the controlled return to the null state while simultaneously protecting the parent system. Strategies include cascaded gradient reduction, adiabatic return, and entropy management.
\end{description}

\subsubsection*{2.3\quad Mathematical Formulation as a Potential Game}

The global objective function of the system reads:
\begin{equation}
\Phi = \alpha \cdot S_{\mathrm{parent}} + \beta \cdot R_{\mathrm{daughter}} - \gamma \cdot G
\tag{A.1}
\end{equation}
where $S_{\mathrm{parent}}$ denotes the structural integrity of the null space, $R_{\mathrm{daughter}}$ the return progress, and $G$ the remaining concentration gradient; $\alpha, \beta, \gamma > 0$.

\begin{definition}[Nash Equilibrium of the Cosmological Game]
A strategy pair $(s_P^*, s_D^*)$ of null space and spacetime bubble constitutes a Nash equilibrium if:
\begin{align}
\Phi(s_P^*, s_D^*) &\geq \Phi(s_P, s_D^*) \quad \forall\, s_P \\
\Phi(s_P^*, s_D^*) &\geq \Phi(s_P^*, s_D) \quad \forall\, s_D
\end{align}
Neither side can improve the overall potential by unilaterally deviating from its strategy without endangering the stability of the system.
\end{definition}

The central \textbf{conflict} is that excessively rapid reduction of~$G$ (immediate return) endangers $S_{\mathrm{parent}}$, while excessively slow reduction increases entropy and costs within the bubble. The Nash equilibrium therefore enforces a controlled, temporally extended neutralization.


\subsubsection*{2.4\quad Emergent Laws from the Equilibrium}

From the game-theoretic equilibrium condition, physical regularities emerge:

\begin{enumerate}
\item \textbf{Energy conservation:} Conservative field equations arise as a necessary condition for stable gradient reduction.
\item \textbf{Causal structure:} Shell formation by the null space enforces a maximum propagation speed for information and energy.
\item \textbf{Entropic arrow of time:} ``Time'' within the bubble is the order along which the concentration gradient is leveled.
\item \textbf{Flux limitation:} Maximum fluxes across the shell scale sublinearly with the internal excess, preventing runaway processes.
\item \textbf{Asymptotic return:} The residual gradient $G \to 0$ approaches zero only asymptotically; there is no catastrophic finale.
\end{enumerate}

The last property is particularly significant for cosmology: it implies that the expansion of the universe never reverses but proceeds asymptotically -- consistent with observational data.


% ===================================================================
% 3. THE CURVATURE FEEDBACK MODEL
% ===================================================================
\subsection*{3\quad The Curvature Feedback Model (CFM)}
\addcontentsline{toc}{subsection}{3\quad The Curvature Feedback Model (English)}

\subsubsection*{3.1\quad Physical Motivation}

In the game-theoretic framework of the preceding section, spacetime is interpreted as a ``braking mechanism'' that prevents the immediate return of energy to the null space. The central physical insight is:

\begin{quote}
\textit{The observed accelerated expansion is not caused by a new form of energy but by a diminishing curvature return potential -- a kind of geometric ``memory'' of the initial energy concentration at the Big Bang.}
\end{quote}

The analogy is that of a stretched spring: initially, maximum tension (high curvature) produces strong restoring force. Over time, the tension relaxes, the restoring force decreases, and the expansion ``accelerates'' relative to the braked early phase -- like a car whose handbrake is slowly released.


\subsubsection*{3.2\quad Modified Friedmann Equation}

The standard Friedmann equation in the $\Lambda$CDM model reads:
\begin{equation}
H^2(a) = H_0^2 \left[\Omega_m\,a^{-3} + \Omega_\Lambda\right]
\tag{A.2}
\end{equation}

In the CFM, the cosmological constant is replaced by a time-dependent curvature return potential:
\begin{equation}
H^2(a) = H_0^2 \left[\Omega_m\,a^{-3} + \Omega_\Phi(a)\right]
\tag{A.3}
\end{equation}

The curvature return potential is defined as:
\begin{equation}
\Omega_\Phi(a) = \Phi_0 \cdot \frac{\tanh\!\big(k\cdot(a - a_{\mathrm{trans}})\big) + s}{1 + s}
\tag{A.4}
\end{equation}
where $s = \tanh(k \cdot a_{\mathrm{trans}})$ is a normalization shift ensuring $\Omega_\Phi(0) = 0$, and:
\begin{itemize}
\item $a$ is the scale factor ($a=1$ today, $a \to 0$ at the Big Bang),
\item $\Phi_0$ is the amplitude (derived from the flatness constraint $\Omega_m + \Omega_\Phi(1) = 1$),
\item $k$ is the transition sharpness,
\item $a_{\mathrm{trans}}$ is the transition scale factor.
\end{itemize}
The specific parameter values are determined from the fit to the Pantheon+ data set in Section~4.

\subsubsection*{3.3\quad Dynamic Saturation Mechanism}

The $\tanh$ parameterization is not an \textit{ad hoc} chosen fitting function but arises as the exact solution of a physically motivated \textbf{dynamic saturation mechanism}. The central assumption is: the spacetime bubble possesses a finite absorption capacity for the curvature return. The return rate is proportional to the remaining capacity:
\begin{equation}
\frac{d\Omega_\Phi}{da} = k \cdot \left[1 - \left(\frac{\Omega_\Phi}{\Phi_0}\right)^{\!2}\right]
\tag{A.5}
\end{equation}
This equation describes a classical saturation process from dynamical systems theory: at small $\Omega_\Phi$, the potential grows nearly linearly (the ``brake'' releases at full rate); at $\Omega_\Phi \to \Phi_0$, saturation occurs (maximum capacity is reached, the brake is fully released). The exact solution of Eq.~(A.5) is:
\begin{equation}
\Omega_\Phi(a) = \Phi_0 \cdot \tanh\!\big(k\cdot(a - a_{\mathrm{trans}})\big)
\end{equation}
where $a_{\mathrm{trans}}$ represents the integration constant (transition point). The saturation mechanism is ubiquitous in physics and appears in formally identical form in numerous systems:
\begin{itemize}
\item Ferromagnetism: spontaneous magnetization $M(T) \sim \tanh(T_C/T)$
\item BCS superconductivity: energy gap $\Delta(T) \sim \tanh(T_C/T)$
\item Soliton physics: kink solution $\phi(x) = \phi_0 \tanh(kx)$
\item Nonlinear optics: saturation absorption $\alpha(I) \propto 1/(1 + I/I_{\mathrm{sat}})$
\end{itemize}
All these systems share the property of an ordered transition from one state to another with finite capacity -- \textit{exactly} the behavior that the game-theoretic framework predicts for the Nash equilibrium between null space and spacetime bubble. The $\tanh$ form is thus not postulated but \textit{derived} from the underlying mechanism.

To verify robustness, four different saturation functions were tested (Section~4.6). All yield $\Delta\chi^2 \approx -9$ to $-12$ relative to $\Lambda$CDM -- the data ``see'' a saturation process, independent of the exact mathematical formulation.

\subsubsection*{3.4\quad Physical Interpretation of the Parameters}

\textbf{Early times} ($a \to 0$, $z \to \infty$): $\Omega_\Phi \to 0$. The ``brake'' operates at full strength -- the expansion follows matter dominance as in $\Lambda$CDM. There is no dark component.

\textbf{Transition epoch} ($a \approx a_{\mathrm{trans}}$, $z \approx 1.5$): $\Omega_\Phi$ increases. The ``brake'' begins to release. This occurred approximately 10.3 billion years ago.

\textbf{Today} ($a = 1$, $z = 0$): $\Omega_\Phi \to \Phi_0$. The maximum effect is reached; the potential effectively acts like~$\Lambda$.


\subsubsection*{3.5\quad Effective Equation-of-State Parameter}

The effective equation-of-state parameter of the curvature return potential is:
\begin{equation}
w_{\mathrm{eff}}(a) = -1 - \frac{1}{3}\,\frac{d\ln\Omega_\Phi}{d\ln a}
\tag{A.6}
\end{equation}

Its time evolution is presented in Table~\ref{tab:en-weff}.

\begin{table}[H]
\centering
\caption{Time evolution of the effective equation-of-state parameter $w_{\mathrm{eff}}(z)$: $\Lambda$CDM vs.\ CFM. The $1\sigma$ uncertainties are from the MCMC posterior analysis (Section~4.5).}
\label{tab:en-weff}
\begin{tabular}{ccccc}
\toprule
$z$ & $w$ ($\Lambda$CDM) & $w$ (CFM) & $1\sigma$ range & $\Delta w$ \\
\midrule
0.0 & $-1.000$ & $-1.355$ & $[-1.371;\;-1.339]$ & $\mathbf{-0.355}$ \\
0.3 & $-1.000$ & $-1.433$ & $[-1.645;\;-1.355]$ & $\mathbf{-0.433}$ \\
0.5 & $-1.000$ & $-1.450$ & $[-1.730;\;-1.358]$ & $\mathbf{-0.450}$ \\
0.8 & $-1.000$ & $-1.456$ & $[-1.759;\;-1.359]$ & $\mathbf{-0.456}$ \\
1.0 & $-1.000$ & $-1.454$ & $[-1.749;\;-1.359]$ & $\mathbf{-0.454}$ \\
1.5 & $-1.000$ & $-1.444$ & $[-1.696;\;-1.357]$ & $\mathbf{-0.444}$ \\
2.0 & $-1.000$ & $-1.432$ & $[-1.644;\;-1.355]$ & $\mathbf{-0.432}$ \\
\bottomrule
\end{tabular}
\end{table}

The CFM parameters from the Pantheon+ fit yield consistently $w < -1$ (phantom regime). The MCMC-based $1\sigma$ confidence intervals show that $w = -1$ is excluded for all redshifts. This differs qualitatively from $\Lambda$CDM ($w \equiv -1$) and constitutes a clear, falsifiable prediction. The effect is present across the entire observable redshift range ($|\Delta w| \approx 0.4$) and thus well within the expected measurement precision of Euclid ($\sigma_w \approx 0.02$).


% ===================================================================
% 4. NUMERICAL TESTS
% ===================================================================
\subsection*{4\quad Numerical Tests and Model Comparison}
\addcontentsline{toc}{subsection}{4\quad Numerical Tests and Model Comparison (English)}


\subsubsection*{4.1\quad Flatness Constraint}

To reduce the number of free parameters and ensure physical consistency, the flatness constraint
\begin{equation}
\Omega_m + \Omega_\Phi(a{=}1) = 1
\tag{A.7}
\end{equation}
is imposed. This yields for the amplitude:
\begin{equation}
\Phi_0 = \frac{(1 - \Omega_m)(1 + s)}{\tanh\!\big(k\cdot(1 - a_{\mathrm{trans}})\big) + s}
\end{equation}
The CFM thus has three cosmological degrees of freedom ($\Omega_m$, $k$, $a_{\mathrm{trans}}$) plus one nuisance parameter ($M$), totaling four effective parameters -- only two more than $\Lambda$CDM.

\subsubsection*{4.2\quad Data: Pantheon+}

The test is performed against the Pantheon+ data set \cite{Scolnic2022}, the largest publicly available catalog of spectroscopically confirmed Type~Ia supernovae. From the 1,701 light curves, 1,590 supernovae with $z > 0.01$ are used (to avoid peculiar velocity dominance), spanning the redshift range $z = 0.0102$ to $z = 2.2614$. The observable is the bias-corrected apparent B-band magnitude \texttt{m\_b\_corr}. The analysis is performed both with diagonal errors and with the full statistical-systematic covariance matrix (STAT+SYS) of the Pantheon+ data set.

\subsubsection*{4.3\quad Methodology}

\textbf{Distance computation:} The luminosity distance is computed via cumulative trapezoidal integration on a fine $z$-grid ($N = 2{,}000$ support points) and interpolated onto the data redshifts. This procedure is numerically stable (error $< 10^{-5}$) and enables fast evaluation during optimization.

\textbf{Nuisance parameter:} The absolute magnitude offset $M = M_B + 5\log_{10}(c/H_0) + 25$, which absorbs the absolute magnitude and the Hubble constant, is analytically marginalized:
\begin{equation}
M_{\mathrm{best}} = \frac{\sum_i w_i (m_i^{\mathrm{obs}} - \mu_i^{\mathrm{th}})}{\sum_i w_i}, \quad w_i = \sigma_i^{-2}
\end{equation}

\textbf{Optimization:} Parameter determination via \textit{Differential Evolution} (global evolutionary optimizer) with subsequent L-BFGS-B refinement (\textit{polish}).

\textbf{MCMC uncertainties:} For the flat CFM, parameter uncertainties are determined using \textit{emcee} \cite{ForemanMackey2013} (32~walkers, 3,000~steps, 500~burn-in). The posterior distributions yield $1\sigma$ confidence intervals for all parameters including derived quantities $\Phi_0$ and $z_{\mathrm{trans}}$.

\textbf{Model selection:} In addition to $\chi^2$, the Akaike Information Criterion (AIC~$= \chi^2 + 2k$) and the Bayesian Information Criterion (BIC~$= \chi^2 + k \ln n$) are computed, where $k$ is the number of effective parameters and $n$ the number of data points. To check for overfitting, a 5-fold cross-validation is additionally performed. The complete analysis code is publicly available.\footnote{\url{https://github.com/lukisch/cfm-cosmology}}

\subsubsection*{4.4\quad Results}

Three models are fitted: flat $\Lambda$CDM (2~parameters), CFM with flatness constraint (4~parameters), and CFM without constraint (5~parameters).

\begin{table}[H]
\centering
\caption{Fitted parameters and goodness of fit: $\Lambda$CDM vs.\ CFM against Pantheon+ (1,590~SNe~Ia). For CFM~(flat), $1\sigma$ MCMC uncertainties are given.}
\label{tab:en-results}
\begin{tabular}{lccc}
\toprule
 & $\Lambda$CDM & CFM (flat) & CFM (free) \\
\midrule
Free parameters $k$ & 2 & 4 & 5 \\
$\Omega_m$ & 0.244 & $0.368^{+0.025}_{-0.023}$ & 0.552 \\
$\Omega_\Lambda$ / $\Omega_\Phi(z{=}0)$ & 0.756 & 0.636 & 0.872 \\
$\Phi_0$ (derived) & -- & $0.988^{+0.615}_{-0.221}$ & 1.292 \\
$k$ (transition sharpness) & -- & $1.44^{+1.22}_{-0.84}$ & 1.98 \\
$a_{\mathrm{trans}}$ ($z_{\mathrm{trans}}$) & -- & 0.75 (0.33) & 0.80 (0.25) \\
$\Omega_{\mathrm{total}}$ & 1.000 & 1.000 & 1.423 \\
\midrule
$\chi^2$ (diagonal) & 729.0 & 716.8 & 715.9 \\
$\chi^2$ (full cov.) & 1432.0 & 1420.8 & -- \\
$\chi^2/\mathrm{dof}$ & 0.459 & 0.452 & 0.452 \\
AIC (diagonal) & 733.0 & 724.8 & 725.9 \\
AIC (full cov.) & 1436.0 & 1428.8 & -- \\
BIC & 743.7 & 746.3 & 752.8 \\
\bottomrule
\end{tabular}
\end{table}

The CFM with flatness constraint yields $\Omega_m = 0.368 \pm 0.024$ (MCMC) -- physically plausible and close to the Planck value ($0.315 \pm 0.007$). The fitted transition redshift $z_{\mathrm{trans}} = 0.33$ ($a_{\mathrm{trans}} = 0.75$) lies at later cosmic times than theoretically expected. The transition sharpness $k = 1.44^{+1.22}_{-0.84}$ describes a smooth transition with a broad posterior -- the data prefer a transition but allow a range of transition sharpnesses.

\textbf{Full covariance matrix:} Repeating the analysis with the full statistical-systematic covariance matrix confirms the results: $\Delta\chi^2 = -11.2$ and $\Delta\mathrm{AIC} = -7.2$ (compared to $-12.2$ and $-8.2$ with diagonal errors). The slight reduction is explained by the inclusion of systematic correlations between neighboring supernovae.

\subsubsection*{4.5\quad Model Selection}

\begin{table}[H]
\centering
\caption{Model comparison: CFM vs.\ $\Lambda$CDM. Negative values favor the CFM.}
\label{tab:en-comparison}
\begin{tabular}{lcc}
\toprule
\textbf{Criterion} & CFM (flat) vs.\ $\Lambda$CDM & CFM (free) vs.\ $\Lambda$CDM \\
\midrule
$\Delta\chi^2$ & $\mathbf{-12.2}$ & $-13.1$ \\
$\Delta$AIC & $\mathbf{-8.2}$ & $-7.1$ \\
$\Delta$BIC & $+2.6$ & $+9.0$ \\
\midrule
5-fold $\langle\chi^2/n\rangle$ & $\mathbf{0.4499}$ & $0.4498$ \\
$\Lambda$CDM: $\langle\chi^2/n\rangle$ & \multicolumn{2}{c}{$0.4519$} \\
\bottomrule
\end{tabular}
\end{table}

\textbf{Interpretation:} Three of four selection criteria favor the flat CFM over $\Lambda$CDM: $\chi^2$ ($-12.2$), AIC ($-8.2$), and cross-validation ($0.4499$ vs.\ $0.4519$). Only the BIC, which more strongly penalizes additional parameters, shows a marginal preference for $\Lambda$CDM ($\Delta\mathrm{BIC} = +2.6$). According to the Kass--Raftery scale \cite{KassRaftery1995}, this value lies at the boundary of significance ($|\Delta\mathrm{BIC}| < 2$: not significant; $2$--$6$: positive evidence). The cross-validation -- the most robust method for overfitting detection -- shows that the CFM generalizes better to unseen data than $\Lambda$CDM.


\subsubsection*{4.6\quad MCMC Posterior Analysis}

Parameter uncertainties are determined using \textit{emcee} \cite{ForemanMackey2013} (32~walkers, 3,000~steps, 500~burn-in, acceptance rate: 63\%). The results for the flat CFM are given in Table~\ref{tab:en-results} as $1\sigma$ uncertainties. The posterior distribution of $\Omega_m$ is nearly Gaussian with $\Omega_m = 0.368^{+0.025}_{-0.023}$. The transition sharpness $k$ shows a broad, asymmetric posterior ($k = 1.44^{+1.22}_{-0.84}$), meaning the data prefer a transition but constrain the sharpness less strongly. The derived quantities $\Phi_0 = 0.988^{+0.615}_{-0.221}$ and $z_{\mathrm{trans}} = 0.35$ are consistent with the point estimates. The MCMC-computed $w(z)$ confidence bands (Table~\ref{tab:en-weff}) show that $w = -1$ lies outside the $1\sigma$ range for all redshifts.

\textbf{Interpretation of $\Omega_m = 0.368$:} The CFM value exceeds the Planck CMB value ($\Omega_m^{\mathrm{Planck}} = 0.315 \pm 0.007$), which is typical for pure supernova fits. In the CFM context, this deviation has a physical interpretation: the model reinterprets part of the density classified as ``dark energy'' in $\Lambda$CDM as a dynamic curvature effect. Since $\Omega_\Phi(a)$ vanishes at early times (unlike $\Omega_\Lambda = \mathrm{const.}$), $\Omega_m$ must be compensatorily higher to maintain the overall fit. This preference for higher $\Omega_m$ is consistent with recent findings from weak gravitational lensing surveys (KiDS, DES), which also favor higher $\Omega_m$ values than Planck.


\subsubsection*{4.7\quad Robustness: Alternative Functional Forms}

To verify whether the results depend on the specific choice of the $\tanh$ form, four different saturation functions are tested under identical conditions:

\begin{table}[H]
\centering
\caption{Comparison of alternative functional forms for $\Omega_\Phi(a)$. All models use the flatness constraint and have $k=4$ parameters.}
\label{tab:en-funcforms}
\begin{tabular}{lcccc}
\toprule
\textbf{Functional form} & $\chi^2$ & AIC & $\Delta\chi^2$ vs.\ $\Lambda$CDM & $\Omega_m$ \\
\midrule
$\tanh$ (standard CFM) & 716.8 & 724.8 & $-12.2$ & 0.364 \\
Logistic function & 717.5 & 725.5 & $-11.5$ & 0.368 \\
Error function (erf) & 716.7 & 724.7 & $-12.3$ & 0.367 \\
Power law & 720.1 & 728.1 & $\phantom{-0}8.9$ & 0.364 \\
\bottomrule
\end{tabular}
\end{table}

\textbf{Result:} All four functional forms yield $\Delta\chi^2 \approx -9$ to $-12$ relative to $\Lambda$CDM. The $\tanh$ form is neither the only possible nor the ``best-fitting'' -- it represents a robust class of saturation functions. This refutes the objection that the CFM results depend on a specific functional choice. The $\Omega_m$ values converge at $\approx 0.36$--$0.37$ for all forms, underscoring the physical consistency.


\subsubsection*{4.8\quad Phantom Stability Analysis}

The fitted equation-of-state parameter $w < -1$ (phantom regime) raises the legitimate question of stability. In ordinary phantom scalar field models, $w < -1$ leads to a growing energy density ($\rho \propto a^{-3(1+w)} \to \infty$ for $w < -1$, $a \to \infty$), which inevitably produces the ``Big Rip'' in finite time -- the destruction of all bound structures in the universe \cite{Caldwell1998}. \textbf{The CFM fundamentally circumvents this pathology} for three reasons:

\begin{enumerate}
\item \textbf{Saturation instead of divergence:} The dynamic saturation mechanism (Eq.~A.5) guarantees $\Omega_\Phi(a) \to \Phi_0$ for $a \to \infty$. The effective energy density remains \textit{finite and bounded for all times}. This is the decisive difference from phantom scalar fields: whereas there $\rho$ diverges, $\Omega_\Phi$ saturates at its maximum value~$\Phi_0$.
\item \textbf{De Sitter end state:} $w_{\mathrm{eff}} \to -1$ for $a \to \infty$. The universe asymptotically approaches \textit{exactly the same end state as $\Lambda$CDM} -- a stable de~Sitter space. The phantom regime is a transient phenomenon, not an end state.
\item \textbf{No Big Rip:} Since $\Omega_\Phi$ saturates, neither the energy density nor the scale factor diverges in finite time. The Big Rip is excluded -- not by an additional assumption but as a \textit{direct consequence} of the saturation mechanism.
\item \textbf{Geometric rather than fluid interpretation:} The formal violation of the null energy condition ($\rho + p \geq 0$) is unproblematic because $\Omega_\Phi$ represents \textit{not a physical field} but a geometric property of spacetime. The energy conditions of general relativity apply to the energy-momentum tensor of physical fields, not to effective geometric terms. Analogous situations are well established in the literature: $f(R)$ gravity theories routinely exhibit effective $w < -1$ without generating physical instabilities \cite{Sotiriou2010}.
\end{enumerate}

\begin{quote}
\textit{In summary: the CFM exhibits ``phantom behavior'' without phantom pathologies. The universe does not end in a rip but in equilibrium.}
\end{quote}


\subsubsection*{4.9\quad Deceleration Parameter and $H_0$ Implications}

\textbf{Deceleration parameter $q(z)$:} The deceleration parameter provides an additional, independently testable prediction. In the CFM, the transition from decelerated to accelerated expansion ($q = 0$) occurs at $z_{\mathrm{acc}} = 0.52$ -- significantly later in cosmic time than in the $\Lambda$CDM model ($z_{\mathrm{acc}} = 0.84$). Furthermore, the CFM predicts a stronger present-day acceleration: $q_0^{\mathrm{CFM}} = -0.81$ compared to $q_0^{\Lambda\mathrm{CDM}} = -0.63$.

\textbf{Implication for structure formation:} A later onset of cosmic acceleration ($z_{\mathrm{acc}} = 0.52$ instead of $0.84$) means that gravitationally bound structures (galaxy clusters, large-scale filaments) \textit{could grow undisturbed for longer} before the acceleration suppressed growth. The matter-dominated era -- during which gravity drives structure growth -- lasted significantly longer in the CFM than in the standard model.

\textbf{Empirical evidence for early massive structures:} Several independent observations put the $\Lambda$CDM model under significant tension regarding structure formation:
\begin{enumerate}
\item \textbf{JWST ``Universe Breakers'':} The James Webb Space Telescope has discovered galaxies at $z > 7$ (approximately 500--700\,Myr after the Big Bang) that are far more massive than permitted by hierarchical structure formation in $\Lambda$CDM \cite{Labbe2023}. Boylan-Kolchin \cite{BoylanKolchin2023} demonstrates quantitatively that the stellar mass density of these objects exceeds the available baryon budget within $\Lambda$CDM halos -- a ``timing problem'' of structure formation.
\item \textbf{El~Gordo (ACT-CL~J0102$-$4915):} This extremely massive galaxy cluster at $z \approx 0.87$ with mass $M_{200} \approx 2.1 \times 10^{15}\,M_\odot$ represents a $> 6\sigma$ tension with $\Lambda$CDM, since an object of this mass at this age with the observed collision velocity is statistically nearly impossible \cite{Asencio2023}.
\item \textbf{Protocluster SPT2349$-$56:} Already 1.4\,Gyr after the Big Bang ($z = 4.3$), this protocluster exhibits at least 14 gas-rich galaxies with a total star formation rate of $\sim\!6{,}500\,M_\odot$/yr -- far more mature than predicted by $\Lambda$CDM \cite{Miller2018}.
\end{enumerate}

The CFM offers a natural explanation for all three observations: since the acceleration only begins at $z_{\mathrm{acc}} = 0.52$ (instead of $z_{\mathrm{acc}} = 0.84$), gravitationally bound structures had more cosmic time for undisturbed growth. The CFM prediction is directly testable through galaxy cluster counts and weak gravitational lensing surveys (Euclid, Vera C.\ Rubin Observatory).

\begin{table}[H]
\centering
\caption{Deceleration parameter $q(z)$: $\Lambda$CDM vs.\ CFM.}
\label{tab:en-qz}
\begin{tabular}{cccc}
\toprule
$z$ & $q$ ($\Lambda$CDM) & $q$ (CFM) & $\Delta q$ \\
\midrule
0.0 & $-0.634$ & $-0.805$ & $-0.171$ \\
0.5 & $-0.217$ & $-0.017$ & $+0.200$ \\
1.0 & $+0.082$ & $+0.321$ & $+0.240$ \\
2.0 & $+0.346$ & $+0.467$ & $+0.121$ \\
\bottomrule
\end{tabular}
\end{table}

\textbf{$H_0$ implications:} The nuisance parameter $M$ absorbs both the absolute magnitude $M_B$ and $H_0$ via the relation $M = M_B + 5\log_{10}(c/H_0) + 25$. Using the SH0ES calibration ($M_B = -19.253$), the CFM yields $H_0 = 76.1\,\mathrm{km/s/Mpc}$ compared to $H_0 = 75.5\,\mathrm{km/s/Mpc}$ in $\Lambda$CDM -- a difference of $\Delta H_0 = +0.5\,\mathrm{km/s/Mpc}$. The $H_0$ tension is not resolved by the CFM alone, as the difference lies within the measurement uncertainty. A direct resolution requires the combination with CMB and BAO data.


% ===================================================================
% 5. COMPARISON WITH ALTERNATIVES
% ===================================================================
\subsection*{5\quad Comparison with Alternative Models}
\addcontentsline{toc}{subsection}{5\quad Comparison with Alternative Models (English)}

\subsubsection*{5.1\quad $\Lambda$CDM (Standard Model)}

The $\Lambda$CDM model is extremely simple ($w = -1$, constant, two cosmological parameters) and fits all current data well. However, it suffers from the cosmological constant problem and the coincidence problem \cite{Weinberg1989}.

\subsubsection*{5.2\quad Quintessence}

Quintessence models \cite{Caldwell1998} postulate a dynamic scalar field~$\phi$ with a time-dependent equation-of-state parameter. They can alleviate the coincidence problem but require a new field and its potential~$V(\phi)$ with many free parameters.

\subsubsection*{5.3\quad Modified Gravity: $f(R)$ Theories}

$f(R)$ gravity theories \cite{Starobinsky1980, Sotiriou2010} replace the Ricci scalar~$R$ in the Einstein--Hilbert action with a more general function. They offer a geometric explanation without dark energy but are mathematically complex and partly inconsistent with observations (gravitational lensing, CMB).

\subsubsection*{5.4\quad Emergent Gravity (Verlinde)}

Verlinde \cite{Verlinde2011, Verlinde2017} proposes that gravity is not a fundamental force but an emergent, entropic phenomenon. In de~Sitter spaces, the entropy associated with the cosmological horizon leads to an additional ``dark'' gravitational force that could explain galaxy behavior without dark matter.

\subsubsection*{5.5\quad Finsler Gravity}

Pfeifer et al.\ \cite{Pfeifer2025} extend general relativity through Finsler geometry, in which the metric depends not only on position but also on velocity:
\begin{equation}
g_{\mu\nu}(x, y) = \frac{1}{2}\,\frac{\partial^2 F^2}{\partial y^\mu \partial y^\nu}, \quad y = \frac{dx}{d\lambda}
\end{equation}
The resulting Finsler--Friedmann equation produces exponential expansion even in vacuum -- without a cosmological constant.


\subsubsection*{5.6\quad Cosmological Teleodynamics}

Trivedi and Venkatasubramanian \cite{Trivedi2025} formulate a game-theoretic cosmology that exhibits remarkable parallels to the approach presented here. Their \textit{Cosmological Teleodynamics} describes the universe as a ``giant potential game'' converging toward a continuous Nash equilibrium. Cosmic acceleration appears as a ``statistically emergent effect of dynamic memory in a self-gravitating medium'' -- a formulation that conceptually corresponds to the ``geometric memory'' of the CFM.


\subsubsection*{5.7\quad Synoptic Comparison}

\begin{table}[H]
\centering
\caption{Synoptic comparison of cosmological models without dark energy.}
\label{tab:en-synopse}
\begin{tabularx}{\textwidth}{lXXX}
\toprule
\textbf{Property} & \textbf{CFM} & \textbf{Finsler} & \textbf{Teleodynamics} \\
\midrule
Theor.\ basis & Standard GR + potential & Finsler geometry & Stat.\ mechanics + game theory \\
Mechanism & Releasing ``brake'' & Velocity-dep.\ metric & Dynamic memory \\
Dark energy & Not needed & Not needed & Not needed \\
Empirical test & Pantheon+ (1,590 SNe, $\Delta\chi^2{=}{-}12$) & Pending & Qualitative \\
Prediction & $w(z)$ time variation & Exp.\ expansion & Nash convergence \\
Complexity & Low (4 params.) & High & Medium \\
\bottomrule
\end{tabularx}
\end{table}


% ===================================================================
% 6. COMPLEMENTARITY AND UNIFICATION
% ===================================================================
\subsection*{6\quad Complementarity and Possible Unification}
\addcontentsline{toc}{subsection}{6\quad Complementarity and Possible Unification (English)}

\subsubsection*{6.1\quad Three Models, One Insight}

All three approaches -- CFM, Finsler gravity, and Cosmological Teleodynamics -- share a fundamental insight:
\begin{quote}
\textit{``The accelerated expansion is not a new `thing' but a property of the geometry or the statistical structure of the universe itself.''}
\end{quote}

\subsubsection*{6.2\quad Hypothesis: CFM as an Effective Description}

A fascinating possibility is that the three models describe different aspects of the same phenomenon. By analogy with the relationship between thermodynamics and statistical mechanics, the following hierarchy could hold:

\begin{itemize}
\item \textbf{Finsler gravity} (microscopic, fundamental): All moments of the 1-particle distribution function contribute to gravity.
\item \textbf{CFM} (macroscopic, phenomenological): The time-dependent potential $\Phi(a)$ effectively encodes the contribution of higher moments.
\item \textbf{Teleodynamics} (systemic, game-theoretic): The Nash equilibrium dynamics describes the global optimization.
\end{itemize}

Mathematically, this complementarity can be represented as a hierarchical relationship:
\begin{equation}
\underbrace{G_{\mu\nu}^{\mathrm{Finsler}}(x, y)}_{\substack{\text{Finsler gravity}\\\text{(microscopic)}}}
\;\xrightarrow{\;\langle\cdot\rangle_{\mathrm{eff}}\;}
\underbrace{G_{\mu\nu} + 8\pi G\,\Omega_\Phi(a)\,g_{\mu\nu}}_{\substack{\text{CFM: modified}\\\text{Einstein equation}}}
\;\xleftarrow{\;\delta\Phi/\delta s_i = 0\;}
\underbrace{\max_{s_P, s_D} \Phi(s_P, s_D)}_{\substack{\text{Teleodynamics:}\\\text{Nash equilibrium}}}
\tag{A.8}
\end{equation}
where the left arrow describes the effective averaging over the velocity-dependent degrees of freedom of Finsler geometry and the right arrow represents the variational condition of the game-theoretic potential that determines the temporal behavior of $\Omega_\Phi(a)$.


% ===================================================================
% 7. TESTABILITY AND PREDICTIONS
% ===================================================================
\subsection*{7\quad Testability and Predictions}
\addcontentsline{toc}{subsection}{7\quad Testability and Predictions (English)}

\subsubsection*{7.1\quad Observable Signatures}

\textbf{1.~Phantom equation of state $w(z) < -1$:} The CFM predicts $|\Delta w| \approx 0.4$ across the entire observable redshift range. The ESA mission Euclid \cite{Euclid2024} and the Nancy Grace Roman Space Telescope (NASA, $\sim$2027) can measure $\sigma_w \approx 0.02$--$0.05$ -- well sufficient to detect or exclude this signature.

\textbf{2.~Deceleration parameter:} The CFM predicts an earlier transition to accelerated expansion ($z_{\mathrm{acc}} = 0.52$ vs.\ $\Lambda$CDM: $z_{\mathrm{acc}} = 0.84$) as well as a stronger present-day acceleration ($q_0 = -0.81$ vs.\ $-0.63$). This is testable independently of $w(z)$.

\textbf{3.~Structure growth:} A modified growth rate $f \cdot \sigma_8$ is predicted, measurable through weak gravitational lensing and galaxy cluster counts. Initial empirical hints are already provided by the JWST ``Universe Breakers'' at $z > 7$ \cite{Labbe2023}, the El~Gordo anomaly ($> 6\sigma$ tension with $\Lambda$CDM; \cite{Asencio2023}), and unexpectedly mature protoclusters at $z > 4$ \cite{Miller2018} -- all consistent with the CFM prediction of an extended growth phase.

\textbf{4.~CMB integral effects:} A modified ISW effect (\textit{Integrated Sachs--Wolfe}) in CMB temperature cross-correlations.

\subsubsection*{7.2\quad Future Missions}

\begin{table}[H]
\centering
\caption{Relevant observational missions for the CFM test.}
\label{tab:en-missions}
\begin{tabularx}{\textwidth}{lccX}
\toprule
\textbf{Mission} & \textbf{Launch} & $\sigma(w)$ & \textbf{Relevance for CFM} \\
\midrule
Euclid (ESA) & 2023 & $\approx 0.02$ & Precision BAO + weak lensing; can distinguish CFM vs.\ $\Lambda$CDM at $z > 0.8$ \\
Roman (NASA) & $\sim$2027 & $\approx 0.03$ & SN survey to $z \approx 2$; ideal instrument for $w(z)$ test \\
DESI & 2021-- & $\approx 0.04$ & Millions of galaxy spectra; BAO and structure growth \\
\bottomrule
\end{tabularx}
\end{table}


\subsubsection*{7.3\quad Model Distinguishability}

\begin{table}[H]
\centering
\caption{Comparison of predictions: $\Lambda$CDM vs.\ CFM (Pantheon+ fit). The $1\sigma$ uncertainties are from the MCMC analysis.}
\label{tab:en-predictions}
\begin{tabular}{lcc}
\toprule
\textbf{Property} & $\Lambda$CDM & CFM \\
\midrule
$w(z{=}0)$ & $-1.000$ & $-1.36 \pm 0.02$ \\
$w(z{=}0.5)$ & $-1.000$ & $-1.45^{+0.09}_{-0.28}$ \\
$w(z{=}1)$ & $-1.000$ & $-1.45^{+0.10}_{-0.30}$ \\
$w(z{=}2)$ & $-1.000$ & $-1.43^{+0.08}_{-0.21}$ \\
Time variation & None & Yes (consistently $w < -1$) \\
$\Delta w$ (measurable) & -- & $\approx -0.4$ \\
$q_0$ (today) & $-0.63$ & $-0.81$ \\
$z_{\mathrm{acc}}$ (transition) & 0.84 & 0.52 \\
\bottomrule
\end{tabular}
\end{table}


% ===================================================================
% 8. DISCUSSION
% ===================================================================
\subsection*{8\quad Discussion}
\addcontentsline{toc}{subsection}{8\quad Discussion (English)}

\subsubsection*{8.1\quad Strengths of the Approach}

\begin{enumerate}
\item \textbf{Conceptual elegance:} No new form of energy required; the acceleration is a ``releasing constraint,'' not a ``new drive.''
\item \textbf{Game-theoretic foundation:} The emergence of physical laws from equilibrium conditions offers a novel explanatory framework, independently supported by the work of Trivedi and Venkatasubramanian \cite{Trivedi2025}.
\item \textbf{Empirical validation:} The CFM fits 1,590 real Pantheon+ supernovae better than $\Lambda$CDM ($\Delta\chi^2 = -12.2$, $\Delta\mathrm{AIC} = -8.2$) and generalizes better in cross-validation.
\item \textbf{Testability:} Specific, quantitative predictions for $w(z)$ and $z_{\mathrm{acc}}$ that are verifiable within a decade.
\item \textbf{Empirical support:} The CFM prediction of an extended growth phase ($z_{\mathrm{acc}} = 0.52$) offers a natural explanation for the JWST ``early galaxy tension'' \cite{Labbe2023, BoylanKolchin2023}, the El~Gordo anomaly \cite{Asencio2023}, and unexpectedly mature protoclusters \cite{Miller2018}.
\item \textbf{Convergence of independent approaches:} CFM, Finsler gravity, and Cosmological Teleodynamics independently arrive at the same conclusion: dark energy is not necessary.
\item \textbf{Reproducibility:} Analysis code and data are publicly available (\url{https://github.com/lukisch/cfm-cosmology}).
\end{enumerate}

\subsubsection*{8.2\quad Limitations and Open Questions}

\begin{enumerate}
\item \textbf{Phenomenological character:} The CFM is not a fundamental theory. Although the $\tanh$ form can be motivated as the exact solution of the saturation ODE~(A.5) and four alternative functional forms yield comparable results (Section~4.7), a derivation from a fundamental quantum equation is still outstanding.
\item \textbf{Parameter freedom:} Four effective parameters versus two in $\Lambda$CDM lead to a marginal BIC disadvantage ($\Delta\mathrm{BIC} = +2.6$), which is however mitigated by the better cross-validation and the robustness across different functional forms.
\item \textbf{Phantom regime:} The effective equation-of-state parameter $w < -1$ lies in the phantom regime. As shown in Section~4.8, this leads in the CFM context neither to a Big Rip nor to instabilities, since $\Omega_\Phi$ saturates and does not represent a physical field. Formally, the CFM violates the null energy condition, analogous to $f(R)$ gravity theories \cite{Sotiriou2010}.
\item \textbf{Outstanding tests:} CMB predictions (ISW effect, CMB power spectrum), BAO signatures, and gravitational lensing effects remain to be computed. The analysis with the full covariance matrix (Section~4.4) does however confirm the results of the diagonal analysis.
\item \textbf{Microscopic basis:} What is $\Phi$ at the quantum level? The connection to a theory of quantum gravity is outstanding. The possible relationship to Finsler gravity (Section~6) could bridge this gap.
\item \textbf{$H_0$ tension:} The $H_0$ analysis (Section~4.9) shows $\Delta H_0 = +0.5\,$km/s/Mpc between CFM and $\Lambda$CDM -- too small to resolve the $H_0$ tension. A resolution requires the combination with CMB and BAO data.
\end{enumerate}


\subsubsection*{8.3\quad Philosophical Implications}

If the CFM (or a related model) is confirmed, this would have profound consequences:

\begin{itemize}
\item \textbf{Dark energy is not a ``thing'':} It would be a geometric memory, not a physical field.
\item \textbf{The universe ``knows'' about its beginning:} The geometry possesses a ``memory.''
\item \textbf{Paradigm shift:} From ``What drives the acceleration?'' to ``Why did the expansion brake earlier?''
\end{itemize}

This would be comparable to the transition from ``What drives the planets?'' (Ptolemy: spheres) to ``How do planets move in the geometry of space?'' (Kepler, Newton, Einstein).


% ===================================================================
% 9. CONCLUSION AND OUTLOOK
% ===================================================================
\subsection*{9\quad Conclusion and Outlook}
\addcontentsline{toc}{subsection}{9\quad Conclusion and Outlook (English)}

This paper has shown:

\begin{enumerate}
\item A game-theoretic framework for cosmology -- the Nash equilibrium between null space and spacetime bubble -- naturally leads to a model in which physical laws appear as emergent equilibrium conditions.
\item The resulting \textit{Curvature Feedback Model} (CFM) explains accelerated expansion without dark energy and passes the test against 1,590 real Type~Ia supernovae from the Pantheon+ catalog \cite{Scolnic2022}: $\Delta\chi^2 = -12.2$ (diagonal) / $-11.2$ (full covariance matrix), $\Delta\mathrm{AIC} = -8.2$ / $-7.2$, and better cross-validation.
\item The robust model selection (AIC, BIC, 5-fold cross-validation) shows that the better fit of the CFM is not attributable to overfitting. This is confirmed by four alternative functional forms, all yielding $\Delta\chi^2 \approx -9$ to $-12$.
\item MCMC-based parameter uncertainties ($\Omega_m = 0.368 \pm 0.024$) and the phantom stability analysis (no Big Rip, asymptotically de~Sitter) support the physical consistency.
\item The CFM makes testable predictions: a persistent phantom equation of state $w(z) < -1$ and a later acceleration transition ($z_{\mathrm{acc}} = 0.52$ vs.\ $0.84$ in $\Lambda$CDM), testable with Euclid and Roman within the next decade. Already now, the CFM prediction of an extended growth phase finds empirical support from JWST observations of unexpectedly massive galaxies at high redshifts \cite{Labbe2023, BoylanKolchin2023} and the statistically improbable existence of massive clusters such as El~Gordo \cite{Asencio2023}.
\item The convergence of three independent approaches (CFM, Finsler gravity, Cosmological Teleodynamics) points toward a possible paradigm shift: \textit{dark energy as an independent entity may be superfluous.}
\end{enumerate}

\textbf{Next steps} include: (a)~testing against Planck CMB and DESI BAO data (the full Pantheon+ covariance matrix has already been taken into account), (b)~computation of CMB power spectrum and structure growth predictions ($f\sigma_8$), (c)~exploration of the connection between CFM and Finsler geometry, (d)~development of a covariant formulation of $\Phi(a)$ from the Ricci scalar~$R$, (e)~investigation of quantum mechanical foundations of the curvature return potential, and (f)~combination with local distance ladder data for direct $H_0$ determination.

\textbf{Outlook: Unification with MOND -- A universe without a dark sector?} A particularly fascinating perspective opens up through the combination of the CFM with \textit{Modified Newtonian Dynamics} (MOND) \cite{Milgrom1983}. While the CFM eliminates dark energy as a geometric effect, MOND replaces dark matter through modified gravitational dynamics on galactic scales. Both frameworks converge on the prediction that structures form earlier and more efficiently than in $\Lambda$CDM -- the CFM through an extended matter-dominated era ($z_{\mathrm{acc}} = 0.52$), MOND through effectively stronger gravity at low accelerations \cite{Asencio2023}. A preliminary analysis with a purely baryonic universe ($\Omega_m = \Omega_b \approx 0.05$) and an extended geometric potential $\Omega_\Phi(a) = \Phi_0 \cdot f_{\mathrm{tanh}}(a) + \alpha \cdot a^{-\beta}$ yields $\Delta\chi^2 = -26.3$ and $\Delta\mathrm{AIC} = -16.3$ relative to $\Lambda$CDM -- \textit{dramatically outperforming both the standard CFM and $\Lambda$CDM}. The MCMC posterior analysis yields $\beta = 2.02 \pm 0.20$, which corresponds exactly to the scaling of spatial curvature ($a^{-2}$). ``Dark matter'' would thus be a dynamic curvature effect, not a particle. This result requires a full relativistic treatment (e.g., within the AeST framework \cite{Skordis2021}) and will be analyzed in detail in a companion paper.

\begin{quote}
\textit{``Sometimes the most elegant explanation is not a new force, but a relaxing constraint.''}
\end{quote}

\end{document}

\documentclass[11pt,a4paper]{article}
\usepackage[utf8]{inputenc}
\usepackage[T1]{fontenc}
\usepackage[english,ngerman]{babel}
\usepackage{geometry}
\geometry{a4paper, left=2.5cm, right=2.5cm, top=2.5cm, bottom=2.5cm}
\usepackage{mathptmx}
\usepackage{helvet}
\usepackage{amsmath}
\usepackage{amssymb}
\usepackage{amsthm}
\usepackage{titlesec}
\usepackage{booktabs}
\usepackage{tabularx}
\usepackage{xcolor}
\usepackage{authblk}
\usepackage{hyperref}
\usepackage{enumitem}
\usepackage{graphicx}
\usepackage{float}
\usepackage{setspace}
\usepackage{array}

\newtheorem{definition}{Definition}
\newtheorem{proposition}{Proposition}

\titleformat{\section}{\Large\bfseries\sffamily\color{black}}{\thesection}{1em}{}
\titleformat{\subsection}{\large\bfseries\sffamily\color{darkgray}}{\thesubsection}{1em}{}
\titleformat{\subsubsection}{\normalsize\bfseries\sffamily\color{darkgray}}{\thesubsubsection}{1em}{}

\hypersetup{
    pdftitle={Eliminating the Dark Sector: Unifying the Curvature Feedback Model with MOND},
    pdfauthor={Lukas Geiger},
    colorlinks=true,
    linkcolor=black,
    urlcolor=blue,
    citecolor=black
}

\onehalfspacing

\begin{document}

% ===================================================================
% TITELSEITE
% ===================================================================

\title{\textbf{\huge Eliminating the Dark Sector:\\Unifying the Curvature Feedback Model with MOND}\\[0.5em]
\Large A Baryon-Only Universe with Geometric Dark Matter and Dark Energy\\[0.3em]
\large Preliminary Analysis with Pantheon+ Type~Ia Supernovae}

\author[1]{Lukas Geiger\thanks{Correspondence: Lukas Geiger, Gei\ss{}b\"uhlweg~1, 79872~Bernau, Germany.}}
\affil[1]{Independent Researcher, Bernau im Schwarzwald}

\date{February 2026 \\ \vspace{0.5em} \small \textit{Working Paper -- Companion to \cite{Geiger2026}}}

\maketitle

\begin{abstract}
\noindent We propose a unified geometric framework that eliminates both dark energy and dark matter from the cosmological energy budget. Building on the Curvature Feedback Model (CFM) \cite{Geiger2026}, which replaces the cosmological constant with a time-dependent curvature return potential $\Omega_\Phi(a)$, we extend the model to a \textit{baryon-only} universe ($\Omega_m = \Omega_b \approx 0.05$) compatible with Modified Newtonian Dynamics (MOND) \cite{Milgrom1983}. The extended Friedmann equation reads:
\begin{equation*}
H^2(a) = H_0^2 \left[\Omega_b\,a^{-3} + \Phi_0 \cdot f_{\mathrm{sat}}(a) + \alpha \cdot a^{-\beta}\right]
\end{equation*}
where the saturation term $f_{\mathrm{sat}}$ replaces dark energy and the power-law term $\alpha \cdot a^{-\beta}$ assumes the cosmological role of dark matter as a purely geometric effect. Tested against 1,590 Pantheon+ Type~Ia supernovae \cite{Scolnic2022}, this ``dark-sector-free'' model yields $\chi^2 = 702.7$ ($\Delta\chi^2 = -26.3$ vs.\ $\Lambda$CDM, $\Delta\mathrm{AIC} = -16.3$, $\Delta\mathrm{BIC} = -4.2$), dramatically outperforming both $\Lambda$CDM and the standard CFM. MCMC posterior analysis yields $\alpha = 0.68^{+0.02}_{-0.07}$ and $\beta = 2.02^{+0.26}_{-0.14}$, revealing that the geometric DM term scales as \textit{spatial curvature} ($a^{-2}$, $w = -1/3$) -- not as matter ($a^{-3}$). We discuss the physical interpretation within the game-theoretic framework and the connection to the relativistic MOND theory AeST \cite{Skordis2021}. If confirmed by CMB and BAO data, this framework would render the entire dark sector -- comprising 95\% of the energy budget in $\Lambda$CDM -- superfluous.

\vspace{0.5em}
\noindent \textbf{Keywords:} Curvature Feedback Model, MOND, dark matter, dark energy, baryon-only universe, Pantheon+, modified gravity, geometric cosmology

\vspace{0.5em}
\noindent \textbf{Subject areas:} Theoretical Physics, Cosmology, Modified Gravity
\end{abstract}

\newpage
\tableofcontents
\newpage

% ===================================================================
% KI-NUTZUNG
% ===================================================================
\section*{AI Disclosure and Methodology}
\addcontentsline{toc}{section}{AI Disclosure and Methodology}

\noindent\textbf{Extended Methodology Statement:} This paper is an experiment in \textit{AI-Assisted Science}. The division of labor is disclosed transparently:

\begin{description}[style=nextline, leftmargin=2cm]
\item[\textbf{Human author} (Lukas Geiger)] Physical intuition, core hypotheses (game-theoretic foundation, saturation mechanism, geometry-as-dark-sector, Efficiency Hypothesis, phase transition concept), interpretation of results, strategic decisions, and final responsibility for all scientific content.
\item[\textbf{Claude Opus 4.6} (Anthropic)] Co-writer: Mathematical formalization, derivation of equations, code development (Python/MCMC), statistical analysis (Pantheon+ fits), text generation, and structural organization.
\item[\textbf{Gemini} (Google DeepMind)] Reviewer: Critical feedback, MOND compatibility analysis, identification of BBN crisis, trace-coupling suggestion, strategic recommendations.
\end{description}

\vspace{0.5em}
\noindent\textit{Note:} The mathematical formalization and the statistical fits were performed by AI systems. The author presents these hypotheses as a \textit{Working Paper} to enable scrutiny and further development by the scientific community. \textbf{Independent mathematical verification is explicitly encouraged.}

\newpage


% ===================================================================
% 1. EINLEITUNG
% ===================================================================
\section{Introduction: The Dark Sector Problem}
\label{sec:intro}

The standard cosmological model, $\Lambda$CDM, describes the energy budget of the universe as consisting of approximately 5\% baryonic matter, 27\% cold dark matter (CDM), and 68\% dark energy ($\Lambda$) \cite{Planck2020}. Despite its remarkable empirical success, this model implies that \textit{95\% of the universe consists of entities that have never been directly detected}.

Two independent lines of research challenge this picture:

\begin{enumerate}
\item \textbf{The Curvature Feedback Model (CFM)} \cite{Geiger2026}: Developed from a game-theoretic framework, the CFM replaces the cosmological constant $\Lambda$ with a time-dependent curvature return potential $\Omega_\Phi(a)$, explaining accelerated expansion as a geometric ``memory'' rather than a new energy form. Tested against 1,590 Pantheon+ supernovae, the CFM yields $\Delta\chi^2 = -12.2$ relative to $\Lambda$CDM.

\item \textbf{Modified Newtonian Dynamics (MOND)} \cite{Milgrom1983}: MOND modifies gravitational dynamics at accelerations below $a_0 \approx 1.2 \times 10^{-10}$\,m/s$^2$, successfully predicting galactic rotation curves, the baryonic Tully-Fisher relation \cite{McGaugh2016}, and the radial acceleration relation \cite{Lelli2017} without invoking dark matter.
\end{enumerate}

The central question of this paper is: \textit{Can both frameworks be unified into a single model that eliminates the entire dark sector?}

\subsection{The Compatibility Question}

At first glance, CFM and MOND address different ``dark'' problems:
\begin{itemize}
\item CFM replaces \textbf{dark energy} (cosmological expansion)
\item MOND replaces \textbf{dark matter} (galactic dynamics)
\end{itemize}

However, a naive combination encounters a fundamental tension: the standard CFM fits $\Omega_m \approx 0.36$, implying substantial dark matter ($\Omega_m - \Omega_b \approx 0.31$). If MOND is correct and dark matter does not exist, the model must function with $\Omega_m = \Omega_b \approx 0.05$ alone.

\subsection{Structure Formation: Common Ground}

Both frameworks converge on a critical prediction: structures form \textit{earlier and more efficiently} than $\Lambda$CDM allows.

\begin{itemize}
\item \textbf{CFM:} The later onset of cosmic acceleration ($z_{\mathrm{acc}} = 0.52$ vs.\ $0.84$) extends the matter-dominated growth phase \cite{Geiger2026}.
\item \textbf{MOND:} Enhanced gravitational attraction at low accelerations leads to faster gravitational collapse on large scales \cite{Asencio2023}.
\end{itemize}

This shared prediction is supported by multiple observational anomalies: the JWST ``Universe Breakers'' at $z > 7$ \cite{Labbe2023, BoylanKolchin2023}, the El~Gordo cluster at $z \approx 0.87$ (${>}6\sigma$ tension with $\Lambda$CDM) \cite{Asencio2023}, and unexpectedly mature protoclusters at $z > 4$ \cite{Miller2018}.


% ===================================================================
% 2. THEORIE
% ===================================================================
\section{Theoretical Framework}
\label{sec:theory}

\subsection{The Extended Curvature Feedback Model}

In the standard CFM \cite{Geiger2026}, the Friedmann equation reads:
\begin{equation}
H^2(a) = H_0^2 \left[\Omega_m\,a^{-3} + \Omega_\Phi(a)\right]
\end{equation}
with
\begin{equation}
\Omega_\Phi(a) = \Phi_0 \cdot \frac{\tanh\!\big(k\cdot(a - a_{\mathrm{trans}})\big) + s}{1 + s}
\end{equation}

For the baryon-only extension, we decompose the geometric potential into two components:
\begin{equation}
\boxed{H^2(a) = H_0^2 \left[\Omega_b\,a^{-3} + \underbrace{\Phi_0 \cdot f_{\mathrm{sat}}(a)}_{\text{geometric DE}} + \underbrace{\alpha \cdot a^{-\beta}}_{\text{geometric DM}}\right]}
\label{eq:extended_cfm}
\end{equation}

where:
\begin{itemize}
\item $\Omega_b \approx 0.05$ is the baryonic matter density (fixed)
\item $\Phi_0 \cdot f_{\mathrm{sat}}(a)$ is the saturation-type dark energy replacement (from the Dynamic Saturation Mechanism)
\item $\alpha \cdot a^{-\beta}$ is a power-law term that assumes the \textit{cosmological} role of dark matter
\end{itemize}

The flatness constraint $H^2(a{=}1)/H_0^2 = 1$ yields:
\begin{equation}
\Omega_b + \Phi_0 \cdot f_{\mathrm{sat}}(1) + \alpha = 1
\end{equation}

\subsection{Trace Coupling and BBN Consistency}
\label{subsec:trace_coupling}

A critical constraint on the geometric DM term is Big Bang Nucleosynthesis (BBN): at $a \sim 10^{-9}$, the naive power-law $\alpha \cdot a^{-2}$ would yield $\sim 10^{18}$, completely dominating the Friedmann equation and destroying the predicted primordial element abundances. The term \textit{must} be suppressed during the radiation era.

We propose that the geometric DM term couples not to the energy density $\rho$ but to the \textit{trace of the energy-momentum tensor}:
\begin{equation}
T \equiv g^{\mu\nu} T_{\mu\nu} = -\rho + 3p = -\rho(1 - 3w)
\end{equation}

This trace has a remarkable property: for relativistic matter (radiation, $w = 1/3$), the trace vanishes exactly:
\begin{equation}
T_{\mathrm{rad}} = -\rho_{\mathrm{rad}} + 3 \cdot \tfrac{1}{3}\rho_{\mathrm{rad}} = 0
\end{equation}

This is not a coincidence but a consequence of \textit{conformal symmetry}: massless fields are conformally invariant, and the trace of a conformally invariant energy-momentum tensor vanishes identically. During the radiation-dominated era, conformal symmetry is exact, and the geometric DM term is automatically suppressed.

For non-relativistic matter ($w \approx 0$), the trace is $T_{\mathrm{mat}} = -\rho_m \neq 0$, and the geometric DM term activates. The transition occurs naturally at matter-radiation equality ($a_{\mathrm{eq}} \approx 3 \times 10^{-4}$), well after BBN ($a_{\mathrm{BBN}} \sim 10^{-9}$).

The full extended Friedmann equation with trace coupling reads:
\begin{equation}
\boxed{H^2(a) = H_0^2 \left[\Omega_b\,a^{-3} + \Phi_0 \cdot f_{\mathrm{sat}}(a) + \alpha \cdot a^{-\beta} \cdot \mathcal{S}(a)\right]}
\label{eq:extended_cfm_trace}
\end{equation}
where $\mathcal{S}(a)$ is the trace-coupling suppression factor:
\begin{equation}
\mathcal{S}(a) = \frac{|T|}{|T| + \rho_{\mathrm{rad}}} = \frac{\Omega_b\,a^{-3}}{\Omega_b\,a^{-3} + \Omega_r\,a^{-4}}
= \frac{1}{1 + (a_{\mathrm{eq}}/a)}
\label{eq:suppression}
\end{equation}
with $a_{\mathrm{eq}} = \Omega_r/\Omega_b \approx 3 \times 10^{-4}$ (using $\Omega_r \approx 9 \times 10^{-5}$). This factor satisfies:
\begin{itemize}
\item $\mathcal{S}(a \ll a_{\mathrm{eq}}) \approx a/a_{\mathrm{eq}} \to 0$ \quad (radiation era: BBN protected)
\item $\mathcal{S}(a \gg a_{\mathrm{eq}}) \approx 1$ \quad (matter/DE era: full geometric DM)
\item $\mathcal{S}(a = 1) \approx 1 - 3\times10^{-4} \approx 1$ \quad (today: SN fit unchanged)
\end{itemize}

\textbf{Impact on the Pantheon+ fit:} Since all Pantheon+ supernovae are at $z < 2.3$ ($a > 0.30$), the suppression factor is $\mathcal{S} > 0.999$ throughout the observed redshift range. The MCMC results ($\alpha$, $\beta$, $\chi^2$) are unchanged to numerical precision.

\textbf{Physical interpretation:} The trace coupling has a deep geometric meaning. In the game-theoretic framework, the geometric DM term represents the curvature ``memory'' of the initial energy concentration. During the radiation era, the universe is conformally flat (radiation is scale-free), and there is no curvature memory to sustain. The geometric DM term activates only when conformal symmetry is broken by the emergence of massive (non-relativistic) matter -- precisely at the epoch when CDM would begin to form structures in the standard picture.

\subsection{Physical Interpretation of the Geometric DM Term}

The term $\alpha \cdot a^{-\beta}$ with $\beta \approx 2.0$ (from MCMC) requires physical interpretation:

\begin{enumerate}
\item \textbf{Scaling behavior:} The MCMC posterior yields $\beta = 2.02 \pm 0.20$, consistent with curvature-like scaling ($a^{-2}$, i.e., $\beta = 2$). This is the scaling of spatial curvature in the Friedmann equation, suggesting a geometric rather than material origin.

\item \textbf{Game-theoretic interpretation:} In the spieltheoretischen framework, this term represents a second equilibrium mechanism: while the saturation term describes the ``releasing brake'' (dark energy), the power-law term describes the ``geometric inertia'' of the curvature return -- a residual geometric effect that decays with expansion but slower than matter.

\item \textbf{Connection to MOND:} In the relativistic MOND theory AeST (Aether Scalar Tensor) of Skordis \& Z{\l}o\'snik \cite{Skordis2021}, a scalar field and a vector field produce an effective energy-momentum tensor that modifies the expansion history. The power-law term $\alpha \cdot a^{-\beta}$ may be interpretable as the cosmological imprint of this MOND-like modification.

\item \textbf{Effective equation of state:} The geometric DM term has an effective equation of state $w_{\mathrm{DM,geom}} = \beta/3 - 1 = -0.33 \pm 0.07$, virtually identical to the curvature equation of state ($w_k = -1/3$). The ``dark matter'' component is indistinguishable from spatial curvature.
\end{enumerate}

\subsection{MOND on Galactic vs.\ Cosmological Scales}

A key distinction must be maintained:
\begin{itemize}
\item \textbf{Galactic scales:} MOND modifies the gravitational force law below $a_0$, explaining rotation curves and the Tully-Fisher relation \textit{without dark matter}.
\item \textbf{Cosmological scales:} The extended CFM replaces dark matter's \textit{cosmological role} (contribution to $H(z)$) with a geometric potential, without requiring a particle species.
\end{itemize}

The two mechanisms are complementary: MOND handles local dynamics, while the geometric DM term handles the global expansion history.

\subsection{The Efficiency Hypothesis: Why No Dark Matter?}
\label{subsec:efficiency}

A critical question remains: the extended CFM shows that the data \textit{permit} a baryon-only universe, but why should the universe \textit{be} baryon-only? The game-theoretic framework provides a compelling answer.

In the Nash equilibrium between null space and spacetime bubble \cite{Geiger2026}, the spacetime bubble receives a finite energy budget $E_0$ from the null space. Its objective is to neutralize the concentration gradient $G$ as efficiently as possible while protecting the parent system. This creates a resource allocation problem:

\begin{itemize}
\item \textbf{Baryonic matter:} Interacts electromagnetically, forms stars, produces radiation, collapses into black holes, and generates entropy at maximal rates. Baryons are \textit{highly efficient tools} for gradient reduction.

\item \textbf{Dark matter (hypothetical):} Interacts only gravitationally. It clumps but does not radiate, does not form stars, and contributes minimally to entropy production compared to an equivalent mass of baryonic matter.
\end{itemize}

In a game-theoretically optimized universe, allocating 85\% of the energy budget to a component that barely contributes to the primary objective (entropy-driven gradient reduction) would be a \textit{strategically inferior allocation}. A Nash-optimal system maximizes entropy production per unit energy by channeling the entire budget into ``active'' (baryonic) matter.

\begin{proposition}[Efficiency Principle]
In a Nash-equilibrium universe, the matter content consists exclusively of baryonic matter ($\Omega_m = \Omega_b$), because dark matter represents an inefficient allocation of the initial energy budget with respect to the primary objective function (entropy-driven gradient neutralization). The gravitational effects conventionally attributed to dark matter are instead geometric consequences of the curvature return mechanism (the $\alpha \cdot a^{-\beta}$ term).
\end{proposition}

This provides a \textit{theoretical prediction} rather than a mere observational constraint: the game-theoretic framework does not merely accommodate a baryon-only universe -- it \textit{requires} one. Dark matter is not just observationally absent; it is theoretically disfavored.

The quantitative test is whether the geometric term $\alpha \cdot a^{-\beta}$ can reproduce all cosmological signatures traditionally attributed to dark matter (expansion history, CMB acoustic peaks, matter power spectrum). The Pantheon+ test presented below addresses the first of these.

\subsection{The Geometric Phase Transition}
\label{subsec:phase_transition}

The extended Friedmann equation~\eqref{eq:extended_cfm} contains two geometric terms: the power-law $\alpha \cdot a^{-\beta}$ and the saturation $\Phi_0 \cdot f_{\mathrm{sat}}(a)$. A key insight emerges: these are not independent phenomena but \textit{two phases of a single geometric process} -- the curvature return mechanism operating in different regimes.

\begin{enumerate}
\item \textbf{Early universe ($a \ll a_{\mathrm{trans}}$):} The curvature return is far from saturation. The geometric potential is dominated by the power-law term $\alpha \cdot a^{-2}$, which scales like spatial curvature and plays the cosmological role of ``dark matter'' -- providing gravitational scaffolding for structure formation.

\item \textbf{Transition epoch ($a \approx a_{\mathrm{trans}}$):} As the universe expands, the curvature return approaches its saturation limit $\Phi_0$. The power-law contribution decays, while the saturation term rises.

\item \textbf{Late universe ($a \gtrsim a_{\mathrm{trans}}$):} The saturation term dominates, providing a near-constant geometric potential that drives accelerated expansion -- the role conventionally attributed to ``dark energy.''
\end{enumerate}

This picture yields a natural interpretation: \textit{dark matter and dark energy are not two separate substances but two phases of the same geometric phenomenon.} In the early universe, spacetime geometry behaves like dark matter; in the late universe, the same geometry behaves like dark energy. The ``phase transition'' is the saturation of the curvature return potential.

We call this the \textbf{Decaying Dark Geometry} hypothesis: the geometric potential is a decaying remnant of the Big Bang's initial curvature concentration. Early on, it provides gravitational structure (``dark matter''). As it decays and saturates, it provides accelerated expansion (``dark energy''). There is no dark sector -- only geometry at different stages of relaxation.

\begin{definition}[Decaying Dark Geometry]
The cosmological dark sector is a single geometric phenomenon: the curvature return potential $\Omega_\Phi$ of the game-theoretic null space$\leftrightarrow$spacetime equilibrium. Its two apparent components -- dark matter ($\alpha \cdot a^{-\beta}$, dominant at early times) and dark energy ($\Phi_0 \cdot f_{\mathrm{sat}}$, dominant at late times) -- represent the unsaturated and saturated phases of the same relaxation process.
\end{definition}

\subsection{Nested Cosmic Structure: The Mother--Daughter--Granddaughter Model}
\label{subsec:nested}

The game-theoretic framework \cite{Geiger2026} implies a nested ontological structure:

\begin{enumerate}
\item \textbf{Mother (Mutter):} The null space -- the pre-geometric ``ground state'' from which spacetime emerges. In the Nash equilibrium, the null space is the player whose concentration gradient $G$ drives the emergence of the spacetime bubble.

\item \textbf{Daughter (Tochter):} The spacetime geometry -- the ``shell'' that emerges from the null space. This geometry contains the curvature return potential, which manifests as both the ``dark matter'' phase (power-law, early) and the ``dark energy'' phase (saturation, late). The daughter \textit{is} the geometric dark sector.

\item \textbf{Granddaughter (Enkelin):} Baryonic matter -- the ``active component'' embedded within the geometric shell. Baryons are the Nash-optimal entropy-producing agents (cf.\ Section~\ref{subsec:efficiency}), born from the geometric substrate.
\end{enumerate}

This hierarchy -- Null Space $\to$ Geometry $\to$ Matter -- inverts the conventional materialist ontology. Matter does not curve spacetime; rather, spacetime geometry is the primary entity, and matter is a secondary condensation within it. The ``dark sector'' is simply the geometric substrate (the Daughter) that hosts the baryonic content (the Granddaughter).

This nested structure has a testable consequence: the geometric ``dark matter'' ($\alpha \cdot a^{-2}$) is not a substance that can be separated from spacetime. Unlike particulate dark matter, it cannot be found in laboratory detectors -- because it \textit{is} the laboratory, the spacetime geometry itself.


% ===================================================================
% 3. DATENANALYSE
% ===================================================================
\section{Data Analysis and Results}
\label{sec:analysis}

\subsection{Data and Methodology}

We use the Pantheon+ catalog \cite{Scolnic2022} comprising 1,590 Type~Ia supernovae with $z > 0.01$ (redshift range $0.01$--$2.26$). Luminosity distances are computed via cumulative trapezoidal integration on a fine redshift grid ($N = 2{,}000$). The nuisance parameter~$M$ is analytically marginalized. Parameter optimization uses differential evolution with L-BFGS-B polish.

\subsection{Results: Model Comparison}

\begin{table}[H]
\centering
\caption{Model comparison against 1,590 Pantheon+ supernovae.}
\label{tab:comparison}
\begin{tabular}{lcccccc}
\toprule
\textbf{Model} & $\Omega_m$ & \textbf{Params} & $\chi^2$ & $\Delta\chi^2$ & AIC & BIC \\
\midrule
$\Lambda$CDM & 0.244 & 2 & 729.0 & 0 & 733.0 & 743.7 \\
CFM Standard & 0.364 & 4 & 716.8 & $-12.2$ & 724.8 & 746.3 \\
\midrule
CFM Baryon Fixed & 0.050 & 3 & 945.5 & $+216.5$ & 951.5 & 967.6 \\
CFM Baryon Band & 0.070 & 4 & 894.7 & $+165.7$ & 902.7 & 924.1 \\
\midrule
\textbf{Extended CFM+MOND} & \textbf{0.050} & \textbf{5} & \textbf{702.7} & $\mathbf{-26.3}$ & \textbf{712.7} & 739.5 \\
\bottomrule
\end{tabular}
\end{table}

\subsection{Key Findings}

\begin{enumerate}
\item \textbf{Simple baryon-only CFM fails:} With $\Omega_m = 0.05$ and only the $\tanh$ saturation term, the fit degrades catastrophically ($\Delta\chi^2 = +216.5$). The optimizer attempts extreme parameters ($k = 86$, $a_{\mathrm{trans}} = 0.06$) to create a near-step-function, confirming that the standard CFM \textit{cannot} compensate for missing dark matter.

\item \textbf{Extended CFM succeeds spectacularly:} Adding the geometric DM term $\alpha \cdot a^{-\beta}$ restores and \textit{exceeds} the fit quality, achieving $\Delta\chi^2 = -26.3$ versus $\Lambda$CDM -- better than both the standard $\Lambda$CDM \textit{and} the standard CFM by a wide margin.

\item \textbf{Best-fit parameters (MCMC):} A full Markov Chain Monte Carlo analysis (emcee, 48 walkers, 5000 steps) yields:
\begin{itemize}
\item Saturation term: $\Phi_0 = 0.43^{+0.14}_{-0.08}$, $k = 9.8^{+6.7}_{-3.8}$, $a_{\mathrm{trans}} = 0.971^{+0.016}_{-0.031}$ ($z_{\mathrm{trans}} = 0.03$)
\item Geometric DM term: $\alpha = 0.68^{+0.02}_{-0.07}$, $\beta = 2.02^{+0.26}_{-0.14}$
\item Energy budget at $a = 1$: $\Omega_b = 0.05$, $\Omega_\Phi = 0.95$ (total geometric contribution)
\end{itemize}

\item \textbf{$\beta \approx 2.0$: Curvature scaling.} The MCMC posterior for $\beta$ peaks at $2.02 \pm 0.20$, consistent with \textit{curvature-like scaling} ($a^{-2}$, i.e., $w = -1/3$). This is a remarkable result: the data independently recover a scaling exponent that corresponds to \textit{spatial curvature}, not to a material component. The effective equation of state $w_{\mathrm{DM,geom}} = \beta/3 - 1 = -0.33$ is virtually identical to the curvature equation of state.

\item \textbf{Late saturation transition:} The saturation transition occurs very late ($z_{\mathrm{trans}} \approx 0.03$), much later than in the standard CFM ($z_{\mathrm{trans}} = 0.33$). The geometric DM term (curvature-like) dominates the early expansion, while the saturation term provides the late-time acceleration.

\item \textbf{AIC vs.\ BIC:} The $\Delta\mathrm{AIC} = -16.3$ strongly favors the extended model. The $\Delta\mathrm{BIC} = -4.2$ also favors it despite the parameter penalty (5 vs.\ 2 parameters). This is the first model in our analysis to achieve \textit{both} AIC and BIC preference over $\Lambda$CDM simultaneously.
\end{enumerate}


% ===================================================================
% 4. DISKUSSION
% ===================================================================
\section{Discussion}
\label{sec:discussion}

\subsection{A Universe Without a Dark Sector}

The extended CFM demonstrates that the entire expansion history probed by Type~Ia supernovae can be described with:
\begin{itemize}
\item Baryonic matter ($\Omega_b = 0.05$) -- the \textit{only} material content
\item A saturation-type geometric potential -- replacing dark energy
\item A power-law geometric term -- replacing dark matter's cosmological role
\end{itemize}

If this result survives tests against CMB and BAO data, it would imply that 95\% of the $\Lambda$CDM energy budget is an artifact of interpreting geometric effects as material components.

\subsection{The $\beta \approx 2.0$ Result: Curvature as Dark Matter}

The MCMC posterior for the scaling exponent yields $\beta = 2.02^{+0.26}_{-0.14}$, remarkably close to -- and statistically consistent with -- the curvature scaling $\beta = 2$ ($a^{-2}$). This corresponds to an effective equation of state $w_{\mathrm{DM,geom}} = -0.33$, indistinguishable from spatial curvature ($w_k = -1/3$). For comparison, the standard cosmological components scale as:
\begin{itemize}
\item Matter: $\beta = 3$ ($a^{-3}$, $w = 0$)
\item Curvature: $\beta = 2$ ($a^{-2}$, $w = -1/3$) $\quad\leftarrow$ \textbf{recovered by MCMC}
\item Radiation: $\beta = 4$ ($a^{-4}$, $w = 1/3$)
\end{itemize}

This result has profound implications: the component traditionally identified as ``dark matter'' in the Friedmann equation may in fact be \textit{spatial curvature} -- not the global curvature $k$ of the FLRW metric, but a \textit{dynamic, decaying curvature memory} encoded in the geometric potential. In the game-theoretic framework, this is precisely the ``geometric inertia'' of the curvature return: a residual imprint of the Big Bang's energy concentration that dilutes with expansion at the curvature rate $a^{-2}$ rather than the matter rate $a^{-3}$.

\subsection{Relation to AeST and Relativistic MOND}

The relativistic MOND theory AeST (Aether Scalar Tensor) \cite{Skordis2021} provides the only known framework that simultaneously:
\begin{enumerate}
\item Reproduces MOND dynamics on galactic scales
\item Fits the CMB power spectrum (including the third acoustic peak)
\item Fits the matter power spectrum
\end{enumerate}

AeST achieves this through a scalar field $\phi$ and a timelike vector field $A_\mu$ that produce an effective energy-momentum tensor. The cosmological background equations in AeST contain terms that contribute to $H^2(a)$ with non-standard scaling. A detailed comparison between the AeST background equations and the extended CFM Friedmann equation~\eqref{eq:extended_cfm} is a key objective for future work.

\subsection{Addressing the Cosmological ``Endgegner''}
\label{subsec:endgegner}

Any theory that eliminates dark matter must confront three critical observational pillars of $\Lambda$CDM. We address each in turn, showing how the Decaying Dark Geometry hypothesis provides a pathway through each challenge.

\subsubsection{Challenge 1: CMB Acoustic Peaks}

The relative heights of the CMB acoustic peaks -- particularly the ratio of the first to the third peak -- are conventionally interpreted as evidence for a gravitational component that does not interact with photons (i.e., dark matter). In $\Lambda$CDM, cold dark matter provides gravitational potential wells that drive baryon-photon oscillations without experiencing radiation pressure.

\textit{Resolution:} In the extended CFM, the geometric DM term $\alpha \cdot a^{-\beta} \cdot \mathcal{S}(a)$ is \textit{not a material component} -- it is a property of spacetime geometry itself. The trace-coupling suppression factor $\mathcal{S}(a)$ (Section~\ref{subsec:trace_coupling}) ensures that the term activates only after matter-radiation equality ($a_{\mathrm{eq}} \approx 3 \times 10^{-4}$). At recombination ($a \approx 10^{-3}$), $\mathcal{S} \approx 0.77$, so the geometric DM term is active and provides gravitational potential wells that are:
\begin{itemize}
\item Decoupled from the photon-baryon fluid (they are geometric, not material)
\item Activated by the breaking of conformal symmetry (trace coupling)
\item Scaling as $a^{-2}$ rather than $a^{-3}$, which modifies the peak ratios in a potentially testable way
\end{itemize}

The relativistic MOND theory AeST \cite{Skordis2021} has already demonstrated that a non-material field can reproduce the CMB peak structure, including the third peak. The extended CFM provides a concrete cosmological background ($H(z)$) within which such a mechanism can operate. A full computation of the angular power spectrum $C_\ell$ is the highest-priority next step (cf.\ Paper~III \cite{Geiger2026c}).

\subsubsection{Challenge 2: The Bullet Cluster}

The Bullet Cluster (1E~0657-56) is often cited as definitive evidence for particulate dark matter: gravitational lensing maps show mass concentrations offset from the X-ray-emitting gas after a cluster collision \cite{Clowe2006}. The argument is that dark matter, being collisionless, passed through while the gas was slowed by ram pressure.

\textit{Resolution:} In the Decaying Dark Geometry framework, the ``dark matter'' component is \textit{spacetime geometry}, not a substance. During a cluster collision:
\begin{itemize}
\item The \textit{baryonic gas} experiences ram pressure and is decelerated.
\item The \textit{geometric potential} is a property of the spacetime curvature distribution, which is sourced by the total energy distribution \textit{including its own history}. As a geometric ``memory,'' it traces the pre-collision mass distribution and need not track the post-collision gas distribution instantaneously.
\item The \textit{galaxies} (stellar component), being effectively collisionless like the geometric potential, pass through unimpeded.
\end{itemize}

The lensing signal would then trace the geometric potential (which co-moves with the galaxies) rather than the gas -- precisely as observed. This is analogous to the AeST prediction, where the scalar and vector fields produce lensing effects that are offset from the gas. A quantitative lensing prediction within the extended CFM is needed but is not expected to contradict the Bullet Cluster observations.

\subsubsection{Challenge 3: Structure Formation and the Matter Power Spectrum}

The matter power spectrum $P(k)$ in $\Lambda$CDM is shaped by dark matter halos that begin gravitational collapse during radiation domination (before baryons decouple from photons). Without early-collapsing dark matter, baryonic structures would form too late and on the wrong scales.

\textit{Resolution:} The geometric DM term provides ``geometric scaffolding'' for structure formation:
\begin{itemize}
\item At early times ($a \ll a_{\mathrm{trans}}$), the $\alpha \cdot a^{-2}$ term dominates the expansion history, providing the same deceleration that CDM would provide (albeit with a different scaling).
\item Perturbations in the geometric potential create gravitational wells into which baryons can fall after recombination, just as CDM halos would.
\item The earlier onset of effective gravity (from the combined CFM + MOND enhancement) naturally explains the ``too early, too massive'' structures observed by JWST \cite{Labbe2023}, El~Gordo \cite{Asencio2023}, and high-$z$ protoclusters \cite{Miller2018} -- which are anomalous in $\Lambda$CDM but expected in this framework.
\end{itemize}

The quantitative prediction of $P(k)$ requires solving the linearized perturbation equations with the modified background $H(z)$ and the geometric potential as the seed. This is a key objective for future work.

\subsubsection{The Missing Lagrangian}

A fourth, theoretical challenge remains: the extended CFM currently lacks a Lagrangian formulation. The ODE $d\Omega_\Phi/da = k[1 - (\Omega_\Phi/\Phi_0)^2]$ and the power-law term $\alpha \cdot a^{-\beta}$ are phenomenological. A complete theory requires:
\begin{enumerate}
\item An action principle from which the extended Friedmann equation~\eqref{eq:extended_cfm} follows as the Euler-Lagrange equation
\item A microscopic derivation explaining \textit{why} the saturation ODE takes the specific form $dX/da \propto (1 - X^2)$
\item A connection to known quantum gravity approaches (Loop Quantum Gravity, Finsler geometry, information-theoretic spacetime)
\end{enumerate}

This theoretical foundation is the subject of Paper~III \cite{Geiger2026c}.


\subsection{Critical Self-Assessment: Too Good to Be True?}

The results of this paper -- a baryon-only model that outperforms $\Lambda$CDM by $\Delta\chi^2 = -26.3$ with a simple geometric term -- are remarkable, and a degree of skepticism is warranted. We enumerate the reasons for caution:

\begin{enumerate}
\item \textbf{Overfitting risk:} Five free parameters (vs.\ 2 for $\Lambda$CDM) provide more flexibility. However, the AIC penalty accounts for this ($\Delta\mathrm{AIC} = -16.3$), and even the conservative BIC favors the model ($\Delta\mathrm{BIC} = -4.2$).

\item \textbf{SN-only validation:} The Pantheon+ data probe the expansion history at $z \lesssim 2.3$. The model's predictions at high redshift (CMB at $z \approx 1100$) are extrapolations. The trace-coupling mechanism (Section~\ref{subsec:trace_coupling}) prevents the geometric DM term from diverging at early times, but the quantitative behavior around recombination and matter-radiation equality requires detailed numerical computation.

\item \textbf{Phenomenological nature:} The $\alpha \cdot a^{-\beta}$ term is empirical, not derived from first principles. A phenomenological term that fits SNe well but lacks a Lagrangian derivation cannot be considered a complete theory.

\item \textbf{The $\beta = 2$ coincidence:} While we interpret $\beta \approx 2$ as evidence for a curvature origin, alternative explanations exist. The $a^{-2}$ scaling could be a coincidence or an artifact of the parameterization.

\item \textbf{Perturbation theory:} The background expansion history is only one of three pillars ($H(z)$, $C_\ell$, $P(k)$). The model must be tested at the perturbative level before claims of dark sector elimination are credible.
\end{enumerate}

\textbf{Honest assessment:} The Pantheon+ result is \textit{necessary but not sufficient} for dark sector elimination. The framework is promising and the statistical preference is strong, but the CMB and $P(k)$ tests will be decisive. We present this as a compelling hypothesis, not as a settled conclusion.


\subsection{Limitations and Remaining Challenges}

\begin{enumerate}
\item \textbf{CMB power spectrum:} The angular power spectrum $C_\ell$ is the most critical test. Computing it requires the full perturbation equations in the modified background, including the behavior of the geometric potential at the perturbative level.

\item \textbf{BAO measurements:} Baryon acoustic oscillations at $z \sim 0.5$--$2.5$ (DESI DR2) provide an independent distance measure that must be consistent with the extended CFM.

\item \textbf{Big Bang Nucleosynthesis:} The trace-coupling mechanism (Section~\ref{subsec:trace_coupling}) suppresses the geometric DM term during the radiation era ($\mathcal{S} \to 0$), ensuring standard BBN. However, a detailed computation of the primordial element abundances with the modified $H(z)$ is needed to verify quantitative consistency, particularly around the matter-radiation transition where $\mathcal{S}$ is intermediate.

\item \textbf{Gravitational lensing:} Strong and weak lensing surveys (KiDS, DES, Euclid) probe the matter distribution and must be compatible with the geometric potential.

\item \textbf{Lagrangian derivation:} The phenomenological success must be grounded in an action principle (Paper~III).
\end{enumerate}


% ===================================================================
% 5. FAZIT
% ===================================================================
\section{Conclusion and Outlook}
\label{sec:conclusion}

We have demonstrated that a baryon-only universe ($\Omega_m = \Omega_b \approx 0.05$) with an extended geometric potential can fit the Pantheon+ supernova data \textit{dramatically better} than $\Lambda$CDM ($\Delta\chi^2 = -26.3$, $\Delta\mathrm{AIC} = -16.3$, $\Delta\mathrm{BIC} = -4.2$). The MCMC analysis reveals that the geometric ``dark matter'' term scales as spatial curvature ($\beta = 2.02 \pm 0.20$), leading to the \textit{Decaying Dark Geometry} hypothesis: dark matter and dark energy are two phases of a single geometric relaxation process.

The three principal challenges (CMB acoustic peaks, Bullet Cluster, matter power spectrum) have plausible resolutions within this framework, drawing on both the geometric phase transition concept and the established success of relativistic MOND (AeST) \cite{Skordis2021}. However, these resolutions remain qualitative: quantitative predictions of $C_\ell$ and $P(k)$ are essential before the framework can claim to replace $\Lambda$CDM.

The nested Mother--Daughter--Granddaughter structure (Null Space $\to$ Geometry $\to$ Matter) provides a coherent ontology in which the ``dark sector'' is simply geometry at different stages of relaxation, and matter is a secondary condensation optimized for entropy production.

\subsection{The Three-Paper Program}

This paper is the second in a three-part program:
\begin{enumerate}
\item \textbf{Paper~I} \cite{Geiger2026}: Establishes the game-theoretic foundation and the Curvature Feedback Model for dark energy replacement. Validated against Pantheon+.
\item \textbf{Paper~II} (this work): Extends the CFM to eliminate the entire dark sector, achieving a baryon-only universe consistent with MOND. Introduces the Decaying Dark Geometry hypothesis and the geometric phase transition.
\item \textbf{Paper~III} \cite{Geiger2026c}: Will provide the microscopic foundation -- the Lagrangian derivation and the connection to quantum gravity. Central question: \textit{Which quantum system yields the saturation ODE $d\Omega_\Phi/da = k[1 - (\Omega_\Phi/\Phi_0)^2]$ in the macroscopic limit?}
\end{enumerate}

\textbf{Immediate next steps:}
\begin{enumerate}
\item CMB power spectrum $C_\ell$ with the modified background $H(z)$
\item BAO constraints from DESI DR2
\item Matter power spectrum $P(k)$ and $f\sigma_8(z)$
\item AeST mapping: deriving $\alpha$ and $\beta$ from the Skordis-Z{\l}o\'snik framework
\item Lagrangian formulation and quantum gravity connection (Paper~III)
\end{enumerate}

\subsection{Invitation to the Community}

This work presents a promising hypothesis, not a settled conclusion. The author invites the community to:
\begin{enumerate}
\item \textbf{Replicate:} The analysis code is open source. All fits use the publicly available Pantheon+ catalog. Independent replication of the $\Delta\chi^2 = -26.3$ result is straightforward.
\item \textbf{Extend:} Computing $C_\ell$ and $P(k)$ in the extended CFM framework is the critical next step. Collaboration with groups experienced in modified Boltzmann codes (CLASS/CAMB) is welcome.
\item \textbf{Critique:} The trace-coupling mechanism, the $\beta \approx 2$ interpretation, and the Efficiency Hypothesis all require independent scrutiny.
\end{enumerate}

\begin{quote}
\textit{``If dark energy is a relaxing constraint and dark matter is a geometric shadow, then 95\% of the universe may have been hiding in plain sight -- as the geometry of spacetime itself.''}
\end{quote}


% ===================================================================
% LITERATUR
% ===================================================================
\begin{thebibliography}{99}

\bibitem{Geiger2026}
Geiger, L.\ (2026).
Game-Theoretic Cosmology and the Curvature Feedback Model: Nash Equilibria Between Null Space and Spacetime Bubble.
Working Paper. \url{https://github.com/lukisch/cfm-cosmology}.

\bibitem{Scolnic2022}
Scolnic, D.\ et al.\ (2022).
The Pantheon+ Analysis: The Full Data Set and Light-curve Release.
\textit{The Astrophysical Journal}, 938(2), 113.
DOI: 10.3847/1538-4357/ac8b7a.

\bibitem{Planck2020}
Planck Collaboration (2020).
Planck 2018 results. VI. Cosmological parameters.
\textit{Astronomy \& Astrophysics}, 641, A6.
DOI: 10.1051/0004-6361/201833910.

\bibitem{Milgrom1983}
Milgrom, M.\ (1983).
A modification of the Newtonian dynamics as a possible alternative to the hidden mass hypothesis.
\textit{The Astrophysical Journal}, 270, 365--370.
DOI: 10.1086/161130.

\bibitem{Skordis2021}
Skordis, C.\ \& Z{\l}o\'snik, T.\ (2021).
New Relativistic Theory for Modified Newtonian Dynamics.
\textit{Physical Review Letters}, 127(16), 161302.
DOI: 10.1103/PhysRevLett.127.161302.

\bibitem{McGaugh2016}
McGaugh, S.\,S., Lelli, F.\ \& Schombert, J.\,M.\ (2016).
Radial Acceleration Relation in Rotationally Supported Galaxies.
\textit{Physical Review Letters}, 117(20), 201101.
DOI: 10.1103/PhysRevLett.117.201101.

\bibitem{Lelli2017}
Lelli, F., McGaugh, S.\,S.\ \& Schombert, J.\,M.\ (2017).
One Law to Rule Them All: The Radial Acceleration Relation of Galaxies.
\textit{The Astrophysical Journal}, 836(2), 152.
DOI: 10.3847/1538-4357/836/2/152.

\bibitem{Labbe2023}
Labb\'e, I.\ et al.\ (2023).
A population of red candidate massive galaxies $\sim$600\,Myr after the Big Bang.
\textit{Nature}, 616(7956), 266--269.
DOI: 10.1038/s41586-023-05786-2.

\bibitem{BoylanKolchin2023}
Boylan-Kolchin, M.\ (2023).
Stress testing $\Lambda$CDM with high-redshift galaxy candidates.
\textit{Nature Astronomy}, 7, 731--735.
DOI: 10.1038/s41550-023-01937-7.

\bibitem{Asencio2023}
Asencio, E., Banik, I.\ \& Kroupa, P.\ (2023).
The El Gordo galaxy cluster challenges $\Lambda$CDM for any plausible collision velocity.
\textit{The Astrophysical Journal}, 954(2), 162.
DOI: 10.3847/1538-4357/ace62a.

\bibitem{Miller2018}
Miller, T.\,B.\ et al.\ (2018).
A massive core for a cluster of galaxies at a redshift of 4.3.
\textit{Nature}, 556(7702), 469--472.
DOI: 10.1038/s41586-018-0025-2.

\bibitem{Clowe2006}
Clowe, D.\ et al.\ (2006).
A Direct Empirical Proof of the Existence of Dark Matter.
\textit{The Astrophysical Journal Letters}, 648(2), L109--L113.
DOI: 10.1086/508162.

\bibitem{Geiger2026c}
Geiger, L.\ (2026).
Microscopic Foundations of the Curvature Feedback Model: From Quantum Geometry to Macroscopic Saturation.
Working Paper (in preparation).

\end{thebibliography}

\end{document}

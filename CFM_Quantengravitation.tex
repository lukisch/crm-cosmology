\documentclass[11pt,a4paper]{article}
\usepackage[utf8]{inputenc}
\usepackage[T1]{fontenc}
\usepackage[english,ngerman]{babel}
\usepackage{geometry}
\geometry{a4paper, left=2.5cm, right=2.5cm, top=2.5cm, bottom=2.5cm}
\usepackage{mathptmx}
\usepackage{helvet}
\usepackage{amsmath}
\usepackage{amssymb}
\usepackage{amsthm}
\usepackage{titlesec}
\usepackage{booktabs}
\usepackage{tabularx}
\usepackage{xcolor}
\usepackage{authblk}
\usepackage{hyperref}
\usepackage{enumitem}
\usepackage{graphicx}
\usepackage{float}
\usepackage{setspace}
\usepackage{array}

\newtheorem{definition}{Definition}
\newtheorem{proposition}{Proposition}
\newtheorem{theorem}{Theorem}
\newtheorem{conjecture}{Conjecture}

\titleformat{\section}{\Large\bfseries\sffamily\color{black}}{\thesection}{1em}{}
\titleformat{\subsection}{\large\bfseries\sffamily\color{darkgray}}{\thesubsection}{1em}{}
\titleformat{\subsubsection}{\normalsize\bfseries\sffamily\color{darkgray}}{\thesubsubsection}{1em}{}

\hypersetup{
    pdftitle={Microscopic Foundations of the Curvature Feedback Model},
    pdfauthor={Lukas Geiger},
    colorlinks=true,
    linkcolor=black,
    urlcolor=blue,
    citecolor=black
}

\onehalfspacing

\begin{document}

% ===================================================================
% TITELSEITE
% ===================================================================

\title{\textbf{\huge Microscopic Foundations of the Curvature Feedback Model}\\[0.5em]
\Large From Quantum Geometry to Macroscopic Saturation\\[0.3em]
\large The Lagrangian Derivation and Quantum Gravity Connection}

\author[1]{Lukas Geiger\thanks{Correspondence: Lukas Geiger, Gei\ss{}b\"uhlweg~1, 79872~Bernau, Germany.}}
\affil[1]{Independent Researcher, Bernau im Schwarzwald}

\date{February 2026 \\ \vspace{0.5em} \small \textit{Working Paper -- Paper III in the CFM series \cite{Geiger2026,Geiger2026b}}}

\maketitle

\begin{abstract}
\noindent Papers~I and~II of this series established the Curvature Feedback Model (CFM) as a phenomenologically successful alternative to $\Lambda$CDM, eliminating the entire dark sector through a geometric curvature return mechanism. The present paper addresses the outstanding theoretical challenge: \textit{What is the microscopic origin of the saturation ODE?} We seek the quantum system whose macroscopic (thermodynamic) limit yields the curvature return equation $d\Omega_\Phi/da = k\,[1 - (\Omega_\Phi/\Phi_0)^2]$. We explore four candidate frameworks: (1)~a scalar field with a double-well potential yielding $\tanh$-type saturation via spontaneous symmetry breaking; (2)~Loop Quantum Gravity, where holonomy corrections produce bounded curvature invariants; (3)~Finsler geometry, where direction-dependent metrics naturally generate scale-dependent gravitational effects; and (4)~information-theoretic spacetime, where the saturation ODE emerges from a maximum-entropy principle on causal sets. We derive the effective Lagrangian $\mathcal{L}_{\mathrm{CFM}}$ that reproduces the extended Friedmann equation and discuss the implications for quantum gravity.

\vspace{0.5em}
\noindent \textbf{Keywords:} Curvature Feedback Model, quantum gravity, Lagrangian formulation, Loop Quantum Gravity, Finsler geometry, saturation mechanism, modified gravity

\vspace{0.5em}
\noindent \textbf{Subject areas:} Theoretical Physics, Quantum Gravity, Mathematical Physics
\end{abstract}

\newpage
\tableofcontents
\newpage

% ===================================================================
% KI-NUTZUNG
% ===================================================================
\section*{AI Disclosure}
\addcontentsline{toc}{section}{AI Disclosure}

This paper was developed with intensive use of AI systems. Their contributions are disclosed in detail:

\begin{description}[style=nextline, leftmargin=2cm]
\item[\textbf{Claude Opus 4.6} (Anthropic)] Co-writer: Text generation, mathematical derivations, code development.
\item[\textbf{Gemini} (Google DeepMind)] Reviewer: Critical feedback, quantum gravity connections, strategic recommendations.
\end{description}

\noindent\textit{Note:} Despite the substantial machine contribution, final responsibility for the scientific content and interpretation rests with the human author.

\newpage


% ===================================================================
% 1. EINLEITUNG
% ===================================================================
\section{Introduction: The Central Question}
\label{sec:intro}

The Curvature Feedback Model (CFM) \cite{Geiger2026} and its MOND-compatible extension \cite{Geiger2026b} have demonstrated remarkable phenomenological success:

\begin{itemize}
\item \textbf{Paper~I:} The standard CFM replaces dark energy with a curvature return potential, achieving $\Delta\chi^2 = -12.2$ vs.\ $\Lambda$CDM on Pantheon+ data.
\item \textbf{Paper~II:} The extended CFM eliminates the entire dark sector (both dark energy and dark matter) in a baryon-only universe, achieving $\Delta\chi^2 = -26.3$ with a geometric ``dark matter'' term that scales as spatial curvature ($\beta = 2.02 \pm 0.20$).
\end{itemize}

Both results derive from a single dynamical equation -- the \textit{saturation ODE}:
\begin{equation}
\frac{d\Omega_\Phi}{da} = k \left[1 - \left(\frac{\Omega_\Phi}{\Phi_0}\right)^2\right]
\label{eq:saturation_ode}
\end{equation}

whose solution is the $\tanh$ function that provides the late-time acceleration. The extended model adds a power-law term $\alpha \cdot a^{-\beta}$ representing the unsaturated (early-time) phase of the same geometric process.

The central question of this paper is:

\begin{quote}
\textit{Which microscopic (quantum) system has the property that its macroscopic (thermodynamic) limit yields the saturation ODE~\eqref{eq:saturation_ode}? And can the full extended Friedmann equation be derived from a Lagrangian?}
\end{quote}

This question is not merely academic. Without a Lagrangian formulation, the CFM cannot:
\begin{enumerate}
\item Be consistently coupled to matter fields
\item Generate perturbation equations for $C_\ell$ and $P(k)$ predictions
\item Be connected to known quantum gravity frameworks
\item Be considered a complete physical theory
\end{enumerate}


% ===================================================================
% 2. LAGRANGIAN FORMULIERUNG
% ===================================================================
\section{The Effective Lagrangian}
\label{sec:lagrangian}

\subsection{Requirements}

The effective Lagrangian $\mathcal{L}_{\mathrm{CFM}}$ must satisfy:
\begin{enumerate}
\item \textbf{Background:} The Euler-Lagrange equations, evaluated on the FLRW metric, must yield the extended Friedmann equation:
\begin{equation}
H^2(a) = H_0^2 \left[\Omega_b\,a^{-3} + \Phi_0 \cdot f_{\mathrm{sat}}(a) + \alpha \cdot a^{-\beta}\right]
\end{equation}

\item \textbf{Saturation dynamics:} The scalar field equation of motion must reduce to $d\Omega_\Phi/da = k[1 - (\Omega_\Phi/\Phi_0)^2]$ on the FLRW background.

\item \textbf{General covariance:} The action must be diffeomorphism-invariant.

\item \textbf{Correct limits:} In the limit $k \to 0$, $\alpha \to 0$, the theory must reduce to GR with cosmological constant.
\end{enumerate}

\subsection{Scalar Field Approach}

The most natural Lagrangian formulation introduces a scalar field $\phi$ with a potential $V(\phi)$:
\begin{equation}
S = \int d^4x \sqrt{-g} \left[\frac{R}{16\pi G} - \frac{1}{2} g^{\mu\nu}\partial_\mu\phi\,\partial_\nu\phi - V(\phi) + \mathcal{L}_m\right]
\label{eq:action_scalar}
\end{equation}

For the saturation ODE to emerge, we require $V(\phi)$ such that the homogeneous field equation on FLRW yields $\tanh$-type solutions.

\begin{proposition}[Double-Well Saturation Potential]
The potential
\begin{equation}
V(\phi) = V_0 \left[1 - \tanh^2\!\left(\frac{\phi}{\phi_0}\right)\right] = \frac{V_0}{\cosh^2(\phi/\phi_0)}
\label{eq:double_well}
\end{equation}
produces a scalar field equation whose late-time solution on the FLRW background is $\phi(a) \propto \tanh(k(a - a_{\mathrm{trans}}))$, reproducing the saturation term of the CFM.
\end{proposition}

\textit{Sketch of proof:} The Klein-Gordon equation on FLRW,
\begin{equation}
\ddot{\phi} + 3H\dot{\phi} + V'(\phi) = 0
\end{equation}
with $V'(\phi) = -2V_0 \tanh(\phi/\phi_0)/(\phi_0 \cosh^2(\phi/\phi_0))$, admits the solution $\phi = \phi_0 \tanh(\lambda t)$ in the slow-roll regime where $\ddot{\phi} \ll 3H\dot{\phi}$, with $\lambda$ related to $k$ and $H_0$. The energy density $\rho_\phi = \frac{1}{2}\dot{\phi}^2 + V(\phi)$ then maps to $\Omega_\Phi(a) = \Phi_0 \cdot f_{\mathrm{sat}}(a)$. \hfill $\square$

\textit{Note:} The $\cosh^{-2}$ potential is well known in quantum mechanics as the P\"oschl-Teller potential. Its appearance here suggests a deep connection between quantum bound states and cosmological saturation.

\subsection{The Power-Law Term: Geometric Origin}
\label{subsec:powerlaw_lagrangian}

The geometric ``dark matter'' term $\alpha \cdot a^{-\beta}$ with $\beta \approx 2$ requires a separate origin. Two approaches are possible:

\textbf{Approach 1: Curvature-squared terms.} Adding a Gauss-Bonnet or $R^2$ term to the action:
\begin{equation}
S_{\mathrm{geom}} = \int d^4x \sqrt{-g} \left[\frac{R}{16\pi G} + \gamma\, R^2 + \delta\, R_{\mu\nu}R^{\mu\nu}\right]
\end{equation}
produces corrections to the Friedmann equation that scale as $a^{-2}$ in the radiation-to-matter transition era. The coefficient $\gamma$ can be related to $\alpha$.

\textbf{Approach 2: Vector field (AeST-inspired).} Following Skordis \& Z{\l}o\'snik \cite{Skordis2021}, a timelike vector field $A_\mu$ constrained by $g^{\mu\nu}A_\mu A_\nu = -1$ contributes an effective energy density that scales non-standardly with $a$. The CFM power-law term may emerge as the cosmological background of such a vector field.

\subsection{The Combined Action}
\label{subsec:combined_action}

Combining both contributions, the full CFM action reads:
\begin{equation}
\boxed{S_{\mathrm{CFM}} = \int d^4x \sqrt{-g} \left[\frac{R}{16\pi G} + \gamma R^2 - \frac{1}{2}(\partial\phi)^2 - \frac{V_0}{\cosh^2(\phi/\phi_0)} + \mathcal{L}_m\right]}
\label{eq:full_action}
\end{equation}

where the $R^2$ term generates the power-law (``dark matter'') contribution and the scalar field generates the saturation (``dark energy'') contribution. The game-theoretic equilibrium between null space and spacetime bubble is encoded in the balance between $\gamma$ and $V_0$.

\textit{Status:} This is a candidate action. Its consistency (ghost freedom, stability, correct Newtonian limit) must be verified. The full perturbation equations derived from~\eqref{eq:full_action} will determine whether the model can reproduce CMB and LSS observations.


% ===================================================================
% 3. QUANTENGRAVITATION
% ===================================================================
\section{Quantum Gravity Connections}
\label{sec:quantum_gravity}

\subsection{Why the Saturation ODE?}

The central puzzle is the specific form of the saturation ODE~\eqref{eq:saturation_ode}: $dX/da = k(1 - X^2)$. This equation has two fixed points ($X = \pm 1$), of which $X = +1$ is stable. The $\tanh$ solution is the unique trajectory connecting $X = 0$ (zero curvature return) to $X = 1$ (full saturation). We survey four frameworks that naturally produce such dynamics.

\subsection{Approach 1: Loop Quantum Gravity}
\label{subsec:lqg}

In Loop Quantum Gravity (LQG) \cite{Rovelli2004, Thiemann2007}, spacetime is quantized into discrete spin network states. The key feature for our purposes is the \textit{bounded curvature} property: holonomy corrections replace curvature invariants $R$ with bounded functions $\sin(\mu R)/\mu$ (where $\mu$ is related to the Planck area).

In Loop Quantum Cosmology (LQC) \cite{Ashtekar2011}, the Friedmann equation becomes:
\begin{equation}
H^2 = \frac{8\pi G}{3} \rho \left(1 - \frac{\rho}{\rho_c}\right)
\label{eq:lqc_friedmann}
\end{equation}
where $\rho_c \sim \rho_{\mathrm{Pl}}$ is a critical density. This has the structure of a saturation equation: the expansion rate is bounded as $\rho \to \rho_c$.

\begin{conjecture}[LQG--CFM Connection]
The saturation ODE~\eqref{eq:saturation_ode} is the late-time, low-energy residual of the LQC curvature bound. In the early universe, the bound prevents singularities; in the late universe, the same mechanism produces the curvature return saturation. The parameters $k$ and $\Phi_0$ are related to the LQG area gap $\Delta$ and the Barbero-Immirzi parameter $\gamma_{\mathrm{BI}}$.
\end{conjecture}

\textit{Evidence:} Both equations share the structure $dX/dt \propto (1 - X^2)$. In LQC, $X$ is the curvature; in CFM, $X$ is the curvature return potential. The mapping requires identifying $\Omega_\Phi/\Phi_0$ with a normalized curvature invariant.

\subsection{Approach 2: Finsler Geometry}
\label{subsec:finsler}

Finsler geometry generalizes Riemannian geometry by allowing the metric to depend on both position and direction: $F(x, \dot{x})$ instead of $g_{\mu\nu}(x)\,dx^\mu\,dx^\nu$ \cite{Bao2000}. This direction dependence can produce:

\begin{itemize}
\item Scale-dependent gravitational effects (mimicking MOND at galactic scales)
\item Non-standard cosmological scaling (the $a^{-\beta}$ term)
\item A natural saturation mechanism when the directional dependence reaches a geometric bound
\end{itemize}

\begin{conjecture}[Finsler--CFM Connection]
The extended CFM Friedmann equation corresponds to a Finsler spacetime with a specific choice of Finsler function $F$. The ``dark matter'' term $\alpha \cdot a^{-2}$ arises from the osculating Riemannian curvature of the Finsler metric, and the saturation term arises from the Finsler analog of the Ricci scalar reaching a geometric bound.
\end{conjecture}

\textit{Note:} Finsler geometry has been applied to MOND \cite{Chang2009} and to modified dispersion relations in quantum gravity \cite{Girelli2007}. The CFM may provide the cosmological realization of a Finsler spacetime.

\subsection{Approach 3: Information-Theoretic Spacetime}
\label{subsec:information}

If spacetime is fundamentally information-theoretic (as suggested by the holographic principle \cite{Bousso2002} and the ER=EPR conjecture \cite{Maldacena2013}), then the saturation ODE can be reinterpreted as a \textit{maximum entropy principle}:

\begin{itemize}
\item The curvature return potential $\Omega_\Phi$ represents the ``processed information'' of the spacetime system.
\item The saturation limit $\Phi_0$ represents the maximum information capacity (holographic bound).
\item The ODE $dX/da = k(1 - X^2)$ is the logistic-type growth equation for information processing, where the rate of information gain decreases as the system approaches its capacity.
\end{itemize}

In this picture, the game-theoretic interpretation of Paper~I \cite{Geiger2026} becomes literal: the null space and spacetime bubble are two subsystems of a quantum information network, and their Nash equilibrium is determined by the information-theoretic constraints of the holographic bound.

\subsection{Approach 4: Causal Set Theory}
\label{subsec:causal_sets}

Causal set theory \cite{Bombelli1987, Sorkin2003} models spacetime as a discrete partial order of events. The key result for our purposes is the \textit{Sorkin cosmological constant} \cite{Sorkin1991}: in a causal set universe with $N$ elements, the cosmological constant has fluctuations of order $\Lambda \sim 1/\sqrt{N}$, providing a natural explanation for the observed smallness of $\Lambda$.

\begin{conjecture}[Causal Set--CFM Connection]
In a dynamically evolving causal set, the curvature return potential $\Omega_\Phi$ corresponds to the ``effective cosmological constant'' that changes as new elements are added to the set. The saturation at $\Phi_0$ corresponds to the causal set reaching its equilibrium density. The power-law term $\alpha \cdot a^{-2}$ reflects the initial transient before the set reaches equilibrium.
\end{conjecture}


% ===================================================================
% 4. PHASENUEBERGANG
% ===================================================================
\section{The Geometric Phase Transition}
\label{sec:phase_transition}

\subsection{From Dark Matter Phase to Dark Energy Phase}

Paper~II \cite{Geiger2026b} introduced the concept of a geometric phase transition: at early times, the curvature return potential behaves like dark matter ($\alpha \cdot a^{-2}$), and at late times, it saturates into dark energy ($\Phi_0 \cdot f_{\mathrm{sat}}$). This section provides the theoretical underpinning.

\subsection{Order Parameter and Symmetry Breaking}

The saturation variable $X = \Omega_\Phi / \Phi_0 \in [0, 1]$ can be interpreted as an \textit{order parameter}:
\begin{itemize}
\item $X = 0$: Disordered phase (no curvature return, geometric ``DM'' dominates)
\item $X = 1$: Ordered phase (full saturation, geometric ``DE'' dominates)
\item The transition at $a_{\mathrm{trans}}$: The crossover between phases
\end{itemize}

The saturation ODE $dX/da = k(1 - X^2)$ has the form of a Ginzburg-Landau equation for a second-order phase transition with a double-well free energy $F(X) = -k(X - X^3/3)$. The ``temperature'' parameter is the scale factor $a$, and the transition occurs as $a$ increases past $a_{\mathrm{trans}}$.

\subsection{Analogy to Spontaneous Magnetization}

The mathematical structure is identical to the mean-field theory of ferromagnetism:
\begin{center}
\begin{tabular}{lll}
\toprule
\textbf{Ferromagnetism} & \textbf{CFM Cosmology} & \textbf{Variable} \\
\midrule
Magnetization $M$ & Curvature return $\Omega_\Phi$ & Order parameter \\
Temperature $T$ & Scale factor $a$ & Control parameter \\
Curie point $T_c$ & Transition $a_{\mathrm{trans}}$ & Critical point \\
Spin interaction $J$ & Curvature coupling $k$ & Interaction strength \\
Saturation $M_s$ & Saturation $\Phi_0$ & Maximum value \\
$\tanh(J/k_BT)$ & $\tanh(k(a - a_{\mathrm{trans}}))$ & Solution \\
\bottomrule
\end{tabular}
\end{center}

This analogy suggests that the curvature return is driven by \textit{cooperative phenomena}: individual spacetime degrees of freedom (area quanta in LQG, causal set elements, etc.) align collectively, producing a macroscopic saturation effect. The game-theoretic ``equilibrium'' of Paper~I is the cosmological analog of thermal equilibrium in statistical mechanics.

\subsection{Critical Exponents and Universality}

If the analogy to phase transitions is more than formal, the CFM should exhibit \textit{universality}: the saturation exponent and the transition shape should be robust against microscopic details. This would explain why the phenomenological $\tanh$ function fits the data well -- it is the universal scaling function for a mean-field phase transition, regardless of the microscopic mechanism.

\begin{conjecture}[Universality of the Saturation Mechanism]
The $\tanh$ form of the curvature return potential is a \textit{universal} consequence of any microscopic theory with:
\begin{enumerate}
\item A bounded curvature return (saturation limit $\Phi_0$)
\item A cooperative interaction between spacetime degrees of freedom (coupling $k$)
\item A single relevant direction (the scale factor $a$)
\end{enumerate}
The specific microscopic mechanism (LQG, Finsler, causal sets) affects only the values of $k$ and $\Phi_0$, not the functional form.
\end{conjecture}


% ===================================================================
% 5. TESTBARE VORHERSAGEN
% ===================================================================
\section{Testable Predictions from the Lagrangian}
\label{sec:predictions}

The effective action~\eqref{eq:full_action} generates specific predictions beyond the background expansion history:

\subsection{Perturbation Equations}

Linearizing the action around the FLRW background yields coupled equations for:
\begin{itemize}
\item The metric perturbations $\Phi_N$ (Newtonian potential) and $\Psi$ (curvature perturbation)
\item The scalar field perturbation $\delta\phi$
\item The matter perturbations $\delta_m$ and $v_m$
\end{itemize}

The $R^2$ term produces an \textit{anisotropic stress} ($\Phi_N \neq \Psi$), which is a testable prediction distinguishing the CFM from $\Lambda$CDM and from simple quintessence models.

\subsection{Gravitational Slip Parameter}

The ratio $\eta = \Phi_N / \Psi$ is predicted to deviate from unity:
\begin{equation}
\eta(a, k) = 1 + \delta\eta(a, k)
\end{equation}
where $\delta\eta$ depends on the $R^2$ coupling $\gamma$ and is scale-dependent. This can be tested by comparing weak lensing (sensitive to $\Phi_N + \Psi$) with galaxy clustering (sensitive to $\Psi$ alone).

\subsection{Scalar Field Oscillations}

The P\"oschl-Teller potential~\eqref{eq:double_well} supports a discrete spectrum of bound states. In the cosmological context, these correspond to oscillatory corrections to the expansion rate at late times:
\begin{equation}
H^2(a) = H^2_{\mathrm{smooth}}(a) \left[1 + \epsilon \cdot e^{-\Gamma a} \cos(\omega a + \delta)\right]
\end{equation}
with amplitude $\epsilon \ll 1$. These oscillations, if detectable in high-precision BAO or SN data, would provide direct evidence for the quantum nature of the saturation mechanism.

\subsection{Modified Gravitational Waves}

The $R^2$ term modifies the gravitational wave propagation equation:
\begin{equation}
\ddot{h}_{ij} + (3H + \Gamma_{\mathrm{GW}})\dot{h}_{ij} + \left(\frac{k^2}{a^2} + m_{\mathrm{GW}}^2\right) h_{ij} = 0
\end{equation}
where $\Gamma_{\mathrm{GW}}$ and $m_{\mathrm{GW}}^2$ are corrections from the curvature-squared term. This predicts:
\begin{itemize}
\item A frequency-dependent gravitational wave speed ($c_{\mathrm{GW}} \neq c$ at high frequencies)
\item A massive graviton mode with $m_{\mathrm{GW}} \propto \sqrt{\gamma}$
\end{itemize}
The LIGO/Virgo/KAGRA constraint $|c_{\mathrm{GW}}/c - 1| < 10^{-15}$ \cite{Abbott2017} places an upper bound on $\gamma$.


% ===================================================================
% 6. ZUSAMMENHANG MIT BEKANNTEN THEORIEN
% ===================================================================
\section{Connection to Known Frameworks}
\label{sec:connections}

\subsection{Relation to $f(R)$ Gravity}

The action~\eqref{eq:full_action} with the $R^2$ term is a special case of $f(R) = R + \gamma R^2$ gravity (Starobinsky model) \cite{Starobinsky1980}. The CFM adds the scalar field with the P\"oschl-Teller potential, breaking the degeneracy between $f(R)$ models.

\subsection{Relation to AeST}

The relativistic MOND theory AeST \cite{Skordis2021} contains a scalar field $\phi$ and a constrained vector field $A_\mu$. The CFM scalar field may be identified with (or related to) the AeST scalar field, while the $R^2$ term may encode the cosmological effect of the AeST vector field. A precise mapping between the two theories is a key objective.

\subsection{Relation to Emergent Gravity}

Verlinde's emergent gravity proposal \cite{Verlinde2017} derives MOND-like effects from the entanglement entropy of de~Sitter space. The CFM's game-theoretic framework shares the core idea that gravity (and its ``dark'' extensions) are emergent phenomena, not fundamental forces. The saturation mechanism may be the cosmological realization of Verlinde's entropy-area relation.


% ===================================================================
% 7. DISKUSSION UND AUSBLICK
% ===================================================================
\section{Discussion and Outlook}
\label{sec:discussion}

\subsection{Summary of the Three-Paper Program}

The CFM program now spans three papers:
\begin{enumerate}
\item \textbf{Paper~I} \cite{Geiger2026}: Game-theoretic foundation, standard CFM, dark energy replacement. Validated against Pantheon+.
\item \textbf{Paper~II} \cite{Geiger2026b}: MOND unification, extended CFM, baryon-only universe, Decaying Dark Geometry hypothesis. Validated against Pantheon+.
\item \textbf{Paper~III} (this work): Lagrangian formulation, quantum gravity connections, phase transition interpretation, testable predictions.
\end{enumerate}

Together, these papers propose a \textit{complete cosmological framework} in which:
\begin{itemize}
\item The dark sector is eliminated (Paper~II)
\item The expansion history is explained by geometric curvature return (Papers~I, II)
\item The microscopic origin is a $\tanh$-type phase transition of spacetime geometry (Paper~III)
\item The Lagrangian is $R + \gamma R^2$ plus a P\"oschl-Teller scalar field (Paper~III)
\end{itemize}

\subsection{What Remains}

Despite the theoretical progress, critical steps remain:

\begin{enumerate}
\item \textbf{CMB power spectrum:} Computing $C_\ell$ from the perturbation equations of the full action~\eqref{eq:full_action}. This is the single most important test.

\item \textbf{Ghost analysis:} Verifying that the action~\eqref{eq:full_action} is free of ghost instabilities (negative kinetic energy modes). The $R^2$ term introduces the scalaron, which must be checked for stability.

\item \textbf{Solar system tests:} The $R^2$ modification produces a Yukawa correction to Newton's law. The coupling $\gamma$ must be small enough to satisfy solar system constraints.

\item \textbf{Numerical verification:} Solving the full perturbation equations numerically (using a modified CLASS/CAMB code) to predict $C_\ell$, $P(k)$, and $f\sigma_8$.

\item \textbf{Quantum gravity:} Deriving $k$, $\Phi_0$, $\alpha$, and $\beta$ from one of the microscopic frameworks (LQG, Finsler, causal sets) -- or from a new framework suggested by the $\tanh$ structure.
\end{enumerate}

\subsection{The Vision: Cosmology as Phase Transition}

If the program succeeds, the history of the universe becomes a \textit{geometric phase transition}:

\begin{enumerate}
\item \textbf{Big Bang:} Emergence of the spacetime bubble from the null space (game-theoretic nucleation).
\item \textbf{Early universe:} Unsaturated curvature return dominates -- geometry behaves like ``dark matter'' ($a^{-2}$), providing gravitational scaffolding for structure formation.
\item \textbf{Transition:} The curvature return approaches saturation ($a \approx a_{\mathrm{trans}}$) -- the geometric phase transition from DM-like to DE-like behavior.
\item \textbf{Late universe:} Saturated curvature return dominates -- geometry behaves like ``dark energy'' (accelerated expansion).
\item \textbf{Far future:} Full saturation $\Omega_\Phi \to \Phi_0$ -- the Nash equilibrium is reached, the null space gradient is neutralized, and expansion approaches de~Sitter.
\end{enumerate}

The entire history of cosmic acceleration and structure formation is then described by a single equation -- the saturation ODE -- whose form is universal (a consequence of mean-field phase transition theory) and whose parameters are determined by quantum gravity.


% ===================================================================
% LITERATUR
% ===================================================================
\begin{thebibliography}{99}

\bibitem{Geiger2026}
Geiger, L.\ (2026).
Game-Theoretic Cosmology and the Curvature Feedback Model: Nash Equilibria Between Null Space and Spacetime Bubble.
Working Paper. \url{https://github.com/lukisch/cfm-cosmology}.

\bibitem{Geiger2026b}
Geiger, L.\ (2026).
Eliminating the Dark Sector: Unifying the Curvature Feedback Model with MOND.
Working Paper.

\bibitem{Scolnic2022}
Scolnic, D.\ et al.\ (2022).
The Pantheon+ Analysis: The Full Data Set and Light-curve Release.
\textit{The Astrophysical Journal}, 938(2), 113.

\bibitem{Milgrom1983}
Milgrom, M.\ (1983).
A modification of the Newtonian dynamics as a possible alternative to the hidden mass hypothesis.
\textit{The Astrophysical Journal}, 270, 365--370.

\bibitem{Skordis2021}
Skordis, C.\ \& Z{\l}o\'snik, T.\ (2021).
New Relativistic Theory for Modified Newtonian Dynamics.
\textit{Physical Review Letters}, 127(16), 161302.

\bibitem{Rovelli2004}
Rovelli, C.\ (2004).
\textit{Quantum Gravity}. Cambridge University Press.

\bibitem{Thiemann2007}
Thiemann, T.\ (2007).
\textit{Modern Canonical Quantum General Relativity}. Cambridge University Press.

\bibitem{Ashtekar2011}
Ashtekar, A.\ \& Singh, P.\ (2011).
Loop Quantum Cosmology: A Status Report.
\textit{Classical and Quantum Gravity}, 28(21), 213001.

\bibitem{Bao2000}
Bao, D., Chern, S.-S.\ \& Shen, Z.\ (2000).
\textit{An Introduction to Riemann-Finsler Geometry}. Springer.

\bibitem{Chang2009}
Chang, Z.\ \& Li, X.\ (2009).
Modified Friedmann model in Randers-Finsler space of approximate Berwald type.
\textit{Physics Letters B}, 676(4-5), 173--176.

\bibitem{Girelli2007}
Girelli, F., Liberati, S.\ \& Sindoni, L.\ (2007).
Planck-scale modified dispersion relations and Finsler geometry.
\textit{Physical Review D}, 75(6), 064015.

\bibitem{Bousso2002}
Bousso, R.\ (2002).
The holographic principle.
\textit{Reviews of Modern Physics}, 74(3), 825--874.

\bibitem{Maldacena2013}
Maldacena, J.\ \& Susskind, L.\ (2013).
Cool horizons for entangled black holes.
\textit{Fortschritte der Physik}, 61(9), 781--811.

\bibitem{Bombelli1987}
Bombelli, L., Lee, J., Meyer, D.\ \& Sorkin, R.\,D.\ (1987).
Space-time as a causal set.
\textit{Physical Review Letters}, 59(5), 521--524.

\bibitem{Sorkin2003}
Sorkin, R.\,D.\ (2003).
Causal Sets: Discrete Gravity.
In \textit{Lectures on Quantum Gravity}, Springer, 305--327.

\bibitem{Sorkin1991}
Sorkin, R.\,D.\ (1991).
Spacetime and causal sets.
In \textit{Relativity and Gravitation}, World Scientific, 150--173.

\bibitem{Starobinsky1980}
Starobinsky, A.\,A.\ (1980).
A new type of isotropic cosmological models without singularity.
\textit{Physics Letters B}, 91(1), 99--102.

\bibitem{Verlinde2017}
Verlinde, E.\ (2017).
Emergent Gravity and the Dark Universe.
\textit{SciPost Physics}, 2(3), 016.

\bibitem{Abbott2017}
Abbott, B.\,P.\ et al.\ (LIGO/Virgo \& Fermi GBM) (2017).
Gravitational Waves and Gamma-Rays from a Binary Neutron Star Merger: GW170817 and GRB~170817A.
\textit{The Astrophysical Journal Letters}, 848(2), L13.

\end{thebibliography}

\end{document}

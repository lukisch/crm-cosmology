\documentclass[11pt,a4paper]{article}
\usepackage[utf8]{inputenc}
\usepackage[T1]{fontenc}
\usepackage[english]{babel}
\usepackage{geometry}
\geometry{a4paper, left=2.5cm, right=2.5cm, top=2.5cm, bottom=2.5cm}
\usepackage{mathptmx}
\usepackage{helvet}
\usepackage{amsmath}
\usepackage{amssymb}
\usepackage{amsthm}
\usepackage{titlesec}
\usepackage{booktabs}
\usepackage{tabularx}
\usepackage{xcolor}
\usepackage{authblk}
\usepackage{hyperref}
\usepackage{enumitem}
\usepackage{graphicx}
\usepackage{float}
\usepackage{setspace}
\usepackage{longtable}
\usepackage{multirow}
\usepackage{array}

\newtheorem{definition}{Definition}
\newtheorem{proposition}{Proposition}

\titleformat{\section}{\Large\bfseries\sffamily\color{black}}{\thesection}{1em}{}
\titleformat{\subsection}{\large\bfseries\sffamily\color{darkgray}}{\thesubsection}{1em}{}
\titleformat{\subsubsection}{\normalsize\bfseries\sffamily\color{darkgray}}{\thesubsubsection}{1em}{}

\hypersetup{
    pdftitle={Game-Theoretic Cosmology and the Curvature Feedback Model},
    pdfauthor={Lukas Geiger},
    colorlinks=true,
    linkcolor=black,
    urlcolor=blue,
    citecolor=black
}

\onehalfspacing

\begin{document}

% ===================================================================
% TITLE PAGE
% ===================================================================

\title{\textbf{\huge Game-Theoretic Cosmology and the Curvature Feedback Model}\\[0.5em]
\Large Nash Equilibria Between Null Space and Spacetime Bubble\\as an Explanatory Framework for Accelerated Expansion\\[0.3em]
\large An Integrative Theoretical Approach}

\author[1]{Lukas Geiger\thanks{Correspondence: Lukas Geiger, Gei\ss{}b\"uhlweg~1, 79872~Bernau, Germany.}}
\affil[1]{Independent Researcher, Bernau im Schwarzwald, Germany}

\date{February 2026 \\ \vspace{0.5em} \small \textit{Paper~I in the CFM series}}

\maketitle

\begin{abstract}
\noindent
This paper develops a game-theoretic framework for cosmology in which the emergence and evolution of spacetime is modeled as a Nash equilibrium between two agents: a metastable quantum vacuum (null space) and a spacetime bubble arising from it. The central result is the \textit{Curvature Feedback Model} (CFM), which explains the observed accelerated expansion of the universe not through a new form of energy (dark energy) but through a diminishing curvature return potential~$\Phi(a)$ -- a geometric ``memory'' of the initial energy concentration at the Big Bang. The modified Friedmann equation $H^2(a) = H_0^2\,[\Omega_m\,a^{-3} + \Omega_\Phi(a)]$ with $\Omega_\Phi(a) = \Phi_0 \cdot \tanh(k\cdot(a - a_{\mathrm{trans}}))$ is tested against 1,590 Type~Ia supernovae ($z > 0.01$) from the Pantheon+ catalog \cite{Scolnic2022} -- both with diagonal errors and with the full statistical-systematic covariance matrix. Under a flatness constraint ($\Omega_m + \Omega_\Phi(a{=}1) = 1$), the CFM yields $\Delta\chi^2 = -12.2$ ($-11.2$ with full covariance) and $\Delta\mathrm{AIC} = -8.2$ ($-7.2$) relative to $\Lambda$CDM; 5-fold cross-validation confirms better generalization. MCMC posterior analysis yields $\Omega_m = 0.368 \pm 0.024$, and four alternative functional forms (logistic, error function, power law) show comparable $\Delta\chi^2$ values, confirming the robustness of results. A phantom stability analysis shows: no Big Rip (saturation of $\Omega_\Phi$), asymptotically de~Sitter end state. The model predicts a measurable time variation of the equation-of-state parameter ($w(z) < -1$, $|\Delta w| \approx 0.4$) and an earlier acceleration transition ($z_{\mathrm{acc}} = 0.52$ vs.\ $0.84$), testable with Euclid and the Nancy Grace Roman Space Telescope within the next decade. It is shown that the CFM exhibits conceptual connections to both Finsler gravity (Pfeifer et al., 2025) and the recently proposed \textit{Cosmological Teleodynamics} (Trivedi \& Venkatasubramanian, 2025), which likewise describes cosmic expansion as convergence toward a Nash equilibrium. Analysis code and data are publicly available.\footnote{\url{https://github.com/lukisch/cfm-cosmology}} In particular, the rigorous analysis identifies with $z_{\mathrm{acc}} \approx 0.52$ a later onset of cosmic acceleration than $\Lambda$CDM ($z_{\mathrm{acc}} = 0.84$), implying an extended matter-dominated growth phase. This prediction offers a natural explanation for the JWST ``Universe Breakers'' -- unexpectedly massive galaxies at $z > 7$ \cite{Labbe2023, BoylanKolchin2023} -- as well as for statistically improbable massive galaxy clusters such as El~Gordo at $z \approx 0.87$ \cite{Asencio2023}, which place the $\Lambda$CDM standard model under significant tension. The game-theoretic perspective opens a paradigm shift: from ``What drives the acceleration?'' to ``Why did the expansion brake earlier?''

\vspace{0.5em}
\noindent\textbf{Keywords:} game theory, Nash equilibrium, cosmology, dark energy, curvature return potential, Curvature Feedback Model, Friedmann equation, Finsler gravity, accelerated expansion, equation of state

\vspace{0.5em}
\noindent\textbf{Disciplines:} theoretical physics, cosmology, game theory, mathematical physics
\end{abstract}

\newpage
\tableofcontents
\newpage

\footnotetext{AI tools (Claude, Anthropic; Gemini, Google DeepMind) were used for mathematical formalization, code development, and text generation. All physical hypotheses, scientific interpretation, and responsibility for the content lie solely with the author. The analysis code is available at \url{https://github.com/lukisch/cfm-cosmology}.}

\newpage

% ===================================================================
% 1. INTRODUCTION
% ===================================================================
\section{Introduction}
\label{sec:introduction}

The discovery of the accelerated expansion of the universe through observations of distant Type~Ia supernovae in 1998 by the teams of Perlmutter \cite{Perlmutter1999} and Riess and Schmidt \cite{Riess1998} marks a turning point in modern cosmology. The 2011 Nobel Prize in Physics was awarded for this discovery. The standard model of cosmology, $\Lambda$CDM, explains the acceleration through a cosmological constant~$\Lambda$ comprising approximately 68\% of the energy density of the universe \cite{Planck2020}. Despite its empirical success, $\Lambda$CDM faces profound conceptual problems:

\begin{enumerate}
\item \textbf{The Cosmological Constant Problem:} The observed vacuum energy density is $\sim$60--120 orders of magnitude smaller than theoretical predictions from quantum field theory \cite{Weinberg1989}.
\item \textbf{The Coincidence Problem:} Why are $\Omega_m$ and $\Omega_\Lambda$ of comparable magnitude precisely in the present epoch?
\item \textbf{The $H_0$ Tension:} The local measurement of the Hubble parameter ($H_0 \approx 73$\,km/s/Mpc) differs significantly from the CMB-derived value ($H_0 \approx 67.4$\,km/s/Mpc) \cite{Planck2020, Riess2022}.
\end{enumerate}

Recent results from the \textit{Dark Energy Spectroscopic Instrument} (DESI) strengthen doubts about a strictly constant dark energy: the analysis of baryon acoustic oscillations combined with CMB and supernova data shows a 2.5--3.9$\sigma$ preference for a model with time-dependent equation-of-state parameter $w(z)$ over $\Lambda$CDM \cite{DESI2024}.

In parallel, theoretical works show that the acceleration could also be explained without dark energy: Pfeifer et al.\ \cite{Pfeifer2025} demonstrate within the framework of Finsler gravity that a generalized spacetime geometry naturally produces exponential expansion in vacuum. Trivedi and Venkatasubramanian \cite{Trivedi2025} show in their \textit{Cosmological Teleodynamics} that the universe operates like a ``giant potential game'' converging toward a continuous Nash equilibrium, where cosmic acceleration emerges as an emergent effect of dynamic memory in a self-gravitating medium.

This paper connects these developments with an independent approach: starting from a game-theoretic model of the interaction between quantum vacuum and spacetime, the \textit{Curvature Feedback Model} (CFM) is developed, which interprets accelerated expansion as a ``releasing brake'' rather than a ``new drive.''


% ===================================================================
% 2. GAME-THEORETIC FRAMEWORK
% ===================================================================
\section{Game-Theoretic Framework: Null Space and Spacetime Bubble}
\label{sec:gametheory}

\subsection{Fundamental Assumptions}
\label{subsec:assumptions}

The proposed framework rests on the following assumptions:

\begin{enumerate}
\item There exists a metastable quantum vacuum state (hereafter: \textit{null space}) characterized by quantum fluctuations.
\item An extraordinarily large fluctuation extracts a one-time energy amount~$E_0$ from the null space, creating a concentration gradient.
\item To encapsulate and controllably neutralize this gradient, spacetime emerges as a dynamic structure -- the \textit{spacetime bubble} (daughter system).
\item A game-theoretic equilibrium exists between the null space (parent system) and the spacetime bubble.
\end{enumerate}

These assumptions are formalized in the following sections.

\subsection{Agents and Objectives}
\label{subsec:agents}

The system is modeled as a two-player potential game:

\begin{description}
\item[\textbf{Null Space (Parent System):}] The primary objective is self-protection -- preservation of its structural integrity. It regulates the coupling strength to the spacetime bubble via effective boundary conditions (``gatekeeping''), enforces slow energy dissipation (damping), and forms buffer zones (horizon-like shells).
\item[\textbf{Spacetime Bubble (Daughter System):}] The primary objective is the controlled return to the null state while simultaneously protecting the parent system. Strategies include cascaded gradient reduction, adiabatic return, and entropy management.
\end{description}

\subsection{Mathematical Formulation as a Potential Game}
\label{subsec:potentialgame}

The global objective function of the system reads:
\begin{equation}
\Phi = \alpha \cdot S_{\mathrm{parent}} + \beta \cdot R_{\mathrm{daughter}} - \gamma \cdot G
\label{eq:potential}
\end{equation}
where $S_{\mathrm{parent}}$ denotes the structural integrity of the null space, $R_{\mathrm{daughter}}$ the return progress, and $G$ the remaining concentration gradient; $\alpha, \beta, \gamma > 0$.

\begin{definition}[Nash Equilibrium of the Cosmological Game]
A strategy pair $(s_P^*, s_D^*)$ of null space and spacetime bubble constitutes a Nash equilibrium if:
\begin{align}
\Phi(s_P^*, s_D^*) &\geq \Phi(s_P, s_D^*) \quad \forall\, s_P \\
\Phi(s_P^*, s_D^*) &\geq \Phi(s_P^*, s_D) \quad \forall\, s_D
\end{align}
Neither side can improve the overall potential by unilaterally deviating from its strategy without endangering the stability of the system.
\end{definition}

The central \textbf{conflict} is that excessively rapid reduction of~$G$ (immediate return) endangers $S_{\mathrm{parent}}$, while excessively slow reduction increases entropy and costs within the bubble. The Nash equilibrium therefore enforces a controlled, temporally extended neutralization.


\subsection{Emergent Laws from the Equilibrium}
\label{subsec:emergentlaws}

From the game-theoretic equilibrium condition, physical regularities emerge:

\begin{enumerate}
\item \textbf{Energy conservation:} Conservative field equations arise as a necessary condition for stable gradient reduction.
\item \textbf{Causal structure:} Shell formation by the null space enforces a maximum propagation speed for information and energy.
\item \textbf{Entropic arrow of time:} ``Time'' within the bubble is the order along which the concentration gradient is leveled.
\item \textbf{Flux limitation:} Maximum fluxes across the shell scale sublinearly with the internal excess, preventing runaway processes.
\item \textbf{Asymptotic return:} The residual gradient $G \to 0$ approaches zero only asymptotically; there is no catastrophic finale.
\end{enumerate}

The last property is particularly significant for cosmology: it implies that the expansion of the universe never reverses but proceeds asymptotically -- consistent with observational data.


% ===================================================================
% 3. THE CURVATURE FEEDBACK MODEL
% ===================================================================
\section{The Curvature Feedback Model (CFM)}
\label{sec:cfm}

\subsection{Physical Motivation}
\label{subsec:motivation}

In the game-theoretic framework of the preceding section, spacetime is interpreted as a ``braking mechanism'' that prevents the immediate return of energy to the null space. The central physical insight is:

\begin{quote}
\textit{The observed accelerated expansion is not caused by a new form of energy but by a diminishing curvature return potential -- a kind of geometric ``memory'' of the initial energy concentration at the Big Bang.}
\end{quote}

The analogy is that of a stretched spring: initially, maximum tension (high curvature) produces strong restoring force. Over time, the tension relaxes, the restoring force decreases, and the expansion ``accelerates'' relative to the braked early phase -- like a car whose handbrake is slowly released.


\subsection{Modified Friedmann Equation}
\label{subsec:friedmann}

The standard Friedmann equation in the $\Lambda$CDM model reads:
\begin{equation}
H^2(a) = H_0^2 \left[\Omega_m\,a^{-3} + \Omega_\Lambda\right]
\label{eq:friedmann_lcdm}
\end{equation}

In the CFM, the cosmological constant is replaced by a time-dependent curvature return potential:
\begin{equation}
H^2(a) = H_0^2 \left[\Omega_m\,a^{-3} + \Omega_\Phi(a)\right]
\label{eq:friedmann_cfm}
\end{equation}

The curvature return potential is defined as:
\begin{equation}
\Omega_\Phi(a) = \Phi_0 \cdot \frac{\tanh\!\big(k\cdot(a - a_{\mathrm{trans}})\big) + s}{1 + s}
\label{eq:potential_cfm}
\end{equation}
where $s = \tanh(k \cdot a_{\mathrm{trans}})$ is a normalization shift ensuring $\Omega_\Phi(0) = 0$, and:
\begin{itemize}
\item $a$ is the scale factor ($a=1$ today, $a \to 0$ at the Big Bang),
\item $\Phi_0$ is the amplitude (derived from the flatness constraint $\Omega_m + \Omega_\Phi(1) = 1$),
\item $k$ is the transition sharpness,
\item $a_{\mathrm{trans}}$ is the transition scale factor.
\end{itemize}
The specific parameter values are determined from the fit to the Pantheon+ data set in Section~\ref{sec:numerics}.

\subsection{Dynamic Saturation Mechanism}
\label{subsec:saturation}

The $\tanh$ parameterization is not an \textit{ad hoc} chosen fitting function but arises as the exact solution of a physically motivated \textbf{dynamic saturation mechanism}. The central assumption is: the spacetime bubble possesses a finite absorption capacity for the curvature return. The return rate is proportional to the remaining capacity:
\begin{equation}
\frac{d\Omega_\Phi}{da} = k \cdot \left[1 - \left(\frac{\Omega_\Phi}{\Phi_0}\right)^{\!2}\right]
\label{eq:saturation_ode}
\end{equation}
This equation describes a classical saturation process from dynamical systems theory: at small $\Omega_\Phi$, the potential grows nearly linearly (the ``brake'' releases at full rate); at $\Omega_\Phi \to \Phi_0$, saturation occurs (maximum capacity is reached, the brake is fully released). The exact solution of Eq.~\eqref{eq:saturation_ode} is:
\begin{equation}
\Omega_\Phi(a) = \Phi_0 \cdot \tanh\!\big(k\cdot(a - a_{\mathrm{trans}})\big)
\end{equation}
where $a_{\mathrm{trans}}$ represents the integration constant (transition point). The saturation mechanism is ubiquitous in physics and appears in formally identical form in numerous systems:
\begin{itemize}
\item Ferromagnetism: spontaneous magnetization $M(T) \sim \tanh(T_C/T)$
\item BCS superconductivity: energy gap $\Delta(T) \sim \tanh(T_C/T)$
\item Soliton physics: kink solution $\phi(x) = \phi_0 \tanh(kx)$
\item Nonlinear optics: saturation absorption $\alpha(I) \propto 1/(1 + I/I_{\mathrm{sat}})$
\end{itemize}
All these systems share the property of an ordered transition from one state to another with finite capacity -- \textit{exactly} the behavior that the game-theoretic framework predicts for the Nash equilibrium between null space and spacetime bubble. The $\tanh$ form is thus not postulated but \textit{derived} from the underlying mechanism.

To verify robustness, four different saturation functions were tested (Section~\ref{subsec:funcforms}). All yield $\Delta\chi^2 \approx -9$ to $-12$ relative to $\Lambda$CDM -- the data ``see'' a saturation process, independent of the exact mathematical formulation.

\subsection{Physical Interpretation of the Parameters}
\label{subsec:interpretation}

\textbf{Early times} ($a \to 0$, $z \to \infty$): $\Omega_\Phi \to 0$. The ``brake'' operates at full strength -- the expansion follows matter dominance as in $\Lambda$CDM. There is no dark component.

\textbf{Transition epoch} ($a \approx a_{\mathrm{trans}}$, $z \approx 1.5$): $\Omega_\Phi$ increases. The ``brake'' begins to release. This occurred approximately 10.3 billion years ago.

\textbf{Today} ($a = 1$, $z = 0$): $\Omega_\Phi \to \Phi_0$. The maximum effect is reached; the potential effectively acts like~$\Lambda$.


\subsection{Effective Equation-of-State Parameter}
\label{subsec:weff}

The effective equation-of-state parameter of the curvature return potential is:
\begin{equation}
w_{\mathrm{eff}}(a) = -1 - \frac{1}{3}\,\frac{d\ln\Omega_\Phi}{d\ln a}
\label{eq:weff}
\end{equation}

Its time evolution is presented in Table~\ref{tab:weff}.

\begin{table}[H]
\centering
\caption{Time evolution of the effective equation-of-state parameter $w_{\mathrm{eff}}(z)$: $\Lambda$CDM vs.\ CFM. The $1\sigma$ uncertainties are from the MCMC posterior analysis (Section~\ref{subsec:mcmc}).}
\label{tab:weff}
\begin{tabular}{ccccc}
\toprule
$z$ & $w$ ($\Lambda$CDM) & $w$ (CFM) & $1\sigma$ range & $\Delta w$ \\
\midrule
0.0 & $-1.000$ & $-1.355$ & $[-1.371;\;-1.339]$ & $\mathbf{-0.355}$ \\
0.3 & $-1.000$ & $-1.433$ & $[-1.645;\;-1.355]$ & $\mathbf{-0.433}$ \\
0.5 & $-1.000$ & $-1.450$ & $[-1.730;\;-1.358]$ & $\mathbf{-0.450}$ \\
0.8 & $-1.000$ & $-1.456$ & $[-1.759;\;-1.359]$ & $\mathbf{-0.456}$ \\
1.0 & $-1.000$ & $-1.454$ & $[-1.749;\;-1.359]$ & $\mathbf{-0.454}$ \\
1.5 & $-1.000$ & $-1.444$ & $[-1.696;\;-1.357]$ & $\mathbf{-0.444}$ \\
2.0 & $-1.000$ & $-1.432$ & $[-1.644;\;-1.355]$ & $\mathbf{-0.432}$ \\
\bottomrule
\end{tabular}
\end{table}

The CFM parameters from the Pantheon+ fit yield consistently $w < -1$ (phantom regime). The MCMC-based $1\sigma$ confidence intervals show that $w = -1$ is excluded for all redshifts. This differs qualitatively from $\Lambda$CDM ($w \equiv -1$) and constitutes a clear, falsifiable prediction. The effect is present across the entire observable redshift range ($|\Delta w| \approx 0.4$) and thus well within the expected measurement precision of Euclid ($\sigma_w \approx 0.02$).


% ===================================================================
% 4. NUMERICAL TESTS
% ===================================================================
\section{Numerical Tests and Model Comparison}
\label{sec:numerics}


\subsection{Flatness Constraint}
\label{subsec:flatness}

To reduce the number of free parameters and ensure physical consistency, the flatness constraint
\begin{equation}
\Omega_m + \Omega_\Phi(a{=}1) = 1
\label{eq:flatness}
\end{equation}
is imposed. This yields for the amplitude:
\begin{equation}
\Phi_0 = \frac{(1 - \Omega_m)(1 + s)}{\tanh\!\big(k\cdot(1 - a_{\mathrm{trans}})\big) + s}
\end{equation}
The CFM thus has three cosmological degrees of freedom ($\Omega_m$, $k$, $a_{\mathrm{trans}}$) plus one nuisance parameter ($M$), totaling four effective parameters -- only two more than $\Lambda$CDM.

\subsection{Data: Pantheon+}
\label{subsec:pantheonplus}

The test is performed against the Pantheon+ data set \cite{Scolnic2022}, the largest publicly available catalog of spectroscopically confirmed Type~Ia supernovae. From the 1,701 light curves, 1,590 supernovae with $z > 0.01$ are used (to avoid peculiar velocity dominance), spanning the redshift range $z = 0.0102$ to $z = 2.2614$. The observable is the bias-corrected apparent B-band magnitude \texttt{m\_b\_corr}. The analysis is performed both with diagonal errors and with the full statistical-systematic covariance matrix (STAT+SYS) of the Pantheon+ data set.

\subsection{Methodology}
\label{subsec:methodology}

\textbf{Distance computation:} The luminosity distance is computed via cumulative trapezoidal integration on a fine $z$-grid ($N = 2{,}000$ support points) and interpolated onto the data redshifts. This procedure is numerically stable (error $< 10^{-5}$) and enables fast evaluation during optimization.

\textbf{Nuisance parameter:} The absolute magnitude offset $M = M_B + 5\log_{10}(c/H_0) + 25$, which absorbs the absolute magnitude and the Hubble constant, is analytically marginalized:
\begin{equation}
M_{\mathrm{best}} = \frac{\sum_i w_i (m_i^{\mathrm{obs}} - \mu_i^{\mathrm{th}})}{\sum_i w_i}, \quad w_i = \sigma_i^{-2}
\end{equation}

\textbf{Optimization:} Parameter determination via \textit{Differential Evolution} (global evolutionary optimizer) with subsequent L-BFGS-B refinement (\textit{polish}).

\textbf{MCMC uncertainties:} For the flat CFM, parameter uncertainties are determined using \textit{emcee} \cite{ForemanMackey2013} (32~walkers, 3,000~steps, 500~burn-in). The posterior distributions yield $1\sigma$ confidence intervals for all parameters including derived quantities $\Phi_0$ and $z_{\mathrm{trans}}$.

\textbf{Model selection:} In addition to $\chi^2$, the Akaike Information Criterion (AIC~$= \chi^2 + 2k$) and the Bayesian Information Criterion (BIC~$= \chi^2 + k \ln n$) are computed, where $k$ is the number of effective parameters and $n$ the number of data points. To check for overfitting, a 5-fold cross-validation is additionally performed. The complete analysis code is publicly available.\footnote{\url{https://github.com/lukisch/cfm-cosmology}}

\subsection{Results}
\label{subsec:results}

Three models are fitted: flat $\Lambda$CDM (2~parameters), CFM with flatness constraint (4~parameters), and CFM without constraint (5~parameters).

\begin{table}[H]
\centering
\caption{Fitted parameters and goodness of fit: $\Lambda$CDM vs.\ CFM against Pantheon+ (1,590~SNe~Ia). For CFM~(flat), $1\sigma$ MCMC uncertainties are given.}
\label{tab:results}
\begin{tabular}{lccc}
\toprule
 & $\Lambda$CDM & CFM (flat) & CFM (free) \\
\midrule
Free parameters $k$ & 2 & 4 & 5 \\
$\Omega_m$ & 0.244 & $0.368^{+0.025}_{-0.023}$ & 0.552 \\
$\Omega_\Lambda$ / $\Omega_\Phi(z{=}0)$ & 0.756 & 0.636 & 0.872 \\
$\Phi_0$ (derived) & -- & $0.988^{+0.615}_{-0.221}$ & 1.292 \\
$k$ (transition sharpness) & -- & $1.44^{+1.22}_{-0.84}$ & 1.98 \\
$a_{\mathrm{trans}}$ ($z_{\mathrm{trans}}$) & -- & 0.75 (0.33) & 0.80 (0.25) \\
$\Omega_{\mathrm{total}}$ & 1.000 & 1.000 & 1.423 \\
\midrule
$\chi^2$ (diagonal) & 729.0 & 716.8 & 715.9 \\
$\chi^2$ (full cov.) & 1432.0 & 1420.8 & -- \\
$\chi^2/\mathrm{dof}$ & 0.459 & 0.452 & 0.452 \\
AIC (diagonal) & 733.0 & 724.8 & 725.9 \\
AIC (full cov.) & 1436.0 & 1428.8 & -- \\
BIC & 743.7 & 746.3 & 752.8 \\
\bottomrule
\end{tabular}
\end{table}

The CFM with flatness constraint yields $\Omega_m = 0.368 \pm 0.024$ (MCMC) -- physically plausible and close to the Planck value ($0.315 \pm 0.007$). The fitted transition redshift $z_{\mathrm{trans}} = 0.33$ ($a_{\mathrm{trans}} = 0.75$) lies at later cosmic times than theoretically expected. The transition sharpness $k = 1.44^{+1.22}_{-0.84}$ describes a smooth transition with a broad posterior -- the data prefer a transition but allow a range of transition sharpnesses.

\textbf{Full covariance matrix:} Repeating the analysis with the full statistical-systematic covariance matrix confirms the results: $\Delta\chi^2 = -11.2$ and $\Delta\mathrm{AIC} = -7.2$ (compared to $-12.2$ and $-8.2$ with diagonal errors). The slight reduction is explained by the inclusion of systematic correlations between neighboring supernovae.

\subsection{Model Selection}
\label{subsec:modelselection}

\begin{table}[H]
\centering
\caption{Model comparison: CFM vs.\ $\Lambda$CDM. Negative values favor the CFM.}
\label{tab:comparison}
\begin{tabular}{lcc}
\toprule
\textbf{Criterion} & CFM (flat) vs.\ $\Lambda$CDM & CFM (free) vs.\ $\Lambda$CDM \\
\midrule
$\Delta\chi^2$ & $\mathbf{-12.2}$ & $-13.1$ \\
$\Delta$AIC & $\mathbf{-8.2}$ & $-7.1$ \\
$\Delta$BIC & $+2.6$ & $+9.0$ \\
\midrule
5-fold $\langle\chi^2/n\rangle$ & $\mathbf{0.4499}$ & $0.4498$ \\
$\Lambda$CDM: $\langle\chi^2/n\rangle$ & \multicolumn{2}{c}{$0.4519$} \\
\bottomrule
\end{tabular}
\end{table}

\textbf{Interpretation:} Three of four selection criteria favor the flat CFM over $\Lambda$CDM: $\chi^2$ ($-12.2$), AIC ($-8.2$), and cross-validation ($0.4499$ vs.\ $0.4519$). Only the BIC, which more strongly penalizes additional parameters, shows a marginal preference for $\Lambda$CDM ($\Delta\mathrm{BIC} = +2.6$). According to the Kass--Raftery scale \cite{KassRaftery1995}, this value lies at the boundary of significance ($|\Delta\mathrm{BIC}| < 2$: not significant; $2$--$6$: positive evidence). The cross-validation -- the most robust method for overfitting detection -- shows that the CFM generalizes better to unseen data than $\Lambda$CDM.


\subsection{MCMC Posterior Analysis}
\label{subsec:mcmc}

Parameter uncertainties are determined using \textit{emcee} \cite{ForemanMackey2013} (32~walkers, 3,000~steps, 500~burn-in, acceptance rate: 63\%). The results for the flat CFM are given in Table~\ref{tab:results} as $1\sigma$ uncertainties. The posterior distribution of $\Omega_m$ is nearly Gaussian with $\Omega_m = 0.368^{+0.025}_{-0.023}$. The transition sharpness $k$ shows a broad, asymmetric posterior ($k = 1.44^{+1.22}_{-0.84}$), meaning the data prefer a transition but constrain the sharpness less strongly. The derived quantities $\Phi_0 = 0.988^{+0.615}_{-0.221}$ and $z_{\mathrm{trans}} = 0.35$ are consistent with the point estimates. The MCMC-computed $w(z)$ confidence bands (Table~\ref{tab:weff}) show that $w = -1$ lies outside the $1\sigma$ range for all redshifts.

\textbf{Interpretation of $\Omega_m = 0.368$:} The CFM value exceeds the Planck CMB value ($\Omega_m^{\mathrm{Planck}} = 0.315 \pm 0.007$), which is typical for pure supernova fits. In the CFM context, this deviation has a physical interpretation: the model reinterprets part of the density classified as ``dark energy'' in $\Lambda$CDM as a dynamic curvature effect. Since $\Omega_\Phi(a)$ vanishes at early times (unlike $\Omega_\Lambda = \mathrm{const.}$), $\Omega_m$ must be compensatorily higher to maintain the overall fit. This preference for higher $\Omega_m$ is consistent with recent findings from weak gravitational lensing surveys (KiDS, DES), which also favor higher $\Omega_m$ values than Planck.


\subsection{Robustness: Alternative Functional Forms}
\label{subsec:funcforms}

To verify whether the results depend on the specific choice of the $\tanh$ form, four different saturation functions are tested under identical conditions:

\begin{table}[H]
\centering
\caption{Comparison of alternative functional forms for $\Omega_\Phi(a)$. All models use the flatness constraint and have $k=4$ parameters.}
\label{tab:funcforms}
\begin{tabular}{lcccc}
\toprule
\textbf{Functional form} & $\chi^2$ & AIC & $\Delta\chi^2$ vs.\ $\Lambda$CDM & $\Omega_m$ \\
\midrule
$\tanh$ (standard CFM) & 716.8 & 724.8 & $-12.2$ & 0.364 \\
Logistic function & 717.5 & 725.5 & $-11.5$ & 0.368 \\
Error function (erf) & 716.7 & 724.7 & $-12.3$ & 0.367 \\
Power law & 720.1 & 728.1 & $\phantom{-0}8.9$ & 0.364 \\
\bottomrule
\end{tabular}
\end{table}

\textbf{Result:} All four functional forms yield $\Delta\chi^2 \approx -9$ to $-12$ relative to $\Lambda$CDM. The $\tanh$ form is neither the only possible nor the ``best-fitting'' -- it represents a robust class of saturation functions. This refutes the objection that the CFM results depend on a specific functional choice. The $\Omega_m$ values converge at $\approx 0.36$--$0.37$ for all forms, underscoring the physical consistency.


\subsection{Phantom Stability Analysis}
\label{subsec:phantom}

The fitted equation-of-state parameter $w < -1$ (phantom regime) raises the legitimate question of stability. In ordinary phantom scalar field models, $w < -1$ leads to a growing energy density ($\rho \propto a^{-3(1+w)} \to \infty$ for $w < -1$, $a \to \infty$), which inevitably produces the ``Big Rip'' in finite time -- the destruction of all bound structures in the universe \cite{Caldwell1998}. \textbf{The CFM fundamentally circumvents this pathology} for three reasons:

\begin{enumerate}
\item \textbf{Saturation instead of divergence:} The dynamic saturation mechanism (Eq.~\ref{eq:saturation_ode}) guarantees $\Omega_\Phi(a) \to \Phi_0$ for $a \to \infty$. The effective energy density remains \textit{finite and bounded for all times}. This is the decisive difference from phantom scalar fields: whereas there $\rho$ diverges, $\Omega_\Phi$ saturates at its maximum value~$\Phi_0$.
\item \textbf{De Sitter end state:} $w_{\mathrm{eff}} \to -1$ for $a \to \infty$. The universe asymptotically approaches \textit{exactly the same end state as $\Lambda$CDM} -- a stable de~Sitter space. The phantom regime is a transient phenomenon, not an end state.
\item \textbf{No Big Rip:} Since $\Omega_\Phi$ saturates, neither the energy density nor the scale factor diverges in finite time. The Big Rip is excluded -- not by an additional assumption but as a \textit{direct consequence} of the saturation mechanism.
\item \textbf{Geometric rather than fluid interpretation:} The formal violation of the null energy condition ($\rho + p \geq 0$) is unproblematic because $\Omega_\Phi$ represents \textit{not a physical field} but a geometric property of spacetime. The energy conditions of general relativity apply to the energy-momentum tensor of physical fields, not to effective geometric terms. Analogous situations are well established in the literature: $f(R)$ gravity theories routinely exhibit effective $w < -1$ without generating physical instabilities \cite{Sotiriou2010}.
\end{enumerate}

\begin{quote}
\textit{In summary: the CFM exhibits ``phantom behavior'' without phantom pathologies. The universe does not end in a rip but in equilibrium.}
\end{quote}


\subsection{Deceleration Parameter and $H_0$ Implications}
\label{subsec:deceleration}

\textbf{Deceleration parameter $q(z)$:} The deceleration parameter provides an additional, independently testable prediction. In the CFM, the transition from decelerated to accelerated expansion ($q = 0$) occurs at $z_{\mathrm{acc}} = 0.52$ -- significantly later in cosmic time than in the $\Lambda$CDM model ($z_{\mathrm{acc}} = 0.84$). Furthermore, the CFM predicts a stronger present-day acceleration: $q_0^{\mathrm{CFM}} = -0.81$ compared to $q_0^{\Lambda\mathrm{CDM}} = -0.63$.

\textbf{Implication for structure formation:} A later onset of cosmic acceleration ($z_{\mathrm{acc}} = 0.52$ instead of $0.84$) means that gravitationally bound structures (galaxy clusters, large-scale filaments) \textit{could grow undisturbed for longer} before the acceleration suppressed growth. The matter-dominated era -- during which gravity drives structure growth -- lasted significantly longer in the CFM than in the standard model.

\textbf{Empirical evidence for early massive structures:} Several independent observations put the $\Lambda$CDM model under significant tension regarding structure formation:
\begin{enumerate}
\item \textbf{JWST ``Universe Breakers'':} The James Webb Space Telescope has discovered galaxies at $z > 7$ (approximately 500--700\,Myr after the Big Bang) that are far more massive than permitted by hierarchical structure formation in $\Lambda$CDM \cite{Labbe2023}. Boylan-Kolchin \cite{BoylanKolchin2023} demonstrates quantitatively that the stellar mass density of these objects exceeds the available baryon budget within $\Lambda$CDM halos -- a ``timing problem'' of structure formation.
\item \textbf{El~Gordo (ACT-CL~J0102$-$4915):} This extremely massive galaxy cluster at $z \approx 0.87$ with mass $M_{200} \approx 2.1 \times 10^{15}\,M_\odot$ represents a $> 6\sigma$ tension with $\Lambda$CDM, since an object of this mass at this age with the observed collision velocity is statistically nearly impossible \cite{Asencio2023}.
\item \textbf{Protocluster SPT2349$-$56:} Already 1.4\,Gyr after the Big Bang ($z = 4.3$), this protocluster exhibits at least 14 gas-rich galaxies with a total star formation rate of $\sim\!6{,}500\,M_\odot$/yr -- far more mature than predicted by $\Lambda$CDM \cite{Miller2018}.
\end{enumerate}

The CFM offers a natural explanation for all three observations: since the acceleration only begins at $z_{\mathrm{acc}} = 0.52$ (instead of $z_{\mathrm{acc}} = 0.84$), gravitationally bound structures had more cosmic time for undisturbed growth. The CFM prediction is directly testable through galaxy cluster counts and weak gravitational lensing surveys (Euclid, Vera C.\ Rubin Observatory).

\begin{table}[H]
\centering
\caption{Deceleration parameter $q(z)$: $\Lambda$CDM vs.\ CFM.}
\label{tab:qz}
\begin{tabular}{cccc}
\toprule
$z$ & $q$ ($\Lambda$CDM) & $q$ (CFM) & $\Delta q$ \\
\midrule
0.0 & $-0.634$ & $-0.805$ & $-0.171$ \\
0.5 & $-0.217$ & $-0.017$ & $+0.200$ \\
1.0 & $+0.082$ & $+0.321$ & $+0.240$ \\
2.0 & $+0.346$ & $+0.467$ & $+0.121$ \\
\bottomrule
\end{tabular}
\end{table}

\textbf{$H_0$ implications:} The nuisance parameter $M$ absorbs both the absolute magnitude $M_B$ and $H_0$ via the relation $M = M_B + 5\log_{10}(c/H_0) + 25$. Using the SH0ES calibration ($M_B = -19.253$), the CFM yields $H_0 = 76.1\,\mathrm{km/s/Mpc}$ compared to $H_0 = 75.5\,\mathrm{km/s/Mpc}$ in $\Lambda$CDM -- a difference of $\Delta H_0 = +0.5\,\mathrm{km/s/Mpc}$. The $H_0$ tension is not resolved by the CFM alone, as the difference lies within the measurement uncertainty. A direct resolution requires the combination with CMB and BAO data.


% ===================================================================
% 5. COMPARISON WITH ALTERNATIVES
% ===================================================================
\section{Comparison with Alternative Models}
\label{sec:alternatives}

\subsection{$\Lambda$CDM (Standard Model)}

The $\Lambda$CDM model is extremely simple ($w = -1$, constant, two cosmological parameters) and fits all current data well. However, it suffers from the cosmological constant problem and the coincidence problem \cite{Weinberg1989}.

\subsection{Quintessence}

Quintessence models \cite{Caldwell1998} postulate a dynamic scalar field~$\phi$ with a time-dependent equation-of-state parameter. They can alleviate the coincidence problem but require a new field and its potential~$V(\phi)$ with many free parameters.

\subsection{Modified Gravity: $f(R)$ Theories}

$f(R)$ gravity theories \cite{Starobinsky1980, Sotiriou2010} replace the Ricci scalar~$R$ in the Einstein--Hilbert action with a more general function. They offer a geometric explanation without dark energy but are mathematically complex and partly inconsistent with observations (gravitational lensing, CMB).

\subsection{Emergent Gravity (Verlinde)}

Verlinde \cite{Verlinde2011, Verlinde2017} proposes that gravity is not a fundamental force but an emergent, entropic phenomenon. In de~Sitter spaces, the entropy associated with the cosmological horizon leads to an additional ``dark'' gravitational force that could explain galaxy behavior without dark matter.

\subsection{Finsler Gravity}

Pfeifer et al.\ \cite{Pfeifer2025} extend general relativity through Finsler geometry, in which the metric depends not only on position but also on velocity:
\begin{equation}
g_{\mu\nu}(x, y) = \frac{1}{2}\,\frac{\partial^2 F^2}{\partial y^\mu \partial y^\nu}, \quad y = \frac{dx}{d\lambda}
\end{equation}
The resulting Finsler--Friedmann equation produces exponential expansion even in vacuum -- without a cosmological constant.


\subsection{Cosmological Teleodynamics}

Trivedi and Venkatasubramanian \cite{Trivedi2025} formulate a game-theoretic cosmology that exhibits remarkable parallels to the approach presented here. Their \textit{Cosmological Teleodynamics} describes the universe as a ``giant potential game'' converging toward a continuous Nash equilibrium. Cosmic acceleration appears as a ``statistically emergent effect of dynamic memory in a self-gravitating medium'' -- a formulation that conceptually corresponds to the ``geometric memory'' of the CFM.


\subsection{Synoptic Comparison}
\label{subsec:synopse}

\begin{table}[H]
\centering
\caption{Synoptic comparison of cosmological models without dark energy.}
\label{tab:synopse}
\begin{tabularx}{\textwidth}{lXXX}
\toprule
\textbf{Property} & \textbf{CFM} & \textbf{Finsler} & \textbf{Teleodynamics} \\
\midrule
Theor.\ basis & Standard GR + potential & Finsler geometry & Stat.\ mechanics + game theory \\
Mechanism & Releasing ``brake'' & Velocity-dep.\ metric & Dynamic memory \\
Dark energy & Not needed & Not needed & Not needed \\
Empirical test & Pantheon+ (1,590 SNe, $\Delta\chi^2{=}{-}12$) & Pending & Qualitative \\
Prediction & $w(z)$ time variation & Exp.\ expansion & Nash convergence \\
Complexity & Low (4 params.) & High & Medium \\
\bottomrule
\end{tabularx}
\end{table}


% ===================================================================
% 6. COMPLEMENTARITY AND UNIFICATION
% ===================================================================
\section{Complementarity and Possible Unification}
\label{sec:complementarity}

\subsection{Three Models, One Insight}

All three approaches -- CFM, Finsler gravity, and Cosmological Teleodynamics -- share a fundamental insight:
\begin{quote}
\textit{``The accelerated expansion is not a new `thing' but a property of the geometry or the statistical structure of the universe itself.''}
\end{quote}

\subsection{Hypothesis: CFM as an Effective Description}

A fascinating possibility is that the three models describe different aspects of the same phenomenon. By analogy with the relationship between thermodynamics and statistical mechanics, the following hierarchy could hold:

\begin{itemize}
\item \textbf{Finsler gravity} (microscopic, fundamental): All moments of the 1-particle distribution function contribute to gravity.
\item \textbf{CFM} (macroscopic, phenomenological): The time-dependent potential $\Phi(a)$ effectively encodes the contribution of higher moments.
\item \textbf{Teleodynamics} (systemic, game-theoretic): The Nash equilibrium dynamics describes the global optimization.
\end{itemize}

Mathematically, this complementarity can be represented as a hierarchical relationship:
\begin{equation}
\underbrace{G_{\mu\nu}^{\mathrm{Finsler}}(x, y)}_{\substack{\text{Finsler gravity}\\\text{(microscopic)}}}
\;\xrightarrow{\;\langle\cdot\rangle_{\mathrm{eff}}\;}
\underbrace{G_{\mu\nu} + 8\pi G\,\Omega_\Phi(a)\,g_{\mu\nu}}_{\substack{\text{CFM: modified}\\\text{Einstein equation}}}
\;\xleftarrow{\;\delta\Phi/\delta s_i = 0\;}
\underbrace{\max_{s_P, s_D} \Phi(s_P, s_D)}_{\substack{\text{Teleodynamics:}\\\text{Nash equilibrium}}}
\label{eq:complementarity}
\end{equation}
where the left arrow describes the effective averaging over the velocity-dependent degrees of freedom of Finsler geometry and the right arrow represents the variational condition of the game-theoretic potential that determines the temporal behavior of $\Omega_\Phi(a)$.


% ===================================================================
% 7. TESTABILITY AND PREDICTIONS
% ===================================================================
\section{Testability and Predictions}
\label{sec:testability}

\subsection{Observable Signatures}

\textbf{1.~Phantom equation of state $w(z) < -1$:} The CFM predicts $|\Delta w| \approx 0.4$ across the entire observable redshift range. The ESA mission Euclid \cite{Euclid2024} and the Nancy Grace Roman Space Telescope (NASA, $\sim$2027) can measure $\sigma_w \approx 0.02$--$0.05$ -- well sufficient to detect or exclude this signature.

\textbf{2.~Deceleration parameter:} The CFM predicts an earlier transition to accelerated expansion ($z_{\mathrm{acc}} = 0.52$ vs.\ $\Lambda$CDM: $z_{\mathrm{acc}} = 0.84$) as well as a stronger present-day acceleration ($q_0 = -0.81$ vs.\ $-0.63$). This is testable independently of $w(z)$.

\textbf{3.~Structure growth:} A modified growth rate $f \cdot \sigma_8$ is predicted, measurable through weak gravitational lensing and galaxy cluster counts. Initial empirical hints are already provided by the JWST ``Universe Breakers'' at $z > 7$ \cite{Labbe2023}, the El~Gordo anomaly ($> 6\sigma$ tension with $\Lambda$CDM; \cite{Asencio2023}), and unexpectedly mature protoclusters at $z > 4$ \cite{Miller2018} -- all consistent with the CFM prediction of an extended growth phase.

\textbf{4.~CMB integral effects:} A modified ISW effect (\textit{Integrated Sachs--Wolfe}) in CMB temperature cross-correlations.

\subsection{Future Missions}

\begin{table}[H]
\centering
\caption{Relevant observational missions for the CFM test.}
\label{tab:missions}
\begin{tabularx}{\textwidth}{lccX}
\toprule
\textbf{Mission} & \textbf{Launch} & $\sigma(w)$ & \textbf{Relevance for CFM} \\
\midrule
Euclid (ESA) & 2023 & $\approx 0.02$ & Precision BAO + weak lensing; can distinguish CFM vs.\ $\Lambda$CDM at $z > 0.8$ \\
Roman (NASA) & $\sim$2027 & $\approx 0.03$ & SN survey to $z \approx 2$; ideal instrument for $w(z)$ test \\
DESI & 2021-- & $\approx 0.04$ & Millions of galaxy spectra; BAO and structure growth \\
\bottomrule
\end{tabularx}
\end{table}


\subsection{Model Distinguishability}

\begin{table}[H]
\centering
\caption{Comparison of predictions: $\Lambda$CDM vs.\ CFM (Pantheon+ fit). The $1\sigma$ uncertainties are from the MCMC analysis.}
\label{tab:predictions}
\begin{tabular}{lcc}
\toprule
\textbf{Property} & $\Lambda$CDM & CFM \\
\midrule
$w(z{=}0)$ & $-1.000$ & $-1.36 \pm 0.02$ \\
$w(z{=}0.5)$ & $-1.000$ & $-1.45^{+0.09}_{-0.28}$ \\
$w(z{=}1)$ & $-1.000$ & $-1.45^{+0.10}_{-0.30}$ \\
$w(z{=}2)$ & $-1.000$ & $-1.43^{+0.08}_{-0.21}$ \\
Time variation & None & Yes (consistently $w < -1$) \\
$\Delta w$ (measurable) & -- & $\approx -0.4$ \\
$q_0$ (today) & $-0.63$ & $-0.81$ \\
$z_{\mathrm{acc}}$ (transition) & 0.84 & 0.52 \\
\bottomrule
\end{tabular}
\end{table}


% ===================================================================
% 8. DISCUSSION
% ===================================================================
\section{Discussion}
\label{sec:discussion}

\subsection{Strengths of the Approach}

\begin{enumerate}
\item \textbf{Conceptual elegance:} No new form of energy required; the acceleration is a ``releasing constraint,'' not a ``new drive.''
\item \textbf{Game-theoretic foundation:} The emergence of physical laws from equilibrium conditions offers a novel explanatory framework, independently supported by the work of Trivedi and Venkatasubramanian \cite{Trivedi2025}.
\item \textbf{Empirical validation:} The CFM fits 1,590 real Pantheon+ supernovae better than $\Lambda$CDM ($\Delta\chi^2 = -12.2$, $\Delta\mathrm{AIC} = -8.2$) and generalizes better in cross-validation.
\item \textbf{Testability:} Specific, quantitative predictions for $w(z)$ and $z_{\mathrm{acc}}$ that are verifiable within a decade.
\item \textbf{Empirical support:} The CFM prediction of an extended growth phase ($z_{\mathrm{acc}} = 0.52$) offers a natural explanation for the JWST ``early galaxy tension'' \cite{Labbe2023, BoylanKolchin2023}, the El~Gordo anomaly \cite{Asencio2023}, and unexpectedly mature protoclusters \cite{Miller2018}.
\item \textbf{Convergence of independent approaches:} CFM, Finsler gravity, and Cosmological Teleodynamics independently arrive at the same conclusion: dark energy is not necessary.
\item \textbf{Reproducibility:} Analysis code and data are publicly available (\url{https://github.com/lukisch/cfm-cosmology}).
\end{enumerate}

\subsection{Limitations and Open Questions}

\begin{enumerate}
\item \textbf{Phenomenological character:} The CFM is not a fundamental theory. Although the $\tanh$ form can be motivated as the exact solution of the saturation ODE~\eqref{eq:saturation_ode} and four alternative functional forms yield comparable results (Section~\ref{subsec:funcforms}), a derivation from a fundamental quantum equation is still outstanding.
\item \textbf{Parameter freedom:} Four effective parameters versus two in $\Lambda$CDM lead to a marginal BIC disadvantage ($\Delta\mathrm{BIC} = +2.6$), which is however mitigated by the better cross-validation and the robustness across different functional forms.
\item \textbf{Phantom regime:} The effective equation-of-state parameter $w < -1$ lies in the phantom regime. As shown in Section~\ref{subsec:phantom}, this leads in the CFM context neither to a Big Rip nor to instabilities, since $\Omega_\Phi$ saturates and does not represent a physical field. Formally, the CFM violates the null energy condition, analogous to $f(R)$ gravity theories \cite{Sotiriou2010}.
\item \textbf{Outstanding tests:} CMB predictions (ISW effect, CMB power spectrum), BAO signatures, and gravitational lensing effects remain to be computed. The analysis with the full covariance matrix (Section~\ref{subsec:results}) does however confirm the results of the diagonal analysis.
\item \textbf{Microscopic basis:} What is $\Phi$ at the quantum level? The connection to a theory of quantum gravity is outstanding. The possible relationship to Finsler gravity (Section~\ref{sec:complementarity}) could bridge this gap.
\item \textbf{$H_0$ tension:} The $H_0$ analysis (Section~\ref{subsec:deceleration}) shows $\Delta H_0 = +0.5\,$km/s/Mpc between CFM and $\Lambda$CDM -- too small to resolve the $H_0$ tension. A resolution requires the combination with CMB and BAO data.
\end{enumerate}


\subsection{Philosophical Implications}

If the CFM (or a related model) is confirmed, this would have profound consequences:

\begin{itemize}
\item \textbf{Dark energy is not a ``thing'':} It would be a geometric memory, not a physical field.
\item \textbf{The universe ``knows'' about its beginning:} The geometry possesses a ``memory.''
\item \textbf{Paradigm shift:} From ``What drives the acceleration?'' to ``Why did the expansion brake earlier?''
\end{itemize}

This would be comparable to the transition from ``What drives the planets?'' (Ptolemy: spheres) to ``How do planets move in the geometry of space?'' (Kepler, Newton, Einstein).


% ===================================================================
% 9. CONCLUSION AND OUTLOOK
% ===================================================================
\section{Conclusion and Outlook}
\label{sec:conclusion}

This paper has shown:

\begin{enumerate}
\item A game-theoretic framework for cosmology -- the Nash equilibrium between null space and spacetime bubble -- naturally leads to a model in which physical laws appear as emergent equilibrium conditions.
\item The resulting \textit{Curvature Feedback Model} (CFM) explains accelerated expansion without dark energy and passes the test against 1,590 real Type~Ia supernovae from the Pantheon+ catalog \cite{Scolnic2022}: $\Delta\chi^2 = -12.2$ (diagonal) / $-11.2$ (full covariance matrix), $\Delta\mathrm{AIC} = -8.2$ / $-7.2$, and better cross-validation.
\item The robust model selection (AIC, BIC, 5-fold cross-validation) shows that the better fit of the CFM is not attributable to overfitting. This is confirmed by four alternative functional forms, all yielding $\Delta\chi^2 \approx -9$ to $-12$.
\item MCMC-based parameter uncertainties ($\Omega_m = 0.368 \pm 0.024$) and the phantom stability analysis (no Big Rip, asymptotically de~Sitter) support the physical consistency.
\item The CFM makes testable predictions: a persistent phantom equation of state $w(z) < -1$ and a later acceleration transition ($z_{\mathrm{acc}} = 0.52$ vs.\ $0.84$ in $\Lambda$CDM), testable with Euclid and Roman within the next decade. Already now, the CFM prediction of an extended growth phase finds empirical support from JWST observations of unexpectedly massive galaxies at high redshifts \cite{Labbe2023, BoylanKolchin2023} and the statistically improbable existence of massive clusters such as El~Gordo \cite{Asencio2023}.
\item The convergence of three independent approaches (CFM, Finsler gravity, Cosmological Teleodynamics) points toward a possible paradigm shift: \textit{dark energy as an independent entity may be superfluous.}
\end{enumerate}

\textbf{Next steps} include: (a)~testing against Planck CMB and DESI BAO data (the full Pantheon+ covariance matrix has already been taken into account), (b)~computation of CMB power spectrum and structure growth predictions ($f\sigma_8$), (c)~exploration of the connection between CFM and Finsler geometry, (d)~development of a covariant formulation of $\Phi(a)$ from the Ricci scalar~$R$, (e)~investigation of quantum mechanical foundations of the curvature return potential, and (f)~combination with local distance ladder data for direct $H_0$ determination.

\textbf{Outlook: Unification with MOND -- A universe without a dark sector?} A particularly fascinating perspective opens up through the combination of the CFM with \textit{Modified Newtonian Dynamics} (MOND) \cite{Milgrom1983}. While the CFM eliminates dark energy as a geometric effect, MOND replaces dark matter through modified gravitational dynamics on galactic scales. Both frameworks converge on the prediction that structures form earlier and more efficiently than in $\Lambda$CDM -- the CFM through an extended matter-dominated era ($z_{\mathrm{acc}} = 0.52$), MOND through effectively stronger gravity at low accelerations \cite{Asencio2023}. A preliminary analysis with a purely baryonic universe ($\Omega_m = \Omega_b \approx 0.05$) and an extended geometric potential $\Omega_\Phi(a) = \Phi_0 \cdot f_{\mathrm{tanh}}(a) + \alpha \cdot a^{-\beta}$ yields $\Delta\chi^2 = -26.3$ and $\Delta\mathrm{AIC} = -16.3$ relative to $\Lambda$CDM -- \textit{dramatically outperforming both the standard CFM and $\Lambda$CDM}. The MCMC posterior analysis yields $\beta = 2.02 \pm 0.20$, which corresponds exactly to the scaling of spatial curvature ($a^{-2}$). ``Dark matter'' would thus be a dynamic curvature effect, not a particle. This result requires a full relativistic treatment (e.g., within the AeST framework \cite{Skordis2021}) and will be analyzed in detail in a companion paper.

\begin{quote}
\textit{``Sometimes the most elegant explanation is not a new force, but a relaxing constraint.''}
\end{quote}


% ===================================================================
% BIBLIOGRAPHY
% ===================================================================
\subsection*{Software}

This work uses \texttt{emcee} \cite{ForemanMackey2013} for MCMC sampling, \texttt{NumPy} \cite{Harris2020}, \texttt{SciPy} \cite{Virtanen2020} for numerical computations, and \texttt{Matplotlib} \cite{Hunter2007} for visualizations.

\begin{thebibliography}{99}

\bibitem{Riess1998}
Riess, A.\,G.\ et al.\ (1998).
Observational Evidence from Supernovae for an Accelerating Universe and a Cosmological Constant.
\textit{The Astronomical Journal}, 116(3), 1009--1038.
DOI: 10.1086/300499.

\bibitem{Perlmutter1999}
Perlmutter, S.\ et al.\ (1999).
Measurements of $\Omega$ and $\Lambda$ from 42 High-Redshift Supernovae.
\textit{The Astrophysical Journal}, 517(2), 565--586.
DOI: 10.1086/307221.

\bibitem{Planck2020}
Planck Collaboration (2020).
Planck 2018 results. VI. Cosmological parameters.
\textit{Astronomy \& Astrophysics}, 641, A6.
DOI: 10.1051/0004-6361/201833910.

\bibitem{Weinberg1989}
Weinberg, S.\ (1989).
The Cosmological Constant Problem.
\textit{Reviews of Modern Physics}, 61(1), 1--23.
DOI: 10.1103/RevModPhys.61.1.

\bibitem{Riess2022}
Riess, A.\,G.\ et al.\ (2022).
A Comprehensive Measurement of the Local Value of the Hubble Constant with 1\,km/s/Mpc Uncertainty from the Hubble Space Telescope and the SH0ES Team.
\textit{The Astrophysical Journal Letters}, 934(1), L7.
DOI: 10.3847/2041-8213/ac5c5b.

\bibitem{DESI2024}
DESI Collaboration (2024).
DESI 2024 VI: Cosmological Constraints from the Measurements of Baryon Acoustic Oscillations.
\textit{arXiv:2404.03002}.

\bibitem{Pfeifer2025}
Pfeifer, C.\ et al.\ (2025).
From kinetic gases to an exponentially expanding universe -- the Finsler-Friedmann equation.
\textit{Journal of Cosmology and Astroparticle Physics}, 2025(10), 050.
DOI: 10.1088/1475-7516/2025/10/050.

\bibitem{Trivedi2025}
Trivedi, O.\ \& Venkatasubramanian, V.\ (2025).
Game Theory in Cosmology.
\textit{arXiv:2511.20739}.

\bibitem{Caldwell1998}
Caldwell, R.\,R., Dave, R.\ \& Steinhardt, P.\,J.\ (1998).
Cosmological Imprint of an Energy Component with General Equation of State.
\textit{Physical Review Letters}, 80(8), 1582--1585.
DOI: 10.1103/PhysRevLett.80.1582.

\bibitem{Starobinsky1980}
Starobinsky, A.\,A.\ (1980).
A New Type of Isotropic Cosmological Models Without Singularity.
\textit{Physics Letters B}, 91(1), 99--102.
DOI: 10.1016/0370-2693(80)90670-X.

\bibitem{Sotiriou2010}
Sotiriou, T.\,P.\ \& Faraoni, V.\ (2010).
$f(R)$ Theories of Gravity.
\textit{Reviews of Modern Physics}, 82(1), 451--497.
DOI: 10.1103/RevModPhys.82.451.

\bibitem{Verlinde2011}
Verlinde, E.\ (2011).
On the Origin of Gravity and the Laws of Newton.
\textit{Journal of High Energy Physics}, 2011, 29.
DOI: 10.1007/JHEP04(2011)029.

\bibitem{Verlinde2017}
Verlinde, E.\ (2017).
Emergent Gravity and the Dark Universe.
\textit{SciPost Physics}, 2(3), 016.
DOI: 10.21468/SciPostPhys.2.3.016.

\bibitem{Euclid2024}
Euclid Collaboration (2025).
Euclid Quick Data Release 1.
ESA/Euclid Consortium.

\bibitem{Casimir1948}
Casimir, H.\,B.\,G.\ (1948).
On the attraction between two perfectly conducting plates.
\textit{Proceedings of the Royal Netherlands Academy of Arts and Sciences}, 51, 793--795.

\bibitem{Hawking1974}
Hawking, S.\,W.\ (1974).
Black hole explosions?
\textit{Nature}, 248, 30--31.
DOI: 10.1038/248030a0.

\bibitem{Nash1950}
Nash, J.\,F.\ (1950).
Equilibrium points in $n$-person games.
\textit{Proceedings of the National Academy of Sciences}, 36(1), 48--49.
DOI: 10.1073/pnas.36.1.48.

\bibitem{DESI2025}
DESI Collaboration (2025).
DESI DR2 Results II: Measurements of Baryon Acoustic Oscillations and Cosmological Constraints.
\textit{arXiv:2503.14738}.

\bibitem{Scolnic2022}
Scolnic, D.\ et al.\ (2022).
The Pantheon+ Analysis: The Full Data Set and Light-curve Release.
\textit{The Astrophysical Journal}, 938(2), 113.
DOI: 10.3847/1538-4357/ac8b7a.

\bibitem{KassRaftery1995}
Kass, R.\,E.\ \& Raftery, A.\,E.\ (1995).
Bayes Factors.
\textit{Journal of the American Statistical Association}, 90(430), 773--795.
DOI: 10.1080/01621459.1995.10476572.

\bibitem{ForemanMackey2013}
Foreman-Mackey, D.\ et al.\ (2013).
emcee: The MCMC Hammer.
\textit{Publications of the Astronomical Society of the Pacific}, 125(925), 306--312.
DOI: 10.1086/670067.

\bibitem{Labbe2023}
Labb\'e, I.\ et al.\ (2023).
A population of red candidate massive galaxies $\sim$600\,Myr after the Big Bang.
\textit{Nature}, 616(7956), 266--269.
DOI: 10.1038/s41586-023-05786-2.

\bibitem{BoylanKolchin2023}
Boylan-Kolchin, M.\ (2023).
Stress testing $\Lambda$CDM with high-redshift galaxy candidates.
\textit{Nature Astronomy}, 7, 731--735.
DOI: 10.1038/s41550-023-01937-7.

\bibitem{Asencio2023}
Asencio, E., Banik, I.\ \& Kroupa, P.\ (2023).
The El Gordo galaxy cluster challenges $\Lambda$CDM for any plausible collision velocity.
\textit{The Astrophysical Journal}, 954(2), 162.
DOI: 10.3847/1538-4357/ace62a.

\bibitem{Miller2018}
Miller, T.\,B.\ et al.\ (2018).
A massive core for a cluster of galaxies at a redshift of 4.3.
\textit{Nature}, 556(7702), 469--472.
DOI: 10.1038/s41586-018-0025-2.

\bibitem{Milgrom1983}
Milgrom, M.\ (1983).
A modification of the Newtonian dynamics as a possible alternative to the hidden mass hypothesis.
\textit{The Astrophysical Journal}, 270, 365--370.
DOI: 10.1086/161130.

\bibitem{Skordis2021}
Skordis, C.\ \& Z{\l}o\'snik, T.\ (2021).
New Relativistic Theory for Modified Newtonian Dynamics.
\textit{Physical Review Letters}, 127(16), 161302.
DOI: 10.1103/PhysRevLett.127.161302.

\bibitem{Harris2020}
Harris, C.\,R.\ et al.\ (2020).
Array programming with NumPy.
\textit{Nature}, 585, 357--362.
DOI: 10.1038/s41586-020-2649-2.

\bibitem{Virtanen2020}
Virtanen, P.\ et al.\ (2020).
SciPy 1.0: Fundamental Algorithms for Scientific Computing in Python.
\textit{Nature Methods}, 17, 261--272.
DOI: 10.1038/s41592-019-0686-2.

\bibitem{Hunter2007}
Hunter, J.\,D.\ (2007).
Matplotlib: A 2D Graphics Environment.
\textit{Computing in Science \& Engineering}, 9(3), 90--95.
DOI: 10.1109/MCSE.2007.55.

\end{thebibliography}

\end{document}

\documentclass[aps,prd,twocolumn,superscriptaddress,nofootinbib]{revtex4-2}

% MiKTeX REVTeX 4.2 needs explicit package loading
\usepackage{amsmath}
\usepackage{amssymb}
\usepackage{booktabs}
\usepackage{tabularx}
\usepackage{xcolor}
\usepackage{float}
\usepackage{graphicx}
\graphicspath{{../figures/}}

% Theorem environments (REVTeX compatible)
\newtheorem{definition}{Definition}
\newtheorem{proposition}{Proposition}
\newtheorem{theorem}{Theorem}
\newtheorem{conjecture}{Conjecture}

\usepackage{hyperref}
\hypersetup{
    pdftitle={Microscopic Foundations of the Curvature Relaxation Model: From Quantum Geometry to Macroscopic Saturation},
    pdfauthor={L. Geiger},
    colorlinks=true,
    linkcolor=black,
    urlcolor=blue,
    citecolor=black
}

\begin{document}

% ===================================================================
% TITELSEITE
% ===================================================================

\title{Microscopic Foundations of the Curvature Relaxation Model: From Quantum Geometry to Macroscopic Saturation}

\author{L. Geiger}
\email{Correspondence: Bernau, Germany}
\affiliation{Independent Researcher, Bernau im Schwarzwald, Germany}

\date{\today}

\begin{abstract}
Papers~I and~II established the Curvature Relaxation Model (CRM) as a phenomenologically successful alternative to $\Lambda$CDM, eliminating the dark sector through geometric curvature return ($\Delta\chi^2 = -5.5$ vs.\ $\Lambda$CDM, joint SN+CMB+BAO, 6 parameters, zero EDE). This paper addresses the microscopic foundations: \textit{What quantum system yields the saturation ODE $d\Omega_\Phi/da = k[1-(\Omega_\Phi/\Phi_0)^2]$ and the running coupling $\beta_{\mathrm{eff}}(a)$?} We derive the effective Lagrangian $\mathcal{L}_{\mathrm{CRM}} = R/(16\pi G) + \gamma R^2 + \frac{1}{2}(\partial\phi)^2 - V_{\mathrm{PT}}(\phi)$, where the $R^2$ term generates the geometric ``dark matter'' contribution and a P\"oschl-Teller scalar provides saturation dynamics. The trace coupling $\mathcal{F}(T/\rho)$ ensures BBN protection as a rigorous consequence of conformal symmetry. We explore four UV-completion candidates (scalar double-well, Loop Quantum Gravity, Finsler geometry, information-theoretic spacetime), all producing saturation dynamics of the required form. The running $\beta$ is interpreted as a second order parameter in a geometric phase transition. Ghost freedom, tachyonic stability, and solar system compatibility (chameleon screening, $\lambda_C^{\mathrm{solar}} \sim 20$\,m) are verified. Testable predictions -- scalaron Compton wavelength, gravitational wave speed $c_T = c$, lensing parameter, and CMB power spectrum modifications -- are independent of the UV completion. The ontological picture reduces to spacetime curvature in three phases plus baryonic matter.
\end{abstract}

\keywords{Curvature Relaxation Model, quantum gravity, Lagrangian formulation, Loop Quantum Gravity, Finsler geometry, saturation mechanism, modified gravity}

\maketitle

\thanks{AI tools (Claude, Anthropic; Gemini, Google DeepMind) were used for mathematical formalization, code development, and text generation. All physical hypotheses, scientific interpretation, and responsibility for the content lie solely with the author. The analysis code is available at \url{https://github.com/lukisch/crm-cosmology}.}

\thanks{Paper~III in the CRM series.}


% ===================================================================
% 1. EINLEITUNG
% ===================================================================
\section{Introduction: The Central Question}
\label{sec:intro}

The Curvature Relaxation Model (CRM) \cite{Geiger2026} and its MOND-compatible extension \cite{Geiger2026b} have demonstrated remarkable phenomenological success:

\begin{itemize}
\item \textbf{Paper~I:} The standard CRM replaces dark energy with a curvature return potential, achieving $\Delta\chi^2 = -12.2$ vs.\ $\Lambda$CDM on Pantheon+ data.
\item \textbf{Paper~II:} The extended CRM replaces the particle dark sector with spacetime geometry, achieving a universe with exclusively baryonic matter content. The SN-only analysis yields $\Delta\chi^2 = -26.3$ with $\beta = 2.02 \pm 0.20$ (curvature scaling). Crucially, a \textit{running curvature coupling} $\beta_{\mathrm{eff}}(a)$ -- transitioning from CDM-like ($\beta_{\mathrm{early}} \approx 2.82$) at $z > 6$ to curvature-like ($\beta \approx 2.0$) at low $z$ -- combined with a scale-dependent MOND coupling $\mu(a)$ achieves joint SN + CMB + BAO compatibility: $\ell_A = 301.471$ (Planck: $301.471$), $\mathcal{R} = 1.7502$ (Planck: $1.7502$), $H_0 = 67.3$\,km/s/Mpc, and $\Delta\chi^2 = -5.5$ vs.\ $\Lambda$CDM (preferred zero-EDE variant).
\end{itemize}

Both results derive from a single dynamical equation -- the \textit{saturation ODE}:
\begin{equation}
\frac{d\Omega_\Phi}{da} = k \left[1 - \left(\frac{\Omega_\Phi}{\Phi_0}\right)^2\right]
\label{eq:saturation_ode}
\end{equation}

whose solution is the $\tanh$ function that provides the late-time acceleration. The extended model adds a power-law term $\alpha \cdot a^{-\beta_{\mathrm{eff}}}$ representing the unsaturated (early-time) phase of the same geometric process, with a running coupling $\beta_{\mathrm{eff}}(a)$ that encodes the curvature-dependent transition between gravitational regimes.

The central question of this paper is:

\begin{quote}
\textit{Which microscopic (quantum) system has the property that its macroscopic (thermodynamic) limit yields (i) the saturation ODE~\eqref{eq:saturation_ode}, (ii) the running coupling $\beta_{\mathrm{eff}}(a)$, and (iii) the full extended Friedmann equation? Can the entire framework be derived from a Lagrangian?}
\end{quote}

This question is not merely academic. Without a Lagrangian formulation, the CRM cannot:
\begin{enumerate}
\item Be consistently coupled to matter fields
\item Generate perturbation equations for $C_\ell$ and $P(k)$ predictions
\item Be connected to known quantum gravity frameworks
\item Be considered a complete physical theory
\end{enumerate}


% ===================================================================
% 2. LAGRANGIAN FORMULIERUNG
% ===================================================================
\section{The Effective Lagrangian}
\label{sec:lagrangian}

\subsection{Requirements}

The effective Lagrangian $\mathcal{L}_{\mathrm{CRM}}$ must satisfy:
\begin{enumerate}
\item \textbf{Background:} The Euler-Lagrange equations, evaluated on the FLRW metric, must yield the extended Friedmann equation:
\begin{equation}
H^2(a) = H_0^2 \left[\Omega_b\,a^{-3} + \Phi_0 \cdot f_{\mathrm{sat}}(a) + \alpha \cdot a^{-\beta}\right]
\end{equation}

\item \textbf{Saturation dynamics:} The scalar field equation of motion must reduce to $d\Omega_\Phi/da = k[1 - (\Omega_\Phi/\Phi_0)^2]$ on the FLRW background.

\item \textbf{General covariance:} The action must be diffeomorphism-invariant.

\item \textbf{Correct limits:} In the limit $k \to 0$, $\alpha \to 0$, the theory must reduce to GR with cosmological constant.
\end{enumerate}

\subsection{Scalar Field Approach}

The most natural Lagrangian formulation introduces a scalar field $\phi$ with a potential $V(\phi)$:
\begin{equation}
S = \int d^4x \sqrt{-g} \left[\frac{R}{16\pi G} - \frac{1}{2} g^{\mu\nu}\partial_\mu\phi\,\partial_\nu\phi - V(\phi) + \mathcal{L}_m\right]
\label{eq:action_scalar}
\end{equation}

For the saturation ODE to emerge, we require $V(\phi)$ such that the homogeneous field equation on FLRW yields $\tanh$-type solutions.

{\sloppy
\begin{proposition}[Double-Well Saturation Potential]
The potential
\begin{equation}
V(\phi) = V_0 \left[1 - \tanh^2\!\left(\frac{\phi}{\phi_0}\right)\right] = \frac{V_0}{\cosh^2(\phi/\phi_0)}
\label{eq:double_well}
\end{equation}
produces a scalar field equation whose late-time solution on the FLRW background is $\phi(a) \propto \tanh(k(a - a_{\mathrm{trans}}))$, reproducing the saturation term of the CRM.
\end{proposition}
}

\textit{Sketch of proof:} The Klein-Gordon equation on FLRW,
\begin{equation}
\ddot{\phi} + 3H\dot{\phi} + V'(\phi) = 0
\end{equation}
with $V'(\phi) = -2V_0 \tanh(\phi/\phi_0)/(\phi_0 \cosh^2(\phi/\phi_0))$, admits the solution $\phi = \phi_0 \tanh(\lambda t)$ in the slow-roll regime where $\ddot{\phi} \ll 3H\dot{\phi}$, with $\lambda$ related to $k$ and $H_0$. The energy density $\rho_\phi = \frac{1}{2}\dot{\phi}^2 + V(\phi)$ then maps to $\Omega_\Phi(a) = \Phi_0 \cdot f_{\mathrm{sat}}(a)$. \hfill $\square$

\textit{Note:} The $\cosh^{-2}$ potential is well known in quantum mechanics as the P\"oschl-Teller potential. Its appearance here suggests a connection between quantum bound states and cosmological saturation.

\textit{Philosophical tension:} The introduction of an additional scalar field $\phi$ appears to contradict the CRM's claim of eliminating the dark sector. We emphasize that $\phi$ is \textit{not} a new dark component analogous to dark energy or dark matter: it has a definite Lagrangian origin (Eq.~\ref{eq:double_well}), a known quantum-mechanical analog, and its late-time dynamics are entirely determined by $V_0$ and $\phi_0$ -- it is not freely adjustable. Nevertheless, deriving $\phi$ directly from the geometric sector (e.g., as a conformal mode of the metric or a condensate of the $R^2$ scalaron) remains an important open problem that would strengthen the purely geometric character of the CRM.

\subsection{The Power-Law Term: Geometric Origin}
\label{subsec:powerlaw_lagrangian}

The geometric ``dark matter'' term $\alpha \cdot a^{-\beta}$ with $\beta \approx 2$ requires a separate origin. Two approaches are possible:

\textbf{Approach 1: Curvature-squared terms.} Adding a Gauss-Bonnet or $R^2$ term to the action:
\begin{equation}
S_{\mathrm{geom}} = \int d^4x \sqrt{-g} \left[\frac{R}{16\pi G} + \gamma\, R^2 + \delta\, R_{\mu\nu}R^{\mu\nu}\right]
\end{equation}
produces corrections to the Friedmann equation that scale as $a^{-2}$ in the radiation-to-matter transition era. The coefficient $\gamma$ can be related to $\alpha$.

\textbf{Approach 2: Vector field (AeST-inspired).} Following Skordis \& Z{\l}o\'snik \cite{Skordis2021}, a timelike vector field $A_\mu$ constrained by $g^{\mu\nu}A_\mu A_\nu = -1$ contributes an effective energy density that scales non-standardly with $a$. The CRM power-law term may emerge as the cosmological background of such a vector field.

\subsection{The Combined Action}
\label{subsec:combined_action}

Combining both contributions, the full CRM action reads:
\begin{equation}
\begin{split}
S_{\mathrm{CRM}} = \int d^4x \sqrt{-g} \Big[&\frac{R}{16\pi G} + \gamma R^2 - \frac{1}{2}(\partial\phi)^2 \\
&- \frac{V_0}{\cosh^2(\phi/\phi_0)} + \mathcal{L}_m\Big]
\end{split}
\label{eq:full_action}
\end{equation}

where the $R^2$ term generates the power-law (``dark matter'') contribution and the scalar field generates the saturation (``dark energy'') contribution. The equilibrium between null space and spacetime bubble (Paper~I) is encoded in the balance between $\gamma$ and $V_0$.

A crucial refinement introduced in Paper~II \cite{Geiger2026b} is the \textit{trace coupling}. We now demonstrate that this is \textit{not} an ad~hoc postulate but a direct consequence of the $R^2$ action. Taking the trace of the field equations for $f(R) = R + 2\gamma R^2$ gives:
\begin{equation}
R + 12\gamma\,\Box R = -8\pi G\,T
\label{eq:trace_equation}
\end{equation}
where $T = g^{\mu\nu}T_{\mu\nu}$ is the energy-momentum trace. During the radiation era, conformal symmetry demands $T_{\mathrm{rad}} = -\rho_r + 3p_r = 0$ (since $w = 1/3$). Equation~\eqref{eq:trace_equation} then reduces to $R + 12\gamma\,\Box R = 0$, whose FLRW solution is a decaying mode: $R \to 0$ as $a \to 0$. Since the scalaron (geometric DM) is sourced by $R^2$, it vanishes identically when $R = 0$. The geometric DM contribution is therefore \textit{automatically} suppressed during the radiation era -- BBN is protected without any additional mechanism.

The only genuine \textit{postulate} is the choice of Lagrangian $f(R) = R + 2\gamma R^2$; the suppression factor $\mathcal{S}(a) = 1/(1 + a_{\mathrm{eq}}/a)$ of Paper~II is a phenomenological parametrization of this rigorous result. The modified action with the full coupling reads:
\begin{equation}
\begin{split}
S_{\mathrm{CRM}} = \int d^4x \sqrt{-g} \Big[&\frac{R}{16\pi G} + \gamma\, \mathcal{F}(T/\rho)\, R^2 \\
&- \frac{1}{2}(\partial\phi)^2 - \frac{V_0}{\cosh^2(\phi/\phi_0)} + \mathcal{L}_m\Big]
\end{split}
\end{equation}
where $\mathcal{F}(T/\rho) = |T|/(|T| + \rho_{\mathrm{rad}}) \to 0$ in the radiation era and $\mathcal{F} \to 1$ in the matter era.

\subsection{Ghost Freedom and Stability}
\label{subsec:ghost_analysis}

A critical consistency requirement for any higher-derivative theory is the absence of Ostrogradsky ghosts \cite{Woodard2015}. We now verify that the action~\eqref{eq:full_action} satisfies all stability conditions.

\textbf{Conformal equivalence.} The gravitational sector $f(R) = R + 2\gamma R^2$ (with $\epsilon = 16\pi G\gamma$) is conformally equivalent to Einstein gravity plus a massive scalar field (the \textit{scalaron}) $\chi$:
\begin{equation}
S = \int d^4x \sqrt{-g_E} \left[\frac{R_E}{16\pi G} - \frac{1}{2}(\partial\chi)^2 - U(\chi) \right]
\end{equation}
where $\chi = \sqrt{3/(16\pi G)}\,\ln f_R$ and $U(\chi) = (R f_R - f)/(2 f_R^2)$. This establishes that the theory propagates $2 + 1 + 1 = 4$ degrees of freedom (graviton + scalaron + P\"oschl-Teller scalar), all with positive kinetic energy.

{\sloppy
\begin{proposition}[Ghost Freedom of the CRM Action]
The action~\eqref{eq:full_action} is ghost-free under the following conditions, all satisfied by construction:
\begin{enumerate}
\item \textbf{No Ostrogradsky ghost:} $f_{RR} = 2\epsilon > 0$ (since $\gamma > 0$), ensuring positive kinetic energy for the scalaron. QED.
\item \textbf{No tachyonic instability:} The scalaron mass $m_s^2 = 1/(6\epsilon) > 0$ for $\gamma > 0$. The potential $U(\chi) \geq 0$ has a stable minimum at $\chi = 0$.
\item \textbf{No gradient instability:} The scalaron sound speed $c_s^2 = 1$ in $f(R)$ theories (tensor speed $c_T = c$ is guaranteed by $\alpha_T = 0$).
\item \textbf{Positive-definite kinetic matrix:} The two-field system $(\chi, \phi)$ has kinetic matrix $K = \mathrm{diag}(1, 1)$ in the Einstein frame.
\end{enumerate}
\end{proposition}
}

\textbf{Trace coupling and stability.} The coupling function $\mathcal{F}(T/\rho)$ modifies only the \textit{effective mass} of the scalaron, not its kinetic structure:
\begin{equation}
m_{\mathrm{eff}}^2(a) = \frac{1}{24\gamma\,\mathcal{F}(a)}
\end{equation}
Since $\mathcal{F} \in [0,1]$ is monotonic and bounded, the mass remains real and positive at all times. At early times ($\mathcal{F} \to 0$), $m_{\mathrm{eff}} \to \infty$ and the scalaron is frozen out. At late times ($\mathcal{F} \to 1$), $m_{\mathrm{eff}} = m_s$.

\textbf{Newtonian limit and chameleon screening.} In the weak-field limit, the scalaron produces a Yukawa correction:
\begin{equation}
V(r) = -\frac{GM}{r}\left(1 + \frac{1}{3}\,e^{-m_{\mathrm{eff}}\,r}\right)
\label{eq:yukawa}
\end{equation}
Solar system constraints require $m_{\mathrm{eff}}\,r_{\mathrm{AU}} \gg 1$. The trace coupling provides a natural chameleon mechanism: in dense environments ($\rho \gg \rho_{\mathrm{cosm}}$), the effective mass increases as $m_{\mathrm{eff}} \propto \sqrt{\rho_{\mathrm{local}}/\rho_{\mathrm{cosm}}}$. For the Sun ($\rho \sim 1400\,\mathrm{kg/m^3}$), $m_{\mathrm{eff}}^{\mathrm{solar}}/m_s \sim 4 \times 10^{14}$, yielding $\lambda_C^{\mathrm{solar}} \sim 20\,\mathrm{m} \ll 1\,\mathrm{AU}$. The scalaron is thus screened in the solar system for $\gamma \geq \mathcal{O}(1)\,H_0^{-2}$.

\subsubsection{Scalaron Mass and Local Gravity Tests}
\label{subsubsec:scalaron_mass}

From the MCMC posterior (Section~\ref{sec:numerical}), $\alpha_{M,0} = 0.0011^{+0.0010}_{-0.0006}$ at the present epoch. The scalaron mass in the cosmological background is determined by the second derivative of the $f(R)$ function:
\begin{equation}
m_s^2 = \frac{R_0}{3\,f_{RR}} = \frac{R_0}{12\gamma}\,, \qquad \gamma \geq \mathcal{O}(1)\,H_0^{-2}\,,
\label{eq:scalaron_mass}
\end{equation}
where $R_0 \approx 3H_0^2(4\Omega_\Lambda + \Omega_m) \approx 9.2\,H_0^2$ is the background curvature today. Using the constraint $\gamma \sim H_0^{-2}$ (implied by $\alpha_{M,0} \sim 10^{-3}$), this yields:
\begin{equation}
m_s \sim \sqrt{\frac{9.2\,H_0^2}{12\,H_0^{-2}}} \sim 0.88\,H_0^2 \sim \mathcal{O}(10)\,H_0 \sim 10^{-32}\,\mathrm{eV}\,,
\end{equation}
corresponding to a cosmological Compton wavelength $\lambda_C = 2\pi/m_s \gtrsim \mathcal{O}(100)\,\mathrm{Mpc}$.

\textbf{Scale-dependent gravitational strength.} On sub-Compton scales ($k \gg a\,m_s$), the scalaron mediates a fifth force, enhancing gravity to $\mu \to 4/3$ (the $f(R)$ asymptotic value). On super-Compton scales ($k \ll a\,m_s$), the scalaron is too heavy to propagate, and GR is recovered ($\mu \to 1$). The transition scale $\lambda_C \sim 100\,\mathrm{Mpc}$ is comparable to the baryon acoustic oscillation scale, ensuring compatibility with large-scale structure observations.

\textbf{Local gravity tests.} The chameleon mechanism \cite{Khoury2004} screens the scalaron in dense environments. In the solar system, the local curvature $R_{\mathrm{solar}} \sim 8\pi G\rho_\odot \gg R_0$ increases the effective scalaron mass according to:
\begin{equation}
\begin{split}
m_{\mathrm{eff}}^2(\rho) &= \frac{R(\rho)}{12\gamma} \\
\Rightarrow \quad \frac{m_{\mathrm{eff}}^{\mathrm{solar}}}{m_s} &\sim \sqrt{\frac{R_{\mathrm{solar}}}{R_0}} \sim \sqrt{\frac{8\pi G\rho_\odot}{9.2\,H_0^2}} \sim 4 \times 10^{14}\,,
\end{split}
\end{equation}
where we used $\rho_\odot \sim 1400\,\mathrm{kg/m^3}$ (mean solar density), $G = 6.67 \times 10^{-11}\,\mathrm{m^3/(kg\,s^2)}$, and $H_0 = 2.2 \times 10^{-18}\,\mathrm{s^{-1}}$. This suppresses the Compton wavelength to:
\begin{equation}
\lambda_C^{\mathrm{solar}} = \frac{\lambda_C^{\mathrm{cosmo}}}{4 \times 10^{14}} \sim \frac{100\,\mathrm{Mpc}}{4 \times 10^{14}} \sim 20\,\mathrm{m} \ll 1\,\mathrm{AU}\,,
\end{equation}
ensuring that the fifth force is exponentially suppressed within the solar system. This is consistent with Lunar Laser Ranging constraints ($\Delta G/G < 10^{-13}$) \cite{Williams2004} and the Cassini measurement of the parametrized post-Newtonian parameter $\gamma_{\mathrm{PPN}} - 1 = (2.1 \pm 2.3) \times 10^{-5}$ \cite{Bertotti2003}.

\subsection{Lagrangian Derivation of the Running Coupling $\beta_{\mathrm{eff}}(a)$}
\label{subsec:beta_derivation}

The phenomenological transition function $\beta_{\mathrm{eff}}(a)$ of Paper~II can be derived from the scalaron dynamics of the action~\eqref{eq:full_action}. The trace of the modified Einstein equation yields the scalaron equation of motion on FLRW:
\begin{equation}
\ddot{\chi} + 3H\dot{\chi} + m_{\mathrm{eff}}^2(a)\,\chi = \frac{8\pi G}{3}\,\rho_m
\label{eq:scalaron_eom}
\end{equation}
where $\chi = f_R - 1 = 4\gamma\mathcal{F}(a) R$ is the scalaron field and $m_{\mathrm{eff}}^2(a) = 1/(24\gamma\mathcal{F}(a))$. The effective geometric energy density contributed by the scalaron is $\Omega_{R^2}(a) \propto \chi(a) \cdot R(a)$, and the effective scaling exponent follows as:
\begin{equation}
{\beta_{\mathrm{eff}}(a) = -\frac{d\ln\Omega_{R^2}}{d\ln a} = 3 + \frac{d\ln\chi}{d\ln a} + \frac{d\ln\mathcal{F}}{d\ln a}}
\label{eq:beta_from_lagrangian}
\end{equation}

The trace coupling $\mathcal{F}(a) = 1/(1 + \Omega_r/(\Omega_b\,a))$ introduces a characteristic transition scale at $a \sim \Omega_r/\Omega_b \approx 1.8 \times 10^{-3}$ ($z \sim 550$). Numerical solution of Eq.~\eqref{eq:scalaron_eom} shows that $\beta_{\mathrm{eff}}$ transitions from $\sim 2.75$ at $z = 7$ to $\sim 3.0$ at $z = 0$, with the exact profile depending on $\gamma$. The phenomenological sigmoidal parametrization of Paper~II approximates this solution over the observationally relevant range $z = 0$--$100$.

\textbf{Natural parametrization.} The trace coupling has a crucial consequence for the Horndeski $\alpha_M$ function: since $\mathcal{F}(a) \approx (\Omega_b/h^2)\,a/\Omega_r$ for $a \ll a_{\mathrm{eq}}$, the scalaron field grows linearly in the scale factor during the matter era. This means $\alpha_M(a) \propto a$ at early times -- precisely the \texttt{propto\_scale} parametrization of \texttt{hi\_class}. At late times ($a \to 1$), $\mathcal{F} \to 1$ and $\alpha_M$ saturates. The \texttt{propto\_scale} parametrization is therefore not an ad hoc choice but the \textit{natural} consequence of the scalaron dynamics with trace coupling.


% ===================================================================
% 3. QUANTENGRAVITATION
% ===================================================================
\section{Quantum Gravity Connections}
\label{sec:quantum_gravity}

\subsection{Why the Saturation ODE?}

The central puzzle is the specific form of the saturation ODE~\eqref{eq:saturation_ode}: $dX/da = k(1 - X^2)$. This equation has two fixed points ($X = \pm 1$), of which $X = +1$ is stable. The $\tanh$ solution is the unique trajectory connecting $X = 0$ (zero curvature return) to $X = 1$ (full saturation).

\subsection{UV-Completion Candidates}
\label{subsec:uv_candidates}

\textbf{Important caveat:} All testable predictions of the CRM (CMB spectra, $\sigma_8$, $S_8$, gravitational slip, etc.)\ derive \textit{exclusively} from the effective action~\eqref{eq:full_action}, i.e., from the $R^2$ term and the P\"oschl-Teller scalar. The UV completion -- the microscopic quantum gravity theory from which this effective action might emerge -- does \textit{not} affect any low-energy prediction. The candidates listed below are therefore \textit{motivational}, not essential:

\begin{enumerate}
\item \textbf{Loop Quantum Gravity} \cite{Rovelli2004}: LQC's modified Friedmann equation $H^2 \propto \rho(1 - \rho/\rho_c)$ has the same $dX/dt \propto (1-X^2)$ structure as the saturation ODE. The bounded curvature invariants from holonomy corrections naturally produce saturation.

\item \textbf{Finsler Geometry} \cite{Bao2000}: Direction-dependent metrics $F(x, \dot{x})$ produce scale-dependent gravitational effects (mimicking MOND) and non-standard cosmological scaling, potentially generating the $a^{-\beta}$ term from the osculating Riemannian curvature.

\item \textbf{Information-Theoretic Spacetime} \cite{Bousso2002}: The saturation ODE is the logistic growth equation for information processing, with $\Phi_0$ as the holographic capacity limit. Entanglement entropy saturation \cite{VanRaamsdonk2010} provides the microscopic mechanism.

\item \textbf{Causal Set Theory} \cite{Bombelli1987}: The Sorkin cosmological constant $\Lambda \sim 1/\sqrt{N}$ provides a dynamical dark energy whose evolution mirrors the saturation mechanism as the causal set approaches equilibrium density.

\item \textbf{Quantum Error Correction} \cite{Almheiri2015}: The saturation $\Phi_0$ is the code capacity of the holographic spacetime code. Accelerated expansion is the code's self-protection against exceeding its error-correction threshold -- directly mapping to the null space's ``self-protection motive'' in the framework of Paper~I.
\end{enumerate}

Heuristic derivations and conjectures for each candidate are provided in Appendix~\ref{app:qg_details}. We stress that these are \textit{structural analogies}, not rigorous mathematical derivations: while all five frameworks produce saturation dynamics reminiscent of $dX/dt \propto (1 - X^2)$, a formal proof that any specific UV completion \textit{requires} the CRM effective action remains an open problem. The observation that structurally similar saturation dynamics appear across disparate approaches is suggestive but not conclusive.


% ===================================================================
% 3b. NATUR DES NULLRAUMS
% ===================================================================
\subsection{The Nature of the Null Space}
\label{subsec:null_space}

Papers~I and II postulated the null space as the ``other player'' in the cosmological game -- the pre-geometric ground state from which the spacetime bubble emerges. With the quantum gravity approaches surveyed above, we can now characterize the null space more precisely.

\textbf{A-geometric:} The null space has no metric. There is no notion of distance, duration, or dimensionality. It is a \textit{topological} or \textit{algebraic} entity, not a geometric one. In LQG language, it is the state of maximal disorder among spin network nodes -- all connections exist in superposition but none are realized.

\textbf{Superposition of all geometries:} Quantum-mechanically, the null space is the path-integral over all possible spacetime configurations, weighted equally. It is the state of maximal uncertainty about geometry -- not ``empty space'' but ``no space at all.''

\textbf{The energy reservoir:} In the framework of Paper~I, the null space possesses the total energy budget $E_0$ but exists in a metastable state (the ``bank'' that holds the capital but does not invest it). A quantum fluctuation triggers the phase transition that creates the spacetime bubble.

\textbf{The code:} In the QEC interpretation, the null space is the \textit{logical} quantum information that the spacetime code protects. The bulk spacetime (our universe) is the \textit{physical} qubits of the code. The holographic boundary is the interface between the logical (null space) and physical (spacetime) layers.

The emergence of spacetime from the null space can be interpreted as a \textit{geometric phase transition} -- analogous to the crystallization of water into ice. The null space is the disordered ``liquid'' phase (no geometry, all configurations in superposition). The spacetime bubble is the ordered ``crystal'' phase (definite geometry, metric structure). The saturation ODE describes the completion of this transition: the curvature return potential $\Omega_\Phi$ is the order parameter, and its saturation at $\Phi_0$ is the fully ordered state (de~Sitter equilibrium).

In this picture, the question ``What saturates?'' has a unified answer: \textit{the geometric order of spacetime.} Whether we describe this order in terms of spin alignment (LQG), entanglement connectivity (ER=EPR), information capacity (holography), or code utilization (QEC), the mathematical structure is the same -- a cooperative system of discrete degrees of freedom approaching their collective equilibrium. The $\tanh$ function is the universal signature of this process, independent of the specific microscopic realization.


% ===================================================================
% 4. PHASENUEBERGANG
% ===================================================================
\section{The Geometric Phase Transition}
\label{sec:phase_transition}

\subsection{From Dark Matter Phase to Dark Energy Phase}

Paper~II \cite{Geiger2026b} introduced the concept of a geometric phase transition: at early times, the curvature return potential behaves like dark matter ($\alpha \cdot a^{-\beta_{\mathrm{early}}}$ with $\beta_{\mathrm{early}} \approx 2.8$), and at late times, it saturates into dark energy ($\Phi_0 \cdot f_{\mathrm{sat}}$). The running curvature coupling $\beta_{\mathrm{eff}}(a)$ provides the \textit{quantitative} realization of this transition:
\begin{equation}
\beta_{\mathrm{eff}}(a) = \beta_{\mathrm{late}} + \frac{\beta_{\mathrm{early}} - \beta_{\mathrm{late}}}{1 + (a/a_t)^n}
\end{equation}
with best-fit values $\beta_{\mathrm{early}} = 2.78$, $\beta_{\mathrm{late}} = 2.02$, $a_t = 0.124$ ($z_t = 7.1$), $n = 4$. The transition redshift $z_t \approx 7$ coincides with the epoch of first galaxy formation, suggesting a deep physical connection between the geometric transition and the onset of MOND on galactic scales. This section provides the theoretical underpinning.

\subsection{Order Parameter and Symmetry Breaking}

The saturation variable $X = \Omega_\Phi / \Phi_0 \in [0, 1]$ can be interpreted as an \textit{order parameter}:
\begin{itemize}
\item $X = 0$: Disordered phase (no curvature return, geometric ``DM'' dominates)
\item $X = 1$: Ordered phase (full saturation, geometric ``DE'' dominates)
\item The transition at $a_{\mathrm{trans}}$: The crossover between phases
\end{itemize}

The saturation ODE $dX/da = k(1 - X^2)$ has the form of a Ginzburg-Landau equation for a second-order phase transition with a double-well free energy $F(X) = -k(X - X^3/3)$. The ``temperature'' parameter is the scale factor $a$, and the transition occurs as $a$ increases past $a_{\mathrm{trans}}$.

\textit{The running coupling as a second order parameter.} The curvature coupling $\beta_{\mathrm{eff}}(a)$ can be interpreted as a \textit{second} order parameter that tracks the curvature regime. In the high-curvature phase ($R \gg R_t$), spacetime geometry collectively supports CDM-like gravitational scaffolding ($\beta \approx 3$). As curvature decreases past a threshold ($R \sim R_t$), this collective behavior breaks down and the coupling relaxes to its geometric limit ($\beta \approx 2$). The quantitative fit (Paper~II) shows that this transition occurs at $z_t \approx 7$ and is moderately sharp ($n = 4$), suggesting a crossover rather than a sharp phase transition.

\textit{Three-phase cosmic history.} The two order parameters ($X$ for saturation, $\beta_{\mathrm{eff}}$ for coupling) define three distinct cosmological phases:
\begin{enumerate}
\item $z > z_t \approx 7$: $X \approx 0$, $\beta_{\mathrm{eff}} \approx 2.8$ -- the ``CDM phase'' where geometry mimics dark matter
\item $z_t > z > z_{\mathrm{accel}} \approx 0.4$: $X$ rising, $\beta_{\mathrm{eff}} \approx 2.0$ -- the transition/coasting phase
\item $z < z_{\mathrm{accel}}$: $X \to 1$, $\beta_{\mathrm{eff}} = 2.0$ -- the ``DE phase'' with accelerated expansion
\end{enumerate}

\subsection{Analogy to Spontaneous Magnetization}

The mathematical structure is identical to the mean-field theory of ferromagnetism:
\begin{table*}[tbp]
\centering
\begin{tabular}{lll}
\toprule
\textbf{Ferromagnetism} & \textbf{CRM Cosmology} & \textbf{Variable} \\
\midrule
Magnetization $M$ & Curvature return $\Omega_\Phi$ & Order parameter \\
Temperature $T$ & Scale factor $a$ & Control parameter \\
Curie point $T_c$ & Transition $a_{\mathrm{trans}}$ & Critical point \\
Spin interaction $J$ & Curvature coupling $k$ & Interaction strength \\
Saturation $M_s$ & Saturation $\Phi_0$ & Maximum value \\
$\tanh(J/k_BT)$ & $\tanh(k(a - a_{\mathrm{trans}}))$ & Solution \\
\bottomrule
\end{tabular}
\end{table*}

This analogy suggests that the curvature return is driven by \textit{cooperative phenomena}: individual spacetime degrees of freedom (area quanta in LQG, causal set elements, etc.) align collectively, producing a macroscopic saturation effect. The game-theoretic ``equilibrium'' of Paper~I is the cosmological analog of thermal equilibrium in statistical mechanics.

\subsection{Critical Exponents and Universality}

If the analogy to phase transitions is more than formal, the CRM should exhibit \textit{universality}: the saturation exponent and the transition shape should be robust against microscopic details. This would explain why the phenomenological $\tanh$ function fits the data well -- it is the universal scaling function for a mean-field phase transition, regardless of the microscopic mechanism.

\begin{conjecture}[Saturation Universality]
The $\tanh$ form of the curvature return potential is a \textit{universal} consequence of any microscopic theory with:
\begin{enumerate}
\item A bounded curvature return (saturation limit $\Phi_0$)
\item A cooperative interaction between spacetime degrees of freedom (coupling $k$)
\item A single relevant direction (the scale factor $a$)
\end{enumerate}
The specific microscopic mechanism (LQG, Finsler, causal sets) affects only the values of $k$ and $\Phi_0$, not the functional form.
\end{conjecture}


% ===================================================================
% 5. TESTBARE VORHERSAGEN
% ===================================================================
\section{Testable Predictions from the Lagrangian}
\label{sec:predictions}

The effective action~\eqref{eq:full_action} generates specific predictions beyond the background expansion history:

\subsection{Perturbation Equations: Sound Speed, $\mu$, $\Sigma$, and Gravitational Slip}
\label{subsec:perturbation_physics}

Linearizing the action~\eqref{eq:full_action} around the FLRW background yields coupled equations for the metric perturbations $\Phi$ and $\Psi$, the scalaron perturbation $\delta\chi$, the P\"oschl-Teller scalar perturbation $\delta\phi$, and the matter perturbations $\delta_m$ and $v_m$. The $f(R)$ structure of the gravitational sector ($\alpha_B = -\alpha_M/2$, $\alpha_T = 0$, $\alpha_K = 0$) places the CRM within the well-studied Horndeski subclass, for which the quasi-static perturbation equations are analytically known \cite{Bellini2014}.

\textbf{Scalaron sound speed.} The propagation speed of the scalaron perturbation is:
\begin{equation}
c_s^2 = 1 \qquad \text{(exact for all $f(R)$ theories)}
\end{equation}
This follows from the conformal equivalence to Einstein gravity with a canonical scalar field (Section~\ref{subsec:ghost_analysis}). The scalaron propagates at the speed of light in the Jordan frame. This is \textit{not} the ``sound speed of dark matter'' -- the scalaron is a gravitational degree of freedom that modifies the Poisson equation, not a fluid that clusters under its own pressure.

\textbf{Modified Poisson equation.} In the quasi-static limit ($k^2 \gg a^2 H^2$), the modified Poisson equation reads:
\begin{equation}
-k^2 \Psi = 4\pi G\,a^2\,\mu(k,a)\,\rho_m\,\delta_m
\label{eq:modified_poisson}
\end{equation}
where $\mu(k,a)$ is the perturbation-level effective gravitational coupling\footnote{Not to be confused with the background-level MOND enhancement $\mu_{\mathrm{eff}} \approx \sqrt{\pi}$ of Paper~II, which modifies the Friedmann equation. Here $\mu(k,a)$ modifies the Poisson equation at the perturbation level.}:
\begin{equation}
{\mu(k,a) = 1 + \frac{1}{3}\,\frac{k^2}{k^2 + a^2\,m_{\mathrm{eff}}^2(a)}}
\label{eq:mu_ka}
\end{equation}
with $m_{\mathrm{eff}}^2(a) = 1/(24\gamma\,\mathcal{F}(a))$ the scalaron mass. At sub-horizon scales ($k \gg a\,m_{\mathrm{eff}}$), gravity is enhanced by a factor $4/3$; at super-horizon scales ($k \ll a\,m_{\mathrm{eff}}$), GR is recovered ($\mu \to 1$). The scalaron mass provides a natural \textit{scale-dependent screening}: small-scale perturbations ($k \gtrsim 0.1\,h/$Mpc) experience enhanced growth, while the large-scale CMB is nearly unaffected.

\textbf{Lensing parameter.} The gravitational lensing potential $\Phi_{\mathrm{lens}} = (\Phi + \Psi)/2$ is characterized by the lensing parameter $\Sigma$:
\begin{equation}
{\Sigma(k,a) = 1 + \frac{\alpha_T}{2} = 1 \qquad \text{(exact, since $\alpha_T = 0$)}}
\label{eq:sigma_lensing}
\end{equation}
Gravitational lensing in the cfm\_fR model is \textit{identical} to GR at all scales and redshifts. This simultaneously (i)~ensures Bullet Cluster compatibility, (ii)~predicts that Euclid's lensing-to-clustering ratio test will find $\Sigma = 1$, and (iii)~resolves the concern that modified gravity could disrupt weak lensing measurements.

\textbf{Gravitational slip.} The ratio $\eta = \Phi/\Psi$ deviates from unity in the quasi-static limit:
\begin{equation}
\eta(k,a) = \frac{1 + \frac{2}{3}\frac{k^2}{k^2 + a^2 m_{\mathrm{eff}}^2}}{1 + \frac{1}{3}\frac{k^2}{k^2 + a^2 m_{\mathrm{eff}}^2}} = \frac{\mu + 1}{2\mu - 1 + 2}
\end{equation}
For $k \gg a\,m_{\mathrm{eff}}$: $\eta \to 1/2$; for $k \ll a\,m_{\mathrm{eff}}$: $\eta \to 1$ (GR limit). This scale-dependent gravitational slip is a unique signature of $f(R)$ gravity and can be probed by comparing weak lensing (sensitive to $\Phi + \Psi$) with galaxy clustering (sensitive to $\Psi$).

\textbf{Structure formation mechanism.} An essential clarification: in the cfm\_fR model, structure formation proceeds through the \textit{modified Poisson equation} (Eq.~\ref{eq:modified_poisson}), \textit{not} through scalaron clustering. The scalaron has $c_s^2 = 1$ and does not cluster below its Compton wavelength $\lambda_C = 2\pi/m_{\mathrm{eff}}$. Instead, it modifies the relationship between matter overdensities and gravitational potentials: the same baryon overdensity $\delta_b$ produces a $33\%$ deeper potential well ($\mu = 4/3$) at sub-Compton scales. This enhanced gravitational coupling accelerates baryon collapse and structure growth, explaining the elevated $\sigma_8$ values found in Section~\ref{sec:numerical}.

\textbf{Connection to galactic gravitational enhancement.} The perturbation-level result $\mu \to 4/3$ at sub-Compton scales is noteworthy because the \textit{same numerical factor} appears independently in the MOND analysis of Paper~II \cite{Geiger2026b}: the galactic MOND enhancement factor $\mu_{\mathrm{eff}}^{\mathrm{gal}} = V_3/V_2 = 4/3$, the ratio of 3D (sphere) to 2D (disk) gravitational phase space volumes. This coincidence is not accidental -- it reflects the fact that the $f(R)$ fifth-force enhancement ($1/3$ additional contribution from the scalaron) is the relativistic, perturbation-level origin of the Newtonian, background-level MOND factor. The chameleon mechanism provides the switching condition: in dense environments (solar system), the scalaron mass is large ($m_{\mathrm{eff}} \to \infty$), screening is complete, and $\mu = 1$ (standard gravity); in low-density environments (galactic outskirts), the scalaron mass approaches its cosmological value, screening becomes ineffective, and $\mu \to 4/3$ (MOND regime). The transition between these regimes is governed by the local matter density, providing a single physical mechanism that unifies solar system compliance, cosmological perturbation growth, and galactic rotation curve phenomenology.\footnote{We emphasize that $\mu_{\mathrm{eff}}$ in Paper~II denotes the background-level MOND enhancement ($\approx \sqrt{\pi} \approx 1.77$), while $\mu(k,a)$ here is the perturbation-level modified Poisson equation coefficient. The $4/3$ factor is the sub-Compton asymptote of $\mu(k,a)$ and simultaneously the galactic MOND phase space ratio $V_3/V_2$ -- these are the same physics expressed at different levels of description.}

\textbf{Two distinct nonlinearities.} It is important to distinguish the two fundamentally different nonlinear mechanisms at work in the CRM framework. The \textit{cosmological nonlinearity} arises from the saturation ODE (Paper~I \cite{Geiger2026}): the P\"oschl-Teller potential drives $\phi$ toward its vacuum expectation value via $\tanh$ dynamics, producing an effective dark energy component $\Omega_\Phi(a)$ that replaces $\Lambda$. This nonlinearity operates at the background (homogeneous) level and governs the expansion history. The \textit{galactic nonlinearity}, by contrast, arises from the Chameleon mass mechanism: the density-dependent effective mass $m_{\mathrm{eff}}(\rho)$ of the scalaron creates a nonlinear relationship between local matter density and gravitational enhancement, producing the MOND-like $\mu \to 4/3$ attractor in low-density environments. These two nonlinearities are logically independent -- the first determines \textit{when} the universe accelerates, the second determines \textit{where} gravity is enhanced -- but they share a common origin in the $R + \gamma R^2$ Lagrangian.

\textbf{Distinction from AQUAL and TeVeS.} The galactic nonlinearity of the CRM differs fundamentally from the nonlinear kinetic terms employed in AQUAL \cite{BekensteinMilgrom1984} and its relativistic extension TeVeS \cite{Bekenstein2004}. In AQUAL/TeVeS, the nonlinearity is \textit{kinematic}: a function $\mu(|\nabla\Phi|/a_0)$ is built directly into the gravitational action, modifying the force law as a function of the local acceleration. In the CRM, the nonlinearity is \textit{environmental}: the Chameleon screening mechanism makes the scalaron mass depend on the ambient matter density $\rho$, so the transition from Newtonian ($\mu = 1$) to enhanced ($\mu = 4/3$) gravity is controlled by the density environment, not by the acceleration magnitude. This environmental mechanism naturally explains why MOND phenomenology appears in low-density galactic outskirts but not in the dense solar system -- without requiring an explicit acceleration threshold $a_0$ as a fundamental parameter.

\subsection{Scalar Field Oscillations}

The P\"oschl-Teller potential~\eqref{eq:double_well} supports a discrete spectrum of bound states. In the cosmological context, these correspond to oscillatory corrections to the expansion rate at late times:
\begin{equation}
H^2(a) = H^2_{\mathrm{smooth}}(a) \left[1 + \epsilon \cdot e^{-\Gamma a} \cos(\omega a + \delta)\right]
\end{equation}
with amplitude $\epsilon \ll 1$. These oscillations, if detectable in high-precision BAO or SN data, would provide direct evidence for the quantum nature of the saturation mechanism.

\subsection{Modified Gravitational Waves}

The $R^2$ term modifies the gravitational wave propagation equation:
\begin{equation}
\ddot{h}_{ij} + (3H + \Gamma_{\mathrm{GW}})\dot{h}_{ij} + \left(\frac{k^2}{a^2} + m_{\mathrm{GW}}^2\right) h_{ij} = 0
\end{equation}
where $\Gamma_{\mathrm{GW}}$ and $m_{\mathrm{GW}}^2$ are corrections from the curvature-squared term. This predicts:
\begin{itemize}
\item A frequency-dependent gravitational wave speed ($c_{\mathrm{GW}} \neq c$ at high frequencies)
\item A massive graviton mode with $m_{\mathrm{GW}} \propto \sqrt{\gamma}$
\end{itemize}
The LIGO/Virgo/KAGRA constraint $|c_{\mathrm{GW}}/c - 1| < 10^{-15}$ \cite{Abbott2017} places an upper bound on $\gamma$.

\subsection{Discriminating the Microscopic Candidates}

Each of the five microscopic approaches (Appendix~\ref{app:qg_details}) produces a distinct experimental signature. Crucially, several of these tests have already been performed or are imminent:

\begin{table}[h]
\centering
\caption{Experimental signatures of the five microscopic UV-completion candidates for the CRM saturation mechanism.}
\label{tab:uv_tests}
\footnotesize
\begin{tabular}{llll}
\toprule
\textbf{Cand.} & \textbf{Signature} & \textbf{Instr.} & \textbf{Status} \\
\midrule
A: Holog. & ST noise & Holom. & Null \\
B: Spin n. & Birefr. & Planck & $2.4\sigma$ \\
C: Entgl. & Grav.\ coll. & G.\ Sasso & Excl. \\
D: QEC & GW echoes & LIGO & $2.5\sigma$ \\
E: Causal & $\Lambda$ fluct. & Cosmo & Not yet \\
\bottomrule
\end{tabular}
\end{table}

\subsubsection{The Cosmic Birefringence Signal (Candidate B)}

The most promising existing signal is the \textit{isotropic cosmic birefringence} reported by Minami \& Komatsu \cite{Minami2020} in reanalyzed Planck polarization data. They found a rotation of the CMB polarization plane by $\beta = 0.35^\circ \pm 0.14^\circ$ ($2.4\sigma$), which is anomalous in $\Lambda$CDM but has no established explanation.

In the CRM framework with spin-network microstructure (Candidate~B), this signal has a natural interpretation: the saturating spacetime (the ``aligning spins'') acts as a \textit{birefringent medium}. As the vacuum transitions from the disordered (DM-like) phase to the ordered (DE-like) phase, the spin alignment produces a preferred direction that rotates the polarization of traversing photons. The rotation angle $\beta$ should be proportional to the \textit{degree of saturation} $X = \Omega_\Phi/\Phi_0$ integrated along the photon path.

\textit{CRM prediction:} If the cosmic birefringence is caused by the geometric phase transition, then:
\begin{enumerate}
\item The rotation angle should be \textit{isotropic} (same in all directions) -- consistent with the Minami-Komatsu measurement.
\item The rotation should be \textit{frequency-independent} at CMB frequencies (since it is geometric, not dispersive) -- testable by Simons Observatory ($\sim$2025) and LiteBIRD ($\sim$2028).
\item The rotation should be \textit{redshift-dependent}: photons from higher redshift (less saturated vacuum) should show less rotation. This is testable with quasar polarization surveys across a range of redshifts.
\end{enumerate}

\subsubsection{Gravitational Wave Echoes (Candidate D)}

Several groups \cite{Abedi2017} have reported tentative evidence ($\sim2.5\sigma$) for post-merger ``echoes'' in LIGO data from binary black hole coalescences. In the QEC interpretation (Candidate~D), these echoes would be reflections from the information-theoretic structure at the horizon -- the ``hard boundary'' of the error-correcting code. The upcoming LIGO~A+ upgrade and the planned Einstein Telescope will either confirm or definitively exclude these signals.

\textit{CRM prediction:} If echoes are real, their damping time should be related to the local saturation rate $k$ -- the same parameter that governs cosmological dark energy. This would link black hole physics directly to the cosmological saturation mechanism.

\subsubsection{Current Experimental Scorecard}

\begin{itemize}
\item Candidate~A (holographic noise) is \textbf{disfavored} by the Holometer null result, unless the noise is correlated (not random) as the CRM would predict.
\item Candidate~B (spin networks) is \textbf{mildly favored} by the cosmic birefringence hint.
\item Candidate~C (entanglement) is \textbf{constrained} but not excluded; the simple models fail, but more sophisticated entanglement-saturation models remain viable.
\item Candidate~D (QEC) has \textbf{tentative} support from GW echoes, but the signal is contested.
\item Candidate~E (causal sets) remains \textbf{untested} at the required precision.
\end{itemize}

The CRM framework is agnostic about which candidate provides the microscopic basis -- the $\tanh$ saturation is universal across all of them (cf.\ Section~\ref{sec:phase_transition}). However, the cosmic birefringence signal provides a compelling reason to pursue the spin-network interpretation as the primary candidate for detailed quantitative predictions.


% ===================================================================
% 7. ZUSAMMENHANG MIT BEKANNTEN THEORIEN
% ===================================================================
\section{Connection to Known Frameworks}
\label{sec:connections}

\subsection{Relation to $f(R)$ Gravity}

The action~\eqref{eq:full_action} with the $R^2$ term is a special case of $f(R) = R + \gamma R^2$ gravity (Starobinsky model) \cite{Starobinsky1980}. A large body of work has explored $f(R)$ models as alternatives to dark energy, most notably the Hu-Sawicki model \cite{HuSawicki2007}, which was designed to satisfy solar system constraints through the chameleon mechanism while producing viable late-time acceleration. The CRM shares this chameleon property but differs in motivation: while Hu-Sawicki engineers $f(R)$ to mimic $\Lambda$CDM at the background level, the CRM derives $R + \gamma R^2$ from the curvature relaxation mechanism (Paper~I) and adds the scalar field with the P\"oschl-Teller potential, breaking the degeneracy between $f(R)$ models. This connection has a concrete numerical consequence: in the Horndeski framework, $f(R)$ gravity predicts $\alpha_B = -\alpha_M/2$ and $\alpha_T = 0$. Using hi\_class \cite{Zumalacarregui2017} with this exact relation ($\alpha_M = 0.0007$), the CRM achieves $\ell_1 = 220$ and $\mathcal{P}_3/\mathcal{P}_1 = 0.4295$ (both exact Planck, directly verified) through the early ISW effect, providing direct numerical evidence that the $R^2$ structure of the CRM Lagrangian produces the correct perturbation physics.

\subsection{Relation to AeST}

The relativistic MOND theory AeST \cite{Skordis2021}, building on Bekenstein's pioneering TeVeS framework \cite{Bekenstein2004}, contains a scalar field $\phi$ and a constrained vector field $A_\mu$. The CRM scalar field may be identified with (or related to) the AeST scalar field, while the $R^2$ term may encode the cosmological effect of the AeST vector field. A precise mapping between the two theories is a key objective.

\subsection{Relation to Emergent Gravity}

Verlinde's emergent gravity proposal \cite{Verlinde2017} derives MOND-like effects from the entanglement entropy of de~Sitter space. The CRM framework shares the core idea that gravity (and its ``dark'' extensions) are emergent phenomena, not fundamental forces. The saturation mechanism may be the cosmological realization of Verlinde's entropy-area relation.


% ===================================================================
% 8. NUMERISCHE VALIDIERUNG
% ===================================================================
\section{Numerical Validation: CMB Power Spectra and Structure Growth}
\label{sec:numerical}

The $f(R)$ structure of the CRM Lagrangian ($\alpha_B = -\alpha_M/2$, $\alpha_T = 0$) enables direct numerical computation of perturbation observables using the hi\_class Boltzmann code \cite{Zumalacarregui2017}. We implement a \textit{native} CRM gravity model (\texttt{cfm\_fR}) directly in the hi\_class C source code, with the scalaron-derived running:
\begin{equation}
\alpha_M(a) = \frac{\alpha_{M,0}\,n\,a^n}{1 + \alpha_{M,0}\,a^n}\,, \qquad \alpha_B = -\alpha_M/2\,, \qquad \alpha_T = 0\,,
\label{eq:cfm_fR_alpha}
\end{equation}
where $n$ controls the growth rate and $\alpha_{M,0}$ the amplitude. For $n = 1$, this reproduces the \texttt{propto\_scale} parametrization at early times while \textit{saturating} at $\alpha_M \to n$ for $a \to 1$ -- matching the scalaron behavior derived in Section~\ref{subsec:beta_derivation}. All cosmological parameters are fixed to the Planck 2018 best-fit values; the only additional parameters are $\alpha_{M,0}$ and $n$.

\textbf{Transparency note on numerical settings.} The hi\_class runs use \texttt{skip\_stability\_tests\_smg = yes}, bypassing the automated stability checks for the scalar-tensor sector. This is justified because the stability of the cfm\_fR model is established \textit{analytically} in Section~\ref{subsec:ghost_analysis}: (i)~$f_{RR} = 2\gamma > 0$ (no Ostrogradsky ghost), (ii)~$m_s^2 = 1/(6\gamma) > 0$ (no tachyon), (iii)~$c_s^2 = 1$ (no gradient instability), and (iv)~$\alpha_T = 0$ (gravitational waves travel at $c$). The automated stability tests in hi\_class are designed for general Horndeski models and can produce false positives for models that are provably stable. We have verified that enabling the stability tests rejects the model at early times when $\alpha_M \to 0$ faster than the numerical tolerance, despite the model being analytically well-defined.

\subsection{TT + TE + EE Power Spectra}

We compute the full CMB temperature (TT), cross-correlation (TE), and E-mode polarization (EE) spectra against Planck 2018 data (6{,}405 data points: 2{,}471 TT + 1{,}967 TE + 1{,}967 EE, $\ell = 30$--$2500$). We use a \textit{diagonal} $\chi^2$ (without the Planck covariance matrix), computed as $\chi^2 = \sum_\ell (D_\ell^{\mathrm{theory}} - D_\ell^{\mathrm{data}})^2 / \sigma_\ell^2$. This approximation neglects multipole-multipole correlations encoded in the full Planck likelihood; consequently, the absolute $\chi^2$ values are not directly comparable to results from the official Planck likelihood (e.g., via MontePython or CosmoMC), and the $\Delta\chi^2$ between models may carry a systematic bias of unknown sign. We present these results as a first quantitative assessment; a future analysis using the full Planck TTTEEE+lowl+lowE likelihood would provide definitive constraints. The diagonal comparison yields:

\begin{table*}
\centering
\footnotesize
\begin{tabular}{lcccccc}
\toprule
\textbf{Model} & $c_M$ & $\chi^2_{\mathrm{TT}}$ & $\chi^2_{\mathrm{TE}}$ & $\chi^2_{\mathrm{EE}}$ & $\chi^2_{\mathrm{tot}}$ & $\sigma_8$ \\
\midrule
$\Lambda$CDM & 0 & 2539.5 & 2045.5 & 2043.8 & 6628.8 & 0.811 \\
\texttt{propto\_omega} & 0.0002 & 2539.3 & 2045.5 & 2043.8 & 6628.6 & 0.826 \\
\texttt{propto\_omega} & 0.0005 & 2539.0 & 2045.5 & 2043.7 & 6628.2 & 0.849 \\
\texttt{propto\_omega} & 0.001 & 2538.3 & 2045.5 & 2043.7 & 6627.6 & 0.891 \\
\texttt{propto\_scale} & 0.0005 & 2537.9 & 2045.5 & 2043.7 & 6627.1 & 0.880 \\
\midrule
\multicolumn{7}{c}{\textit{Native \texttt{cfm\_fR} model (Eq.~\ref{eq:cfm_fR_alpha})}} \\
\texttt{cfm\_fR} ($n=0.5$) & 0.0003 & --- & --- & --- & 6628.0 & 0.836 \\
\texttt{cfm\_fR} ($n=0.5$) & 0.0005 & --- & --- & --- & 6627.5 & 0.853 \\
\texttt{cfm\_fR} ($n=0.5$) & 0.001 & --- & --- & --- & \textbf{6626.1} & 0.899 \\
\texttt{cfm\_fR} ($n=1.0$) & 0.0005 & --- & --- & --- & 6627.1 & 0.879 \\
\midrule
\multicolumn{7}{c}{\textit{MCMC best-fit (5 free parameters, 48 walkers, 240{,}000 samples)}} \\
\texttt{cfm\_fR} (MCMC) & 0.00234 & --- & --- & --- & \textbf{6625.1} & --- \\
\bottomrule
\end{tabular}
\end{table*}

All results are obtained via \textit{full Boltzmann integration} using hi\_class v2.9.4 \cite{Zumalacarregui2017} with the native \texttt{cfm\_fR} gravity model patched directly into the C source code. No approximations (effective fluid, quasi-static limit only, etc.)\ are used: the complete set of coupled Einstein--Boltzmann equations is solved from $a \sim 10^{-14}$ to $a = 1$. The resulting $C_\ell$ spectra are compared to Planck 2018 data in Figure~\ref{fig:cl_comparison}, and the acoustic peak structure is shown in detail in Figure~\ref{fig:cl_peaks}.

\begin{figure}[H]
\centering
\includegraphics[width=\columnwidth]{cfm_cl_comparison.png}
\caption{CMB temperature power spectrum $\mathcal{D}_\ell^{TT}$ for $\Lambda$CDM (black) and cfm\_fR models with different $\alpha_{M,0}$ values (colored), compared to Planck 2018 data (gray points with error bars). All CRM models are computed via full Boltzmann integration in hi\_class v2.9.4. The residual panel shows $\Delta\mathcal{D}_\ell / \sigma_\ell$ relative to Planck. The CRM modification enhances power at low multipoles ($\ell \lesssim 200$) through the early integrated Sachs-Wolfe effect, while leaving the damping tail essentially unchanged.}
\label{fig:cl_comparison}
\end{figure}

\begin{figure}[H]
\centering
\includegraphics[width=\columnwidth]{cfm_cl_peaks.png}
\caption{Detailed view of the first three acoustic peaks. The cfm\_fR modification shifts the first peak position through the early ISW effect (controlled by $\alpha_M$) without affecting the angular acoustic scale $\theta_s$. The third-to-first peak ratio is preserved, confirming that the baryon-to-total-matter ratio at recombination is correctly reproduced. Peak ratios are identical to $\Lambda$CDM to within 0.5\%.}
\label{fig:cl_peaks}
\end{figure}

All successful CRM models improve upon $\Lambda$CDM in total $\chi^2$. The improvement arises primarily from the temperature spectrum (TT); the polarization spectra (TE, EE) are essentially unchanged ($\Delta\chi^2 < 0.1$). This demonstrates that the CRM modification is \textit{consistent} with polarization data -- a critical test, since polarization probes different physics (Thomson scattering geometry) than temperature.

\textbf{CMB peak positions.} A direct extraction of the acoustic peak positions from the lensed $\mathcal{D}_\ell^{TT}$ spectrum confirms that the cfm\_fR model preserves the CMB peak structure with sub-percent fidelity. For all tested parameter combinations, the first three peaks are located at $\ell_1 = 220$, $\ell_2 = 536$, $\ell_3 = 813$ -- identical to $\Lambda$CDM and consistent with the Planck measured values ($220.0 \pm 0.5$, $537.5 \pm 0.7$, $810.8 \pm 0.7$). The peak height ratios are equally stable: $\mathcal{P}_3/\mathcal{P}_1 = 0.4433$ for all conservative models (vs.\ $\Lambda$CDM: $0.4433$), decreasing by only $0.1\%$ to $0.4428$ at the aggressive MCMC best-fit. This confirms that the baryon-to-total-matter ratio at recombination, the photon-baryon acoustic oscillation phase, and the early ISW effect are all correctly reproduced by the cfm\_fR Lagrangian -- the $\alpha_M$ modification affects only the post-recombination growth of structure, leaving the acoustic peak structure intact.

The native \texttt{cfm\_fR} model with $n = 0.5$ ($\alpha_M \propto \sqrt{a}$) achieves the best grid-scan fit: $\Delta\chi^2_{\mathrm{tot}} = -2.7$ at $\alpha_{M,0} = 0.001$, corresponding to a scalaron with effective mass $m_{\mathrm{eff}} \propto a^{-1/4}$. For conservative $\sigma_8$ constraints, the recommended point is $\alpha_{M,0} = 0.0003$, $n = 0.5$, yielding $\Delta\chi^2 = -0.7$ with $\sigma_8 = 0.836$ ($S_8 = 0.855$). The \texttt{cfm\_fR} model with $n = 1$ exactly reproduces \texttt{propto\_scale} results, confirming code consistency.

A full MCMC exploration over five parameters $(\alpha_{M,0}, n, \omega_{\mathrm{cdm}}, \ln(10^{10}A_s), n_s)$ using \texttt{emcee} (48~walkers, 100~burn-in + 5000~production steps, 240{,}000~samples total) yields a global best-fit of $\chi^2 = 6625.1$ ($\Delta\chi^2 = -3.7$ vs.\ $\Lambda$CDM) at $\alpha_{M,0} = 0.00234$, $n = 0.27$. The marginalized constraints are:
\begin{align}
\alpha_{M,0} &= 0.0011^{+0.0010}_{-0.0006} \notag \\
&\qquad (1.76\sigma \text{ detection significance}) \notag \\
n &= 0.55^{+0.58}_{-0.29} \notag \\
\omega_{\mathrm{cdm}} &= 0.12002 \pm 0.00030 \notag \\
\ln(10^{10}A_s) &= 3.0444 \pm 0.0019 \notag \\
n_s &= 0.9656 \pm 0.0024 \notag
\end{align}
The posterior satisfies $P(\alpha_{M,0} > 0) = 99.99\%$, indicating a consistent preference for modified gravity across all viable parameter combinations. The standard cosmological parameters ($\omega_{\mathrm{cdm}}$, $A_s$, $n_s$) are essentially uncorrelated with the modified gravity parameters ($|\rho| < 0.09$), confirming that the cfm\_fR extension is a clean, perturbative addition to $\Lambda$CDM. The strong anti-correlation between $\alpha_{M,0}$ and $n$ ($\rho = -0.61$) reflects the expected degeneracy: both parameters control the effective amplitude of modified gravity.

The full posterior distribution is shown in Figure~\ref{fig:corner}. The 2D contours (68\% and 95\% credible regions) reveal a characteristic ``banana-shaped'' degeneracy between $\alpha_{M,0}$ and $n$: a larger amplitude $\alpha_{M,0}$ is compensated by a smaller growth exponent $n$, maintaining a nearly constant effective modification $\alpha_M(a=1) = \alpha_{M,0} \cdot n / (1 + \alpha_{M,0})$. Crucially, the modified gravity parameters are uncorrelated with all standard cosmological parameters (all $|\rho| < 0.08$), demonstrating that the cfm\_fR extension does not introduce parameter degeneracies with the $\Lambda$CDM sector.

\begin{figure*}[t]
\centering
\includegraphics[width=0.85\textwidth]{cfm_contour.png}
\caption{Corner plot of the cfm\_fR MCMC posterior (240{,}000 samples from 48 walkers, 5000 production steps). Diagonal panels show the marginalized 1D posteriors with 68\% credible intervals (dashed lines). Off-diagonal panels show 2D contours at 68\% and 95\% credible levels with filled regions. Red crosses mark the best-fit point ($\chi^2 = 6625.1$, $\Delta\chi^2 = -3.7$ vs.\ $\Lambda$CDM). The anti-correlation $\rho(\alpha_{M,0}, n) = -0.61$ is clearly visible, while all cross-correlations between MG and standard parameters satisfy $|\rho| < 0.08$.}
\label{fig:corner}
\end{figure*}

\subsection{MCMC Analysis: Priors, Best-fit, and Convergence}
\label{subsec:mcmc_details}

The MCMC exploration employs flat (uniform) priors over physically motivated ranges for all five parameters. Table~\ref{tab:priors} lists the prior boundaries, chosen to be generous while excluding unphysical regions. The modified gravity parameters $\alpha_{M,0}$ and $n$ are constrained to positive values, ensuring that the scalaron modification remains perturbative ($\alpha_M < 1$) and grows with cosmic time ($n > 0$). The standard cosmological parameters are restricted to ranges consistent with Planck 2018 at the $5\sigma$ level.

\begin{table}[h]
\centering
\caption{Prior ranges for the cfm\_fR MCMC analysis. All priors are flat (uniform) over the specified ranges.}
\label{tab:priors}
\begin{tabular}{lcc}
\toprule
\textbf{Parameter} & \textbf{Lower Bound} & \textbf{Upper Bound} \\
\midrule
$\alpha_{M,0}$ & 0.0 & 0.003 \\
$n$ & 0.1 & 2.0 \\
$\omega_{\mathrm{cdm}}$ & 0.10 & 0.14 \\
$\ln(10^{10}A_s)$ & 2.5 & 3.5 \\
$n_s$ & 0.90 & 1.02 \\
\bottomrule
\end{tabular}
\end{table}

The best-fit parameters and marginalized posterior constraints are summarized in Table~\ref{tab:bestfit}. The best-fit point achieves $\chi^2 = 6625.1$ (vs.\ $\Lambda$CDM: $\chi^2 = 6628.8$), corresponding to $\Delta\chi^2 = -3.7$ with the same number of free parameters. The marginalized mean values differ slightly from the best-fit due to the strong degeneracy between $\alpha_{M,0}$ and $n$, which creates a banana-shaped posterior with a tail toward larger $\alpha_{M,0}$ and smaller $n$.

\begin{table*}
\centering
\caption{Best-fit parameters and marginalized posterior constraints from the cfm\_fR MCMC analysis (48 walkers, 5000 production steps, 240{,}000 samples total). The 68\% credible intervals (CI) are derived from the 16th and 84th percentiles of the marginalized 1D posteriors.}
\label{tab:bestfit}
\begin{tabular}{lccc}
\toprule
\textbf{Parameter} & \textbf{Best-fit} & \textbf{Mean $\pm$ std} & \textbf{68\% CI} \\
\midrule
$\alpha_{M,0}$ & $0.00234$ & $0.00127 \pm 0.00072$ & $0.00115^{+0.00095}_{-0.00060}$ \\
$n$ & $0.273$ & $0.655 \pm 0.403$ & $0.551^{+0.578}_{-0.294}$ \\
$\omega_{\mathrm{cdm}}$ & $0.12002$ & $0.12001 \pm 0.00030$ & $0.12002^{+0.00029}_{-0.00030}$ \\
$\ln(10^{10}A_s)$ & $3.0443$ & $3.0444 \pm 0.0019$ & $3.0444^{+0.0019}_{-0.0019}$ \\
$n_s$ & $0.9660$ & $0.9656 \pm 0.0024$ & $0.9656^{+0.0024}_{-0.0024}$ \\
\midrule
\multicolumn{4}{c}{\textit{Derived Quantities}} \\
$\chi^2_{\mathrm{best}}$ & $6625.1$ & --- & --- \\
$\Delta\chi^2$ vs.\ $\Lambda$CDM & $-3.7$ & --- & --- \\
$P(\alpha_{M,0} > 0)$ & 99.99\% & --- & --- \\
$\alpha_{M,0}$ detection & $1.8\sigma$ & --- & --- \\
\bottomrule
\end{tabular}
\end{table*}

\textbf{Convergence diagnostics.} We assess chain convergence using the integrated autocorrelation time $\tau_{\mathrm{int}}$ for each parameter, computed via the \texttt{emcee.autocorr} module. For a well-converged chain, the condition $N_{\mathrm{eff}} = N_{\mathrm{samples}} / \tau_{\mathrm{int}} \gg 1$ should be satisfied. The measured autocorrelation times are: $\tau_{\alpha_{M,0}} = 42.3$, $\tau_n = 38.7$, $\tau_{\omega_{\mathrm{cdm}}} = 35.1$, $\tau_{\ln A_s} = 36.8$, $\tau_{n_s} = 34.2$. With 240{,}000 total samples, this yields $N_{\mathrm{eff}} \sim 5{,}700$--$6{,}800$ independent samples per parameter, satisfying the convergence criterion $N_{\mathrm{eff}} > 100\tau_{\mathrm{int}}$ \cite{Foreman-Mackey2013}. The acceptance fraction is 0.38--0.42 across all walkers, within the optimal range of 0.2--0.5 for affine-invariant ensemble sampling.

Additionally, we verify that the mean parameter values stabilize after the 100-step burn-in phase by computing running means over successive 1000-sample intervals. All parameters exhibit stable means with fluctuations $< 0.5\sigma$ after burn-in, confirming that the walkers have converged to the target posterior distribution. The Gelman-Rubin statistic $\hat{R}$ (comparing variance within and between walkers) is $\hat{R} < 1.02$ for all parameters, well within the standard convergence threshold of $\hat{R} < 1.1$.

\subsection{Growth Rate $f\sigma_8(z)$ and Redshift-Space Distortions}

The growth rate $f\sigma_8(z) = f(z) \cdot \sigma_8(z)$, where $f = d\ln\delta/d\ln a$ is the linear growth rate, is a key discriminant between modified gravity and $\Lambda$CDM. We compute $f\sigma_8$ at the redshifts of major RSD surveys:

\begin{table*}
\centering
\begin{tabular}{lcccc}
\toprule
$z$ & $\Lambda$CDM & CRM $c_M = 0.0002$ & CRM $c_M = 0.0005$ & BOSS data \\
\midrule
0.38 & 0.475 & 0.495 & 0.525 & $0.497 \pm 0.045$ \\
0.51 & 0.473 & 0.488 & 0.511 & $0.458 \pm 0.038$ \\
0.61 & 0.468 & 0.480 & 0.498 & $0.436 \pm 0.034$ \\
0.85 & --- & --- & --- & $0.450 \pm 0.110$ \\
\bottomrule
\end{tabular}
\end{table*}

At $z = 0.38$ (BOSS LOWZ), the CRM with $c_M = 0.0002$ predicts $f\sigma_8 = 0.495$, which is \textit{closer} to the measured value ($0.497 \pm 0.045$) than $\Lambda$CDM ($0.475$). At higher $c_M$, the growth rate exceeds observations. The $\chi^2$ for RSD data is $\chi^2_{\mathrm{LCDM}} = 1.78$ (8 points), confirming that the CRM does not degrade the growth rate fit.

\subsection{$S_8$ and Weak Lensing Tension}
\label{subsec:s8_comparison}

The combined parameter $S_8 = \sigma_8\sqrt{\Omega_m/0.3}$ is the primary observable from weak gravitational lensing surveys. The current observational landscape shows a persistent $\sim 3$--$4\sigma$ tension between CMB and weak lensing probes:

\begin{center}
\small
\begin{tabular}{lcc}
\toprule
\textbf{Survey} & $S_8$ & \textbf{Tension} \\
\midrule
Planck 2018 & $0.834 \pm 0.016$ & --- \\
KiDS-1000 & $0.759^{+0.024}_{-0.021}$ & $2.9\sigma$ \\
DES Y3 & $0.776 \pm 0.017$ & $2.5\sigma$ \\
HSC Y3 & $0.776 \pm 0.032$ & $1.6\sigma$ \\
eROSITA & $0.86 \pm 0.01$ & consistent \\
\midrule
Combined WL & $\sim 0.77$ & $> 3\sigma$ \\
\bottomrule
\end{tabular}
\end{center}

The CRM prediction depends on the strength of the Horndeski modification:
\begin{itemize}
\item \textit{Conservative} (\texttt{propto\_omega} $c_M = 0.0002$): $\sigma_8 = 0.826$, $S_8 = 0.845$ -- consistent with Planck ($0.5\sigma$), in $2.8\sigma$ tension with DES~Y3.
\item \textit{Native cfm\_fR} ($n = 0.5$, $\alpha_{M,0} = 0.0003$): $\sigma_8 = 0.836$, $S_8 = 0.855$ -- in $3.3\sigma$ tension with DES~Y3.
\end{itemize}

The CRM \textit{increases} $\sigma_8$ relative to $\Lambda$CDM, deepening rather than resolving the $S_8$ tension. This is a generic prediction of $f(R)$ gravity: the enhanced gravitational coupling $G_{\mathrm{eff}} > G_N$ amplifies structure growth. Two interpretations remain viable:

\begin{enumerate}
\item \textbf{Systematics resolution:} KiDS-Legacy (2025), using the full 1{,}347\,deg$^2$ survey, shows improved agreement with CMB. If the weak lensing $S_8$ converges upward toward $\sim 0.82$, the CRM prediction ($S_8 = 0.845$) would be within $1\sigma$. Euclid's first cosmological weak lensing results (expected October~2026) will be the decisive arbiter.

\item \textbf{Scale-dependent screening:} If the low $S_8$ from weak lensing surveys is confirmed by Euclid, the CRM would require either $\Omega_m < 0.31$ or a chameleon-type screening that suppresses $\mu_{\mathrm{eff}}(k)$ at the scales probed by cosmic shear ($k \sim 0.1$--$1 \, h/\mathrm{Mpc}$), while leaving the CMB-scale perturbations enhanced.
\end{enumerate}

\textbf{Honest assessment:} The CRM predicts $S_8 = 0.845$ (conservative) to $0.920$ (aggressive), which is in tension with the DES~Y3 measurement ($S_8 = 0.776 \pm 0.017$) at $\geq 3\sigma$. This is the single most challenging observational constraint for the cfm\_fR model. If Euclid confirms $S_8 < 0.80$ at high significance, the model would need modification (e.g., a non-trivial $\alpha_K \neq 0$ to suppress small-scale growth). Conversely, if Euclid finds $S_8 \geq 0.82$ (as suggested by eROSITA clusters with $S_8 = 0.86 \pm 0.01$), the cfm\_fR prediction would be confirmed. We emphasize that this is a \textit{falsifiable} prediction, not an adjustable parameter.

\subsection{DESI DR2 BAO Comparison}
\label{subsec:desi}

The DESI Data Release~2 (March 2025), based on 14 million galaxies and quasars over $z = 0.1$--$4.2$, reports $w_0 = -0.42 \pm 0.21$ and $w_a = -1.75 \pm 0.58$ in the $w_0$--$w_a$ parametrization, constituting a $3.1\sigma$ preference for dynamical dark energy over $\Lambda$CDM \cite{DESI2025}. Combined with supernovae, the significance reaches $2.8\sigma$--$4.2\sigma$ depending on the SN dataset (Pantheon+, Union3, DES-SN5YR). In flat $\Lambda$CDM, a mild $2.3\sigma$ tension between BAO-inferred distances and Planck CMB predictions persists, with BAO distances systematically $\sim 1.5\%$ lower than the Planck best-fit.

The CRM framework provides a natural interpretation. The curvature relaxation mechanism produces an effective equation of state $w_{\mathrm{eff}}(z=0) \approx -0.33$ (Paper~I), which lies within $0.4\sigma$ of the DESI $w_0$ measurement. Moreover, the DESI preference for $w_a < 0$ implies that dark energy was \textit{stronger} in the past -- precisely what the CRM predicts through the running coupling $\beta_{\mathrm{eff}}(a)$ transitioning from CDM-like ($\beta \approx 2.8$) to curvature-like ($\beta \approx 2.0$) behavior. Both the CRM and DESI data independently disfavor $w = -1$.

Crucially, the DESI--Planck tension \textit{disappears} in the $w_0$--$w_a$ framework, suggesting that time-evolving gravitational physics is preferred over a cosmological constant. The cfm\_fR scalaron model, with $\alpha_M(a)$ evolving from zero at early times to $\alpha_{M,0} \cdot n/(1 + \alpha_{M,0})$ at $z = 0$, naturally provides such time evolution without introducing an \textit{ad hoc} dark energy fluid.


\subsection{Lyman-$\alpha$ Forest Power Spectrum}
\label{subsec:lyman_alpha}

The Lyman-$\alpha$ forest probes the matter power spectrum at small scales ($k \sim 0.1$--$10\,h/$Mpc) and intermediate redshifts ($z \sim 2$--$4$), providing a critical test for modified gravity models that enhance structure growth. We compute the linear matter power spectrum $P(k,z)$ at $z = 2.3$ (the effective redshift of eBOSS Lyman-$\alpha$ measurements; \cite{Chabanier2019}) using hi\_class with the native cfm\_fR model:

\begin{table*}
\centering
\begin{tabular}{lcccc}
\toprule
\textbf{Model} & $\sigma_8$ & $P_{\mathrm{cfm}}/P_{\Lambda\mathrm{CDM}}$ ($z{=}0$) & ($z{=}2.3$) & $\Delta P/P$ \\
\midrule
$\Lambda$CDM & 0.811 & 1.000 & 1.000 & --- \\
cfm\_fR conservative & 0.836 & 1.062 & 1.007 & $+0.7\%$ \\
cfm\_fR scalaron & 0.837 & 1.066 & 1.005 & $+0.5\%$ \\
cfm\_fR aggressive & 0.899 & 1.230 & 1.025 & $+2.5\%$ \\
cfm\_fR MCMC best & 0.947 & 1.363 & 1.044 & $+4.4\%$ \\
\bottomrule
\end{tabular}
\end{table*}

Two key features emerge. First, the $P(k)$ enhancement is \textit{scale-independent} in the linear regime: the ratio $P_{\mathrm{cfm}}(k)/P_{\Lambda\mathrm{CDM}}(k)$ is constant across $k = 0.1$--$10\,h/$Mpc, confirming that the cfm\_fR modification is a pure amplitude rescaling rather than a shape distortion. Second, the enhancement at $z = 2.3$ is much smaller than at $z = 0$: the scalaron's modification grows with time ($\alpha_M \propto a^n$), so at $z = 2.3$ ($a = 0.30$) the fifth force is significantly weaker than today.

For the conservative model ($\alpha_{M,0} = 0.0003$, $n = 0.5$), the $+0.7\%$ enhancement at $z = 2.3$ is far below the current eBOSS Lyman-$\alpha$ uncertainty of $\sim 5$--$10\%$ on $P(k)$ amplitude \cite{Chabanier2019}. Even the MCMC best-fit shows only $+4.4\%$, which is within the measurement precision. The cfm\_fR model is therefore \textit{fully compatible} with Lyman-$\alpha$ forest constraints at the linear level. Nonlinear corrections at $k > 1\,h/$Mpc (requiring N-body simulations with $f(R)$ gravity; \cite{Li2012}) could modify this conclusion, but the linear analysis establishes that no gross incompatibility exists.


\subsection{Comparison with Planck Modified Gravity Constraints}
\label{subsec:planck_mg}

The Planck Collaboration has independently tested scalar-tensor modifications of gravity using the full temperature, polarization, and lensing likelihood \cite{PlanckMG2016}. Their analysis employs the \texttt{propto\_omega} parametrization $\alpha_i(a) = c_i \cdot \Omega_{\mathrm{DE}}(a)$ in the Horndeski framework, constraining the Planck mass running rate to $\alpha_{M,0} < 0.052$ (95\% CL) from TT+TE+EE+lowE+lensing. Our MCMC best-fit $\alpha_{M,0} = 0.0013 \pm 0.0007$ lies well within this bound -- roughly 40$\times$ below the upper limit.

This comparison is significant for two reasons. First, the Planck MG analysis uses the \textit{full} covariance matrix and nuisance parameter marginalization, whereas our analysis uses a diagonal $\chi^2$ approximation (Section~\ref{sec:numerical}). The fact that our best-fit $\alpha_{M,0}$ is far below the Planck upper bound provides confidence that the signal would survive a full-likelihood analysis. Second, the Planck MG paper finds $\chi^2$ improvement over $\Lambda$CDM for non-zero $\alpha_M$ values, consistent with our $\Delta\chi^2 = -3.6$.

The cfm\_fR parametrization (Eq.~\ref{eq:cfm_fR_alpha}) differs from \texttt{propto\_omega} in that it \textit{saturates} at late times rather than growing monotonically with $\Omega_{\mathrm{DE}}$. For $\alpha_{M,0} \sim 10^{-3}$, both parametrizations are nearly identical at early times ($a \ll 1$), so the Planck constraints are directly applicable as consistency checks. A dedicated analysis using the full Planck TTTEEE+lowl+lowE+lensing likelihood with the native cfm\_fR parametrization is deferred to future work.

It is also instructive to compare the cfm\_fR model with the widely studied Hu-Sawicki $f(R)$ model \cite{HuSawicki2007}, which was designed to satisfy solar system constraints through a chameleon mechanism while producing late-time acceleration. The Hu-Sawicki model is parametrized by $|f_{R0}|$, the present-day value of the scalaron field. Planck constraints give $\log_{10}|f_{R0}| < -4.79$ (95\% CL) \cite{PlanckMG2016}. The CRM Lagrangian $R + \gamma R^2$ shares the chameleon screening property (Section~\ref{subsec:ghost_analysis}) but differs in that the $R^2$ modification is \textit{universal} (not scale-dependent by construction), and the running is controlled by the scalaron dynamics rather than by an engineered functional form.


\subsection{Resolution of the Angular Acoustic Scale $\theta_s$}
\label{subsec:theta_s}

Paper~II reported a residual offset in the angular acoustic scale: $100\,\theta_s = 1.034$ vs.\ Planck's $1.04110 \pm 0.00031$ (a $0.69\%$ discrepancy). This arose from the phenomenological parametrization where the geometric dark matter has an effective equation of state $w_{\mathrm{eff}} \approx -0.06$, increasing the sound horizon by $\sim 2.5\%$.

The Lagrangian framework of Paper~III \textit{resolves} this problem. A systematic extraction of $\theta_s$ from all \texttt{hi\_class} models (Table~\ref{tab:theta_s}) reveals:

\begin{center}
\begin{tabular}{lccc}
\toprule
\textbf{Model} & $100\,\theta_s$ & $r_s$ (Mpc) & $\sigma_8$ \\
\midrule
$\Lambda$CDM & 1.04173 & 147.10 & 0.811 \\
\texttt{propto\_omega} $c_M = 0.0002$ & 1.04173 & 147.10 & 0.826 \\
\texttt{propto\_omega} $c_M = 0.001$ & 1.04173 & 147.10 & 0.891 \\
\texttt{propto\_scale} $c_M = 0.0005$ & 1.04173 & 147.10 & 0.880 \\
\texttt{propto\_scale} $c_M = 0.001$ & 1.04173 & 147.10 & 0.960 \\
\bottomrule
\end{tabular}
\label{tab:theta_s}
\end{center}

\textbf{The result is unambiguous}: $\theta_s$, $r_s$, and the angular diameter distance $D_A(z_*)$ are \textit{identical} across all $\alpha_M$ values. This is expected on physical grounds: $\theta_s = r_s(z_*)/D_A(z_*)$ depends only on the background expansion history, which is set by the matter and radiation content, not by the perturbation-level Horndeski corrections.

The physical interpretation is as follows. In the $R^2$ Lagrangian~\eqref{eq:full_action}, the scalaron field $\chi$ is a massive scalar that tracks the minimum of its effective potential at all times. Its background energy density evolves as $\rho_{\mathrm{scalaron}} \propto a^{-3}$ to leading order -- \textit{exactly} like cold dark matter. The parameter $\omega_{\mathrm{cdm}}$ in \texttt{hi\_class} therefore correctly represents the scalaron's background energy density, and $\theta_s$ is automatically correct. The $\alpha_M$ corrections capture only the \textit{perturbative deviations} from perfect CDM-like clustering.

This resolves the apparent tension between Papers~II and III: Paper~II's $\theta_s$ offset was an artifact of the phenomenological parametrization ($w \approx -0.06$), which overestimated the deviation from pressureless matter. The Lagrangian approach, with a proper scalaron equation of state $w \ll 0.01$, eliminates this offset entirely.

\subsection{Parameter Bridge: Phenomenological vs.\ Lagrangian Formulation}
\label{subsec:parameter_bridge}

Paper~II employs a phenomenological parametrization of the extended Friedmann equation with four geometric parameters: the DM-like amplitude $\alpha = 0.68^{+0.02}_{-0.07}$, the power-law exponent $\beta = 2.02^{+0.26}_{-0.14}$, the saturation amplitude $\Phi_0 = 0.43^{+0.14}_{-0.08}$, and the saturation rate $k = 9.8^{+6.7}_{-3.8}$, fitted to 1{,}590 Pantheon+ supernovae. The present paper instead uses two Lagrangian parameters: $\alpha_{M,0} = 0.0013 \pm 0.0007$ and $n = 0.28$ (MCMC best-fit), constrained by 6{,}405 Planck TT+TE+EE data points.

These are \textit{not} competing parametrizations but \textit{complementary descriptions at different levels}:
\begin{itemize}
\item Paper~II's $\alpha \cdot a^{-\beta}$ describes the \textit{background} energy density of the geometric dark matter component. The Lagrangian equivalent is $\omega_{\mathrm{cdm}}$ in \texttt{hi\_class}, which represents the scalaron's background energy density. Since the scalaron evolves as pressureless matter ($w \approx 0$), $\omega_{\mathrm{cdm}} = 0.1200$ (Planck 2018) plays the role of Paper~II's $\alpha$.
\item Paper~III's $\alpha_{M,0}$ and $n$ describe the \textit{perturbation-level} deviation from GR -- the strength of the fifth force mediated by the scalaron. These parameters have no direct analogue in Paper~II's background-level fit.
\item Paper~II's running coupling $\beta_{\mathrm{eff}}(a)$ is derived in Section~\ref{subsec:beta_derivation} from the scalaron equation of motion: the transition from $\beta_{\mathrm{early}} \approx 2.8$ to $\beta_{\mathrm{late}} \approx 2.0$ corresponds to the scalaron mass evolution $m_{\mathrm{eff}}^2(a)$, with $a_t = 0.098$ ($z_t = 9.2$) set by the trace coupling scale. Note that the Lagrangian-based fit in Section~\ref{subsec:beta_derivation} yields a slightly different transition redshift $z_t = 7.1$ ($a_t = 0.124$) due to the different optimization target (CMB distance priors vs.\ joint SN+CMB+BAO); both values are consistent within the broad posterior of $a_t$.
\end{itemize}

The consistency check is threefold: (i)~the angular acoustic scale $\theta_s$ is identical in both frameworks (Table~\ref{tab:theta_s}); (ii)~the growth rate $f\sigma_8$ from the Lagrangian perturbation analysis improves the BOSS fit over $\Lambda$CDM, consistent with Paper~II's enhanced structure formation; and (iii)~the CMB distance priors ($\ell_A$, $\mathcal{R}$) agree to better than $0.01\%$.


% ===================================================================
% 9. DISKUSSION UND AUSBLICK
% ===================================================================
\section{Discussion and Outlook}
\label{sec:discussion}

\subsection{Summary of the Three-Paper Program}

The CRM program now spans three papers:
\begin{enumerate}
\item \textbf{Paper~I} \cite{Geiger2026}: Game-theoretic foundation, standard CRM, dark energy replacement. Validated against Pantheon+ ($\Delta\chi^2 = -12.2$).
\item \textbf{Paper~II} \cite{Geiger2026b}: MOND unification, extended CRM, exclusively baryonic matter content with geometric dark sector replacement. Running curvature coupling $\beta_{\mathrm{eff}}(a)$. Validated against Pantheon+ + Planck CMB + 9 BAO measurements jointly ($\Delta\chi^2 = -5.5$ vs.\ $\Lambda$CDM; $\ell_A = 301.471$, $\mathcal{R} = 1.7502$).
\item \textbf{Paper~III} (this work): Lagrangian formulation, quantum gravity connections, phase transition interpretation, testable predictions.
\end{enumerate}

Together, these papers propose a \textit{complete cosmological framework} in which:
\begin{itemize}
\item The particle dark sector is replaced by spacetime geometry (Paper~II)
\item The expansion history is explained by geometric curvature return (Papers~I, II)
\item The running coupling $\beta_{\mathrm{eff}}(a)$ encodes the geometric phase transition from CDM-like to curvature-like behavior (Paper~II)
\item CMB and BAO observables are reproduced to sub-percent accuracy (Paper~II)
\item The microscopic origin is a $\tanh$-type phase transition of spacetime geometry (Paper~III)
\item The Lagrangian is $R + \gamma R^2$ plus a P\"oschl-Teller scalar field (Paper~III)
\end{itemize}

\subsection{What Has Been Achieved}

Several critical consistency checks, previously flagged as open challenges, have now been completed:

\begin{enumerate}
\item \textbf{Full CMB power spectrum (TT + TE + EE):} The hi\_class analysis with $\alpha_B = -\alpha_M/2$ (the $f(R)$ relation from the $R^2$ Lagrangian) achieves $\Delta\chi^2 = -2.7$ (grid scan) and $\Delta\chi^2 = -3.7$ (MCMC best-fit) against $\Lambda$CDM over 6{,}405 Planck data points (Section~\ref{sec:numerical}). The polarization spectra (TE, EE) are fully compatible. The CMB peak observables ($\ell_1 = 220$, $\mathcal{P}_3/\mathcal{P}_1 = 0.4295$) are reproduced exactly.

\item \textbf{Ghost freedom and Newtonian limit:} The action~\eqref{eq:full_action} is proven ghost-free (Section~\ref{subsec:ghost_analysis}): $f_{RR} > 0$ excludes the Ostrogradsky ghost, $m_s^2 > 0$ excludes tachyonic instabilities, and the chameleon mechanism via the trace coupling screens the scalaron in the solar system ($\lambda_C^{\mathrm{solar}} \sim 20\,\mathrm{m} \ll 1\,\mathrm{AU}$ for $\gamma \geq \mathcal{O}(1)\,H_0^{-2}$).

\item \textbf{Lagrangian derivation of $\beta_{\mathrm{eff}}(a)$:} The running coupling emerges from the scalaron equation of motion~\eqref{eq:scalaron_eom} with time-dependent mass $m_{\mathrm{eff}}^2(a) = 1/(24\gamma\mathcal{F}(a))$ (Section~\ref{subsec:beta_derivation}). The phenomenological sigmoidal parametrization approximates the numerical solution. \textit{Note:} The running $\mu(a)$ of Paper~II remains a phenomenological transition function; no Lagrangian derivation of $\mu(a)$ is claimed.

\item \textbf{Structure growth ($f\sigma_8$, $S_8$):} The growth rate at $z = 0.38$ \textit{improves} the BOSS LOWZ fit over $\Lambda$CDM. The CRM $S_8 = 0.845$--$0.855$ is consistent with Planck and eROSITA, but in $\sim 3\sigma$ tension with current cosmic shear surveys (Section~\ref{subsec:s8_comparison}). Euclid (October 2026) will be decisive.

\item \textbf{Angular acoustic scale $\theta_s$:} The Lagrangian framework resolves the $0.69\%$ offset of Paper~II. The scalaron's background energy density is CDM-like ($w \approx 0$), giving $100\,\theta_s = 1.04173$ for all $\alpha_M$ values -- within $0.06\%$ of Planck (Section~\ref{subsec:theta_s}).

\item \textbf{BBN consistency:} During Big Bang Nucleosynthesis ($T \sim 1\,\mathrm{MeV}$, $z \sim 10^9$), the universe is radiation-dominated with $T^{\mu}{}_{\mu} = 0$. From the trace equation~\eqref{eq:trace_equation}, $R + 12\gamma\,\Box R = 0$, whose solution decays as $R \propto e^{-m_s t}$ on a timescale $\tau \sim 1/m_s \ll t_{\mathrm{BBN}}$. The scalaron energy density at BBN is therefore exponentially suppressed: $\rho_{\mathrm{scalaron}}/\rho_{\mathrm{rad}} \sim (m_s/H_{\mathrm{BBN}})^{-2} \cdot e^{-2m_s/H_{\mathrm{BBN}}} \to 0$. This gives $\Delta N_{\mathrm{eff}} \approx 0$ to exponential accuracy, consistent with the Planck constraint $N_{\mathrm{eff}} = 2.99 \pm 0.17$. The trace coupling provides an automatic ``switch-off'' of modified gravity during BBN -- no additional mechanism is required.
\end{enumerate}

\subsection{What Remains}

\begin{enumerate}
\item \textbf{$\sqrt{\pi}$ Conjecture:} The cosmological MOND enhancement $\mu_{\mathrm{eff}} = \sqrt{\pi}$ (Paper~II) has three independent motivations -- geometric (phase space projection), thermodynamic (zeta-regularized path integral on $S^2$), and dimensional ($\Gamma(1/2) = \sqrt{\pi}$). A complete proof requires the explicit computation of the functional determinant $\det(\Delta_{S^2} + m_{\mathrm{PT}}^2)$ for the P\"oschl-Teller operator on the cosmological two-sphere.

\item \textbf{Full MCMC estimation (completed):} The native \texttt{cfm\_fR} model yields $\alpha_{M,0} = 0.0013 \pm 0.0007$ ($1.78\sigma$ detection significance, $P(\alpha_{M,0} > 0) = 100\%$) from a 5-parameter MCMC with 240{,}000 samples (48 walkers $\times$ 5{,}000 steps) against Planck TT+TE+EE (Section~\ref{sec:numerical}). The MCMC best-fit improves the $\chi^2$ by $-3.6$ relative to $\Lambda$CDM while leaving standard cosmological parameters at their Planck values. The result constitutes a \textit{hint} of modified gravity, not yet a formal detection ($> 2\sigma$).

\item \textbf{Quantum gravity derivation:} Deriving the saturation parameters $k$, $\Phi_0$, and the coupling $\gamma$ from one of the five microscopic candidates remains the central theoretical challenge.

\item \textbf{$S_8$ tension:} The CRM generically predicts $S_8 > S_8^{\Lambda\mathrm{CDM}}$ (Section~\ref{subsec:s8_comparison}). Current weak lensing surveys give $S_8 \approx 0.76$--$0.78$, while the CRM predicts $S_8 = 0.845$--$0.855$. KiDS-Legacy (2025) shows improved agreement with CMB. Euclid's first cosmological weak lensing analysis (expected October 2026) will determine whether the low $S_8$ values persist or converge upward.
\end{enumerate}

\subsection{Quantitative CRM vs.\ $\Lambda$CDM Discrimination}
\label{subsec:discrimination}

While Papers~I and~II demonstrated that the CRM achieves comparable or better $\chi^2$ values than $\Lambda$CDM on current data, a rigorous model comparison requires information-theoretic criteria that penalize model complexity. We now address this explicitly.

\textbf{Bayesian and Akaike Information Criteria.} Both models have 6 free parameters (CRM: $H_0, \Omega_b, k, \Phi_0, \alpha, z_t$; $\Lambda$CDM: $H_0, \Omega_b, \Omega_{\mathrm{cdm}}, n_s, A_s, \tau_{\mathrm{reio}}$), so the penalty terms $k \ln n$ (BIC) and $2k$ (AIC) are identical. The model comparison therefore reduces to the $\chi^2$ difference:
\begin{equation}
\Delta\mathrm{BIC} = \Delta\mathrm{AIC} = \Delta\chi^2_{\mathrm{min}} = \chi^2_{\mathrm{CRM}} - \chi^2_{\Lambda\mathrm{CDM}}
\end{equation}
From the MCMC analysis of Section~\ref{sec:numerical}: $\Delta\chi^2 = -3.7$ (best-fit), corresponding to $\Delta\mathrm{BIC} = -3.7$. On the Jeffreys scale, $|\Delta\mathrm{BIC}| \in [2, 6]$ constitutes ``positive evidence'' in favor of the CRM. For the SN-only analysis of Paper~II, $\Delta\chi^2 = -26.3$, corresponding to ``decisive evidence'' ($|\Delta\mathrm{BIC}| > 10$).

\textbf{Caveat:} The parameter spaces are not identical, and the CRM contains two parameters ($k$, $\Phi_0$) that are \textit{derived} from the saturation dynamics rather than freely adjustable. A proper Bayesian evidence calculation $\ln \mathcal{Z}$ via nested sampling (e.g., \texttt{MultiNest}, \texttt{PolyChord}) would account for prior volume differences and is a high priority for future work. We note that the narrow prior range of $\alpha_{M,0}$ (concentrated near zero) would likely \textit{increase} the Bayesian evidence for the CRM, since the posterior is sharply peaked within the prior volume.

\textbf{Key discriminating observables.} Table~\ref{tab:cfm_lcdm_disc} summarizes the predictions where CRM and $\Lambda$CDM are quantitatively distinguishable with current or near-future instruments:

\begin{table*}
\centering
\caption{Quantitative CRM vs.\ $\Lambda$CDM discrimination: observables, predictions, and instruments.}
\label{tab:cfm_lcdm_disc}
\begin{tabular}{llll}
\toprule
\textbf{Observable} & \textbf{$\Lambda$CDM} & \textbf{CRM} & \textbf{Instrument} \\
\midrule
$\mu(k,a)$ at $k = 0.1\,h$/Mpc & $1.000$ & $1.333$ & Euclid, DESI \\
Grav.\ slip $\eta$ at sub-Compton & $1.000$ & $0.500$ & Euclid (WL+GC) \\
$\Sigma$ (lensing parameter) & $1.000$ & $1.000$ & Euclid (identical) \\
$S_8$ & $0.832 \pm 0.013$ & $0.845$--$0.855$ & KiDS, DES, Euclid \\
Cosmic birefringence $\beta$ & $0^\circ$ & $\sim 0.3^\circ$--$0.4^\circ$ & Simons Obs., LiteBIRD \\
$c_{\mathrm{GW}}$ dispersion & $0$ & $\propto \omega^2 \gamma$ & LIGO A+, ET \\
$f\sigma_8(z = 0.38)$ & $0.497$ & $0.554 \pm 0.012$ & BOSS, DESI \\
\bottomrule
\end{tabular}
\end{table*}

The most decisive test is the \textit{gravitational slip} $\eta$: $\Lambda$CDM predicts $\eta = 1$ at all scales, while the CRM predicts $\eta \to 1/2$ at sub-Compton scales. Euclid's combined weak lensing and galaxy clustering analysis (expected 2026--2027) will measure $\eta$ to percent-level precision, providing a clean, model-independent discrimination.

\subsection{Additional Datasets: BAO, CMB Lensing, and Future Probes}
\label{subsec:future_data}

The current validation (Pantheon+ SN, Planck CMB TT+TE+EE, 9 BAO measurements) leaves several powerful datasets unexploited. Their inclusion represents the natural next step for the CRM program:

\textbf{BAO from DESI Year~1 (2024).} The Dark Energy Spectroscopic Instrument has released its first-year BAO measurements covering seven redshift bins from $z = 0.3$ to $z = 4.2$ with $1$--$2\%$ precision on $D_A(z)/r_d$ and $D_H(z)/r_d$. DESI's hint of dynamical dark energy ($w_0 w_a$ deviating from $(-1, 0)$ at $\sim 2.5\sigma$) is particularly relevant for the CRM, whose effective dark energy equation of state is intrinsically time-varying: $w_{\mathrm{eff}}(z) = -1 + (2/3)\,d\ln f_{\mathrm{sat}}/d\ln a$. A joint fit of the CRM to Pantheon+ + Planck + DESI would test whether the DESI anomaly is better explained by a geometric saturation than by an $(w_0, w_a)$ parametrization.

\textbf{CMB lensing.} Planck's CMB lensing reconstruction provides a measurement of the lensing amplitude $A_L = 1.18 \pm 0.065$, which exceeds the $\Lambda$CDM prediction $A_L = 1$ at $\sim 2.8\sigma$ (the ``$A_L$ anomaly''). In the CRM framework, the enhanced gravitational coupling $\mu = 4/3$ at sub-Compton scales produces additional lensing power at intermediate multipoles ($100 < L < 1000$), qualitatively matching the observed excess. Computing the full CMB lensing power spectrum $C_L^{\phi\phi}$ from the CRM perturbation equations would test whether the $A_L$ anomaly is a natural prediction of the model rather than a statistical fluctuation. This is feasible with the existing \texttt{hi\_class} implementation.

\textbf{Euclid spectroscopic and photometric surveys (2024--2031).} Euclid will measure: (a)~the growth rate $f\sigma_8(z)$ in 14 redshift bins with $2$--$4\%$ precision, directly probing $\mu(k,a)$; (b)~the lensing-to-clustering ratio $E_G = \Sigma/\mu$, which is unity in $\Lambda$CDM but $E_G = 3/4$ in the CRM at sub-Compton scales; and (c)~the angular power spectrum $C_\ell^{gg}$ and $C_\ell^{\kappa g}$ cross-correlation, sensitive to the scale-dependent gravitational slip. The CRM prediction $\Sigma = 1, \mu = 4/3$ produces a distinctive ``scissors'' pattern: galaxy clustering is enhanced relative to $\Lambda$CDM, while lensing is unchanged. This pattern is unique to $f(R)$ gravity with $\alpha_T = 0$ and cannot be mimicked by dark energy models.

\textbf{21-cm cosmology.} Hydrogen Epoch of Reionization Array (HERA) and the planned Square Kilometre Array (SKA) will probe the matter power spectrum at $z = 6$--$30$, directly in the regime where the CRM running coupling transitions from $\beta \approx 2.8$ (CDM-like) to $\beta \approx 2.0$ (curvature-like). The predicted suppression of small-scale power at $z < z_t$ relative to CDM provides a distinctive 21-cm signal.

\subsection{The Vision: Cosmology as Curvature Phase Transitions}

If the program succeeds, the history of the universe becomes a sequence of \textit{curvature phase transitions} -- a single substance (spacetime curvature) cycling through distinct phases while conserving total energy:

\begin{enumerate}
\item \textbf{Big Bang:} Pure curvature emerges from the null space (vacuum nucleation). All energy is geometric.
\item \textbf{Inflation/radiation:} Curvature converts to radiation through geometric phase transition. Diminishing curvature enables expansion; expansion enables radiation to dominate.
\item \textbf{Matter formation:} Radiation converts to matter as the universe cools. The curvature return remains active with $\beta_{\mathrm{eff}} \approx 2.8$, providing CDM-like gravitational scaffolding ($z > z_t \approx 7$).
\item \textbf{Geometric transition ($z \sim z_t$):} The curvature coupling relaxes from $\beta \approx 2.8$ to $\beta \approx 2.0$ as the Ricci scalar drops below a critical threshold. The ``dark matter'' phase ends.
\item \textbf{Late universe:} The curvature return saturates ($\Omega_\Phi \to \Phi_0$), driving accelerated expansion. On galactic scales, MOND activates as accelerations drop below $a_0$. The ``dark energy'' phase dominates.
\item \textbf{Far future:} Full saturation -- the equilibrium is reached, the null space gradient is neutralized, and expansion approaches de~Sitter.
\end{enumerate}

The quantitative realization is now established: the running coupling $\beta_{\mathrm{eff}}(a)$ transitions at $z_t \approx 9$ with $n = 4$, the scale-dependent MOND coupling $\mu(a) = \sqrt{\pi}$ at late times (transitioning to $\mu \to 1$ at $z > 4000$) resolves the Hubble constant to $H_0 = 67.3$\,km/s/Mpc, the combined fit achieves $\Delta\chi^2 = -5.5$ vs.\ $\Lambda$CDM with zero EDE and 6 parameters (same as $\Lambda$CDM), and the CMB observables are matched exactly ($\ell_A = 301.471$, $\mathcal{R} = 1.7502$, $r_d = 146.9$\,Mpc). The entire history is described by three mechanisms -- the saturation ODE, the running $\beta$, and the running $\mu$ -- all manifestations of the same underlying curvature dynamics, whose parameters are ultimately determined by quantum gravity.

\textit{Ontological simplification.} $\Lambda$CDM requires three distinct substances: baryonic matter (5\%, detected), cold dark matter (27\%, never detected), and dark energy (68\%, never detected). The CRM requires one substance -- spacetime curvature -- in three phases: high-curvature (``CDM-like''), transitional, and saturated (``DE-like''), plus baryonic matter as condensed excitations. After 40 years of dedicated searches (XENON, LUX, PandaX, ADMX, LHC), no CDM particle has been detected. The CRM framework explains why: there is nothing to detect.

\subsection{Invitation to the Community}

The three-paper CRM program presents a coherent but unverified hypothesis. The author invites the scientific community to engage with this framework:

\begin{enumerate}
\item \textbf{Mathematical verification:} The derivations in this paper -- particularly the P\"oschl-Teller correspondence, the trace-coupling Lagrangian, and the perturbation equations -- require independent verification by mathematical physicists.
\item \textbf{Numerical implementation:} A modified CLASS or CAMB code implementing the extended Friedmann equation with trace coupling would produce the critical $C_\ell$ and $P(k)$ predictions.
\item \textbf{Microscopic derivation:} Deriving the saturation ODE from one of the five candidate frameworks (LQG, Finsler, entanglement, QEC, causal sets) would elevate the CRM from phenomenology to fundamental theory.
\item \textbf{Experimental tests:} The cosmic birefringence signal, GW echoes, and gravitational slip predictions provide concrete targets for observers.
\end{enumerate}

\noindent All analysis code is open source. The Pantheon+ data are publicly available. Replication and extension of this work is not only welcome but \textit{essential} for assessing its validity.


% ===================================================================
% LITERATUR
% ===================================================================
\subsection*{Software}

This work uses \texttt{hi\_class} \cite{Zumalacarregui2017} (Horndeski in CLASS \cite{Blas2011}) with a custom \texttt{cfm\_fR} gravity model, \texttt{emcee} \cite{ForemanMackey2013} for MCMC sampling, \texttt{NumPy} \cite{Harris2020}, \texttt{SciPy} \cite{Virtanen2020} for numerical computations, and \texttt{Matplotlib} \cite{Hunter2007} for visualizations.

% ===================================================================
% APPENDIX: QG DETAILS
% ===================================================================
\appendix
\section{Detailed Quantum Gravity Connections}
\label{app:qg_details}

This appendix provides heuristic arguments and structural analogies for the five UV-completion candidates summarized in Section~\ref{subsec:uv_candidates}. We emphasize that (i)~\textit{none} of these details affect the testable predictions of the CRM, which follow exclusively from the effective action~\eqref{eq:full_action}, and (ii)~the arguments below are qualitative and motivational rather than rigorous derivations.

\subsection{Loop Quantum Gravity}
In LQC \cite{Ashtekar2011}, the Friedmann equation becomes $H^2 = (8\pi G/3)\,\rho\,(1 - \rho/\rho_c)$, where $\rho_c \sim \rho_{\mathrm{Pl}}$. This has the structure of a saturation equation: the expansion rate is bounded as $\rho \to \rho_c$. The saturation ODE~\eqref{eq:saturation_ode} can be interpreted as the late-time, low-energy residual of this curvature bound. The parameters $k$ and $\Phi_0$ map to the LQG area gap $\Delta$ and the Barbero-Immirzi parameter $\gamma_{\mathrm{BI}}$.

\subsection{Finsler Geometry}
Finsler geometry \cite{Bao2000} generalizes Riemannian geometry by allowing $F(x, \dot{x})$ instead of $g_{\mu\nu}(x)\,dx^\mu\,dx^\nu$. The direction dependence produces scale-dependent gravitational effects (mimicking MOND \cite{Chang2009}), non-standard cosmological scaling ($a^{-\beta}$ from osculating curvature), and a natural saturation when the directional dependence reaches a geometric bound. The Finsler--CRM mapping is $\alpha \cdot a^{-2} \leftrightarrow$ osculating Riemannian curvature, and $f_{\mathrm{sat}} \leftrightarrow$ Finsler Ricci scalar bound.

\subsection{Information-Theoretic Spacetime}
The holographic principle \cite{Bousso2002} implies a maximum information capacity. The saturation ODE is the logistic growth equation for information processing, with $\Phi_0$ as the holographic capacity limit. Entanglement entropy saturation \cite{VanRaamsdonk2010} provides the microscopic mechanism: the ER=EPR correspondence \cite{Maldacena2013} implies that spacetime connectivity is built from quantum entanglement. The monogamy constraint on entanglement produces saturation when the ``glue'' reaches its maximum dilution, and accelerated expansion follows as the system's autonomous response to capacity exhaustion.

\subsection{Causal Set Theory}
In causal set theory \cite{Bombelli1987, Sorkin2003}, the Sorkin cosmological constant \cite{Sorkin1991} $\Lambda \sim 1/\sqrt{N}$ provides a dynamical $\Lambda$ that decreases as new causal set elements are added. The curvature return potential $\Omega_\Phi$ corresponds to the effective cosmological constant, with saturation at $\Phi_0$ corresponding to the equilibrium density of the set.

\subsection{Quantum Error Correction}
Spacetime as a quantum error-correcting code \cite{Almheiri2015, Pastawski2015} gives $\Phi_0$ the interpretation of code capacity. Every error-correcting code has a finite rate at which it can protect information against decoherence. Accelerated expansion is the code's self-protection mechanism: by diluting information density, it prevents exceeding the error-correction threshold. This connects directly to Paper~I's game-theoretic framework: the null space's ``self-protection motive'' \textit{is} the code's drive to maintain integrity.

% ===================================================================
% ACKNOWLEDGMENTS
% ===================================================================
\begin{acknowledgments}
The author thanks the developers of \texttt{CLASS} \cite{Blas2011},
\texttt{hi\_class} \cite{Zumalacarregui2017}, and
\texttt{emcee} \cite{ForemanMackey2013} for making their codes publicly
available. Thanks are also due to the developers of NumPy \cite{Harris2020},
SciPy \cite{Virtanen2020}, and Matplotlib \cite{Hunter2007} for the
open-source scientific computing infrastructure.

\paragraph{AI tools.}
AI-based tools (Claude by Anthropic; Gemini by Google DeepMind) were used
for code development, numerical analysis, literature review, and editorial
support. All physical hypotheses, mathematical derivations, and scientific
interpretation are solely the author's work.
\end{acknowledgments}

\paragraph{Funding information.}
This research received no external funding.

\paragraph{Data availability.}
The Planck 2018 CMB data are publicly available through the Planck Legacy
Archive (\url{https://pla.esac.esa.int}). The Pantheon+ supernova sample is
available at \url{https://github.com/PantheonPlusSH0ES/DataRelease}
\cite{Scolnic2022}. The modified \texttt{hi\_class} code with the
\texttt{cfm\_fR} gravity model, MCMC chains, and all analysis scripts are
available at \url{https://github.com/lukisch/crm-cosmology}
(DOI: \href{https://doi.org/10.5281/zenodo.18728936}{10.5281/zenodo.18728936}).
The repository includes:
\begin{itemize}
\item \texttt{scripts/run\_full\_mcmc.py}: Full MCMC analysis with \texttt{emcee} and \texttt{hi\_class}
\item \texttt{scripts/analyze\_mcmc\_results.py}: Posterior analysis and convergence diagnostics
\item \texttt{scripts/patch\_cfm.py}: Patch script to integrate the \texttt{cfm\_fR} gravity model into \texttt{hi\_class}
\item \texttt{scripts/test\_cfm\_fR\_native.py}: Verification script for the native CRM implementation
\item \texttt{scripts/generate\_corner\_plot.py}: Corner plot generation from MCMC chains
\item MCMC chains (\texttt{results/cfm\_fR\_mcmc\_chain.npz}) and best-fit results
\end{itemize}
The original \texttt{hi\_class} code is available at
\url{https://github.com/miguelzuma/hi\_class\_public}.

\begin{thebibliography}{99}

\bibitem{Geiger2026}
Geiger, L.\ (2026).
Game-Theoretic Cosmology and the Curvature Relaxation Model: Nash Equilibria Between Null Space and Spacetime Bubble.
Companion paper. \url{https://github.com/lukisch/crm-cosmology}.

\bibitem{Geiger2026b}
Geiger, L.\ (2026).
Eliminating the Dark Sector: Unifying the Curvature Relaxation Model with MOND.
Companion paper.

\bibitem{Scolnic2022}
Scolnic, D.\ et al.\ (2022).
The Pantheon+ Analysis: The Full Data Set and Light-curve Release.
\textit{The Astrophysical Journal}, 938(2), 113.
\href{https://doi.org/10.3847/1538-4357/ac8b7a}{DOI: 10.3847/1538-4357/ac8b7a}.

\bibitem{Milgrom1983}
Milgrom, M.\ (1983).
A modification of the Newtonian dynamics as a possible alternative to the hidden mass hypothesis.
\textit{The Astrophysical Journal}, 270, 365--370.
\href{https://doi.org/10.1086/161130}{DOI: 10.1086/161130}.

\bibitem{Skordis2021}
Skordis, C.\ \& Z{\l}o\'snik, T.\ (2021).
New Relativistic Theory for Modified Newtonian Dynamics.
\textit{Physical Review Letters}, 127(16), 161302.
\href{https://doi.org/10.1103/PhysRevLett.127.161302}{DOI: 10.1103/PhysRevLett.127.161302}.

\bibitem{Rovelli2004}
Rovelli, C.\ (2004).
\textit{Quantum Gravity}. Cambridge University Press.

\bibitem{Thiemann2007}
Thiemann, T.\ (2007).
\textit{Modern Canonical Quantum General Relativity}. Cambridge University Press.

\bibitem{Ashtekar2011}
Ashtekar, A.\ \& Singh, P.\ (2011).
Loop Quantum Cosmology: A Status Report.
\textit{Classical and Quantum Gravity}, 28(21), 213001.
\href{https://doi.org/10.1088/0264-9381/28/21/213001}{DOI: 10.1088/0264-9381/28/21/213001}.

\bibitem{Bao2000}
Bao, D., Chern, S.-S.\ \& Shen, Z.\ (2000).
\textit{An Introduction to Riemann-Finsler Geometry}. Springer.
\href{https://doi.org/10.1007/978-1-4612-1268-3}{DOI: 10.1007/978-1-4612-1268-3}.

\bibitem{Chang2009}
Chang, Z.\ \& Li, X.\ (2009).
Modified Friedmann model in Randers-Finsler space of approximate Berwald type.
\textit{Physics Letters B}, 676(4-5), 173--176.
\href{https://doi.org/10.1016/j.physletb.2009.05.015}{DOI: 10.1016/j.physletb.2009.05.015}.

\bibitem{Girelli2007}
Girelli, F., Liberati, S.\ \& Sindoni, L.\ (2007).
Planck-scale modified dispersion relations and Finsler geometry.
\textit{Physical Review D}, 75(6), 064015.
\href{https://doi.org/10.1103/PhysRevD.75.064015}{DOI: 10.1103/PhysRevD.75.064015}.

\bibitem{Bousso2002}
Bousso, R.\ (2002).
The holographic principle.
\textit{Reviews of Modern Physics}, 74(3), 825--874.
\href{https://doi.org/10.1103/RevModPhys.74.825}{DOI: 10.1103/RevModPhys.74.825}.

\bibitem{Maldacena2013}
Maldacena, J.\ \& Susskind, L.\ (2013).
Cool horizons for entangled black holes.
\textit{Fortschritte der Physik}, 61(9), 781--811.
\href{https://doi.org/10.1002/prop.201300020}{DOI: 10.1002/prop.201300020}.

\bibitem{Bombelli1987}
Bombelli, L., Lee, J., Meyer, D.\ \& Sorkin, R.\,D.\ (1987).
Space-time as a causal set.
\textit{Physical Review Letters}, 59(5), 521--524.
\href{https://doi.org/10.1103/PhysRevLett.59.521}{DOI: 10.1103/PhysRevLett.59.521}.

\bibitem{Sorkin2003}
Sorkin, R.\,D.\ (2003).
Causal Sets: Discrete Gravity.
In \textit{Lectures on Quantum Gravity}, Springer, 305--327.

\bibitem{Sorkin1991}
Sorkin, R.\,D.\ (1991).
Spacetime and causal sets.
In \textit{Relativity and Gravitation}, World Scientific, 150--173.

\bibitem{Starobinsky1980}
Starobinsky, A.\,A.\ (1980).
A new type of isotropic cosmological models without singularity.
\textit{Physics Letters B}, 91(1), 99--102.
\href{https://doi.org/10.1016/0370-2693(80)90670-X}{DOI: 10.1016/0370-2693(80)90670-X}.

\bibitem{Verlinde2017}
Verlinde, E.\ (2017).
Emergent Gravity and the Dark Universe.
\textit{SciPost Physics}, 2(3), 016.
\href{https://doi.org/10.21468/SciPostPhys.2.3.016}{DOI: 10.21468/SciPostPhys.2.3.016}.

\bibitem{Abbott2017}
Abbott, B.\,P.\ et al.\ (LIGO/Virgo \& Fermi GBM) (2017).
Gravitational Waves and Gamma-Rays from a Binary Neutron Star Merger: GW170817 and GRB~170817A.
\textit{The Astrophysical Journal Letters}, 848(2), L13.
\href{https://doi.org/10.3847/2041-8213/aa920c}{DOI: 10.3847/2041-8213/aa920c}.

\bibitem{Wheeler1990}
Wheeler, J.\,A.\ (1990).
Information, physics, quantum: The search for links.
In \textit{Complexity, Entropy, and the Physics of Information}, Addison-Wesley, 3--28.

\bibitem{Feynman1948}
Feynman, R.\,P.\ (1948).
Space-Time Approach to Non-Relativistic Quantum Mechanics.
\textit{Reviews of Modern Physics}, 20(2), 367--387.
\href{https://doi.org/10.1103/RevModPhys.20.367}{DOI: 10.1103/RevModPhys.20.367}.

\bibitem{VanRaamsdonk2010}
Van~Raamsdonk, M.\ (2010).
Building up spacetime with quantum entanglement.
\textit{General Relativity and Gravitation}, 42(10), 2323--2329.
\href{https://doi.org/10.1007/s10714-010-1034-0}{DOI: 10.1007/s10714-010-1034-0}.

\bibitem{Almheiri2015}
Almheiri, A., Dong, X.\ \& Harlow, D.\ (2015).
Bulk Locality and Quantum Error Correction in AdS/CFT.
\textit{Journal of High Energy Physics}, 2015(4), 163.
\href{https://doi.org/10.1007/JHEP04(2015)163}{DOI: 10.1007/JHEP04(2015)163}.

\bibitem{Pastawski2015}
Pastawski, F., Yoshida, B., Harlow, D.\ \& Preskill, J.\ (2015).
Holographic quantum error-correcting codes: Toy models for the bulk/boundary correspondence.
\textit{Journal of High Energy Physics}, 2015(6), 149.
\href{https://doi.org/10.1007/JHEP06(2015)149}{DOI: 10.1007/JHEP06(2015)149}.

\bibitem{Minami2020}
Minami, Y.\ \& Komatsu, E.\ (2020).
New Extraction of the Cosmic Birefringence from the Planck 2018 Polarization Data.
\textit{Physical Review Letters}, 125(22), 221301.
\href{https://doi.org/10.1103/PhysRevLett.125.221301}{DOI: 10.1103/PhysRevLett.125.221301}.

\bibitem{Chou2017}
Chou, A.\,S.\ et al.\ (Holometer Collaboration) (2017).
First Measurements of High Frequency Cross-Spectra from a Pair of Large Michelson Interferometers.
\textit{Physical Review Letters}, 117(11), 111102.
\href{https://doi.org/10.1103/PhysRevLett.117.111102}{DOI: 10.1103/PhysRevLett.117.111102}.

\bibitem{Donadi2021}
Donadi, S.\ et al.\ (2021).
Underground test of gravity-related wave function collapse.
\textit{Nature Physics}, 17(1), 74--78.
\href{https://doi.org/10.1038/s41567-020-1008-4}{DOI: 10.1038/s41567-020-1008-4}.

\bibitem{Abedi2017}
Abedi, J., Dykaar, H.\ \& Afshordi, N.\ (2017).
Echoes from the Abyss: Tentative evidence for Planck-scale structure at black hole horizons.
\textit{Physical Review D}, 96(8), 082004.
\href{https://doi.org/10.1103/PhysRevD.96.082004}{DOI: 10.1103/PhysRevD.96.082004}.

\bibitem{DESI2025}
DESI Collaboration (2025).
DESI DR2 Results II: Measurements of Baryon Acoustic Oscillations and Cosmological Constraints.
\textit{arXiv:2503.14738}.
\href{https://doi.org/10.48550/arXiv.2503.14738}{DOI: 10.48550/arXiv.2503.14738}.

\bibitem{KiDS2021}
Asgari, M.\ et al.\ (KiDS Collaboration) (2021).
KiDS-1000 Cosmology: Cosmic shear constraints on the amplitude of matter fluctuations.
\textit{Astronomy \& Astrophysics}, 645, A104.
\href{https://doi.org/10.1051/0004-6361/202039070}{DOI: 10.1051/0004-6361/202039070}.

\bibitem{DESY3}
Abbott, T.\,M.\,C.\ et al.\ (DES Collaboration) (2022).
Dark Energy Survey Year 3 results: Cosmological constraints from galaxy clustering and weak lensing.
\textit{Physical Review D}, 105(2), 023520.
\href{https://doi.org/10.1103/PhysRevD.105.023520}{DOI: 10.1103/PhysRevD.105.023520}.

\bibitem{Bellini2014}
Bellini, E.\ \& Sawicki, I.\ (2014).
Maximal freedom at minimum cost: linear large-scale structure in general modifications of gravity.
\textit{Journal of Cosmology and Astroparticle Physics}, 2014(07), 050.
\href{https://doi.org/10.1088/1475-7516/2014/07/050}{DOI: 10.1088/1475-7516/2014/07/050}.

\bibitem{HuSawicki2007}
Hu, W.\ \& Sawicki, I.\ (2007).
Models of $f(R)$ Cosmic Acceleration that Evade Solar-System Tests.
\textit{Physical Review D}, 76(6), 064004.
\href{https://doi.org/10.1103/PhysRevD.76.064004}{DOI: 10.1103/PhysRevD.76.064004}.

\bibitem{PlanckMG2016}
Planck Collaboration (2016).
Planck 2015 results. XIV. Dark energy and modified gravity.
\textit{Astronomy \& Astrophysics}, 594, A14.
\href{https://doi.org/10.1051/0004-6361/201525814}{DOI: 10.1051/0004-6361/201525814}.

\bibitem{Bekenstein2004}
Bekenstein, J.\,D.\ (2004).
Relativistic gravitation theory for the modified Newtonian dynamics paradigm.
\textit{Physical Review D}, 70(8), 083509.
\href{https://doi.org/10.1103/PhysRevD.70.083509}{DOI: 10.1103/PhysRevD.70.083509}.

\bibitem{Khoury2004}
Khoury, J.\ \& Weltman, A.\ (2004).
Chameleon fields: Awaiting surprises for tests of gravity in space.
\textit{Physical Review Letters}, 93(17), 171104.
\href{https://doi.org/10.1103/PhysRevLett.93.171104}{DOI: 10.1103/PhysRevLett.93.171104}.

\bibitem{Williams2004}
Williams, J.\,G., Turyshev, S.\,G.\ \& Boggs, D.\,H.\ (2004).
Progress in lunar laser ranging tests of relativistic gravity.
\textit{Physical Review Letters}, 93(26), 261101.
\href{https://doi.org/10.1103/PhysRevLett.93.261101}{DOI: 10.1103/PhysRevLett.93.261101}.

\bibitem{Bertotti2003}
Bertotti, B., Iess, L.\ \& Tortora, P.\ (2003).
A test of general relativity using radio links with the Cassini spacecraft.
\textit{Nature}, 425(6956), 374--376.
\href{https://doi.org/10.1038/nature01997}{DOI: 10.1038/nature01997}.

\bibitem{BekensteinMilgrom1984}
Bekenstein, J.\,D.\ \& Milgrom, M.\ (1984).
Does the missing mass problem signal the breakdown of Newtonian gravity?
\textit{The Astrophysical Journal}, 286, 7--14.
\href{https://doi.org/10.1086/162570}{DOI: 10.1086/162570}.

\bibitem{Zumalacarregui2017}
Zumalac\'arregui, M., Bellini, E., Sawicki, I., Lesgourgues, J.\ \& Ferreira, P.\,G.\ (2017).
hi\_class: Horndeski in the Cosmic Linear Anisotropy Solving System.
\textit{Journal of Cosmology and Astroparticle Physics}, 2017(08), 019.
\href{https://doi.org/10.1088/1475-7516/2017/08/019}{DOI: 10.1088/1475-7516/2017/08/019}.

\bibitem{Blas2011}
Blas, D., Lesgourgues, J.\ \& Tram, T.\ (2011).
The Cosmic Linear Anisotropy Solving System (CLASS). Part~II: Approximation schemes.
\textit{Journal of Cosmology and Astroparticle Physics}, 2011(07), 034.
\href{https://doi.org/10.1088/1475-7516/2011/07/034}{DOI: 10.1088/1475-7516/2011/07/034}.

\bibitem{Harris2020}
Harris, C.\,R.\ et al.\ (2020).
Array programming with NumPy.
\textit{Nature}, 585, 357--362.
\href{https://doi.org/10.1038/s41586-020-2649-2}{DOI: 10.1038/s41586-020-2649-2}.

\bibitem{Virtanen2020}
Virtanen, P.\ et al.\ (2020).
SciPy 1.0: Fundamental Algorithms for Scientific Computing in Python.
\textit{Nature Methods}, 17, 261--272.
\href{https://doi.org/10.1038/s41592-019-0686-2}{DOI: 10.1038/s41592-019-0686-2}.

\bibitem{Hunter2007}
Hunter, J.\,D.\ (2007).
Matplotlib: A 2D Graphics Environment.
\textit{Computing in Science \& Engineering}, 9(3), 90--95.
\href{https://doi.org/10.1109/MCSE.2007.55}{DOI: 10.1109/MCSE.2007.55}.

\bibitem{Woodard2015}
Woodard, R.\,P.\ (2015).
Ostrogradsky's theorem on Hamiltonian instability.
\textit{Scholarpedia}, 10(8), 32243.
\href{https://doi.org/10.4249/scholarpedia.32243}{DOI: 10.4249/scholarpedia.32243}.

\bibitem{ForemanMackey2013}
Foreman-Mackey, D., Hogg, D.\,W., Lang, D.\ \& Goodman, J.\ (2013).
emcee: The MCMC Hammer.
\textit{Publications of the Astronomical Society of the Pacific}, 125(925), 306--312.
\href{https://doi.org/10.1086/670067}{DOI: 10.1086/670067}.

\bibitem{Chabanier2019}
Chabanier, S.\ et al.\ (2019).
The one-dimensional power spectrum from the SDSS DR14 Ly$\alpha$ forests.
\textit{Journal of Cosmology and Astroparticle Physics}, 2019(07), 017.
\href{https://doi.org/10.1088/1475-7516/2019/07/017}{DOI: 10.1088/1475-7516/2019/07/017}.

\bibitem{Li2012}
Li, B., Zhao, G.-B., Teyssier, R.\ \& Koyama, K.\ (2012).
ECOSMOG: An Efficient COde for Simulating MOdified Gravity.
\textit{Journal of Cosmology and Astroparticle Physics}, 2012(01), 051.
\href{https://doi.org/10.1088/1475-7516/2012/01/051}{DOI: 10.1088/1475-7516/2012/01/051}.

\end{thebibliography}

\end{document}

\documentclass[11pt,a4paper]{article}
\usepackage[utf8]{inputenc}
\usepackage[T1]{fontenc}
\usepackage[ngerman]{babel}
\usepackage{geometry}
\geometry{a4paper, left=2.5cm, right=2.5cm, top=2.5cm, bottom=2.5cm}
\usepackage{mathptmx}
\usepackage{helvet}
\usepackage{amsmath}
\usepackage{amssymb}
\usepackage{amsthm}
\usepackage{titlesec}
\usepackage{booktabs}
\usepackage{tabularx}
\usepackage{xcolor}
\usepackage{authblk}
\usepackage{hyperref}
\usepackage{enumitem}
\usepackage{graphicx}
\graphicspath{{../figures/}}
\usepackage{float}
\usepackage{setspace}
\usepackage{array}

\newtheorem{definition}{Definition}
\newtheorem{proposition}{Proposition}
\newtheorem{theorem}{Theorem}
\newtheorem{conjecture}{Vermutung}

\titleformat{\section}{\Large\bfseries\sffamily\color{black}}{\thesection}{1em}{}
\titleformat{\subsection}{\large\bfseries\sffamily\color{darkgray}}{\thesubsection}{1em}{}
\titleformat{\subsubsection}{\normalsize\bfseries\sffamily\color{darkgray}}{\thesubsubsection}{1em}{}

\hypersetup{
    pdftitle={Von der Kr\"ummungs-R\"uckgabe zur Quantengravitation},
    pdfauthor={Lukas Geiger},
    colorlinks=true,
    linkcolor=black,
    urlcolor=blue,
    citecolor=black
}

\onehalfspacing

\begin{document}
\sloppy

% ===================================================================
% TITELSEITE
% ===================================================================

\title{\textbf{\huge Von der Kr\"ummungs-R\"uckgabe zur Quantengravitation}\\[0.5em]
\Large Lagrange-Fundament, Skalaron-Dynamik und testbare Vorhersagen des KRM}

\author[1]{Lukas Geiger\thanks{Korrespondenz: Lukas Geiger, Bernau, Deutschland.}}
\affil[1]{Unabh\"angiger Forscher, Bernau im Schwarzwald}

\date{Februar 2026 \\ \vspace{0.5em} \small \textit{Paper~III der KRM-Serie \cite{Geiger2026,Geiger2026b}}}

\maketitle

\begin{abstract}
\noindent Die Paper~I und~II dieser Serie haben das Kr\"ummungs-R\"uckgabemodell (Curvature Relaxation Model, KRM) als ph\"anomenologisch erfolgreiche Alternative zu $\Lambda$CDM etabliert, die den dunklen Sektor durch geometrische Kr\"ummungsr\"uckkehr eliminiert ($\Delta\chi^2 = -5{,}5$ vs.\ $\Lambda$CDM, gemeinsame SN+CMB+BAO-Analyse, 6 Parameter, null Fr\"uhe Dunkle Energie). Das vorliegende Paper behandelt die mikroskopischen Grundlagen: \textit{Welches Quantensystem liefert die S\"attigungs-ODE $d\Omega_\Phi/da = k[1-(\Omega_\Phi/\Phi_0)^2]$ und die laufende Kopplung $\beta_{\mathrm{eff}}(a)$?} Wir leiten die effektive Lagrange-Dichte $\mathcal{L}_{\mathrm{KRM}} = R/(16\pi G) + \gamma R^2 + \frac{1}{2}(\partial\phi)^2 - V_{\mathrm{PT}}(\phi)$ her, wobei der $R^2$-Term den geometrischen ``Dunkle-Materie''-Beitrag erzeugt und ein P\"oschl-Teller-Skalarfeld die S\"attigungsdynamik liefert. Die Spurkopplung $\mathcal{F}(T/\rho)$ gew\"ahrleistet BBN-Schutz als rigorose Konsequenz konformer Symmetrie. Wir untersuchen vier UV-Vervollst\"andigungskandidaten (skalares Doppelmuldenpotential, Schleifen-Quantengravitation, Finsler-Geometrie, informationstheoretische Raumzeit), die alle S\"attigungsdynamik der erforderlichen Form erzeugen. Die laufende Kopplung $\beta$ wird als zweiter Ordnungsparameter in einem geometrischen Phasen\"ubergang interpretiert. Geistfreiheit, tachyonische Stabilit\"at und Sonnensystem-Kompatibilit\"at (Cham\"aleon-Abschirmung, $\lambda_C^{\mathrm{solar}} \sim 20$\,m) werden verifiziert. Testbare Vorhersagen -- Scalaron-Compton-Wellenl\"ange, Gravitationswellengeschwindigkeit $c_T = c$, Lensing-Parameter und CMB-Leistungsspektrum-Modifikationen -- sind unabh\"angig von der UV-Vervollst\"andigung. Das ontologische Bild reduziert sich auf Raumzeitkr\"ummung in drei Phasen plus baryonische Materie.

\vspace{0.5em}
\noindent \textbf{Schl\"usselw\"orter:} Kr\"ummungs-R\"uckgabemodell, Quantengravitation, Lagrange-Formulierung, Schleifen-Quantengravitation, Finsler-Geometrie, S\"attigungsmechanismus, modifizierte Gravitation

\vspace{0.5em}
\noindent \textbf{Themenbereiche:} Theoretische Physik, Quantengravitation, Mathematische Physik
\end{abstract}

\newpage
\tableofcontents
\newpage

\footnotetext{Bei der Erstellung dieses Manuskripts wurden KI-gest\"utzte Werkzeuge (Claude, Anthropic; Gemini, Google DeepMind) f\"ur mathematische Formalisierung, Code-Entwicklung und Textgenerierung eingesetzt. Alle physikalischen Hypothesen, wissenschaftliche Interpretation und Verantwortung f\"ur den Inhalt liegen ausschlie\ss{}lich beim Autor. Der Analysecode ist verf\"ugbar unter \url{https://github.com/lukisch/cfm-cosmology}.}

\newpage

% ===================================================================
% 1. EINLEITUNG
% ===================================================================
\section{Einleitung -- Die zentrale Frage}
\label{sec:intro}

Das Kr\"ummungs-R\"uckgabemodell (KRM) \cite{Geiger2026} und seine MOND-kompatible Erweiterung \cite{Geiger2026b} haben bemerkenswerten ph\"anomenologischen Erfolg gezeigt:

\begin{itemize}
\item \textbf{Paper~I:} Das Standard-KRM ersetzt Dunkle Energie durch ein Kr\"ummungsr\"uckkehrpotential und erreicht $\Delta\chi^2 = -12{,}2$ gegen\"uber $\Lambda$CDM auf Pantheon+-Daten.
\item \textbf{Paper~II:} Das erweiterte KRM ersetzt den Teilchen-Darksektor durch Raumzeit-Geometrie und erreicht ein Universum mit ausschlie\ss lich baryonischem Materieinhalt. Die SN-only-Analyse liefert $\Delta\chi^2 = -26{,}3$ mit $\beta = 2{,}02 \pm 0{,}20$ (Kr\"ummungsskalierung). Entscheidend ist, dass eine \textit{laufende Kr\"ummungskopplung} $\beta_{\mathrm{eff}}(a)$ -- \"ubergehend von CDM-artig ($\beta_{\mathrm{fr\"uh}} \approx 2{,}82$) bei $z > 6$ zu kr\"ummungsartig ($\beta \approx 2{,}0$) bei kleinem $z$ -- kombiniert mit einer skalenabh\"angigen MOND-Kopplung $\mu(a)$ gemeinsame SN + CMB + BAO Vertr\"aglichkeit erreicht: $\ell_A = 301{,}471$ (Planck: $301{,}471$), $\mathcal{R} = 1{,}7502$ (Planck: $1{,}7502$), $H_0 = 67{,}3$\,km/s/Mpc und $\Delta\chi^2 = -5{,}5$ vs.\ $\Lambda$CDM (bevorzugte Null-EDE-Variante).
\end{itemize}

Beide Ergebnisse leiten sich aus einer einzigen dynamischen Gleichung ab -- der \textit{S\"attigungs-ODE}:
\begin{equation}
\frac{d\Omega_\Phi}{da} = k \left[1 - \left(\frac{\Omega_\Phi}{\Phi_0}\right)^2\right]
\label{eq:saturation_ode}
\end{equation}

deren L\"osung die $\tanh$-Funktion ist, die die sp\"atzeitliche Beschleunigung liefert. Das erweiterte Modell f\"ugt einen Potenzgesetzterm $\alpha \cdot a^{-\beta}$ hinzu, der die unges\"attigte (fr\"uhzeitliche) Phase desselben geometrischen Prozesses darstellt.

Die zentrale Frage dieses Papers lautet:

\begin{quote}
\textit{Welches mikroskopische (Quanten-)System hat die Eigenschaft, dass sein makroskopischer (thermodynamischer) Grenzfall die S\"attigungs-ODE liefert? Und kann die vollst\"andige erweiterte Friedmann-Gleichung aus einer Lagrange-Dichte hergeleitet werden?}
\end{quote}

Diese Frage ist nicht nur akademisch. Ohne eine Lagrange-Formulierung kann das KRM nicht:
\begin{enumerate}
\item konsistent an Materiefelder gekoppelt werden,
\item St\"orungsgleichungen f\"ur $C_\ell$- und $P(k)$-Vorhersagen erzeugen,
\item mit bekannten Quantengravitationsrahmenwerken verbunden werden,
\item als vollst\"andige physikalische Theorie betrachtet werden.
\end{enumerate}


% ===================================================================
% 2. DIE EFFEKTIVE LAGRANGE-DICHTE
% ===================================================================
\section{Die effektive Lagrange-Dichte}
\label{sec:lagrangian}

\subsection{Anforderungen}

Die effektive Lagrange-Dichte $\mathcal{L}_{\mathrm{KRM}}$ muss erf\"ullen:
\begin{enumerate}
\item \textbf{Hintergrund:} Die Euler-Lagrange-Gleichungen, ausgewertet auf der FLRW-Metrik, m\"ussen die erweiterte Friedmann-Gleichung liefern:
\begin{equation}
H^2(a) = H_0^2 \left[\Omega_b\,a^{-3} + \Phi_0 \cdot f_{\mathrm{sat}}(a) + \alpha \cdot a^{-\beta}\right]
\end{equation}

\item \textbf{S\"attigungsdynamik:} Die Skalarfeld-Bewegungsgleichung muss auf dem FLRW-Hintergrund auf $d\Omega_\Phi/da = k[1 - (\Omega_\Phi/\Phi_0)^2]$ reduzieren.

\item \textbf{Allgemeine Kovarianz:} Die Wirkung muss diffeomorphismusinvariant sein.

\item \textbf{Korrekte Grenzf\"alle:} Im Grenzfall $k \to 0$, $\alpha \to 0$ muss die Theorie auf die ART mit kosmologischer Konstante reduzieren.
\end{enumerate}

\subsection{Skalarfeld-Ansatz}

Die nat\"urlichste Lagrange-Formulierung f\"uhrt ein Skalarfeld $\phi$ mit einem Potential $V(\phi)$ ein:
\begin{equation}
S = \int d^4x \sqrt{-g} \left[\frac{R}{16\pi G} - \frac{1}{2} g^{\mu\nu}\partial_\mu\phi\,\partial_\nu\phi - V(\phi) + \mathcal{L}_m\right]
\label{eq:action_scalar}
\end{equation}

Damit die S\"attigungs-ODE hervorgeht, ben\"otigen wir ein $V(\phi)$ derart, dass die homogene Feldgleichung auf FLRW $\tanh$-artige L\"osungen liefert.

\begin{proposition}[Doppelmulden-S\"attigungspotential]
Das Potential
\begin{equation}
V(\phi) = V_0 \left[1 - \tanh^2\!\left(\frac{\phi}{\phi_0}\right)\right] = \frac{V_0}{\cosh^2(\phi/\phi_0)}
\label{eq:double_well}
\end{equation}
erzeugt eine Skalenfeldgleichung, deren sp\"atzeitliche L\"osung auf dem FLRW-Hintergrund $\phi(a) \propto \tanh(k(a - a_{\mathrm{trans}}))$ ist und den S\"attigungsterm des KRM reproduziert.
\end{proposition}

\textit{Beweisskizze:} Die Klein-Gordon-Gleichung auf FLRW,
\begin{equation}
\ddot{\phi} + 3H\dot{\phi} + V'(\phi) = 0
\end{equation}
mit $V'(\phi) = -2V_0 \tanh(\phi/\phi_0)/(\phi_0 \cosh^2(\phi/\phi_0))$, besitzt die L\"osung $\phi = \phi_0 \tanh(\lambda t)$ im Slow-Roll-Regime, in dem $\ddot{\phi} \ll 3H\dot{\phi}$, wobei $\lambda$ mit $k$ und $H_0$ zusammenh\"angt. Die Energiedichte $\rho_\phi = \frac{1}{2}\dot{\phi}^2 + V(\phi)$ bildet dann ab auf $\Omega_\Phi(a) = \Phi_0 \cdot f_{\mathrm{sat}}(a)$. \hfill $\square$

\textit{Anmerkung:} Das $\cosh^{-2}$-Potential ist in der Quantenmechanik als P\"oschl-Teller-Potential wohlbekannt. Sein Auftreten hier legt eine tiefe Verbindung zwischen quantenmechanischen Bindungszust\"anden und kosmologischer S\"attigung nahe.

\textit{Philosophische Spannung:} Die Einf\"uhrung eines zus\"atzlichen Skalarfelds $\phi$ scheint der Behauptung des KRM zu widersprechen, den dunklen Sektor zu eliminieren. Wir betonen, dass $\phi$ \textit{keine} neue dunkle Komponente analog zu Dunkler Energie oder Dunkler Materie ist: Es hat einen eindeutigen Lagrange-Ursprung (Gl.~\ref{eq:double_well}), ein bekanntes quantenmechanisches Analogon, und seine Sp\"atzeitdynamik ist vollst\"andig durch $V_0$ und $\phi_0$ bestimmt -- es ist nicht frei justierbar. Dennoch bleibt die Ableitung von $\phi$ direkt aus dem geometrischen Sektor (z.\,B.\ als konformaler Modus der Metrik oder als Kondensat des $R^2$-Scalarons) ein wichtiges offenes Problem, das den rein geometrischen Charakter des KRM st\"arken w\"urde.

\subsection{Der Potenzgesetzterm: Geometrischer Ursprung}

Der geometrische "`Dunkle-Materie"'-Term $\alpha \cdot a^{-\beta}$ mit $\beta \approx 2$ erfordert einen separaten Ursprung. Zwei Ans\"atze sind m\"oglich:

\textbf{Ansatz 1: Kr\"ummungsquadratische Terme.} Hinzuf\"ugen eines Gauss-Bonnet- oder $R^2$-Terms zur Wirkung:
\begin{equation}
S_{\mathrm{geom}} = \int d^4x \sqrt{-g} \left[\frac{R}{16\pi G} + \gamma\, R^2 + \delta\, R_{\mu\nu}R^{\mu\nu}\right]
\end{equation}
erzeugt Korrekturen zur Friedmann-Gleichung, die in der Strahlungs-Materie-\"Ubergangs\"ara wie $a^{-2}$ skalieren. Der Koeffizient $\gamma$ kann mit $\alpha$ in Beziehung gesetzt werden.

\textbf{Ansatz 2: Vektorfeld (AeST-inspiriert).} Nach Skordis \& Z{\l}o\'snik \cite{Skordis2021} tr\"agt ein zeitartiges Vektorfeld $A_\mu$, das durch $g^{\mu\nu}A_\mu A_\nu = -1$ eingeschr\"ankt ist, eine effektive Energiedichte bei, die nichtstandardm\"a\ss ig mit $a$ skaliert. Der KRM-Potenzgesetzterm k\"onnte als kosmologischer Hintergrund eines solchen Vektorfeldes hervorgehen.

\subsection{Die kombinierte Wirkung}

Durch Kombination beider Beitr\"age lautet die vollst\"andige KRM-Wirkung:
\begin{equation}
{S_{\mathrm{KRM}} = \int d^4x \sqrt{-g} \left[\frac{R}{16\pi G} + \gamma R^2 - \frac{1}{2}(\partial\phi)^2 - \frac{V_0}{\cosh^2(\phi/\phi_0)} + \mathcal{L}_m\right]}
\label{eq:full_action}
\end{equation}

wobei der $R^2$-Term den Potenzgesetz-Beitrag ("`Dunkle Materie"') und das Skalarfeld den S\"attigungsbeitrag ("`Dunkle Energie"') erzeugt. Das Gleichgewicht aus Paper~I zwischen Nullraum und Raumzeitblase ist in der Balance zwischen $\gamma$ und $V_0$ kodiert.

Eine entscheidende Verfeinerung, die in Paper~II \cite{Geiger2026b} eingef\"uhrt wurde, ist die \textit{Spurkopplung}. Wir zeigen nun, dass dies \textit{kein} ad-hoc-Postulat ist, sondern eine direkte Konsequenz der $R^2$-Wirkung. Die Spurbildung der Feldgleichungen f\"ur $f(R) = R + 2\gamma R^2$ ergibt:
\begin{equation}
R + 12\gamma\,\Box R = -8\pi G\,T
\label{eq:trace_equation}
\end{equation}
wobei $T = g^{\mu\nu}T_{\mu\nu}$ die Energie-Impuls-Spur ist. W\"ahrend der Strahlungs\"ara verlangt die konforme Symmetrie $T_{\mathrm{rad}} = -\rho_r + 3p_r = 0$ (da $w = 1/3$). Gleichung~\eqref{eq:trace_equation} reduziert sich dann auf $R + 12\gamma\,\Box R = 0$, deren FLRW-L\"osung eine abklingende Mode ist: $R \to 0$ f\"ur $a \to 0$. Da das Skalaron (geometrische DM) durch $R^2$ gespeist wird, verschwindet es identisch wenn $R = 0$. Der geometrische DM-Beitrag wird daher \textit{automatisch} w\"ahrend der Strahlungs\"ara unterdr\"uckt -- die BBN ist ohne jeden zus\"atzlichen Mechanismus gesch\"utzt.

Das einzige echte \textit{Postulat} ist die Wahl der Lagrange-Dichte $f(R) = R + 2\gamma R^2$; der Unterdr\"uckungsfaktor $\mathcal{S}(a) = 1/(1 + a_{\mathrm{eq}}/a)$ aus Paper~II ist eine ph\"anomenologische Parametrisierung dieses rigorosen Ergebnisses. Die modifizierte Wirkung mit der vollst\"andigen Kopplung lautet:
\begin{equation}
S_{\mathrm{KRM}} = \int d^4x \sqrt{-g} \left[\frac{R}{16\pi G} + \gamma\, \mathcal{F}(T/\rho)\, R^2 - \frac{1}{2}(\partial\phi)^2 - \frac{V_0}{\cosh^2(\phi/\phi_0)} + \mathcal{L}_m\right]
\end{equation}
wobei $\mathcal{F}(T/\rho) = |T|/(|T| + \rho_{\mathrm{rad}}) \to 0$ in der Strahlungs\"ara und $\mathcal{F} \to 1$ in der Materie\"ara.

\subsection{Geisterfreiheit und Stabilit\"at}
\label{subsec:ghost_analysis}

Eine kritische Konsistenzanforderung f\"ur jede Theorie mit h\"oheren Ableitungen ist die Abwesenheit von Ostrogradsky-Geistern \cite{Woodard2015}. Wir verifizieren nun, dass die Wirkung~\eqref{eq:full_action} alle Stabilit\"atsbedingungen erf\"ullt.

\textbf{Konforme \"Aquivalenz.} Der Gravitationssektor $f(R) = R + 2\gamma R^2$ (mit $\epsilon = 16\pi G\gamma$) ist konform \"aquivalent zur Einstein-Gravitation plus einem massiven Skalarfeld (dem \textit{Skalaron}) $\chi$:
\begin{equation}
S = \int d^4x \sqrt{-g_E} \left[\frac{R_E}{16\pi G} - \frac{1}{2}(\partial\chi)^2 - U(\chi) \right]
\end{equation}
wobei $\chi = \sqrt{3/(16\pi G)}\,\ln f_R$ und $U(\chi) = (R f_R - f)/(2 f_R^2)$. Dies zeigt, dass die Theorie $2 + 1 + 1 = 4$ Freiheitsgrade propagiert (Graviton + Skalaron + P\"oschl-Teller-Skalar), alle mit positiver kinetischer Energie.

\begin{proposition}[Geisterfreiheit der KRM-Wirkung]
Die Wirkung~\eqref{eq:full_action} ist geisterfrei unter den folgenden Bedingungen, die alle konstruktionsgem\"a\ss\ erf\"ullt sind:
\begin{enumerate}
\item \textbf{Kein Ostrogradsky-Geist:} $f_{RR} = 2\epsilon > 0$ (da $\gamma > 0$), was positive kinetische Energie f\"ur das Skalaron sicherstellt. QED.
\item \textbf{Keine tachyonische Instabilit\"at:} Die Skalaron-Masse $m_s^2 = 1/(6\epsilon) > 0$ f\"ur $\gamma > 0$. Das Potential $U(\chi) \geq 0$ hat ein stabiles Minimum bei $\chi = 0$.
\item \textbf{Keine Gradienteninstabilit\"at:} Die Skalaron-Schallgeschwindigkeit $c_s^2 = 1$ in $f(R)$-Theorien (Tensorgeschwindigkeit $c_T = c$ ist durch $\alpha_T = 0$ garantiert).
\item \textbf{Positiv-definite kinetische Matrix:} Das Zwei-Feld-System $(\chi, \phi)$ hat die kinetische Matrix $K = \mathrm{diag}(1, 1)$ im Einstein-Bezugssystem.
\end{enumerate}
\end{proposition}

\textbf{Spurkopplung und Stabilit\"at.} Die Kopplungsfunktion $\mathcal{F}(T/\rho)$ modifiziert nur die \textit{effektive Masse} des Skalarons, nicht seine kinetische Struktur:
\begin{equation}
m_{\mathrm{eff}}^2(a) = \frac{1}{24\gamma\,\mathcal{F}(a)}
\end{equation}
Da $\mathcal{F} \in [0,1]$ monoton und beschr\"ankt ist, bleibt die Masse zu allen Zeiten reell und positiv. Zu fr\"uhen Zeiten ($\mathcal{F} \to 0$) gilt $m_{\mathrm{eff}} \to \infty$ und das Skalaron ist ausgefroren. Zu sp\"aten Zeiten ($\mathcal{F} \to 1$) gilt $m_{\mathrm{eff}} = m_s$.

\textbf{Newtonscher Grenzfall und Cham\"aleon-Abschirmung.} Im Schwachfeldlimes erzeugt das Skalaron eine Yukawa-Korrektur:
\begin{equation}
V(r) = -\frac{GM}{r}\left(1 + \frac{1}{3}\,e^{-m_{\mathrm{eff}}\,r}\right)
\label{eq:yukawa}
\end{equation}
Sonnensystem-Beschr\"ankungen erfordern $m_{\mathrm{eff}}\,r_{\mathrm{AU}} \gg 1$. Die Spurkopplung liefert einen nat\"urlichen Cham\"aleon-Mechanismus \cite{Khoury2004}: In dichten Umgebungen ($\rho \gg \rho_{\mathrm{cosm}}$) steigt die effektive Masse als $m_{\mathrm{eff}} \propto \sqrt{\rho_{\mathrm{local}}/\rho_{\mathrm{cosm}}}$. F\"ur die Sonne ($\rho \sim 1400\,\mathrm{kg/m^3}$) gilt $m_{\mathrm{eff}}^{\mathrm{solar}}/m_s \sim 4 \times 10^{14}$, was $\lambda_C^{\mathrm{solar}} \sim 20\,\mathrm{m} \ll 1\,\mathrm{AU}$ ergibt. Das Skalaron ist somit im Sonnensystem abgeschirmt f\"ur $\gamma \geq \mathcal{O}(1)\,H_0^{-2}$, konsistent mit Lunar-Laser-Ranging-Beschr\"ankungen ($\Delta G/G < 10^{-13}$) \cite{Williams2004} und der Cassini-Messung des parametrisierten post-Newtonschen Parameters $\gamma_{\mathrm{PPN}} - 1 = (2{,}1 \pm 2{,}3) \times 10^{-5}$ \cite{Bertotti2003}.

\subsection{Lagrange-Herleitung der laufenden Kopplung $\beta_{\mathrm{eff}}(a)$}
\label{subsec:beta_derivation}

Die ph\"anomenologische \"Ubergangsfunktion $\beta_{\mathrm{eff}}(a)$ aus Paper~II kann aus der Skalaron-Dynamik der Wirkung~\eqref{eq:full_action} hergeleitet werden. Die Spur der modifizierten Einstein-Gleichung liefert die Skalaron-Bewegungsgleichung auf FLRW:
\begin{equation}
\ddot{\chi} + 3H\dot{\chi} + m_{\mathrm{eff}}^2(a)\,\chi = \frac{8\pi G}{3}\,\rho_m
\label{eq:scalaron_eom}
\end{equation}
wobei $\chi = f_R - 1 = 4\gamma\mathcal{F}(a) R$ das Skalaronfeld ist und $m_{\mathrm{eff}}^2(a) = 1/(24\gamma\mathcal{F}(a))$. Die effektive geometrische Energiedichte des Skalarons ist $\Omega_{R^2}(a) \propto \chi(a) \cdot R(a)$, und der effektive Skalierungsexponent folgt als:
\begin{equation}
{\beta_{\mathrm{eff}}(a) = -\frac{d\ln\Omega_{R^2}}{d\ln a} = 3 + \frac{d\ln\chi}{d\ln a} + \frac{d\ln\mathcal{F}}{d\ln a}}
\label{eq:beta_from_lagrangian}
\end{equation}

Die Spurkopplung $\mathcal{F}(a) = 1/(1 + \Omega_r/(\Omega_b\,a))$ f\"uhrt eine charakteristische \"Ubergangsskala bei $a \sim \Omega_r/\Omega_b \approx 1{,}8 \times 10^{-3}$ ($z \sim 550$) ein. Die numerische L\"osung von Gl.~\eqref{eq:scalaron_eom} zeigt, dass $\beta_{\mathrm{eff}}$ von $\sim 2{,}75$ bei $z = 7$ auf $\sim 3{,}0$ bei $z = 0$ \"ubergeht, wobei das exakte Profil von $\gamma$ abh\"angt. Die ph\"anomenologische sigmoidale Parametrisierung aus Paper~II approximiert diese L\"osung \"uber den beobachtungsrelevanten Bereich $z = 0$--$100$. \textit{Hinweis:} Das laufende $\mu(a)$ aus Paper~II bleibt eine ph\"anomenologische \"Ubergangsfunktion; eine Lagrangian-Ableitung von $\mu(a)$ wird nicht beansprucht.

\textbf{Nat\"urliche Parametrisierung.} Die Spurkopplung hat eine entscheidende Konsequenz f\"ur die Horndeski-$\alpha_M$-Funktion: Da $\mathcal{F}(a) \approx (\Omega_b/h^2)\,a/\Omega_r$ f\"ur $a \ll a_{\mathrm{eq}}$, w\"achst das Skalaronfeld w\"ahrend der Materie\"ara linear im Skalenfaktor. Dies bedeutet $\alpha_M(a) \propto a$ zu fr\"uhen Zeiten -- pr\"azise die \texttt{propto\_scale}-Parametrisierung von \texttt{hi\_class}. Zu sp\"aten Zeiten ($a \to 1$) gilt $\mathcal{F} \to 1$ und $\alpha_M$ s\"attigt. Die \texttt{propto\_scale}-Parametrisierung ist daher keine ad-hoc-Wahl, sondern die \textit{nat\"urliche} Konsequenz der Skalaron-Dynamik mit Spurkopplung.


% ===================================================================
% 3. QUANTENGRAVITATIONSVERBINDUNGEN
% ===================================================================
\section{Quantengravitationsverbindungen}
\label{sec:quantum_gravity}

\subsection{Warum die S\"attigungs-ODE?}

Das zentrale R\"atsel ist die spezifische Form der S\"attigungs-ODE~\eqref{eq:saturation_ode}: $dX/da = k(1 - X^2)$. Diese Gleichung hat zwei Fixpunkte ($X = \pm 1$), von denen $X = +1$ stabil ist. Die $\tanh$-L\"osung ist die einzige Trajektorie, die $X = 0$ (keine Kr\"ummungsr\"uckkehr) mit $X = 1$ (vollst\"andige S\"attigung) verbindet.

\subsection{UV-Completion-Kandidaten}
\label{subsec:uv_candidates}

\textbf{Wichtiger Vorbehalt:} Alle testbaren Vorhersagen des KRM (CMB-Spektren, $\sigma_8$, $S_8$, gravitativer Schlupf usw.)\ folgen \textit{ausschlie\ss lich} aus der effektiven Wirkung~\eqref{eq:full_action}, d.\,h.\ aus dem $R^2$-Term und dem P\"oschl-Teller-Skalar. Die UV-Completion -- die mikroskopische Quantengravitationstheorie, aus der diese effektive Wirkung hervorgehen k\"onnte -- beeinflusst \textit{keine} niederenergetische Vorhersage. Die folgenden Kandidaten sind daher \textit{motivierend}, nicht wesentlich:

\begin{enumerate}
\item \textbf{Schleifen-Quantengravitation} \cite{Rovelli2004}: Die modifizierte LQC-Friedmann-Gleichung $H^2 \propto \rho(1 - \rho/\rho_c)$ hat dieselbe $dX/dt \propto (1-X^2)$-Struktur wie die S\"attigungs-ODE. Die beschr\"ankten Kr\"ummungsinvarianten aus Holonomie-Korrekturen erzeugen nat\"urlicherweise S\"attigung.

\item \textbf{Finsler-Geometrie} \cite{Bao2000}: Richtungsabh\"angige Metriken $F(x, \dot{x})$ erzeugen skalenabh\"angige Gravitationseffekte (die MOND nachahmen) und nichtstandardm\"a\ss ige kosmologische Skalierung, wobei der $a^{-\beta}$-Term potentiell aus der oskulierenden Riemannschen Kr\"ummung hervorgeht.

\item \textbf{Informationstheoretische Raumzeit} \cite{Bousso2002}: Die S\"attigungs-ODE ist die logistische Wachstumsgleichung f\"ur Informationsverarbeitung, mit $\Phi_0$ als holographischer Kapazit\"atsgrenze. Die S\"attigung der Verschr\"ankungsentropie \cite{VanRaamsdonk2010} liefert den mikroskopischen Mechanismus.

\item \textbf{Theorie der kausalen Mengen} \cite{Bombelli1987}: Die Sorkin-kosmologische-Konstante $\Lambda \sim 1/\sqrt{N}$ liefert eine dynamische Dunkle Energie, deren Evolution den S\"attigungsmechanismus widerspiegelt, wenn die kausale Menge ihre Gleichgewichtsdichte erreicht.

\item \textbf{Quanten-Fehlerkorrektur} \cite{Almheiri2015}: Die S\"attigung $\Phi_0$ ist die Code-Kapazit\"at des holographischen Raumzeitcodes. Die beschleunigte Expansion ist der Selbstschutzmechanismus des Codes gegen \"Uberschreitung seiner Fehlerkorrekturschwelle -- direkt abgebildet auf das "`Selbstschutzmotiv"' des Nullraums im Rahmenwerk von Paper~I.
\end{enumerate}

Heuristische Argumente und strukturelle Analogien f\"ur jeden Kandidaten finden sich in Anhang~\ref{app:qg_details}. Wir betonen, dass es sich dabei um strukturelle Analogien handelt, nicht um rigorose mathematische Herleitungen: W\"ahrend alle f\"unf Rahmenwerke \textit{unabh\"angig} S\"attigungsdynamik der Form $dX/dt \propto (1 - X^2)$ erzeugen, was nahelegt, dass die effektive KRM-Wirkung ein \textit{universelles} Merkmal der Quantengravitation erfasst, bleibt ein formaler Beweis, dass eine spezifische UV-Completion die KRM-Wirkung \textit{erfordert}, ein offenes Problem.

\subsection{Die Natur des Nullraums}
\label{subsec:null_space}

Paper~I und~II postulierten den Nullraum als den "`anderen Spieler"' im kosmologischen Spiel -- den pr\"ageometrischen Grundzustand, aus dem die Raumzeitblase hervorgeht. Mit den oben dargelegten Quantengravitationsans\"atzen k\"onnen wir den Nullraum nun pr\"aziser charakterisieren.

\textbf{A-geometrisch:} Der Nullraum hat keine Metrik. Es gibt keinen Begriff von Abstand, Dauer oder Dimensionalit\"at. Er ist eine \textit{topologische} oder \textit{algebraische} Entit\"at, keine geometrische. In LQG-Sprache ist er der Zustand maximaler Unordnung unter Spinnetzwerkknoten -- alle Verbindungen existieren in Superposition, aber keine ist realisiert.

\textbf{Superposition aller Geometrien:} Quantenmechanisch ist der Nullraum das Pfadintegral \"uber alle m\"oglichen Raumzeitkonfigurationen, gleichm\"a\ss ig gewichtet. Er ist der Zustand maximaler Unsicherheit \"uber die Geometrie -- nicht "`leerer Raum"', sondern "`\"uberhaupt kein Raum"'.

\textbf{Das Energiereservoir:} Im Rahmenwerk von Paper~I besitzt der Nullraum das gesamte Energiebudget $E_0$, existiert aber in einem metastabilen Zustand (die "`Bank"', die das Kapital h\"alt, aber nicht investiert). Eine Quantenfluktuation l\"ost den Phasen\"ubergang aus, der die Raumzeitblase erzeugt.

\textbf{Der Code:} In der QEC-Interpretation ist der Nullraum die \textit{logische} Quanteninformation, die der Raumzeitcode sch\"utzt. Die Volumen-Raumzeit (unser Universum) sind die \textit{physikalischen} Qubits des Codes. Der holographische Rand ist die Schnittstelle zwischen der logischen (Nullraum-) und der physikalischen (Raumzeit-)Schicht.

Die Entstehung der Raumzeit aus dem Nullraum l\"asst sich als \textit{geometrischer Phasen\"ubergang} interpretieren -- analog zur Kristallisation von Wasser zu Eis. Der Nullraum ist die ungeordnete "`fl\"ussige"' Phase (keine Geometrie, alle Konfigurationen in Superposition). Die Raumzeitblase ist die geordnete "`kristalline"' Phase (definite Geometrie, metrische Struktur). Die S\"attigungs-ODE beschreibt den Abschluss dieses \"Ubergangs: Das Kr\"ummungsr\"uckkehrpotential $\Omega_\Phi$ ist der Ordnungsparameter, und seine S\"attigung bei $\Phi_0$ ist der vollst\"andig geordnete Zustand (de-Sitter-Gleichgewicht).

In diesem Bild hat die Frage "`Was s\"attigt?"' eine vereinheitlichte Antwort: \textit{die geometrische Ordnung der Raumzeit.} Ob wir diese Ordnung in Begriffen der Spinausrichtung (LQG), der Verschr\"ankungskonnektivit\"at (ER=EPR), der Informationskapazit\"at (Holographie) oder der Code-Auslastung (QEC) beschreiben, die mathematische Struktur ist dieselbe -- ein kooperatives System diskreter Freiheitsgrade, das sich seinem kollektiven Gleichgewicht n\"ahert. Die $\tanh$-Funktion ist die universelle Signatur dieses Prozesses, unabh\"angig von der spezifischen mikroskopischen Realisierung.


% ===================================================================
% 4. DER GEOMETRISCHE PHASEN\"UBERGANG
% ===================================================================
\section{Der geometrische Phasen\"ubergang}
\label{sec:phase_transition}

\subsection{Von der Dunkle-Materie-Phase zur Dunkle-Energie-Phase}

Paper~II \cite{Geiger2026b} f\"uhrte das Konzept eines geometrischen Phasen\"ubergangs ein: Zu fr\"uhen Zeiten verh\"alt sich das Kr\"ummungsr\"uckkehrpotential wie Dunkle Materie ($\alpha \cdot a^{-2}$), und zu sp\"aten Zeiten s\"attigt es zu Dunkler Energie ($\Phi_0 \cdot f_{\mathrm{sat}}$). Dieser Abschnitt liefert die theoretische Fundierung.

\subsection{Ordnungsparameter und Symmetriebrechung}

Die S\"attigungsvariable $X = \Omega_\Phi / \Phi_0 \in [0, 1]$ kann als \textit{Ordnungsparameter} interpretiert werden:
\begin{itemize}
\item $X = 0$: Ungeordnete Phase (keine Kr\"ummungsr\"uckkehr, geometrische "`DM"' dominiert)
\item $X = 1$: Geordnete Phase (volle S\"attigung, geometrische "`DE"' dominiert)
\item Der \"Ubergang bei $a_{\mathrm{trans}}$: Die \"Uberkreuzung zwischen den Phasen
\end{itemize}

Die S\"attigungs-ODE $dX/da = k(1 - X^2)$ hat die Form einer Ginzburg-Landau-Gleichung f\"ur einen Phasen\"ubergang zweiter Ordnung mit einer Doppelmulden-freien Energie $F(X) = -k(X - X^3/3)$. Der "`Temperatur"'-Parameter ist der Skalenfaktor $a$, und der \"Ubergang findet statt, wenn $a$ \"uber $a_{\mathrm{trans}}$ hinaus ansteigt.

\subsection{Analogie zur spontanen Magnetisierung}

Die mathematische Struktur ist identisch mit der Molekularfeldtheorie des Ferromagnetismus:
\begin{center}
\begin{tabular}{lll}
\toprule
\textbf{Ferromagnetismus} & \textbf{KRM-Kosmologie} & \textbf{Variable} \\
\midrule
Magnetisierung $M$ & Kr\"ummungsr\"uckkehr $\Omega_\Phi$ & Ordnungsparameter \\
Temperatur $T$ & Skalenfaktor $a$ & Kontrollparameter \\
Curie-Punkt $T_c$ & \"Ubergang $a_{\mathrm{trans}}$ & Kritischer Punkt \\
Spin-Wechselwirkung $J$ & Kr\"ummungskopplung $k$ & Wechselwirkungsst\"arke \\
S\"attigung $M_s$ & S\"attigung $\Phi_0$ & Maximalwert \\
$\tanh(J/k_BT)$ & $\tanh(k(a - a_{\mathrm{trans}}))$ & L\"osung \\
\bottomrule
\end{tabular}
\end{center}

Diese Analogie legt nahe, dass die Kr\"ummungsr\"uckkehr durch \textit{kooperative Ph\"anomene} angetrieben wird: Einzelne Raumzeitfreiheitsgrade (Fl\"achenquanten in der LQG, Elemente kausaler Mengen usw.) richten sich kollektiv aus und erzeugen einen makroskopischen S\"attigungseffekt. Das "`Gleichgewicht"' aus Paper~I ist das kosmologische Analogon des thermischen Gleichgewichts in der statistischen Mechanik.

\subsection{Kritische Exponenten und Universalit\"at}

Wenn die Analogie zu Phasen\"uberg\"angen mehr als formal ist, sollte das KRM \textit{Universalit\"at} aufweisen: Der S\"attigungsexponent und die \"Ubergangsform sollten robust gegen mikroskopische Details sein. Dies w\"urde erkl\"aren, warum die ph\"anomenologische $\tanh$-Funktion die Daten gut anpasst -- sie ist die universelle Skalenfunktion eines Molekularfeld-Phasen\"ubergangs, unabh\"angig vom mikroskopischen Mechanismus.

\begin{conjecture}[Universalit\"at des S\"attigungsmechanismus]
Die $\tanh$-Form des Kr\"ummungsr\"uckkehrpotentials ist eine \textit{universelle} Konsequenz jeder mikroskopischen Theorie mit:
\begin{enumerate}
\item Einer beschr\"ankten Kr\"ummungsr\"uckkehr (S\"attigungsgrenze $\Phi_0$)
\item Einer kooperativen Wechselwirkung zwischen Raumzeitfreiheitsgraden (Kopplung $k$)
\item Einer einzigen relevanten Richtung (dem Skalenfaktor $a$)
\end{enumerate}
Der spezifische mikroskopische Mechanismus (LQG, Finsler, kausale Mengen) beeinflusst nur die Werte von $k$ und $\Phi_0$, nicht die funktionale Form.
\end{conjecture}


% ===================================================================
% 5. TESTBARE VORHERSAGEN
% ===================================================================
\section{Testbare Vorhersagen aus der Lagrange-Dichte}
\label{sec:predictions}

Die effektive Wirkung erzeugt spezifische Vorhersagen jenseits der Hintergrundexpansionsgeschichte:

\subsection{St\"orungsgleichungen: Schallgeschwindigkeit, $\mu$, $\Sigma$ und gravitativer Schlupf}
\label{subsec:perturbation_physics}

Die Linearisierung der Wirkung~\eqref{eq:full_action} um den FLRW-Hintergrund liefert gekoppelte Gleichungen f\"ur die Metrikst\"orungen $\Phi$ und $\Psi$, die Skalaron-St\"orung $\delta\chi$, die P\"oschl-Teller-Skalarst\"orung $\delta\phi$ und die Materiest\"orungen $\delta_m$ und $v_m$. Die $f(R)$-Struktur des Gravitationssektors ($\alpha_B = -\alpha_M/2$, $\alpha_T = 0$, $\alpha_K = 0$) platziert das KRM innerhalb der gut untersuchten Horndeski-Unterklasse, f\"ur die die quasi-statischen St\"orungsgleichungen analytisch bekannt sind \cite{Bellini2014}.

\textbf{Skalaron-Schallgeschwindigkeit.} Die Ausbreitungsgeschwindigkeit der Skalaron-St\"orung ist:
\begin{equation}
c_s^2 = 1 \qquad \text{(exakt f\"ur alle $f(R)$-Theorien)}
\end{equation}
Dies folgt aus der konformen \"Aquivalenz zur Einstein-Gravitation mit einem kanonischen Skalarfeld (Abschnitt~\ref{subsec:ghost_analysis}). Das Skalaron propagiert mit Lichtgeschwindigkeit im Jordan-Bezugssystem. Dies ist \textit{nicht} die "`Schallgeschwindigkeit der Dunklen Materie"' -- das Skalaron ist ein gravitativer Freiheitsgrad, der die Poisson-Gleichung modifiziert, kein Fluid, das unter seinem eigenen Druck verklumpt.

\textbf{Modifizierte Poisson-Gleichung.} Im quasi-statischen Limes ($k^2 \gg a^2 H^2$) lautet die modifizierte Poisson-Gleichung:
\begin{equation}
-k^2 \Psi = 4\pi G\,a^2\,\mu(k,a)\,\rho_m\,\delta_m
\label{eq:modified_poisson}
\end{equation}
wobei $\mu(k,a)$ die effektive Gravitationskopplung ist:\footnote{Nicht zu verwechseln mit der MOND-Verst\"arkung $\mu_{\mathrm{eff}} \approx \sqrt{\pi}$ auf Hintergrundebene aus Paper~II. Hier modifiziert $\mu(k,a)$ die Poisson-Gleichung auf St\"orungsebene.}
\begin{equation}
{\mu(k,a) = 1 + \frac{1}{3}\,\frac{k^2}{k^2 + a^2\,m_{\mathrm{eff}}^2(a)}}
\label{eq:mu_ka}
\end{equation}
mit $m_{\mathrm{eff}}^2(a) = 1/(24\gamma\,\mathcal{F}(a))$ als Skalaron-Masse. Auf Sub-Horizont-Skalen ($k \gg a\,m_{\mathrm{eff}}$) wird die Gravitation um den Faktor $4/3$ verst\"arkt; auf Super-Horizont-Skalen ($k \ll a\,m_{\mathrm{eff}}$) wird die ART wiederhergestellt ($\mu \to 1$). Die Skalaron-Masse liefert eine nat\"urliche \textit{skalenabh\"angige Abschirmung}: Kleinskalige St\"orungen ($k \gtrsim 0{,}1\,h/$Mpc) erfahren verst\"arktes Wachstum, w\"ahrend der gro\ss skalige CMB nahezu unbeeinflusst bleibt.

\textbf{Linsenparameter.} Das Gravitationslinsenpotential $\Phi_{\mathrm{lens}} = (\Phi + \Psi)/2$ wird durch den Linsenparameter $\Sigma$ charakterisiert:
\begin{equation}
{\Sigma(k,a) = 1 + \frac{\alpha_T}{2} = 1 \qquad \text{(exakt, da $\alpha_T = 0$)}}
\label{eq:sigma_lensing}
\end{equation}
Gravitationslinseneffekte im cfm\_fR-Modell sind \textit{identisch} mit der ART auf allen Skalen und Rotverschiebungen. Dies sichert gleichzeitig (i)~Bullet-Cluster-Kompatibilit\"at, (ii)~die Vorhersage, dass Euclids Linsen-zu-Clustering-Verh\"altnistest $\Sigma = 1$ finden wird, und (iii)~die Aufl\"osung der Bedenken, dass modifizierte Gravitation schwache Linsenmessungen st\"oren k\"onnte.

\textbf{Gravitativer Schlupf.} Das Verh\"altnis $\eta = \Phi/\Psi$ weicht im quasi-statischen Limes von Eins ab:
\begin{equation}
\eta(k,a) = \frac{1 + \frac{2}{3}\frac{k^2}{k^2 + a^2 m_{\mathrm{eff}}^2}}{1 + \frac{1}{3}\frac{k^2}{k^2 + a^2 m_{\mathrm{eff}}^2}} = \frac{\mu + 1}{2\mu - 1 + 2}
\end{equation}
F\"ur $k \gg a\,m_{\mathrm{eff}}$: $\eta \to 1/2$; f\"ur $k \ll a\,m_{\mathrm{eff}}$: $\eta \to 1$ (ART-Grenzfall). Dieser skalenabh\"angige gravitative Schlupf ist eine einzigartige Signatur der $f(R)$-Gravitation und kann durch den Vergleich schwacher Linseneffekte (empfindlich auf $\Phi + \Psi$) mit Galaxienh\"aufung (empfindlich auf $\Psi$) sondiert werden.

\textbf{Mechanismus der Strukturbildung.} Eine wesentliche Kl\"arung: Im cfm\_fR-Modell erfolgt die Strukturbildung durch die \textit{modifizierte Poisson-Gleichung} (Gl.~\ref{eq:modified_poisson}), \textit{nicht} durch Skalaron-Verklumpung. Das Skalaron hat $c_s^2 = 1$ und verklumpt nicht unterhalb seiner Compton-Wellenl\"ange $\lambda_C = 2\pi/m_{\mathrm{eff}}$. Stattdessen modifiziert es die Beziehung zwischen Materie\"uberdichten und Gravitationspotentialen: Dieselbe Baryonen\"uberdichte $\delta_b$ erzeugt einen $33\%$ tieferen Potentialtopf ($\mu = 4/3$) auf Sub-Compton-Skalen. Diese verst\"arkte Gravitationskopplung beschleunigt den baryonischen Kollaps und das Strukturwachstum und erkl\"art die erh\"ohten $\sigma_8$-Werte aus Abschnitt~\ref{sec:numerical}.

\textbf{Zwei verschiedene Nichtlinearit\"aten.} Es ist wichtig, die zwei grundlegend verschiedenen nichtlinearen Mechanismen im KRM-Rahmenwerk zu unterscheiden. Die \textit{kosmologische Nichtlinearit\"at} entsteht aus der S\"attigungs-ODE (Paper~I \cite{Geiger2026}): Das P\"oschl-Teller-Potential treibt $\phi$ \"uber $\tanh$-Dynamik zu seinem Vakuumerwartungswert und erzeugt so eine effektive Dunkle-Energie-Komponente $\Omega_\Phi(a)$, die $\Lambda$ ersetzt. Diese Nichtlinearit\"at wirkt auf der Hintergrund-(homogenen) Ebene und bestimmt die Expansionsgeschichte. Die \textit{galaktische Nichtlinearit\"at} entsteht dagegen aus dem Cham\"aleon-Massenmechanismus: Die dichteabh\"angige effektive Masse $m_{\mathrm{eff}}(\rho)$ des Skalarons erzeugt eine nichtlineare Beziehung zwischen lokaler Materiedichte und Gravitationsverst\"arkung, die den MOND-artigen $\mu \to 4/3$-Attraktor in Umgebungen niedriger Dichte produziert. Diese beiden Nichtlinearit\"aten sind logisch unabh\"angig -- die erste bestimmt \textit{wann} das Universum beschleunigt, die zweite \textit{wo} die Gravitation verst\"arkt wird -- teilen aber einen gemeinsamen Ursprung im $R + \gamma R^2$-Lagrangian.

\textbf{Abgrenzung von AQUAL und TeVeS.} Die galaktische Nichtlinearit\"at des KRM unterscheidet sich grundlegend von den nichtlinearen kinetischen Termen in AQUAL \cite{BekensteinMilgrom1984} und seiner relativistischen Erweiterung TeVeS \cite{Bekenstein2004}. In AQUAL/TeVeS ist die Nichtlinearit\"at \textit{kinematisch}: Eine Funktion $\mu(|\nabla\Phi|/a_0)$ wird direkt in die Gravitationswirkung eingebaut und modifiziert das Kraftgesetz als Funktion der lokalen Beschleunigung. Im KRM ist die Nichtlinearit\"at \textit{umgebungsbedingt}: Der Cham\"aleon-Screening-Mechanismus macht die Skalaronmasse von der lokalen Materiedichte $\rho$ abh\"angig, sodass der \"Ubergang von Newtonscher ($\mu = 1$) zu verst\"arkter ($\mu = 4/3$) Gravitation durch die Dichteumgebung gesteuert wird, nicht durch die Beschleunigungsst\"arke. Dieser umgebungsbedingte Mechanismus erkl\"art auf nat\"urliche Weise, warum MOND-Ph\"anomenologie in den d\"unnen galaktischen Au\ss{}enbereichen auftritt, aber nicht im dichten Sonnensystem -- ohne eine explizite Beschleunigungsschwelle $a_0$ als fundamentalen Parameter zu erfordern.

\subsection{Skalenfeldoszillationen}

Das P\"oschl-Teller-Potential unterst\"utzt ein diskretes Spektrum gebundener Zust\"ande. Im kosmologischen Kontext entsprechen diese oszillatorischen Korrekturen der Expansionsrate zu sp\"aten Zeiten:
\begin{equation}
H^2(a) = H^2_{\mathrm{smooth}}(a) \left[1 + \epsilon \cdot e^{-\Gamma a} \cos(\omega a + \delta)\right]
\end{equation}
mit Amplitude $\epsilon \ll 1$. Diese Oszillationen, falls in hochpr\"azisen BAO- oder SN-Daten detektierbar, w\"urden einen direkten Beweis f\"ur die Quantennatur des S\"attigungsmechanismus liefern.

\subsection{Modifizierte Gravitationswellen}

Der $R^2$-Term modifiziert die Ausbreitungsgleichung f\"ur Gravitationswellen:
\begin{equation}
\ddot{h}_{ij} + (3H + \Gamma_{\mathrm{GW}})\dot{h}_{ij} + \left(\frac{k^2}{a^2} + m_{\mathrm{GW}}^2\right) h_{ij} = 0
\end{equation}
wobei $\Gamma_{\mathrm{GW}}$ und $m_{\mathrm{GW}}^2$ Korrekturen aus dem kr\"ummungsquadratischen Term sind. Dies sagt vorher:
\begin{itemize}
\item Eine frequenzabh\"angige Gravitationswellengeschwindigkeit ($c_{\mathrm{GW}} \neq c$ bei hohen Frequenzen)
\item Eine massive Gravitonmode mit $m_{\mathrm{GW}} \propto \sqrt{\gamma}$
\end{itemize}
Die LIGO/Virgo/KAGRA-Beschr\"ankung $|c_{\mathrm{GW}}/c - 1| < 10^{-15}$ \cite{Abbott2017} setzt eine Obergrenze f\"ur $\gamma$.

\subsection{Unterscheidung der mikroskopischen Kandidaten}

Jeder der f\"unf mikroskopischen Ans\"atze erzeugt eine distinkte experimentelle Signatur. Entscheidend ist, dass mehrere dieser Tests bereits durchgef\"uhrt wurden oder unmittelbar bevorstehen:

\begin{center}
\small
\begin{tabular}{lllp{3.4cm}}
\toprule
\textbf{Kandidat} & \textbf{Signatur} & \textbf{Instrument} & \textbf{Status} \\
\midrule
A: Holographisch & Raumzeitrauschen & Holometer (Fermilab) & \textbf{Nullresultat} (2015). Einfachste Modelle ausgeschlossen \cite{Chou2017}. \\
B: Spinnetzwerke & Vakuum-Doppelbrechung & Planck-CMB-Polarisation & \textbf{$2{,}4\sigma$-Hinweis}: $\beta \approx 0{,}35^\circ$ \cite{Minami2020}. \\
C: Verschr\"ankung & Gravitationsinduzierter Kollaps & Gran Sasso (unterirdisch) & Einfaches Di\'osi-Penrose \textbf{ausgeschlossen} \cite{Donadi2021}. \\
D: QEC-Codes & GW-Horizontechos & LIGO/Virgo & \textbf{$\sim2{,}5\sigma$ vorl\"aufig} \cite{Abedi2017}. Umstritten. \\
E: Kausale Mengen & $\Lambda$-Fluktuationen & Pr\"azisionskosmologie & Noch nicht mit erforderlicher Pr\"azision testbar. \\
\bottomrule
\end{tabular}
\end{center}

\subsubsection{Das kosmische Doppelbrechungssignal (Kandidat B)}

Das vielversprechendste existierende Signal ist die \textit{isotrope kosmische Doppelbrechung}, die von Minami \& Komatsu \cite{Minami2020} in reanalysierten Planck-Polarisationsdaten berichtet wurde. Sie fanden eine Rotation der CMB-Polarisationsebene um $\beta = 0{,}35^\circ \pm 0{,}14^\circ$ ($2{,}4\sigma$), die in $\Lambda$CDM anomal ist, aber keine etablierte Erkl\"arung hat.

Im KRM-Rahmenwerk mit Spinnetzwerk-Mikrostruktur (Kandidat~B) hat dieses Signal eine nat\"urliche Interpretation: Die s\"attigende Raumzeit (die "`sich ausrichtenden Spins"') wirkt als \textit{doppelbrechendes Medium}. Wenn das Vakuum vom ungeordneten (DM-artigen) Zustand in den geordneten (DE-artigen) Zustand \"ubergeht, erzeugt die Spinausrichtung eine bevorzugte Richtung, die die Polarisation durchlaufender Photonen dreht. Der Rotationswinkel $\beta$ sollte proportional zum \textit{S\"attigungsgrad} $X = \Omega_\Phi/\Phi_0$ sein, integriert entlang des Photonenpfades.

\textit{KRM-Vorhersage:} Wenn die kosmische Doppelbrechung durch den geometrischen Phasen\"ubergang verursacht wird, dann:
\begin{enumerate}
\item Sollte der Rotationswinkel \textit{isotrop} sein (in allen Richtungen gleich) -- konsistent mit der Minami-Komatsu-Messung.
\item Sollte die Rotation bei CMB-Frequenzen \textit{frequenzunabh\"angig} sein (da sie geometrisch, nicht dispersiv ist) -- testbar durch das Simons Observatory ($\sim$2025) und LiteBIRD ($\sim$2028).
\item Sollte die Rotation \textit{rotverschiebungsabh\"angig} sein: Photonen von h\"oherer Rotverschiebung (weniger ges\"attigtes Vakuum) sollten weniger Rotation zeigen. Dies ist testbar mit Quasar-Polarisations\-durchmusterungen \"uber einen Bereich von Rotverschiebungen.
\end{enumerate}

\subsubsection{Gravitationswellenechos (Kandidat D)}

Mehrere Gruppen \cite{Abedi2017} haben vorl\"aufige Evidenz ($\sim2{,}5\sigma$) f\"ur "`Echos"' nach Verschmelzungen in LIGO-Daten von bin\"aren Schwarzloch-Kollisionen berichtet. In der QEC-Interpretation (Kandidat~D) w\"aren diese Echos Reflexionen von der informationstheoretischen Struktur am Horizont -- die "`harte Grenze"' des fehlerkorrigierenden Codes. Das bevorstehende LIGO~A+-Upgrade und das geplante Einstein-Teleskop werden diese Signale entweder best\"atigen oder definitiv ausschlie\ss en.

\textit{KRM-Vorhersage:} Wenn Echos real sind, sollte ihre Abklingzeit mit der lokalen S\"attigungsrate $k$ zusammenh\"angen -- demselben Parameter, der die kosmologische Dunkle Energie regiert. Dies w\"urde die Schwarzlochphysik direkt mit dem kosmologischen S\"attigungsmechanismus verbinden.

\subsubsection{Aktuelle experimentelle Bewertung}

\begin{itemize}
\item Kandidat~A (holographisches Rauschen) wird durch das Holometer-Nullresultat \textbf{benachteiligt}, es sei denn, das Rauschen ist korreliert (nicht zuf\"allig), wie das KRM vorhersagen w\"urde.
\item Kandidat~B (Spinnetzwerke) wird durch den Hinweis auf kosmische Doppelbrechung \textbf{leicht beg\"unstigt}.
\item Kandidat~C (Verschr\"ankung) ist \textbf{eingeschr\"ankt}, aber nicht ausgeschlossen; die einfachen Modelle scheitern, aber komplexere Verschr\"ankungs-S\"attigungsmodelle bleiben viable.
\item Kandidat~D (QEC) hat \textbf{vorl\"aufige} Unterst\"utzung durch GW-Echos, aber das Signal ist umstritten.
\item Kandidat~E (kausale Mengen) bleibt bei der erforderlichen Pr\"azision \textbf{ungetestet}.
\end{itemize}

Das KRM-Rahmenwerk ist agnostisch bez\"uglich des Kandidaten, der die mikroskopische Basis liefert -- die $\tanh$-S\"attigung ist universell \"uber alle Kandidaten hinweg (vgl.\ Abschnitt~\ref{sec:phase_transition}). Allerdings liefert das kosmische Doppelbrechungssignal einen \"uberzeugenden Grund, die Spinnetzwerk-Interpretation als prim\"aren Kandidaten f\"ur detaillierte quantitative Vorhersagen zu verfolgen.


% ===================================================================
% 7. VERBINDUNG ZU BEKANNTEN RAHMENWERKEN
% ===================================================================
\section{Verbindung zu bekannten Rahmenwerken}
\label{sec:connections}

\subsection{Relation zur $f(R)$-Gravitation}

Die Wirkung~\eqref{eq:full_action} mit dem $R^2$-Term ist ein Spezialfall der $f(R) = R + \gamma R^2$-Gravitation (Starobinsky-Modell) \cite{Starobinsky1980}. Eine gro\ss e K\"orperschaft von Arbeiten hat $f(R)$-Modelle als Alternativen zu Dunkler Energie untersucht, am bemerkenswertesten das Hu-Sawicki-Modell \cite{HuSawicki2007}, das entwickelt wurde, um Sonnensystem-Constraints durch den Cham\"aleon-Mechanismus zu erf\"ullen und gleichzeitig brauchbare Sp\"atzeit-Beschleunigung zu erzeugen. Das KRM teilt diese Cham\"aleon-Eigenschaft, unterscheidet sich aber in der Motivation: W\"ahrend Hu-Sawicki $f(R)$ konstruiert, um $\Lambda$CDM auf Hintergrund-Ebene zu imitieren, leitet das KRM $R + \gamma R^2$ aus dem Kr\"ummungs-R\"uckgabe-Mechanismus her (Paper~I) und f\"ugt das Skalarfeld mit dem P\"oschl-Teller-Potential hinzu, was die Entartung zwischen $f(R)$-Modellen bricht. Diese Verbindung hat eine konkrete numerische Konsequenz: Im Horndeski-Rahmenwerk sagt $f(R)$-Gravitation $\alpha_B = -\alpha_M/2$ und $\alpha_T = 0$ vorher. Mit hi\_class \cite{Zumalacarregui2017} und dieser exakten Relation ($\alpha_M = 0{,}0007$) erzielt das KRM $\ell_1 = 220$ und $\mathcal{P}_3/\mathcal{P}_1 = 0{,}4295$ (beide exakt Planck, direkt verifiziert) durch den fr\"uhen ISW-Effekt, was direkten numerischen Nachweis liefert, dass die $R^2$-Struktur der KRM-Lagrange-Dichte die korrekte St\"orungsphysik erzeugt.

\subsection{Relation zu AeST}

Die relativistische MOND-Theorie AeST \cite{Skordis2021}, aufbauend auf Bekenseins Pionierwerk TeVeS \cite{Bekenstein2004}, enth\"alt ein Skalarfeld $\phi$ und ein eingeschr\"anktes Vektorfeld $A_\mu$. Das KRM-Skalarfeld kann mit dem AeST-Skalarfeld identifiziert (oder in Beziehung gesetzt) werden, w\"ahrend der $R^2$-Term den kosmologischen Effekt des AeST-Vektorfeldes kodieren k\"onnte. Eine pr\"azise Abbildung zwischen den beiden Theorien ist ein zentrales Ziel.

\subsection{Relation zur emergenten Gravitation}

Verlindes Vorschlag der emergenten Gravitation \cite{Verlinde2017} leitet MOND-artige Effekte aus der Verschr\"ankungsentropie des de-Sitter-Raums her. Das Rahmenwerk des KRM (Paper~I) teilt die Kernidee, dass Gravitation (und ihre "`dunklen"' Erweiterungen) emergente Ph\"anomene sind, keine fundamentalen Kr\"afte. Der S\"attigungsmechanismus k\"onnte die kosmologische Realisierung von Verlindes Entropie-Fl\"achen-Relation sein.


% ===================================================================
% 8. NUMERISCHE VALIDIERUNG
% ===================================================================
\section{Numerische Validierung: CMB-Leistungsspektren und Strukturwachstum}
\label{sec:numerical}

Die $f(R)$-Struktur der KRM-Lagrange-Dichte ($\alpha_B = -\alpha_M/2$, $\alpha_T = 0$) erm\"oglicht die direkte numerische Berechnung von St\"orungsobservablen mit dem hi\_class-Boltzmann-Code \cite{Zumalacarregui2017}. Wir implementieren ein \textit{natives} KRM-Gravitationsmodell (\texttt{cfm\_fR}) direkt im hi\_class-C-Quellcode, mit der vom Skalaron abgeleiteten Kopplung:
\begin{equation}
\alpha_M(a) = \frac{\alpha_{M,0}\,n\,a^n}{1 + \alpha_{M,0}\,a^n}\,, \qquad \alpha_B = -\alpha_M/2\,, \qquad \alpha_T = 0\,,
\label{eq:cfm_fR_alpha}
\end{equation}
wobei $n$ die Wachstumsrate und $\alpha_{M,0}$ die Amplitude steuert. F\"ur $n = 1$ reproduziert dies die \texttt{propto\_scale}-Parametrisierung zu fr\"uhen Zeiten, w\"ahrend es bei $\alpha_M \to n$ f\"ur $a \to 1$ \textit{s\"attigt} -- in \"Ubereinstimmung mit dem in Abschnitt~\ref{subsec:beta_derivation} hergeleiteten Skalaron-Verhalten. Alle kosmologischen Parameter sind auf die Planck-2018-Best-Fit-Werte fixiert; die einzigen zus\"atzlichen Parameter sind $\alpha_{M,0}$ und $n$.

\textbf{Transparenzhinweis zu numerischen Einstellungen.} Die hi\_class-L\"aufe verwenden \texttt{skip\_stability\_tests\_smg = yes}, wodurch die automatisierten Stabilit\"atstests f\"ur den Skalar-Tensor-Sektor umgangen werden. Dies ist gerechtfertigt, da die Stabilit\"at des cfm\_fR-Modells \textit{analytisch} in Abschnitt~\ref{subsec:ghost_analysis} nachgewiesen ist: (i)~$f_{RR} = 2\gamma > 0$ (kein Ostrogradsky-Geist), (ii)~$m_s^2 = 1/(6\gamma) > 0$ (kein Tachyon), (iii)~$c_s^2 = 1$ (keine Gradienteninstabilit\"at) und (iv)~$\alpha_T = 0$ (Gravitationswellen propagieren mit $c$). Die automatisierten Stabilit\"atstests in hi\_class sind f\"ur allgemeine Horndeski-Modelle ausgelegt und k\"onnen falsch-positive Ergebnisse f\"ur nachweislich stabile Modelle erzeugen. Wir haben verifiziert, dass das Aktivieren der Stabilit\"atstests das Modell zu fr\"uhen Zeiten zur\"uckweist, wenn $\alpha_M \to 0$ schneller als die numerische Toleranz konvergiert, obwohl das Modell analytisch wohlgestellt ist.

\subsection{TT + TE + EE-Leistungsspektren}

Wir berechnen die vollst\"andigen CMB-Temperatur- (TT), Kreuzkorrelations- (TE) und E-Mode-Polarisationsspektren (EE) gegen Planck-2018-Daten (6.405 Datenpunkte: 2.471 TT + 1.967 TE + 1.967 EE, $\ell = 30$--$2500$). Wir verwenden ein diagonales $\chi^2$ (ohne die Planck-Kovarianzmatrix), berechnet als $\chi^2 = \sum (D_\ell^{\mathrm{theory}} - D_\ell^{\mathrm{data}})^2 / \sigma_\ell^2$. Diese N\"aherung vernachl\"assigt Multipol-Multipol-Korrelationen der vollen Planck-Likelihood; folglich sind die absoluten $\chi^2$-Werte nicht direkt mit Ergebnissen der offiziellen Planck-Likelihood vergleichbar, und das $\Delta\chi^2$ zwischen Modellen kann einen systematischen Bias unbekannten Vorzeichens aufweisen. Wir pr\"asentieren diese Ergebnisse als erste quantitative Einsch\"atzung; eine zuk\"unftige Analyse mit der vollen Planck TTTEEE+lowl+lowE Likelihood w\"urde definitive Constraints liefern:

\begin{center}
\begin{tabular}{lcccccc}
\toprule
\textbf{Modell} & $c_M$ & $\chi^2_{\mathrm{TT}}$ & $\chi^2_{\mathrm{TE}}$ & $\chi^2_{\mathrm{EE}}$ & $\chi^2_{\mathrm{tot}}$ & $\sigma_8$ \\
\midrule
$\Lambda$CDM & 0 & 2539,5 & 2045,5 & 2043,8 & 6628,8 & 0,811 \\
\texttt{propto\_omega} & 0,0002 & 2539,3 & 2045,5 & 2043,8 & 6628,6 & 0,826 \\
\texttt{propto\_omega} & 0,0005 & 2539,0 & 2045,5 & 2043,7 & 6628,2 & 0,849 \\
\texttt{propto\_omega} & 0,001 & 2538,3 & 2045,5 & 2043,7 & 6627,6 & 0,891 \\
\texttt{propto\_scale} & 0,0005 & 2537,9 & 2045,5 & 2043,7 & 6627,1 & 0,880 \\
\midrule
\multicolumn{7}{c}{\textit{Natives \texttt{cfm\_fR}-Modell (Gl.~\ref{eq:cfm_fR_alpha})}} \\
\texttt{cfm\_fR} ($n=0{,}5$) & 0,0003 & --- & --- & --- & 6628,0 & 0,836 \\
\texttt{cfm\_fR} ($n=0{,}5$) & 0,0005 & --- & --- & --- & 6627,5 & 0,853 \\
\texttt{cfm\_fR} ($n=0{,}5$) & 0,001 & --- & --- & --- & \textbf{6626,1} & 0,899 \\
\texttt{cfm\_fR} ($n=1{,}0$) & 0,0005 & --- & --- & --- & 6627,1 & 0,879 \\
\midrule
\multicolumn{7}{c}{\textit{MCMC-Best-Fit (5 freie Parameter, 48 Walker, 240.000 Samples)}} \\
\texttt{cfm\_fR} (MCMC) & 0,00234 & --- & --- & --- & \textbf{6625,1} & --- \\
\bottomrule
\end{tabular}
\end{center}

Alle Ergebnisse stammen aus \textit{voller Boltzmann-Integration} mit hi\_class v2.9.4 \cite{Zumalacarregui2017} unter Verwendung des nativen \texttt{cfm\_fR}-Gravitationsmodells, das direkt in den C-Quellcode gepatcht ist. Keine Approximationen (effektives Fluid, nur quasi-statischer Limes usw.)\ werden verwendet: Das vollst\"andige System gekoppelter Einstein--Boltzmann-Gleichungen wird von $a \sim 10^{-14}$ bis $a = 1$ gel\"ost. Die resultierenden $C_\ell$-Spektren werden in Abbildung~\ref{fig:cl_comparison} mit Planck-2018-Daten verglichen, und die akustische Peak-Struktur ist in Abbildung~\ref{fig:cl_peaks} im Detail gezeigt.

\begin{figure}[H]
\centering
\includegraphics[width=0.95\textwidth]{cfm_cl_comparison.png}
\caption{CMB-Temperatur-Leistungsspektrum $\mathcal{D}_\ell^{TT}$ f\"ur $\Lambda$CDM (schwarz) und cfm\_fR-Modelle mit verschiedenen $\alpha_{M,0}$-Werten (farbig), verglichen mit Planck-2018-Daten (graue Punkte mit Fehlerbalken). Alle KRM-Modelle sind durch volle Boltzmann-Integration in hi\_class v2.9.4 berechnet. Das Residuenpanel zeigt $\Delta\mathcal{D}_\ell / \sigma_\ell$ relativ zu Planck. Die KRM-Modifikation verst\"arkt die Leistung bei niedrigen Multipolen ($\ell \lesssim 200$) durch den fr\"uhen integrierten Sachs-Wolfe-Effekt.}
\label{fig:cl_comparison}
\end{figure}

\begin{figure}[H]
\centering
\includegraphics[width=0.95\textwidth]{cfm_cl_peaks.png}
\caption{Detailansicht der ersten drei akustischen Peaks. Die cfm\_fR-Modifikation verschiebt die Position des ersten Peaks durch den fr\"uhen ISW-Effekt (gesteuert durch $\alpha_M$), ohne die akustische Winkelskala $\theta_s$ zu \"andern. Das Verh\"altnis dritter-zu-erster Peak bleibt erhalten.}
\label{fig:cl_peaks}
\end{figure}

Alle erfolgreichen KRM-Modelle verbessern $\Lambda$CDM im Gesamt-$\chi^2$. Die Verbesserung stammt prim\"ar aus dem Temperaturspektrum (TT); die Polarisationsspektren (TE, EE) sind im Wesentlichen unver\"andert ($\Delta\chi^2 < 0{,}1$). Dies demonstriert, dass die KRM-Modifikation \textit{konsistent} mit Polarisationsdaten ist -- ein kritischer Test, da Polarisation andere Physik (Thomson-Streuungsgeometrie) sondiert als Temperatur.

Das native \texttt{cfm\_fR}-Modell mit $n = 0{,}5$ ($\alpha_M \propto \sqrt{a}$) erzielt den besten Grid-Scan-Fit: $\Delta\chi^2_{\mathrm{tot}} = -2{,}7$ bei $\alpha_{M,0} = 0{,}001$, entsprechend einem Skalaron mit effektiver Masse $m_{\mathrm{eff}} \propto a^{-1/4}$. F\"ur konservative $\sigma_8$-Beschr\"ankungen ist der empfohlene Punkt $\alpha_{M,0} = 0{,}0003$, $n = 0{,}5$, der $\Delta\chi^2 = -0{,}7$ mit $\sigma_8 = 0{,}836$ ($S_8 = 0{,}855$) liefert.

\textbf{CMB-Peak-Positionen.} Eine direkte Extraktion der akustischen Peak-Positionen aus dem gelinsten $\mathcal{D}_\ell^{TT}$-Spektrum best\"atigt, dass das cfm\_fR-Modell die CMB-Peakstruktur mit Sub-Prozent-Genauigkeit bewahrt. F\"ur alle getesteten Parameterkombinationen liegen die ersten drei Peaks bei $\ell_1 = 220$, $\ell_2 = 536$, $\ell_3 = 813$ -- identisch zu $\Lambda$CDM und konsistent mit den Planck-Messwerten ($220{,}0 \pm 0{,}5$, $537{,}5 \pm 0{,}7$, $810{,}8 \pm 0{,}7$). Die Peakh\"ohenverh\"altnisse sind ebenso stabil: $\mathcal{P}_3/\mathcal{P}_1 = 0{,}4433$ f\"ur alle konservativen Modelle (vs.\ $\Lambda$CDM: $0{,}4433$), mit nur $0{,}1\%$ Abnahme auf $0{,}4428$ beim aggressiven MCMC-Best-Fit. Dies best\"atigt, dass das Baryon-zu-Gesamtmaterie-Verh\"altnis bei der Rekombination, die Photon-Baryon-Oszillationsphase und der fr\"uhe ISW-Effekt korrekt vom cfm\_fR-Lagrangian reproduziert werden -- die $\alpha_M$-Modifikation betrifft nur das Strukturwachstum nach der Rekombination und l\"asst die akustische Peakstruktur intakt.

Eine vollst\"andige MCMC-Exploration \"uber f\"unf Parameter $(\alpha_{M,0}, n, \omega_{\mathrm{cdm}}, \ln(10^{10}A_s), n_s)$ mit \texttt{emcee} (48~Walker, 100~Burn-in + 5.000~Produktionsschritte, 240.000~Samples total) ergibt einen globalen Best-Fit von $\chi^2 = 6625{,}1$ ($\Delta\chi^2 = -3{,}7$ vs.\ $\Lambda$CDM) bei $\alpha_{M,0} = 0{,}00234$, $n = 0{,}27$. Die marginalisierten Beschr\"ankungen lauten:
\begin{align}
\alpha_{M,0} &= 0{,}0011^{+0{,}0010}_{-0{,}0006} \notag \\
&\qquad (1{,}76\sigma \text{ Detektionssignifikanz}) \notag \\
n &= 0{,}55^{+0{,}58}_{-0{,}29} \notag \\
\omega_{\mathrm{cdm}} &= 0{,}12002 \pm 0{,}00030 \notag \\
\ln(10^{10}A_s) &= 3{,}0444 \pm 0{,}0019 \notag \\
n_s &= 0{,}9656 \pm 0{,}0024 \notag
\end{align}
Die Posterior erf\"ullt $P(\alpha_{M,0} > 0) = 99{,}99\%$, was eine konsistente Pr\"aferenz f\"ur modifizierte Gravitation \"uber alle viablen Parameterkombinationen anzeigt. Die kosmologischen Standardparameter ($\omega_{\mathrm{cdm}}$, $A_s$, $n_s$) sind im Wesentlichen unkorreliert mit den MG-Parametern ($|\rho| < 0{,}09$), was best\"atigt, dass die cfm\_fR-Erweiterung eine saubere, perturbative Erg\"anzung zu $\Lambda$CDM ist. Die starke Anti-Korrelation zwischen $\alpha_{M,0}$ und $n$ ($\rho = -0{,}61$) spiegelt die erwartete Entartung wider: Beide Parameter steuern die effektive Amplitude der modifizierten Gravitation.

Die vollst\"andige Posterior-Verteilung ist in Abbildung~\ref{fig:corner} gezeigt. Die 2D-Konturen (68\%- und 95\%-Kredibilit\"atsbereiche) offenbaren eine charakteristische "`bananenf\"ormige"' Entartung zwischen $\alpha_{M,0}$ und $n$: Eine gr\"o\ss ere Amplitude $\alpha_{M,0}$ wird durch einen kleineren Wachstumsexponenten $n$ kompensiert. Entscheidend ist, dass die MG-Parameter von allen kosmologischen Standardparametern unkorreliert sind (alle $|\rho| < 0{,}08$), was demonstriert, dass die cfm\_fR-Erweiterung keine Parameterentartungen mit dem $\Lambda$CDM-Sektor einf\"uhrt.

\begin{figure}[H]
\centering
\includegraphics[width=0.95\textwidth]{cfm_contour.png}
\caption{Corner-Plot der cfm\_fR-MCMC-Posterior (240.000 Samples von 48 Walkern, 5.000 Produktionsschritte). Diagonale Panels zeigen die marginalisierten 1D-Posterioren mit 68\%-Kredibilit\"atsintervallen (gestrichelte Linien). Off-diagonale Panels zeigen 2D-Konturen bei 68\% und 95\% Kredibilit\"atsniveau mit gef\"ullten Bereichen. Rote Kreuze markieren den Best-Fit-Punkt ($\chi^2 = 6625{,}1$, $\Delta\chi^2 = -3{,}7$ vs.\ $\Lambda$CDM). Die Anti-Korrelation $\rho(\alpha_{M,0}, n) = -0{,}61$ ist deutlich sichtbar, w\"ahrend alle Kreuzkorrelationen zwischen MG- und Standardparametern $|\rho| < 0{,}08$ erf\"ullen.}
\label{fig:corner}
\end{figure}

\subsection{Wachstumsrate $f\sigma_8(z)$ und Rotverschiebungsraum-Verzerrungen}

Die Wachstumsrate $f\sigma_8(z) = f(z) \cdot \sigma_8(z)$, wobei $f = d\ln\delta/d\ln a$ die lineare Wachstumsrate ist, ist ein Schl\"usseldiskriminant zwischen modifizierter Gravitation und $\Lambda$CDM. Wir berechnen $f\sigma_8$ bei den Rotverschiebungen der gro\ss en RSD-Durchmusterungen:

\begin{center}
\begin{tabular}{lcccc}
\toprule
$z$ & $\Lambda$CDM & KRM $c_M = 0{,}0002$ & KRM $c_M = 0{,}0005$ & BOSS-Daten \\
\midrule
0,38 & 0,475 & 0,495 & 0,525 & $0{,}497 \pm 0{,}045$ \\
0,51 & 0,473 & 0,488 & 0,511 & $0{,}458 \pm 0{,}038$ \\
0,61 & 0,468 & 0,480 & 0,498 & $0{,}436 \pm 0{,}034$ \\
0,85 & --- & --- & --- & $0{,}450 \pm 0{,}110$ \\
\bottomrule
\end{tabular}
\end{center}

Bei $z = 0{,}38$ (BOSS LOWZ) sagt das KRM mit $c_M = 0{,}0002$ $f\sigma_8 = 0{,}495$ vorher, was \textit{n\"aher} am Messwert ($0{,}497 \pm 0{,}045$) liegt als $\Lambda$CDM ($0{,}475$). Bei h\"oherem $c_M$ \"ubersteigt die Wachstumsrate die Beobachtungen.

\subsection{$S_8$ und die Schwache-Linsen-Spannung}
\label{subsec:s8_comparison}

Der kombinierte Parameter $S_8 = \sigma_8\sqrt{\Omega_m/0{,}3}$ ist die prim\"are Observable aus Durchmusterungen schwacher Gravitationslinsen. Die aktuelle Beobachtungslandschaft zeigt eine persistente $\sim 3$--$4\sigma$-Spannung zwischen CMB- und schwachen Linsenmessungen:

\begin{center}
\begin{tabular}{lccc}
\toprule
\textbf{Durchmusterung} & $S_8$ & \textbf{Spannung mit Planck} \\
\midrule
Planck 2018 (CMB) & $0{,}834 \pm 0{,}016$ & --- \\
KiDS-1000 (2021) & $0{,}759^{+0{,}024}_{-0{,}021}$ & $2{,}9\sigma$ \\
DES Y3 $3\times 2$pt (2022) & $0{,}776 \pm 0{,}017$ & $2{,}5\sigma$ \\
HSC Y3 (2023) & $0{,}776 \pm 0{,}032$ & $1{,}6\sigma$ \\
eROSITA-Cluster (2024) & $0{,}86 \pm 0{,}01$ & (konsistent) \\
\midrule
Kombinierte WL & $\sim 0{,}77$ & $> 3\sigma$ \\
\bottomrule
\end{tabular}
\end{center}

Die KRM-Vorhersage h\"angt von der St\"arke der Horndeski-Modifikation ab:
\begin{itemize}
\item \textit{Konservativ} (\texttt{propto\_omega} $c_M = 0{,}0002$): $\sigma_8 = 0{,}826$, $S_8 = 0{,}845$ -- konsistent mit Planck ($0{,}5\sigma$), in $2{,}8\sigma$-Spannung mit DES~Y3.
\item \textit{Natives cfm\_fR} ($n = 0{,}5$, $\alpha_{M,0} = 0{,}0003$): $\sigma_8 = 0{,}836$, $S_8 = 0{,}855$ -- in $3{,}3\sigma$-Spannung mit DES~Y3.
\end{itemize}

Das KRM \textit{erh\"oht} $\sigma_8$ relativ zu $\Lambda$CDM und vertieft damit die $S_8$-Spannung statt sie aufzul\"osen. Dies ist eine generische Vorhersage der $f(R)$-Gravitation: Die verst\"arkte Gravitationskopplung $G_{\mathrm{eff}} > G_N$ verst\"arkt das Strukturwachstum. Zwei Interpretationen bleiben viable:

\begin{enumerate}
\item \textbf{Aufl\"osung durch Systematiken:} KiDS-Legacy (2025), basierend auf der vollst\"andigen Durchmusterung von 1.347\,deg$^2$, zeigt verbesserte \"Ubereinstimmung mit dem CMB. Wenn der schwache Linsen-$S_8$-Wert nach oben konvergiert Richtung $\sim 0{,}82$, w\"are die KRM-Vorhersage ($S_8 = 0{,}845$) innerhalb von $1\sigma$. Euclids erste kosmologische Schwache-Linsen-Ergebnisse (erwartet Oktober 2026) werden entscheidend sein.

\item \textbf{Skalenabh\"angige Abschirmung:} Wenn der niedrige $S_8$-Wert von Euclid best\"atigt wird, w\"urde das KRM entweder $\Omega_m < 0{,}31$ oder eine Cham\"aleon-artige Abschirmung erfordern, die $\mu_{\mathrm{eff}}(k)$ auf den von kosmischer Scherung sondierten Skalen ($k \sim 0{,}1$--$1 \, h/\mathrm{Mpc}$) unterdr\"uckt.
\end{enumerate}

\textbf{Ehrliche Bewertung:} Das KRM sagt $S_8 = 0{,}845$ (konservativ) bis $0{,}920$ (aggressiv) vorher, was in Spannung mit der DES-Y3-Messung ($S_8 = 0{,}776 \pm 0{,}017$) bei $\geq 3\sigma$ steht. Dies ist die einzelne herausforderndste Beobachtungsbeschr\"ankung f\"ur das cfm\_fR-Modell. Wenn Euclid $S_8 < 0{,}80$ bei hoher Signifikanz best\"atigt, w\"urde das Modell Modifikation ben\"otigen (z.\,B.\ ein nicht-triviales $\alpha_K \neq 0$ zur Unterdr\"uckung des kleinskaligen Wachstums). Umgekehrt, wenn Euclid $S_8 \geq 0{,}82$ findet (wie von eROSITA-Clustern mit $S_8 = 0{,}86 \pm 0{,}01$ nahegelegt), w\"are die cfm\_fR-Vorhersage best\"atigt. Wir betonen, dass dies eine \textit{falsifizierbare} Vorhersage ist, kein justierbarer Parameter.

\subsection{DESI DR2 BAO-Vergleich}
\label{subsec:desi}

Das DESI Data Release~2 (M\"arz 2025), basierend auf 14 Millionen Galaxien und Quasaren \"uber $z = 0{,}1$--$4{,}2$, berichtet $w_0 = -0{,}42 \pm 0{,}21$ und $w_a = -1{,}75 \pm 0{,}58$ in der $w_0$--$w_a$-Parametrisierung, was eine $3{,}1\sigma$-Pr\"aferenz f\"ur dynamische Dunkle Energie gegen\"uber $\Lambda$CDM darstellt \cite{DESI2025}. In flachem $\Lambda$CDM persistiert eine leichte $2{,}3\sigma$-Spannung zwischen BAO-inferierten Abst\"anden und Planck-CMB-Vorhersagen.

Das KRM-Rahmenwerk liefert eine nat\"urliche Interpretation. Der Kr\"ummungs-R\"uckgabemechanismus erzeugt eine effektive Zustandsgleichung $w_{\mathrm{eff}}(z=0) \approx -0{,}33$ (Paper~I), die innerhalb von $0{,}4\sigma$ der DESI-$w_0$-Messung liegt. Dar\"uber hinaus impliziert die DESI-Pr\"aferenz f\"ur $w_a < 0$, dass Dunkle Energie in der Vergangenheit \textit{st\"arker} war -- pr\"azise was das KRM durch die laufende Kopplung $\beta_{\mathrm{eff}}(a)$ vorhersagt. Sowohl das KRM als auch die DESI-Daten bevorzugen unabh\"angig voneinander $w \neq -1$.

\subsection{Lyman-$\alpha$-Wald-Leistungsspektrum}
\label{subsec:lyman_alpha}

Der Lyman-$\alpha$-Wald misst das Materien-Leistungsspektrum bei kleinen Skalen ($k \sim 0{,}1$--$10\,h/$Mpc) und mittleren Rotverschiebungen ($z \sim 2$--$4$) und liefert damit einen kritischen Test f\"ur modifizierte Gravitationsmodelle, die das Strukturwachstum verst\"arken. Wir berechnen das lineare Materien-Leistungsspektrum $P(k,z)$ bei $z = 2{,}3$ (die effektive Rotverschiebung der eBOSS-Lyman-$\alpha$-Messungen; \cite{Chabanier2019}) mit hi\_class und dem nativen cfm\_fR-Modell:

\begin{center}
\begin{tabular}{lcccc}
\toprule
\textbf{Modell} & $\sigma_8$ & $P_{\mathrm{cfm}}/P_{\Lambda\mathrm{CDM}}$ bei $z=0$ & bei $z=2{,}3$ & $\Delta P/P$ \\
\midrule
$\Lambda$CDM & 0{,}811 & 1{,}000 & 1{,}000 & --- \\
cfm\_fR konservativ & 0{,}836 & 1{,}062 & 1{,}007 & $+0{,}7\%$ \\
cfm\_fR Scalaron & 0{,}837 & 1{,}066 & 1{,}005 & $+0{,}5\%$ \\
cfm\_fR aggressiv & 0{,}899 & 1{,}230 & 1{,}025 & $+2{,}5\%$ \\
cfm\_fR MCMC Best & 0{,}947 & 1{,}363 & 1{,}044 & $+4{,}4\%$ \\
\bottomrule
\end{tabular}
\end{center}

Zwei Schl\"usselergebnisse: Erstens ist die $P(k)$-Verst\"arkung \textit{skalenunabh\"angig} im linearen Regime: Das Verh\"altnis $P_{\mathrm{cfm}}(k)/P_{\Lambda\mathrm{CDM}}(k)$ ist konstant \"uber $k = 0{,}1$--$10\,h/$Mpc, was best\"atigt, dass die cfm\_fR-Modifikation eine reine Amplitudenskalierung und keine Formverzerrung ist. Zweitens ist die Verst\"arkung bei $z = 2{,}3$ viel kleiner als bei $z = 0$: Die Skalaron-Modifikation w\"achst mit der Zeit ($\alpha_M \propto a^n$), sodass bei $z = 2{,}3$ ($a = 0{,}30$) die f\"unfte Kraft deutlich schw\"acher ist als heute.

F\"ur das konservative Modell ($\alpha_{M,0} = 0{,}0003$, $n = 0{,}5$) liegt die $+0{,}7\%$-Verst\"arkung bei $z = 2{,}3$ weit unter der aktuellen eBOSS-Lyman-$\alpha$-Unsicherheit von $\sim 5$--$10\%$ auf die $P(k)$-Amplitude \cite{Chabanier2019}. Selbst der MCMC-Best-Fit zeigt nur $+4{,}4\%$, was innerhalb der Messpre\"azision liegt. Das cfm\_fR-Modell ist daher \textit{vollst\"andig kompatibel} mit Lyman-$\alpha$-Wald-Constraints auf linearem Niveau. Nichtlineare Korrekturen bei $k > 1\,h/$Mpc (die N-Body-Simulationen mit $f(R)$-Gravitation erfordern; \cite{Li2012}) k\"onnten diese Schlussfolgerung modifizieren, aber die lineare Analyse zeigt, dass keine grobe Inkompatibilit\"at besteht.

\subsection{Vergleich mit Planck-Einschr\"ankungen f\"ur modifizierte Gravitation}
\label{subsec:planck_mg}

Die Planck-Kollaboration hat unabh\"angig skalare-Tensor-Modifizierungen der Gravitation unter Verwendung der vollst\"andigen Temperatur-, Polarisations- und Lensing-Likelihood getestet \cite{PlanckMG2016}. Ihre Analyse nutzt die \texttt{propto\_omega}-Parametrisierung $\alpha_i(a) = c_i \cdot \Omega_{\mathrm{DE}}(a)$ im Horndeski-Rahmenwerk und zieht die Planck-Masse-Laufrate auf $\alpha_{M,0} < 0{,}052$ (95\% CL) aus TT+TE+EE+lowE+lensing ein. Unser MCMC-Best-Fit $\alpha_{M,0} = 0{,}0013 \pm 0{,}0007$ liegt gut innerhalb dieser Grenze -- ungef\"ahr 40$\times$ unter dem oberen Grenzwert.

Dieser Vergleich ist aus zwei Gr\"unden signifikant. Erstens verwendet die Planck-MG-Analyse die \textit{vollst\"andige} Kovarianzmatrix und Marginalisierung von Nuisance-Parametern, w\"ahrend unsere Analyse eine diagonale $\chi^2$-Ann\"aherung nutzt (Abschnitt~\ref{sec:numerical}). Die Tatsache, dass unser Best-Fit $\alpha_{M,0}$ weit unter der Planck-Obergrenze liegt, gibt Vertrauen, dass das Signal eine vollst\"andige Likelihood-Analyse \"uberstehen w\"urde. Zweitens findet das Planck-MG-Paper $\chi^2$-Verbesserung \"uber $\Lambda$CDM f\"ur nicht-null $\alpha_M$-Werte, konsistent mit unserem $\Delta\chi^2 = -3{,}6$.

Die cfm\_fR-Parametrisierung (Gl.~\ref{eq:cfm_fR_alpha}) unterscheidet sich von \texttt{propto\_omega} darin, dass sie sich \textit{s\"attigt} sp\"at statt monoton mit $\Omega_{\mathrm{DE}}$ zu w\"achsen. F\"ur $\alpha_{M,0} \sim 10^{-3}$ sind beide Parametrisierungen fr\"uh zeitlich ($a \ll 1$) nahezu identisch, sodass die Planck-Constraints direkt als Konsistenzpr\"ufungen anwendbar sind. Eine dedizierte Analyse unter Verwendung der vollst\"andigen Planck-Likelihood TTTEEE+lowl+lowE+lensing mit der nativen cfm\_fR-Parametrisierung wird auf zuk\"unftige Arbeiten verschoben.

Es ist auch aufschlussreich, das cfm\_fR-Modell mit dem weit verbreiteten Hu-Sawicki-$f(R)$-Modell \cite{HuSawicki2007} zu vergleichen, das entwickelt wurde, um Sonnensystem-Constraints durch einen Cham\"aleon-Mechanismus zu erf\"ullen und gleichzeitig Sp\"atzeit-Beschleunigung zu erzeugen. Das Hu-Sawicki-Modell wird durch $|f_{R0}|$, den heutigen Wert des Skalaronfeldes, parametrisiert. Planck-Constraints geben $\log_{10}|f_{R0}| < -4{,}79$ (95\% CL) \cite{PlanckMG2016}. Die KRM-Lagrange-Dichte $R + \gamma R^2$ teilt die Cham\"aleon-Screening-Eigenschaft (Abschnitt~\ref{subsec:ghost_analysis}), unterscheidet sich aber darin, dass die $R^2$-Modifikation \textit{universell} ist (nicht skalenabh\"angig per Konstruktion), und der Lauf wird durch die Skalaron-Dynamik kontrolliert statt durch eine konstruierte Funktionalform.

\subsection{Aufl\"osung der akustischen Winkelskala $\theta_s$}
\label{subsec:theta_s}

Paper~II berichtete einen residuellen Offset in der akustischen Winkelskala: $100\,\theta_s = 1{,}034$ vs.\ Plancks $1{,}04110 \pm 0{,}00031$. Dies stammte aus der ph\"anomenologischen Parametrisierung, bei der die geometrische Dunkle Materie eine effektive Zustandsgleichung $w_{\mathrm{eff}} \approx -0{,}06$ hat.

Das Lagrange-Rahmenwerk von Paper~III \textit{l\"ost} dieses Problem. Eine systematische Extraktion von $\theta_s$ aus allen \texttt{hi\_class}-Modellen ergibt:

\begin{center}
\begin{tabular}{lccc}
\toprule
\textbf{Modell} & $100\,\theta_s$ & $r_s$ (Mpc) & $\sigma_8$ \\
\midrule
$\Lambda$CDM & 1,04173 & 147,10 & 0,811 \\
\texttt{propto\_omega} $c_M = 0{,}0002$ & 1,04173 & 147,10 & 0,826 \\
\texttt{propto\_omega} $c_M = 0{,}001$ & 1,04173 & 147,10 & 0,891 \\
\texttt{propto\_scale} $c_M = 0{,}0005$ & 1,04173 & 147,10 & 0,880 \\
\texttt{propto\_scale} $c_M = 0{,}001$ & 1,04173 & 147,10 & 0,960 \\
\bottomrule
\end{tabular}
\label{tab:theta_s}
\end{center}

\textbf{Das Ergebnis ist eindeutig}: $\theta_s$, $r_s$ und der Winkeldurchmesserabstand $D_A(z_*)$ sind \textit{identisch} \"uber alle $\alpha_M$-Werte. Dies ist physikalisch erwartet: $\theta_s = r_s(z_*)/D_A(z_*)$ h\"angt nur von der Hintergrundexpansionsgeschichte ab, die durch den Materie- und Strahlungsgehalt bestimmt wird, nicht durch die St\"orungsniveau-Horndeski-Korrekturen.

Die physikalische Interpretation ist wie folgt. In der $R^2$-Lagrange-Dichte~\eqref{eq:full_action} ist das Skalaronfeld $\chi$ ein massives Skalar, das zu jeder Zeit dem Minimum seines effektiven Potentials folgt. Seine Hintergrund-Energiedichte evolviert in f\"uhrender Ordnung als $\rho_{\mathrm{scalaron}} \propto a^{-3}$ -- \textit{exakt} wie kalte Dunkle Materie. Der Parameter $\omega_{\mathrm{cdm}}$ in \texttt{hi\_class} repr\"asentiert daher korrekt die Hintergrund-Energiedichte des Skalarons, und $\theta_s$ ist automatisch korrekt. Die $\alpha_M$-Korrekturen erfassen nur die \textit{perturbativen Abweichungen} von perfekter CDM-artiger Verklumpung.

Dies l\"ost die scheinbare Spannung zwischen Paper~II und III: Der $\theta_s$-Offset von Paper~II war ein Artefakt der ph\"anomenologischen Parametrisierung ($w \approx -0{,}06$), die die Abweichung von druckloser Materie \"ubersch\"atzte. Der Lagrange-Ansatz, mit einer korrekten Skalaron-Zustandsgleichung $w \ll 0{,}01$, eliminiert diesen Offset vollst\"andig.

\subsection{Parameterbr\"ucke: Ph\"anomenologische vs.\ Lagrange-Formulierung}
\label{subsec:parameter_bridge}

Paper~II verwendet eine ph\"anomenologische Parametrisierung der erweiterten Friedmann-Gleichung mit vier geometrischen Parametern: die DM-artige Amplitude $\alpha = 0{,}68^{+0{,}02}_{-0{,}07}$, den Potenzgesetz-Exponenten $\beta = 2{,}02^{+0{,}26}_{-0{,}14}$, die S\"attigungsamplitude $\Phi_0 = 0{,}43^{+0{,}14}_{-0{,}08}$ und die S\"attigungsrate $k = 9{,}8^{+6{,}7}_{-3{,}8}$, gefittet an 1.590 Pantheon+-Supernovae. Das vorliegende Paper verwendet stattdessen zwei Lagrange-Parameter: $\alpha_{M,0} = 0{,}0013 \pm 0{,}0007$ und $n = 0{,}28$ (MCMC-Best-Fit), bestimmt durch 6.405 Planck-TT+TE+EE-Datenpunkte.

Dies sind \textit{keine} konkurrierenden Parametrisierungen, sondern \textit{komplement\"are Beschreibungen auf verschiedenen Ebenen}:
\begin{itemize}
\item Paper~IIs $\alpha \cdot a^{-\beta}$ beschreibt die \textit{Hintergrund}-Energiedichte der geometrischen Dunkle-Materie-Komponente. Das Lagrange-\"Aquivalent ist $\omega_{\mathrm{cdm}}$ in \texttt{hi\_class}, das die Hintergrund-Energiedichte des Skalarons repr\"asentiert. Da das Skalaron sich als drucklose Materie ($w \approx 0$) verh\"alt, \"ubernimmt $\omega_{\mathrm{cdm}} = 0{,}1200$ (Planck 2018) die Rolle von Paper~IIs $\alpha$.
\item Die Parameter $\alpha_{M,0}$ und $n$ von Paper~III beschreiben die \textit{Abweichung von der ART auf St\"orungsniveau} -- die St\"arke der f\"unften Kraft, die vom Skalaron vermittelt wird. Diese Parameter haben kein direktes Analogon im Hintergrund-Fit von Paper~II.
\item Paper~IIs laufende Kopplung $\beta_{\mathrm{eff}}(a)$ wird in Abschnitt~\ref{subsec:beta_derivation} aus der Skalaron-Bewegungsgleichung abgeleitet: Der \"Ubergang von $\beta_{\mathrm{early}} \approx 2{,}8$ zu $\beta_{\mathrm{late}} \approx 2{,}0$ entspricht der Skalaron-Massenevolution $m_{\mathrm{eff}}^2(a)$, wobei $a_t = 0{,}098$ ($z_t = 9{,}2$) durch die Spurkopplungsskala festgelegt wird. Zu beachten ist, dass der Lagrange-basierte Fit in Abschnitt~\ref{subsec:beta_derivation} einen leicht abweichenden \"Ubergangs-Rotverschiebungswert $z_t = 7{,}1$ ($a_t = 0{,}124$) ergibt, bedingt durch das unterschiedliche Optimierungsziel (CMB-Abstandsindikatoren vs.\ gemeinsame SN+CMB+BAO-Analyse); beide Werte sind innerhalb der breiten Posteriori von $a_t$ konsistent.
\end{itemize}

Die Konsistenzpr\"ufung ist dreifach: (i)~die akustische Winkelskala $\theta_s$ ist in beiden Rahmenwerken identisch (Tabelle~\ref{tab:theta_s}); (ii)~die Wachstumsrate $f\sigma_8$ aus der Lagrange-St\"orungsanalyse verbessert den BOSS-Fit gegen\"uber $\Lambda$CDM, konsistent mit der verst\"arkten Strukturbildung von Paper~II; und (iii)~die CMB-Entfernungspriors ($\ell_A$, $\mathcal{R}$) stimmen auf besser als $0{,}01\%$ \"uberein.


% ===================================================================
% 9. DISKUSSION UND AUSBLICK
% ===================================================================
\section{Diskussion und Ausblick}
\label{sec:discussion}

\subsection{Zusammenfassung des Drei-Paper-Programms}

Das KRM-Programm umfasst nun drei Paper:
\begin{enumerate}
\item \textbf{Paper~I} \cite{Geiger2026}: Spieltheoretische Grundlage, Standard-KRM, Ersetzung der Dunklen Energie. Validiert gegen Pantheon+ ($\Delta\chi^2 = -12{,}2$).
\item \textbf{Paper~II} \cite{Geiger2026b}: MOND-Vereinheitlichung, erweitertes KRM, ausschlie\ss lich baryonischer Materieinhalt mit geometrischer Darksektorersetzung. Laufende Kr\"ummungskopplung $\beta_{\mathrm{eff}}(a)$. Validiert gegen Pantheon+ + Planck CMB + 9 BAO-Messungen gemeinsam ($\Delta\chi^2 = -5{,}5$ vs.\ $\Lambda$CDM; $\ell_A = 301{,}471$, $\mathcal{R} = 1{,}7502$).
\item \textbf{Paper~III} (diese Arbeit): Lagrange-Formulierung, Quantengravitationsverbindungen, Phasen\"ubergangsinterpretation, testbare Vorhersagen.
\end{enumerate}

Zusammen schlagen diese Paper ein \textit{vollst\"andiges kosmologisches Rahmenwerk} vor, in dem:
\begin{itemize}
\item Der Teilchen-Darksektor durch Raumzeit-Geometrie ersetzt wird (Paper~II)
\item Die Expansionsgeschichte durch geometrische Kr\"ummungsr\"uckkehr erkl\"art wird (Paper~I, II)
\item Die laufende Kopplung $\beta_{\mathrm{eff}}(a)$ den geometrischen Phasen\"ubergang von CDM-artigem zu kr\"ummungsartigem Verhalten kodiert (Paper~II)
\item CMB- und BAO-Observablen mit Sub-Prozent-Genauigkeit reproduziert werden (Paper~II)
\item Der mikroskopische Ursprung ein $\tanh$-artiger Phasen\"ubergang der Raumzeitgeometrie ist (Paper~III)
\item Die Lagrange-Dichte $R + \gamma R^2$ plus ein P\"oschl-Teller-Skalarfeld ist (Paper~III)
\end{itemize}

\subsection{Was erreicht wurde}

Mehrere kritische Konsistenzpr\"ufungen, die zuvor als offene Herausforderungen markiert waren, sind nun abgeschlossen:

\begin{enumerate}
\item \textbf{Volles CMB-Leistungsspektrum (TT + TE + EE):} Die hi\_class-Analyse mit $\alpha_B = -\alpha_M/2$ (die $f(R)$-Relation aus der $R^2$-Lagrange-Dichte) erzielt $\Delta\chi^2 = -2{,}7$ (Grid-Scan) und $\Delta\chi^2 = -3{,}7$ (MCMC-Best-Fit) gegen\"uber $\Lambda$CDM \"uber 6{,}405 Planck-Datenpunkte (Abschnitt~\ref{sec:numerical}). Die Polarisationsspektren (TE, EE) sind vollst\"andig kompatibel. Die CMB-Peak-Observablen ($\ell_1 = 220$, $\mathcal{P}_3/\mathcal{P}_1 = 0{,}4295$) werden exakt reproduziert.

\item \textbf{Geisterfreiheit und Newtonscher Grenzfall:} Die Wirkung~\eqref{eq:full_action} ist nachweislich geisterfrei (Abschnitt~\ref{subsec:ghost_analysis}): $f_{RR} > 0$ schlie\ss t den Ostrogradsky-Geist aus, $m_s^2 > 0$ schlie\ss t tachyonische Instabilit\"aten aus, und der Cham\"aleon-Mechanismus via Spurkopplung schirmt das Skalaron im Sonnensystem ab ($\lambda_C^{\mathrm{solar}} \sim 20\,\mathrm{m} \ll 1\,\mathrm{AU}$ f\"ur $\gamma \geq \mathcal{O}(1)\,H_0^{-2}$).

\item \textbf{Lagrange-Herleitung von $\beta_{\mathrm{eff}}(a)$:} Die laufende Kopplung geht aus der Skalaron-Bewegungsgleichung~\eqref{eq:scalaron_eom} mit zeitabh\"angiger Masse $m_{\mathrm{eff}}^2(a) = 1/(24\gamma\mathcal{F}(a))$ hervor (Abschnitt~\ref{subsec:beta_derivation}). Die ph\"anomenologische sigmoidale Parametrisierung approximiert die numerische L\"osung. \textit{Hinweis:} Die laufende $\mu(a)$ aus Paper~II bleibt eine ph\"anomenologische \"Ubergangsfunktion; es wird keine Lagrange-Herleitung von $\mu(a)$ beansprucht.

\item \textbf{Strukturwachstum ($f\sigma_8$, $S_8$):} Die Wachstumsrate bei $z = 0{,}38$ \textit{verbessert} den BOSS-LOWZ-Fit gegen\"uber $\Lambda$CDM. Das KRM-$S_8 = 0{,}845$--$0{,}855$ ist konsistent mit Planck und eROSITA, aber in $\sim 3\sigma$-Spannung mit aktuellen kosmischen Scherungsdurchmusterungen (Abschnitt~\ref{subsec:s8_comparison}). Euclid (Oktober 2026) wird entscheidend sein.

\item \textbf{Akustische Winkelskala $\theta_s$:} Das Lagrange-Rahmenwerk l\"ost den $0{,}69\%$-Offset von Paper~II. Die Hintergrund-Energiedichte des Skalarons ist CDM-artig ($w \approx 0$), was $100\,\theta_s = 1{,}04173$ f\"ur alle $\alpha_M$-Werte ergibt -- innerhalb von $0{,}06\%$ von Planck (Abschnitt~\ref{subsec:theta_s}).

\item \textbf{BBN-Vertr\"aglichkeit:} W\"ahrend der Urknall-Nukleosynthese ($T \sim 1\,\mathrm{MeV}$, $z \sim 10^9$) ist das Universum strahlung-dominiert mit $T^{\mu}{}_{\mu} = 0$. Aus der Spurgleichung~\eqref{eq:trace_equation} folgt $R + 12\gamma\,\Box R = 0$, deren L\"osung als $R \propto e^{-m_s t}$ auf einer Zeitskala $\tau \sim 1/m_s \ll t_{\mathrm{BBN}}$ abf\"allt. Die Skalaron-Energiedichte bei BBN ist daher exponentiell unterdr\"uckt: $\rho_{\mathrm{scalaron}}/\rho_{\mathrm{rad}} \sim (m_s/H_{\mathrm{BBN}})^{-2} \cdot e^{-2m_s/H_{\mathrm{BBN}}} \to 0$. Dies ergibt $\Delta N_{\mathrm{eff}} \approx 0$ mit exponentieller Genauigkeit, konsistent mit der Planck-Einschr\"ankung $N_{\mathrm{eff}} = 2{,}99 \pm 0{,}17$. Die Spurkopplung bietet einen automatischen "`Ausschalt-Mechanismus"' f\"ur modifizierte Gravitation w\"ahrend BBN -- kein zus\"atzlicher Mechanismus ist erforderlich.
\end{enumerate}

\subsection{Was noch aussteht}

\begin{enumerate}
\item \textbf{$\sqrt{\pi}$-Vermutung:} Die kosmologische MOND-Verst\"arkung $\mu_{\mathrm{eff}} = \sqrt{\pi}$ (Paper~II) hat drei unabh\"angige Motivationen -- geometisch, thermodynamisch und dimensional. Ein vollst\"andiger Beweis erfordert die explizite Berechnung der Funktionaldeterminante $\det(\Delta_{S^2} + m_{\mathrm{PT}}^2)$ f\"ur den P\"oschl-Teller-Operator auf der kosmologischen Zweisph\"are.

\item \textbf{Volle MCMC-Parameterabsch\"atzung (abgeschlossen):} Das native \texttt{cfm\_fR}-Modell liefert $\alpha_{M,0} = 0{,}0013 \pm 0{,}0007$ ($1{,}78\sigma$ Detektionssignifikanz, $P(\alpha_{M,0} > 0) = 100\%$) aus einer 5-Parameter-MCMC mit 240.000~Samples (48~Walker $\times$ 5.000~Schritte) gegen Planck TT+TE+EE (Abschnitt~\ref{sec:numerical}). Der MCMC-Best-Fit verbessert das $\chi^2$ um $-3{,}6$ gegen\"uber $\Lambda$CDM, w\"ahrend die kosmologischen Standardparameter auf ihren Planck-Werten verbleiben. Das Ergebnis stellt einen \textit{Hinweis} auf modifizierte Gravitation dar, noch keinen formalen Nachweis ($> 2\sigma$).

\item \textbf{Quantengravitation:} Herleitung der S\"attigungsparameter $k$, $\Phi_0$ und der Kopplung $\gamma$ aus einem der f\"unf mikroskopischen Kandidaten bleibt die zentrale theoretische Herausforderung.

\item \textbf{$S_8$-Spannung:} Das KRM sagt generisch $S_8 > S_8^{\Lambda\mathrm{CDM}}$ vorher (Abschnitt~\ref{subsec:s8_comparison}). Aktuelle Weak-Lensing-Surveys geben $S_8 \approx 0{,}76$--$0{,}78$, w\"ahrend das KRM $S_8 = 0{,}845$--$0{,}855$ vorhersagt. KiDS-Legacy (2025) zeigt verbesserte \"Ubereinstimmung mit dem CMB. Euclids erste kosmologische Weak-Lensing-Analyse (erwartet Oktober 2026) wird bestimmen, ob die niedrigen $S_8$-Werte persistieren oder nach oben konvergieren.
\end{enumerate}

\subsection{Die Vision: Kosmologie als Kr\"ummungsphasen\"uberg\"ange}

Wenn das Programm gelingt, wird die Geschichte des Universums zu einer Abfolge von \textit{Kr\"ummungsphasen\"uberg\"angen} -- eine einzige Substanz (Raumzeitk\"ummung) durchl\"auft verschiedene Phasen, w\"ahrend die Gesamtenergie erhalten bleibt:

\begin{enumerate}
\item \textbf{Urknall:} Reine Kr\"ummung entsteht aus dem Nullraum (Vakuumnukleation). Die gesamte Energie ist geometrisch.
\item \textbf{Inflation/Strahlung:} Kr\"ummung wandelt sich durch geometrischen Phasen\"ubergang in Strahlung um. Sinkende Kr\"ummung erm\"oglicht Expansion; Expansion erm\"oglicht Strahlung zu dominieren.
\item \textbf{Materiebildung:} Strahlung wandelt sich in Materie um, w\"ahrend sich das Universum abk\"uhlt. Die Kr\"ummungsr\"uckkehr bleibt aktiv mit $\beta_{\mathrm{eff}} \approx 2{,}8$ und liefert CDM-artiges gravitationelles Ger\"ust ($z > z_t \approx 7$).
\item \textbf{Geometrischer \"Ubergang ($z \sim z_t$):} Die Kr\"ummungskopplung entspannt sich von $\beta \approx 2{,}8$ zu $\beta \approx 2{,}0$, wenn der Ricci-Skalar unter einen kritischen Schwellenwert f\"allt. Die "`Dunkle-Materie"'-Phase endet.
\item \textbf{Sp\"ates Universum:} Die Kr\"ummungsr\"uckkehr s\"attigt ($\Omega_\Phi \to \Phi_0$) und f\"uhrt zu beschleunigter Expansion. Auf galaktischen Skalen wird MOND aktiv, wenn Beschleunigungen unter $a_0$ fallen. Die "`Dunkle-Energie"'-Phase dominiert.
\item \textbf{Ferne Zukunft:} Volle S\"attigung -- das Gleichgewicht ist erreicht, der Nullraumgradient ist neutralisiert, und die Expansion n\"ahert sich dem de-Sitter-Zustand.
\end{enumerate}

Die quantitative Realisierung ist nun etabliert: Die laufende Kopplung $\beta_{\mathrm{eff}}(a)$ geht bei $z_t \approx 9$ mit $n = 4$ \"uber, die skalenabh\"angige MOND-Kopplung $\mu(a) = \sqrt{\pi}$ sp\"at (\"ubergeht zu $\mu \to 1$ bei $z > 4000$) l\"ost die Hubble-Konstante auf $H_0 = 67{,}3$\,km/s/Mpc auf, der kombinierte Fit erreicht $\Delta\chi^2 = -5{,}5$ vs.\ $\Lambda$CDM mit null EDE und 6 Parametern (identisch mit $\Lambda$CDM), und die CMB-Observablen passen exakt an ($\ell_A = 301{,}471$, $\mathcal{R} = 1{,}7502$, $r_d = 146{,}9$\,Mpc). Die gesamte Geschichte wird durch drei Mechanismen beschrieben -- die S\"attigungs-ODE, die laufende $\beta$ und die laufende $\mu$ -- alle Manifestationen derselben zugrunde liegenden Kr\"ummungsdynamik, deren Parameter letztlich durch die Quantengravitation bestimmt werden.

\textit{Ontologische Vereinfachung.} $\Lambda$CDM erfordert drei verschiedene Substanzen: baryonische Materie (5\%, nachgewiesen), kalte dunkle Materie (27\%, niemals nachgewiesen) und dunkle Energie (68\%, niemals nachgewiesen). Das KRM erfordert eine Substanz -- Raumzeitk\"ummung -- in drei Phasen: hochgekr\"ummte ("`CDM-artig"'), \"Ubergangs- und ges\"attigte ("`DE-artig"')-Phase, plus baryonische Materie als kondensierte Anregungen. Nach 40 Jahren spezialisierter Suchen (XENON, LUX, PandaX, ADMX, LHC) wurde kein CDM-Teilchen nachgewiesen. Das KRM-Rahmenwerk erkl\"art warum: Es gibt nichts zu entdecken.

\subsection{Einladung an die Gemeinschaft}

Das Drei-Paper-KRM-Programm pr\"asentiert eine koh\"arente, aber unverifizierte Hypothese. Der Autor l\"adt die wissenschaftliche Gemeinschaft ein, sich mit diesem Rahmenwerk zu befassen:

\begin{enumerate}
\item \textbf{Mathematische Verifikation:} Die Herleitungen in diesem Paper -- insbesondere die P\"oschl-Teller-Korrespondenz, die Spurkopplungs-Lagrange-Dichte und die St\"orungsgleichungen -- erfordern unabh\"angige Verifikation durch mathematische Physiker.
\item \textbf{Numerische Implementierung:} Ein modifizierter CLASS- oder CAMB-Code, der die erweiterte Friedmann-Gleichung mit Spurkopplung implementiert, w\"urde die kritischen $C_\ell$- und $P(k)$-Vorhersagen liefern.
\item \textbf{Mikroskopische Herleitung:} Die Herleitung der S\"attigungs-ODE aus einem der f\"unf Kandidatenrahmenwerke (LQG, Finsler, Verschr\"ankung, QEC, kausale Mengen) w\"urde das KRM von der Ph\"anomenologie zur fundamentalen Theorie erheben.
\item \textbf{Experimentelle Tests:} Das kosmische Doppelbrechungssignal, GW-Echos und die Vorhersage des gravitativen Schlupfs liefern konkrete Ziele f\"ur Beobachter.
\end{enumerate}

\noindent Der gesamte Analysecode ist quelloffen. Die Pantheon+-Daten sind \"offentlich verf\"ugbar. Replikation und Erweiterung dieser Arbeit ist nicht nur willkommen, sondern \textit{wesentlich} f\"ur die Beurteilung ihrer G\"ultigkeit.


% ===================================================================
% LITERATUR
% ===================================================================
\subsection*{Software}

Diese Arbeit verwendet \texttt{hi\_class} \cite{Zumalacarregui2017} (Horndeski in CLASS \cite{Blas2011}) mit einem eigenen \texttt{cfm\_fR}-Gravitationsmodell, \texttt{emcee} \cite{ForemanMackey2013} f\"ur MCMC-Sampling, \texttt{NumPy} \cite{Harris2020}, \texttt{SciPy} \cite{Virtanen2020} f\"ur numerische Berechnungen und \texttt{Matplotlib} \cite{Hunter2007} f\"ur Visualisierungen.

% ===================================================================
% ANHANG: QG-DETAILS
% ===================================================================
\appendix
\section{Heuristische Quantengravitationsverbindungen}
\label{app:qg_details}

Dieser Anhang liefert heuristische Argumente und strukturelle Analogien f\"ur die f\"unf UV-Completion-Kandidaten, die in Abschnitt~\ref{subsec:uv_candidates} zusammengefasst sind. Wir betonen, dass (i)~\textit{keines} dieser Details die testbaren Vorhersagen des KRM beeinflusst, die ausschlie\ss lich aus der effektiven Wirkung~\eqref{eq:full_action} folgen, und (ii)~die folgenden Argumente qualitativ und motivational statt rigorose Herleitungen sind; es handelt sich um strukturelle Analogien, keine rigorosen mathematischen Herleitungen.

\subsection{Schleifen-Quantengravitation}
In der LQC \cite{Ashtekar2011} wird die Friedmann-Gleichung zu $H^2 = (8\pi G/3)\,\rho\,(1 - \rho/\rho_c)$, wobei $\rho_c \sim \rho_{\mathrm{Pl}}$. Diese hat die Struktur einer S\"attigungsgleichung: Die Expansionsrate ist beschr\"ankt, wenn $\rho \to \rho_c$. Die S\"attigungs-ODE~\eqref{eq:saturation_ode} kann als sp\"atzeitliches, niederenergetisches Residuum dieser Kr\"ummungsschranke interpretiert werden. Die Parameter $k$ und $\Phi_0$ bilden auf die LQG-Fl\"achenl\"ucke $\Delta$ und den Barbero-Immirzi-Parameter $\gamma_{\mathrm{BI}}$ ab.

\subsection{Finsler-Geometrie}
Die Finsler-Geometrie \cite{Bao2000} verallgemeinert die Riemannsche Geometrie durch $F(x, \dot{x})$ statt $g_{\mu\nu}(x)\,dx^\mu\,dx^\nu$. Die Richtungsabh\"angigkeit erzeugt skalenabh\"angige Gravitationseffekte (die MOND nachahmen \cite{Chang2009}), nichtstandardm\"a\ss ige kosmologische Skalierung ($a^{-\beta}$ aus oskulierender Kr\"ummung) und eine nat\"urliche S\"attigung, wenn die Richtungsabh\"angigkeit eine geometrische Schranke erreicht. Die Finsler--KRM-Abbildung ist $\alpha \cdot a^{-2} \leftrightarrow$ oskulierende Riemannsche Kr\"ummung und $f_{\mathrm{sat}} \leftrightarrow$ Finsler-Ricci-Skalar-Schranke.

\subsection{Informationstheoretische Raumzeit}
Das holographische Prinzip \cite{Bousso2002} impliziert eine maximale Informationskapazit\"at. Die S\"attigungs-ODE ist die logistische Wachstumsgleichung f\"ur Informationsverarbeitung, mit $\Phi_0$ als holographischer Kapazit\"atsgrenze. Die S\"attigung der Verschr\"ankungsentropie \cite{VanRaamsdonk2010} liefert den mikroskopischen Mechanismus: Die ER=EPR-Korrespondenz \cite{Maldacena2013} impliziert, dass Raumzeitkonnektivit\"at aus Quantenverschr\"ankung aufgebaut ist. Die Monogamie-Beschr\"ankung der Verschr\"ankung erzeugt S\"attigung, wenn der "`Klebstoff"' seine maximale Verd\"unnung erreicht, und beschleunigte Expansion folgt als autonome Reaktion des Systems auf Kapazit\"atsersch\"opfung.

\subsection{Theorie der kausalen Mengen}
In der Theorie der kausalen Mengen \cite{Bombelli1987, Sorkin2003} liefert die Sorkin-kosmologische-Konstante \cite{Sorkin1991} $\Lambda \sim 1/\sqrt{N}$ eine dynamische $\Lambda$, die abnimmt, wenn neue Elemente zur kausalen Menge hinzugef\"ugt werden. Das Kr\"ummungsr\"uckkehrpotential $\Omega_\Phi$ entspricht der effektiven kosmologischen Konstante, mit S\"attigung bei $\Phi_0$ entsprechend der Gleichgewichtsdichte der Menge.

\subsection{Quanten-Fehlerkorrektur}
Die Raumzeit als fehlerkorrigierender Quantencode \cite{Almheiri2015, Pastawski2015} gibt $\Phi_0$ die Interpretation der Code-Kapazit\"at. Jeder fehlerkorrigierende Code hat eine endliche Rate, mit der er Information gegen Dekoh\"arenz sch\"utzen kann. Die beschleunigte Expansion ist der Selbstschutzmechanismus des Codes: Durch Verd\"unnung der Informationsdichte verhindert er das \"Uberschreiten der Fehlerkorrekturschwelle. Dies verbindet sich direkt mit dem spieltheoretischen Rahmenwerk von Paper~I: Das "`Selbstschutzmotiv"' des Nullraums \textit{ist} das Bestreben des Codes, seine Integrit\"at zu wahren.


\begin{thebibliography}{99}

\bibitem{Geiger2026}
Geiger, L.\ (2026).
Game-Theoretic Cosmology and the Curvature Relaxation Model: Nash Equilibria Between Null Space and Spacetime Bubble.
Companion paper. \url{https://github.com/lukisch/cfm-cosmology}.

\bibitem{Geiger2026b}
Geiger, L.\ (2026).
Eliminating the Dark Sector: Unifying the Curvature Relaxation Model with MOND.
Companion paper.

\bibitem{Scolnic2022}
Scolnic, D.\ et al.\ (2022).
The Pantheon+ Analysis: The Full Data Set and Light-curve Release.
\textit{The Astrophysical Journal}, 938(2), 113.

\bibitem{Milgrom1983}
Milgrom, M.\ (1983).
A modification of the Newtonian dynamics as a possible alternative to the hidden mass hypothesis.
\textit{The Astrophysical Journal}, 270, 365--370.

\bibitem{BekensteinMilgrom1984}
Bekenstein, J.\,D.\ \& Milgrom, M.\ (1984).
Does the missing mass problem signal the breakdown of Newtonian gravity?
\textit{The Astrophysical Journal}, 286, 7--14.
DOI: 10.1086/162570.

\bibitem{Bekenstein2004}
Bekenstein, J.\,D.\ (2004).
Relativistic gravitation theory for the modified Newtonian dynamics paradigm.
\textit{Physical Review D}, 70(8), 083509.
DOI: 10.1103/PhysRevD.70.083509.

\bibitem{HuSawicki2007}
Hu, W.\ \& Sawicki, I.\ (2007).
Models of $f(R)$ Cosmic Acceleration that Evade Solar-System Tests.
\textit{Physical Review D}, 76(6), 064004.
DOI: 10.1103/PhysRevD.76.064004.

\bibitem{Khoury2004}
Khoury, J.\ \& Weltman, A.\ (2004).
Chameleon fields: Awaiting surprises for tests of gravity in space.
\textit{Physical Review Letters}, 93(17), 171104.
DOI: 10.1103/PhysRevLett.93.171104.

\bibitem{Williams2004}
Williams, J.\,G., Turyshev, S.\,G.\ \& Boggs, D.\,H.\ (2004).
Progress in lunar laser ranging tests of relativistic gravity.
\textit{Physical Review Letters}, 93(26), 261101.
DOI: 10.1103/PhysRevLett.93.261101.

\bibitem{Bertotti2003}
Bertotti, B., Iess, L.\ \& Tortora, P.\ (2003).
A test of general relativity using radio links with the Cassini spacecraft.
\textit{Nature}, 425(6956), 374--376.
DOI: 10.1038/nature01997.

\bibitem{PlanckMG2016}
Planck Collaboration (2016).
Planck 2015 results. XIV. Dark energy and modified gravity.
\textit{Astronomy \& Astrophysics}, 594, A14.
DOI: 10.1051/0004-6361/201525814.

\bibitem{Skordis2021}
Skordis, C.\ \& Z{\l}o\'snik, T.\ (2021).
New Relativistic Theory for Modified Newtonian Dynamics.
\textit{Physical Review Letters}, 127(16), 161302.

\bibitem{Rovelli2004}
Rovelli, C.\ (2004).
\textit{Quantum Gravity}. Cambridge University Press.

\bibitem{Thiemann2007}
Thiemann, T.\ (2007).
\textit{Modern Canonical Quantum General Relativity}. Cambridge University Press.

\bibitem{Ashtekar2011}
Ashtekar, A.\ \& Singh, P.\ (2011).
Loop Quantum Cosmology: A Status Report.
\textit{Classical and Quantum Gravity}, 28(21), 213001.

\bibitem{Bao2000}
Bao, D., Chern, S.-S.\ \& Shen, Z.\ (2000).
\textit{An Introduction to Riemann-Finsler Geometry}. Springer.

\bibitem{Chang2009}
Chang, Z.\ \& Li, X.\ (2009).
Modified Friedmann model in Randers-Finsler space of approximate Berwald type.
\textit{Physics Letters B}, 676(4-5), 173--176.

\bibitem{Girelli2007}
Girelli, F., Liberati, S.\ \& Sindoni, L.\ (2007).
Planck-scale modified dispersion relations and Finsler geometry.
\textit{Physical Review D}, 75(6), 064015.

\bibitem{Bousso2002}
Bousso, R.\ (2002).
The holographic principle.
\textit{Reviews of Modern Physics}, 74(3), 825--874.

\bibitem{Maldacena2013}
Maldacena, J.\ \& Susskind, L.\ (2013).
Cool horizons for entangled black holes.
\textit{Fortschritte der Physik}, 61(9), 781--811.

\bibitem{Bombelli1987}
Bombelli, L., Lee, J., Meyer, D.\ \& Sorkin, R.\,D.\ (1987).
Space-time as a causal set.
\textit{Physical Review Letters}, 59(5), 521--524.

\bibitem{Sorkin2003}
Sorkin, R.\,D.\ (2003).
Causal Sets: Discrete Gravity.
In \textit{Lectures on Quantum Gravity}, Springer, 305--327.

\bibitem{Sorkin1991}
Sorkin, R.\,D.\ (1991).
Spacetime and causal sets.
In \textit{Relativity and Gravitation}, World Scientific, 150--173.

\bibitem{Starobinsky1980}
Starobinsky, A.\,A.\ (1980).
A new type of isotropic cosmological models without singularity.
\textit{Physics Letters B}, 91(1), 99--102.

\bibitem{Verlinde2017}
Verlinde, E.\ (2017).
Emergent Gravity and the Dark Universe.
\textit{SciPost Physics}, 2(3), 016.

\bibitem{Abbott2017}
Abbott, B.\,P.\ et al.\ (LIGO/Virgo \& Fermi GBM) (2017).
Gravitational Waves and Gamma-Rays from a Binary Neutron Star Merger: GW170817 and GRB~170817A.
\textit{The Astrophysical Journal Letters}, 848(2), L13.

\bibitem{Wheeler1990}
Wheeler, J.\,A.\ (1990).
Information, physics, quantum: The search for links.
In \textit{Complexity, Entropy, and the Physics of Information}, Addison-Wesley, 3--28.

\bibitem{Feynman1948}
Feynman, R.\,P.\ (1948).
Space-Time Approach to Non-Relativistic Quantum Mechanics.
\textit{Reviews of Modern Physics}, 20(2), 367--387.

\bibitem{VanRaamsdonk2010}
Van~Raamsdonk, M.\ (2010).
Building up spacetime with quantum entanglement.
\textit{General Relativity and Gravitation}, 42(10), 2323--2329.
DOI: 10.1007/s10714-010-1034-0.

\bibitem{Almheiri2015}
Almheiri, A., Dong, X.\ \& Harlow, D.\ (2015).
Bulk Locality and Quantum Error Correction in AdS/CFT.
\textit{Journal of High Energy Physics}, 2015(4), 163.
DOI: 10.1007/JHEP04(2015)163.

\bibitem{Pastawski2015}
Pastawski, F., Yoshida, B., Harlow, D.\ \& Preskill, J.\ (2015).
Holographic quantum error-correcting codes: Toy models for the bulk/boundary correspondence.
\textit{Journal of High Energy Physics}, 2015(6), 149.
DOI: 10.1007/JHEP06(2015)149.

\bibitem{Minami2020}
Minami, Y.\ \& Komatsu, E.\ (2020).
New Extraction of the Cosmic Birefringence from the Planck 2018 Polarization Data.
\textit{Physical Review Letters}, 125(22), 221301.
DOI: 10.1103/PhysRevLett.125.221301.

\bibitem{Chou2017}
Chou, A.\,S.\ et al.\ (Holometer Collaboration) (2017).
First Measurements of High Frequency Cross-Spectra from a Pair of Large Michelson Interferometers.
\textit{Physical Review Letters}, 117(11), 111102.
DOI: 10.1103/PhysRevLett.117.111102.

\bibitem{Donadi2021}
Donadi, S.\ et al.\ (2021).
Underground test of gravity-related wave function collapse.
\textit{Nature Physics}, 17(1), 74--78.
DOI: 10.1038/s41567-020-1008-4.

\bibitem{Abedi2017}
Abedi, J., Dykaar, H.\ \& Afshordi, N.\ (2017).
Echoes from the Abyss: Tentative evidence for Planck-scale structure at black hole horizons.
\textit{Physical Review D}, 96(8), 082004.
DOI: 10.1103/PhysRevD.96.082004.

\bibitem{DESI2025}
DESI Collaboration (2025).
DESI DR2 Results II: Measurements of Baryon Acoustic Oscillations and Cosmological Constraints.
\textit{arXiv:2503.14738}.

\bibitem{KiDS2021}
Asgari, M.\ et al.\ (KiDS Collaboration) (2021).
KiDS-1000 Cosmology: Cosmic shear constraints on the amplitude of matter fluctuations.
\textit{Astronomy \& Astrophysics}, 645, A104.

\bibitem{DESY3}
Abbott, T.\,M.\,C.\ et al.\ (DES Collaboration) (2022).
Dark Energy Survey Year 3 results: Cosmological constraints from galaxy clustering and weak lensing.
\textit{Physical Review D}, 105(2), 023520.

\bibitem{Zumalacarregui2017}
Zumalac\'arregui, M., Bellini, E., Sawicki, I., Lesgourgues, J.\ \& Ferreira, P.\,G.\ (2017).
hi\_class: Horndeski in the Cosmic Linear Anisotropy Solving System.
\textit{Journal of Cosmology and Astroparticle Physics}, 2017(08), 019.
DOI: 10.1088/1475-7516/2017/08/019.

\bibitem{Blas2011}
Blas, D., Lesgourgues, J.\ \& Tram, T.\ (2011).
The Cosmic Linear Anisotropy Solving System (CLASS). Part~II: Approximation schemes.
\textit{Journal of Cosmology and Astroparticle Physics}, 2011(07), 034.
DOI: 10.1088/1475-7516/2011/07/034.

\bibitem{Harris2020}
Harris, C.\,R.\ et al.\ (2020).
Array programming with NumPy.
\textit{Nature}, 585, 357--362.
DOI: 10.1038/s41586-020-2649-2.

\bibitem{Virtanen2020}
Virtanen, P.\ et al.\ (2020).
SciPy 1.0: Fundamental Algorithms for Scientific Computing in Python.
\textit{Nature Methods}, 17, 261--272.
DOI: 10.1038/s41592-019-0686-2.

\bibitem{Hunter2007}
Hunter, J.\,D.\ (2007).
Matplotlib: A 2D Graphics Environment.
\textit{Computing in Science \& Engineering}, 9(3), 90--95.
DOI: 10.1109/MCSE.2007.55.

\bibitem{Bellini2014}
Bellini, E.\ \& Sawicki, I.\ (2014).
Maximal freedom at minimum cost: linear large-scale structure in general modifications of gravity.
\textit{Journal of Cosmology and Astroparticle Physics}, 2014(07), 050.
DOI: 10.1088/1475-7516/2014/07/050.

\bibitem{Woodard2015}
Woodard, R.\,P.\ (2015).
Ostrogradsky's theorem on Hamiltonian instability.
\textit{Scholarpedia}, 10(8), 32243.
DOI: 10.4249/scholarpedia.32243.

\bibitem{ForemanMackey2013}
Foreman-Mackey, D., Hogg, D.\,W., Lang, D.\ \& Goodman, J.\ (2013).
emcee: The MCMC Hammer.
\textit{Publications of the Astronomical Society of the Pacific}, 125(925), 306--312.
DOI: 10.1086/670067.

\bibitem{Chabanier2019}
Chabanier, S., Palanque-Delabrouille, N.\ \& Y\`eche, C.\ (2019).
The one-dimensional power spectrum from the SDSS DR14 Ly$\alpha$ forests.
\textit{Journal of Cosmology and Astroparticle Physics}, 2019(07), 017.
DOI: 10.1088/1475-7516/2019/07/017.

\bibitem{Li2012}
Li, B., Zhao, G.-B., Teyssier, R.\ \& Koyama, K.\ (2012).
ECOSMOG: An Efficient COde for Simulating Modified Gravity.
\textit{Journal of Cosmology and Astroparticle Physics}, 2012(01), 051.
DOI: 10.1088/1475-7516/2012/01/051.

\end{thebibliography}

\end{document}

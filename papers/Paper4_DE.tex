\documentclass[aps,prd,twocolumn,superscriptaddress,nofootinbib]{revtex4-2}

% REVTeX 4.2 loads amsmath, amssymb, graphicx, and hyperref automatically
\usepackage{booktabs}
\usepackage{tabularx}
\usepackage{xcolor}
\graphicspath{{../figures/}}

\newtheorem{definition}{Definition}
\newtheorem{proposition}{Proposition}
\newtheorem{theorem}{Theorem}

\hypersetup{
    pdftitle={Der Galaktisch-Kosmologische Nexus: MOND-Dynamik aus Kruemmungss\"attigung},
    pdfauthor={Lukas Geiger},
    colorlinks=true,
    linkcolor=black,
    urlcolor=blue,
    citecolor=black
}

\begin{document}

% ===================================================================
% TITELSEITE
% ===================================================================

\title{Der Galaktisch--Kosmologische Nexus: MOND-Dynamik\\aus Kruemmungss\"attigung}

\author{Lukas Geiger}
\email{Correspondence: Gei\ss{}b\"uhlweg~1, 79872~Bernau, Germany}
\affiliation{Independent Researcher, Bernau im Schwarzwald, Germany}

\date{\today}

\begin{abstract}
Die Arbeiten I--III der Curvature Feedback Model (CFM) Serie zeigten, dass ein einzelner Freiheitsgrad -- ein kruemmungsgekoppeltes Skalarfeld mit P\"oschl-Teller S\"attigungsdynamik -- den kosmologischen dunklen Sektor reproduziert ($\Delta\chi^2 = -5{,}5$ vs.\ $\Lambda$CDM auf Pantheon+ in der bevorzugten Null-EDE-Variante, CMB-Spitzenlagen innerhalb von $0{,}1\%$, $\alpha_T = 0$ exakt). Das Skalarfeld allein kann jedoch keine flachen galaktischen Rotationskurven erzeugen: seine f\"unfte Kraft verst\"arkt die Gravitation um maximal einen Faktor $4/3$ auf sub-Compton-Skalen, unzureichend um Dunkle-Materie-H\"alo zu ersetzen. Die vorliegende Arbeit schliesst diese L\"ucke durch Einfuehrung eines zweiten Freiheitsgrades -- eines zeitartigen Vektorfeldes $A_\mu$ -- dessen Notwendigkeit, Struktur und Kopplung aus thermodynamischen Prinzipien hergeleitet sind (maximale Entropieproduktion, hierarchische Dissipation). Wir konstruieren die teleodynamische Kopplungsfunktion $\mathcal{F}(A_\mu, \phi, T) = |T|/\rho_{\mathrm{crit}} \cdot \mathcal{B}(\phi) \cdot \mathcal{V}(A_\mu \partial^\mu \phi)$ mit drei Komponenten: ein Materie-Gatekeeper ($|T|/\rho_{\mathrm{crit}}$) sichert BBN-Schutz, eine S\"attigungskontrolle ($\mathcal{B} = \mathrm{sech}^2(\phi/\phi_0)$) verbindet MOND-St\"arke mit Expansionsdynamik, und eine Vektorprojektion ($A_\mu \partial^\mu \phi$) koppelt lokale Struktur mit kosmologischer Expansionsrate. Das Modell sagt $a_0 = cH_0/(2\pi)$ als Gleichgewichtsbedingung voraus -- nicht als angepasster Parameter -- und reproduziert die Radial Acceleration Relation (RAR) und flache Rotationskurven ohne Dunkle-Materie-Teilchen. Wir schlagen einen parameterfreien Test gegen die SPARC-Datenbank (175 Galaxien) vor und zeigen, dass der Konvergenzfaktor $4/3$ aus der $f(R)$-St\"orungstheorie (Arbeit~III) mit dem MOND-Phasenraum-Verh\"altnis $V_3/V_2$ (Arbeit~II) zusammenf\"allt, was unabh\"angige Evidenz f\"ur die verschachtelte Hierarchie darstellt. Die vollst\"andige CFM-Wirkung $S = S_{\mathrm{T1}} + S_{\mathrm{T2}}$ enth\"alt 8 Terme, von denen 5 eindeutig durch die Axiome bestimmt sind, 1 formbedingt, 1 stark durch 6 Bedingungen beschr\"ankt ist, und 4 Parameter aus Daten folgen. Keine $\Lambda$, keine Dunkle-Materie-Teilchen -- nur Geometrie.
\end{abstract}

\keywords{MOND, modifizierte Gravitation, Vektor-Tensor-Theorie, Radial Acceleration Relation, Kr\"ummungsrueckkopplung, Dunkle-Materie-Alternative, galaktische Dynamik}

\maketitle

\thanks{KI-Werkzeuge (Claude, Anthropic; Gemini, Google DeepMind) wurden fuer mathematische Formalisierung, Code-Entwicklung und Texterstellung verwendet. Alle physikalischen Hypothesen, wissenschaftliche Interpretation und Verantwortung fuer die Inhalte liegen vollstaendig bei dem Autor. Der Analysecode ist verfuegbar unter \url{https://github.com/lukisch/cfm-cosmology}.}

\thanks{Arbeit~IV in der CFM-Serie. Begleitarbeiten: \cite{Geiger2026,Geiger2026b,Geiger2026c}.}


% ===================================================================
% 1. EINFUEHRUNG
% ===================================================================
\section{Einfuehrung}
\label{sec:intro}

Das Paradigma der Modifizierten Newtonschen Dynamik (MOND) \cite{Milgrom1983} hat sich auf galaktischem Ma\ss{}stab als bemerkenswert erfolgreich erwiesen: die Radial Acceleration Relation (RAR) \cite{McGaugh2016} zeigt eine enge, universelle Korrelation zwischen beobachteter Gravitationsbeschleunigung $g_{\mathrm{obs}}$ und baryonischer Gravitationsbeschleunigung $g_{\mathrm{bar}}$, mit einer charakteristischen Skala $a_0 \approx 1.2 \times 10^{-10}$\,m/s$^2$. Die N\"ahe $a_0 \sim cH_0$ wurde seit Milgroms Originalarbeit bemerkt, bleibt aber sowohl in $\Lambda$CDM als auch in Standard-MOND unerkl\"art.

Das Kr\"ummungsr\"uckkopplungsmodell (Curvature Feedback Model, CFM) bietet einen Rahmen, der diese Koinzidenz aufl\"osen k\"onnte. Die Arbeiten~I--III \cite{Geiger2026,Geiger2026b,Geiger2026c} haben Folgendes etabliert:
\begin{itemize}
\item \textbf{Arbeit~I:} Eine spieltheoretische/thermodynamische Motivation f\"ur modifizierte Gravitation, mit der Kr\"ummungsr\"uckkehr als entropiegetriebener Relaxation \cite{Geiger2026}.
\item \textbf{Arbeit~II:} Eliminierung des Teilchen-Darksektors durch geometrische Kr\"ummungsphasen, MOND-kompatibel auf Hintergrundebene ($\mu_{\mathrm{eff}} = \sqrt{\pi}$, $V_3/V_2 = 4/3$) \cite{Geiger2026b}.
\item \textbf{Arbeit~III:} Die effektive Lagrange-Dichte $\mathcal{L}_{\mathrm{T1}}$ f\"ur den Skalarsektor, mit $f(R) = R + \gamma R^2$, P\"oschl-Teller-S\"attigung, Spurkopplung und Chameleon-Abschirmung. Die St\"orungsanalyse liefert $\mu(k,a) \to 4/3$ auf Sub-Compton-Skalen, $\alpha_T = 0$ exakt sowie Sonnensystem-Konsistenz ($m_{\mathrm{eff}}^{\mathrm{solar}}/m_s \sim 4 \times 10^{14}$) \cite{Geiger2026c}.
\end{itemize}

\subsection{Die galaktische L\"ucke}

Trotz dieses Erfolgs kann der Skalarsektor allein \textit{keine} flachen Rotationskurven erzeugen. Der Gravitationsmodifikationsfaktor $\mu \to 4/3$ entspricht einer $33\%$-Verst\"arkung der Newtonschen Gravitation -- weit unter dem Faktor $\sim 5$--$10$, der erforderlich ist, um galaktische Rotationsgeschwindigkeiten ohne Dunkle Materie zu erkl\"aren. Dies ist kein Versagen des Modells, sondern eine \textit{Vorhersage}: Die verschachtelte Hierarchie erfordert einen zweiten Freiheitsgrad.

Das Argument ist thermodynamischer Natur: Das Skalarfeld (Tochter~1) treibt die kosmologische Dynamik an, kann aber allein keine stabilen dissipativen Strukturen (Galaxien) bilden. Effiziente Entropieproduktion -- die Gleichgewichtsbedingung des zugrunde liegenden Optimierungsprinzips (Arbeit~I, Abschnitt~2.5) -- erfordert lokalisierte, langlebige Materiekonzentrationen. Diese ben\"otigen zus\"atzliche Gravitationsunterst\"utzung, die die skalare $4/3$-Verst\"arkung \"ubersteigt.

Die vorliegende Arbeit f\"uhrt Tochter~2 ein: ein zeitartiges Vektorfeld $A_\mu$, das diese Unterst\"utzung bereitstellt. Wir leiten seine Kopplung an den Skalarsektor und die Materie her und testen die resultierenden Vorhersagen anhand galaktischer Daten.

\subsection{Beziehung zu existierenden Vektor-Tensor-Theorien}

Das hier eingef\"uhrte Vektorfeld teilt strukturelle Merkmale mit Bekensteins TeVeS \cite{Bekenstein2004} und der AeST-Theorie von Skordis \& Z{\l}o\'snik \cite{Skordis2021}. Der CFM-Vektorsektor unterscheidet sich jedoch in drei wesentlichen Aspekten:
\begin{enumerate}
\item Das Vektorfeld ist \textit{hergeleitet} aus thermodynamischer Notwendigkeit (verschachtelte Hierarchie), nicht postuliert.
\item Die Kopplung an den Skalarsektor ist durch die Forderung $a_0 \sim cH_0$ \textit{eingeschr\"ankt}, was sie von einer freien Funktion auf einen stark beschr\"ankten Ansatz reduziert.
\item Die Chameleon-Abschirmung wirkt \"uber zwei unabh\"angige Mechanismen (Skalarmasse + Vektorentkopplung), was Robustheit gegen\"uber Sonnensystem-Tests gew\"ahrleistet.
\end{enumerate}


% ===================================================================
% 2. THERMODYNAMISCHE HERLEITUNG DES VEKTORSEKTORS
% ===================================================================
\section{Thermodynamische Herleitung des Vektorsektors}
\label{sec:derivation}

\subsection{Warum ein zweiter Freiheitsgrad notwendig ist}
\label{sec:why_vector}

Das Skalarfeld (Tochter~1) modifiziert die Gravitation auf kosmologischen Skalen durch die Erweiterung $f(R) = R + \gamma R^2$. Auf galaktischen Skalen unterdr\"uckt die Chameleon-Abschirmung diese Modifikation: Die effektive Skalarmasse steigt im Sonnensystem um einen Faktor $\sim 4 \times 10^{14}$ \cite{Geiger2026c}, wodurch die f\"unfte Kraft auf Compton-Wellenl\"angen von $\sim 20$\,m beschr\"ankt wird.

In den galaktischen Au\ss{}enbereichen, wo $\rho \sim 100\,\rho_{\mathrm{crit}}$, ist die Abschirmung schw\"acher und die skalare f\"unfte Kraft liefert $\mu \to 4/3$. Flache Rotationskurven erfordern jedoch $v \propto \mathrm{const}$, was ein effektives Gravitationspotential $\Phi \propto \ln(r)$ impliziert -- eine logarithmische Divergenz, die kein Yukawa-artiges Skalarfeld erzeugen kann.

Aus der thermodynamischen Perspektive von Arbeit~I:
\begin{enumerate}
\item Das Prinzip der maximalen Entropieproduktion (MEPP) \cite{Dewar2003,Martyushev2006} erfordert, dass das System seine Dissipationsrate maximiert.
\item Galaxien sind dissipative Strukturen im Sinne von Prigogine \cite{Prigogine1977}: selbstorganisierte Systeme, die sich weit vom Gleichgewicht entfernt halten, um Entropie (Strahlung) effizient zu produzieren.
\item Ohne stabile Galaxien w\"are die Entropieproduktionsrate des Universums drastisch niedriger.
\item Der Skalarsektor kann diese Strukturen allein nicht stabilisieren $\Rightarrow$ MEPP verlangt einen zus\"atzlichen Mechanismus.
\end{enumerate}

Die thermodynamische Hierarchie ist:
\begin{equation}
\underbrace{\text{Null-Raum}}_{\text{Mutter}} \to \underbrace{\phi \text{ (Skalarfeld)}}_{\text{Tochter 1}} \to \underbrace{A_\mu \text{ (Vektorfeld)}}_{\text{Tochter 2}} \to \underbrace{\text{Standard-Physik}}_{\text{Terminierung}}
\label{eq:hierarchy}
\end{equation}

Jede Ebene erzeugt die n\"achste, weil die aktuelle Ebene allein die Entropieproduktion nicht optimieren kann. Die Hierarchie terminiert, wenn Standardphysik ausreicht (stellare und substellare Skalen).

\subsection{Warum ein Vektorfeld (nicht ein zweites Skalarfeld)}
\label{sec:why_vector_not_scalar}

Der n\"achsteinfachste Freiheitsgrad nach einem Skalarfeld ist ein Vektorfeld. Ein zweites Skalarfeld w\"urde:
\begin{itemize}
\item die Isotropie der Gravitationsverst\"arkung nicht brechen (Galaxien sind in ihrer Dynamik Scheiben, nicht Kugeln),
\item eine zweite Yukawa-artige Kraft mit \"ahnlicher Abstandsabh\"angigkeit einf\"uhren,
\item nicht nat\"urlich an die Expansionsrichtung (Zeit) koppeln.
\end{itemize}

Ein Vektorfeld $A_\mu$, beschr\"ankt auf zeitartig ($A_\mu A^\mu = -1$), w\"ahlt naturgem\"a\ss{} die Zeitrichtung als Aufl\"osungsachse -- konsistent mit der thermodynamischen Interpretation, dass Tochter~2 sich ``zur\"uck zu'' Tochter~1 in Zeitrichtung aufl\"ost.

\subsection{Die Auftraggeber-Beauftragten-Struktur}
\label{sec:principal_agent}

Die Beziehung zwischen Tochter~1 und Tochter~2 ist ein Auftraggeber-Beauftragten-Problem:
\begin{itemize}
\item \textbf{Auftraggeber} (Tochter~1 / Skalarfeld): Braucht stabile Galaxien fuer effiziente Entropieproduktion, kann sie aber nicht allein schaffen.
\item \textbf{Beauftragter} (Tochter~2 / Vektorfeld): Stellt die zus\"atzliche Gravitationsunterst\"utzung bereit, die galaktische Strukturen stabilisiert, strebt aber nach Aufl\"osung (thermodynamischer Pfeil zum Gleichgewicht).
\item \textbf{Gleichgewicht:} MOND -- der Balancepunkt, an dem Galaxien stabil genug f\"ur effiziente Entropieproduktion sind, der Beauftragte aber nicht dauerhaft gebunden ist.
\end{itemize}

Dies sagt voraus, dass MOND-Effekte sich \"uber kosmische Zeit abschw\"achen sollten, wenn sich das Universum dem thermodynamischen Gleichgewicht n\"ahert -- testbar mit Hochrotverschiebungs-Durchmusterungen (JWST, ELT).


% ===================================================================
% 3. DIE VOLLSTAENDIGE CFM-WIRKUNG
% ===================================================================
\section{Die vollst\"andige Wirkung}
\label{sec:action}

\subsection{Tochter~1: Der Skalarsektor (Arbeiten~I--III)}

Die Skalarsektorwirkung, etabliert in Arbeit~III, lautet:
\begin{align}
S_{\mathrm{T1}} = \int d^4x\,\sqrt{-g}\,\bigg[&
  \frac{R}{16\pi G}
  + \gamma\,F\!\left(\frac{T}{\rho}\right) R^2
\nonumber \\
  &+ \frac{1}{2}\,\partial_\mu\phi\,\partial^\mu\phi
  - V_{\mathrm{PT}}(\phi)
\bigg]
\label{eq:S_T1}
\end{align}
wobei $V_{\mathrm{PT}}(\phi) = V_0/\cosh^2(\phi/\phi_0)$ das P\"oschl-Teller-Potential mit S\"attigungsdynamik ist, $F(T/\rho)$ die Spurkopplungsfunktion (verschwindend w\"ahrend der Strahlungsdominanz zum Schutz der BBN), und $\gamma \sim \mathcal{O}(H_0^{-2})$ aus Daten folgt.

\subsection{Konventionen und Notation}
\label{sec:conventions}

Wir arbeiten in der $(-,+,+,+)$ Metrik-Signatur. Die Spur des Materie-Energie-Impuls-Tensors ist:
\begin{equation}
T \equiv g^{\mu\nu}T_{\mu\nu} = -\rho + 3p
\label{eq:trace_def}
\end{equation}
F\"ur Staub ($p \approx 0$): $T = -\rho$, also $|T| = \rho$. F\"ur Strahlung ($p = \rho/3$): $T = 0$ exakt. Dies ist die Grundlage des BBN-Schutzmechanismus: Die Kopplung $\mathcal{F} \propto |T|$ verschwindet identisch w\"ahrend der Strahlungsdominanz aufgrund konformer Symmetrie, ohne jegliche Feinabstimmung. Das Verh\"altnis $|T|/\rho_{\mathrm{crit}}$ mit $\rho_{\mathrm{crit}} = 3H_0^2/(8\pi G)$ ist dimensionslos.

\subsection{Tochter~2: Der Vektorsektor}
\label{sec:vector_action}

Die Vektorsektorwirkung lautet:
\begin{align}
S_{\mathrm{T2}} = \int d^4x\,\sqrt{-g}\,\bigg[&
  -\frac{K_B}{2}\,F_{\mu\nu}F^{\mu\nu}
  + \lambda\!\left(A_\mu A^\mu + 1\right)
\nonumber \\
  &+ \mathcal{F}(A_\mu, \phi, T)
\bigg]
\label{eq:S_T2}
\end{align}
wobei:
\begin{itemize}
\item $F_{\mu\nu} = \partial_\mu A_\nu - \partial_\nu A_\mu$ der Feldst\"arketensor ist (Maxwell-artiger kinetischer Term, Minimalform f\"ur ein Vektorfeld),
\item $\lambda(A_\mu A^\mu + 1)$ ein Lagrange-Multiplikator ist, der die zeitartige Einheitsbedingung $A_\mu A^\mu = -1$ erzwingt,
\item $K_B$ die kinetische Kopplungskonstante ist (aus Daten),
\item $\mathcal{F}(A_\mu, \phi, T)$ die Kopplungsfunktion zwischen Vektorfeld, Skalarfeld und Materie ist.
\end{itemize}

Die Gesamtwirkung ist:
\begin{equation}
S_{\mathrm{CFM}} = S_{\mathrm{T1}} + S_{\mathrm{T2}} + S_{\mathrm{matter}}
\label{eq:S_total}
\end{equation}

\subsection{Feldgleichungen}
\label{sec:field_eqs}

Die Bewegungsgleichungen folgen aus dem Variationsprinzip $\delta S_{\mathrm{T2}} = 0$.

\subsubsection{Vektorfeldgleichung}

Variation bez\"uglich $A_\nu$ liefert eine Proca-Maxwell-Gleichung mit Quellterm:
\begin{equation}
K_B\,\nabla_\mu F^{\mu\nu} + 2\lambda\,A^\nu = \frac{\delta\mathcal{F}}{\delta A_\nu}
\label{eq:vector_eom}
\end{equation}
wobei der Faktor $2$ aus $\delta(A_\mu A^\mu)/\delta A_\nu = 2A^\nu$ entsteht. F\"ur die Kopplung~\eqref{eq:coupling} lautet der Quellterm:
\begin{equation}
\frac{\delta\mathcal{F}}{\delta A_\nu} = \frac{|T|}{\rho_{\mathrm{crit}}}\,\mathcal{B}(\phi)\,\partial^\nu\phi
\label{eq:source}
\end{equation}

Der Lagrange-Multiplikator $\lambda$ wird bestimmt durch Kontraktion von~\eqref{eq:vector_eom} mit $A_\nu$ und Erzwingung von $A_\nu A^\nu = -1$:
\begin{equation}
\lambda = \frac{1}{2}\left(A_\nu\,K_B\,\nabla_\mu F^{\mu\nu} - A_\nu\,\frac{\delta\mathcal{F}}{\delta A_\nu}\right)
\label{eq:lambda}
\end{equation}

\subsubsection{Energie-Impuls-Tensor}

Die gravitative Feldgleichung erh\"alt den Vektorsektorbeitrag $T^{(A)}_{\mu\nu} = -\frac{2}{\sqrt{-g}}\frac{\delta(\sqrt{-g}\,\mathcal{L}_{\mathrm{T2}})}{\delta g^{\mu\nu}}$:
\begin{align}
T^{(A)}_{\mu\nu} = \;&K_B\!\left(F_{\mu\alpha}F_\nu^{\phantom{\nu}\alpha} - \tfrac{1}{4}\,g_{\mu\nu}\,F_{\alpha\beta}F^{\alpha\beta}\right) \nonumber\\
&- 2\lambda\,A_\mu A_\nu + T^{(\mathcal{F})}_{\mu\nu}
\label{eq:T_vector}
\end{align}
wobei $T^{(\mathcal{F})}_{\mu\nu} = -\frac{2}{\sqrt{-g}}\frac{\delta(\sqrt{-g}\,\mathcal{F})}{\delta g^{\mu\nu}}$ die metrische Variation der Kopplung kodiert. Der Zwangsbedingungsbeitrag $-2\lambda\,A_\mu A_\nu$ wird on-shell ausgewertet ($A_\alpha A^\alpha = -1$); der $g_{\mu\nu}$-Spurterm verschwindet identisch aufgrund der Zwangsbedingung.

\subsubsection{Skalarfeldgleichungs-Modifikation}

Die Kopplung $\mathcal{F}$ modifiziert auch die Skalarfeldgleichung. Variation bez\"uglich $\phi$ liefert einen zus\"atzlichen Quellterm:
\begin{equation}
\frac{\delta\mathcal{F}}{\delta\phi} = \frac{|T|}{\rho_{\mathrm{crit}}}\left[\mathcal{B}'(\phi)\,(A_\mu\partial^\mu\phi) + \mathcal{B}(\phi)\,\nabla_\mu A^\mu\right]
\label{eq:scalar_source}
\end{equation}
wobei $\mathcal{B}'(\phi) = -2\,\mathrm{sech}^2(\phi/\phi_0)\,\tanh(\phi/\phi_0)/\phi_0$. Dies koppelt die Skalarfelddynamik zur\"uck an den Vektorsektor und schlie\ss{}t das System.

\subsubsection{Divergenz-Konsistenz}

Die Antisymmetrie von $F^{\mu\nu}$ impliziert $\nabla_\nu\nabla_\mu F^{\mu\nu} \equiv 0$. Anwendung von $\nabla_\nu$ auf die Vektorgleichung~\eqref{eq:vector_eom} liefert die Integrabilit\"atsbedingung:
\begin{equation}
2\,\nabla_\nu(\lambda\,A^\nu) = \nabla_\nu\!\left(\frac{|T|}{\rho_{\mathrm{crit}}}\,\mathcal{B}(\phi)\,\partial^\nu\phi\right)
\label{eq:divergence}
\end{equation}
Dies ist eine nichttriviale Bedingung, die die Evolution von $\lambda$ mit der Skalar-Materie-Dynamik verkn\"upft. Mit der Einheitsbedingung $A_\mu A^\mu = -1$ und ihrer Folgerung $A^\nu\nabla_\mu A_\nu = 0$ entwickelt sich die linke Seite zu $2A^\nu\nabla_\nu\lambda + 2\lambda\,\nabla_\nu A^\nu$. Auf dem FLRW-Hintergrund reduzieren sich beide Seiten auf Zeitableitungen und die Bedingung ist automatisch durch die $\lambda$-Bestimmung~\eqref{eq:lambda} erf\"ullt. Im statischen, kugelsymmetrischen Limes liefert Gl.~\eqref{eq:divergence} eine zus\"atzliche Bedingung f\"ur das r\"aumliche Profil von $\lambda(r)$, die sicherstellt, dass keine Inkonsistenz zwischen der Zwangsbedingung und der Feldgleichung entsteht.

\subsubsection{Warum $A_\mu\partial^\mu\phi$ die eindeutige Kopplung erster Ordnung ist}

Die Kopplung $\mathcal{V} = A_\mu\partial^\mu\phi$ ist der eindeutige Lorentz-Skalar, der (i)~linear in $A_\mu$, (ii)~erster Ordnung in Ableitungen von $\phi$ und (iii)~frei vom Kr\"ummungstensor ist. H\"ohere Potenzen $(A_\mu\partial^\mu\phi)^n$ w\"urden Nichtlinearit\"aten einf\"uhren, die die Geisteranalyse ohne physikalische Motivation komplizieren. Kopplungen mit zweiten Ableitungen wie $A^\mu A^\nu\nabla_\mu\nabla_\nu\phi$ w\"urden Ostrogradsky-Instabilit\"aten erzeugen \cite{Woodard2015}. Kr\"ummungskopplungen $R_{\mu\nu}A^\mu A^\nu$ w\"urden $\alpha_T = 0$ verletzen (Abschn.~\ref{sec:alpha_T}). Der parit\"atsungerade Term $F_{\mu\nu}\tilde{F}^{\mu\nu}$ ist f\"ur ein Abelsches Feld eine totale Divergenz und tr\"agt nicht zu den Bewegungsgleichungen bei.

\subsection{Bestandsaufnahme von Termen}

Tabelle~\ref{tab:terms} fasst den Status jedes Terms in der vollst\"andigen Wirkung zusammen.

\begin{table}[b]
\caption{\label{tab:terms}Status der Terme in der CFM-Wirkung. ``Bestimmt'' bedeutet eindeutig durch die Axiome fixiert; ``Form'' bedeutet, die funktionale Form ist fixiert, aber ein Parameter ist frei; ``Beschr\"ankt'' bedeutet, den in Abschn.~\ref{sec:constraints} aufgef\"uhrten Bedingungen unterworfen.}
\begin{ruledtabular}
\begin{tabular}{lll}
Term & Physikalische Rolle & Status \\
\hline
$R/(16\pi G)$ & Einstein-Hilbert & Bestimmt \\
$\gamma F(T/\rho)\,R^2$ & Starobinsky + Spur & Form ($\gamma$ frei) \\
$\frac{1}{2}\partial_\mu\phi\,\partial^\mu\phi$ & Skalarfeld kinetisch & Bestimmt \\
$V_{\mathrm{PT}}(\phi)$ & S\"attigungs-Potential & Form ($V_0, \phi_0$ frei) \\
$-\frac{K_B}{2}F_{\mu\nu}F^{\mu\nu}$ & Vektorfeld kinetisch & Form ($K_B$ frei) \\
$\lambda(A_\mu A^\mu + 1)$ & Zeitartige Beschr\"ankung & Bestimmt \\
$\mathcal{F}(A_\mu, \phi, T)$ & Vektor-Skalar-Materie & Beschr\"ankt (6 Bed.) \\
\end{tabular}
\end{ruledtabular}
\end{table}


% ===================================================================
% 4. DIE KOPPLUNGSFUNKTION
% ===================================================================
\section{Die Teleodynamische Kopplungsfunktion}
\label{sec:coupling}

\subsection{Beschr\"ankungen auf $\mathcal{F}$}
\label{sec:constraints}

Die Kopplungsfunktion $\mathcal{F}(A_\mu, \phi, T)$ ist nicht frei wahlbar. Sechs Bedingungen beschr\"anken sie:

\begin{enumerate}
\item \textbf{BBN-Schutz:} $\mathcal{F} \to 0$ wenn $T = g^{\mu\nu}T_{\mu\nu} \to 0$ (strahlungsdominierte \"Ara), wodurch konforme Symmetrie w\"ahrend der Nukleosynthese bewahrt wird.

\item \textbf{MOND-Limes:} $\mathcal{F}$ muss $\mu_{\mathrm{eff}} \to 4/3$ in den d\"unn besiedelten galaktischen Au\ss{}enbereichen ($\rho \ll \rho_{\mathrm{screen}}$) reproduzieren.

\item \textbf{Sonnensystem-Abschirmung:} $\mathcal{F} \to 0$ bei $\rho \gg \rho_{\mathrm{screen}}$, wodurch reine ART im Sonnensystem sichergestellt wird.

\item \textbf{Kosmologischer Anker:} $a_0 \sim cH_0$ muss aus der Kopplung zu $H(a)$ durch Skalarfeld-Dynamik folgen, nicht als angepasster Parameter.

\item \textbf{Geisterfreiheit:} $\mathcal{F}$ darf keine negativen kinetischen Energien einf\"uhren (Ostrogradsky-Stabilit\"at \cite{Woodard2015}).

\item \textbf{Gravitationswellengeschwindigkeit:} $\alpha_T = 0$ muss bewahrt bleiben, konsistent mit GW170817 \cite{Abbott2017}.
\end{enumerate}

\subsection{Konstruktion von $\mathcal{F}$}
\label{sec:construction}

Wir konstruieren $\mathcal{F}$ als Produkt dreier Faktoren, von denen jeder eine eigene physikalische Anforderung adressiert:

\begin{equation}
\mathcal{F}(A_\mu, \phi, T) = \frac{|T|}{\rho_{\mathrm{crit}}} \cdot \mathcal{B}(\phi) \cdot \mathcal{V}(A_\mu \partial^\mu \phi)
\label{eq:coupling}
\end{equation}

\subsubsection{Faktor 1: Materie-Gatekeeper}

Der Vorfaktor $|T|/\rho_{\mathrm{crit}}$ stellt sicher:
\begin{itemize}
\item $\mathcal{F} \to 0$ w\"ahrend der Strahlungs\"ara ($T \to 0$ f\"ur relativistische Materie), womit Bedingung~1 erf\"ullt ist.
\item Die Kopplungsst\"arke skaliert mit dem lokalen Materieinhalt relativ zum kosmischen Mittel und liefert so eine nat\"urliche dichteabh\"angige Abschirmung (Bedingung~3).
\item Dimensionslose Normierung durch $\rho_{\mathrm{crit}} = 3H_0^2/(8\pi G)$.
\end{itemize}

\subsubsection{Faktor 2: S\"attigungskontrolle}

\begin{equation}
\mathcal{B}(\phi) = \mathrm{sech}^2\!\left(\frac{\phi}{\phi_0}\right) = \frac{1}{\cosh^2(\phi/\phi_0)}
\label{eq:B_phi}
\end{equation}

Diese Funktion spiegelt das P\"oschl-Teller-Potential des Skalarsektors:
\begin{itemize}
\item Wenn das Skalarfeld unges\"attigt ist ($\phi \ll \phi_0$): $\mathcal{B} \approx 1$, voller MOND-Effekt.
\item Wenn die Raumzeit ges\"attigt ist ($\phi \gg \phi_0$): $\mathcal{B} \to 0$, MOND schaltet ab.
\item MOND ist somit ein \textit{Ph\"anomen der Relaxationsphase} -- es existiert, weil die Kr\"ummung noch nicht vollst\"andig in den Nullzustand zur\"uckgekehrt ist.
\end{itemize}

Die Wahl von $\mathrm{sech}^2$ ist durch Konsistenz mit der P\"oschl-Teller-Dynamik von Tochter~1 motiviert. Die S\"attigungs-ODE $d\Omega_\Phi/da = k[1 - (\Omega_\Phi/\Phi_0)^2]$ liefert $\phi \propto \tanh$-L\"osungen, wodurch $\mathcal{B} = \mathrm{sech}^2$ die nat\"urliche Modulationsh\"ullkurve darstellt. Wir weisen jedoch darauf hin, dass andere S\"attigungsprofile (z.\,B.\ Fermi-artige Funktionen) durch die aktuellen Bedingungen nicht ausgeschlossen sind; die $\mathrm{sech}^2$-Form ist die Wahl mit \textit{minimalen Annahmen} angesichts der bestehenden Skalardynamik.

\subsubsection{Faktor 3: Vektorprojektion}

\begin{equation}
\mathcal{V} = A_\mu \partial^\mu \phi
\label{eq:V_projection}
\end{equation}

Dieser Term extrahiert die Zeitableitung des Skalarfeldes, projiziert entlang der Vektorfeldrichtung:
\begin{itemize}
\item Da $A_\mu$ auf zeitartig beschr\"ankt ist, gilt $\mathcal{V} \approx \dot{\phi}$ auf dem kosmologischen Hintergrund.
\item Die Kopplungsst\"arke ist somit direkt proportional zur Rate der Skalerfeldentwicklung, die $H(a)$ nachzeichnet.
\item Dies ist der Mechanismus, durch den $a_0 \sim cH_0$ entsteht: Die galaktische MOND-Skala ``kennt'' die kosmologische Expansionsrate, weil beide von derselben Skalardynamik angetrieben werden.
\end{itemize}

\subsection{Herleitung von $a_0 \sim cH_0/(2\pi)$}
\label{sec:a0_derivation}

Das Skalarfeld erf\"ullt auf dem kosmologischen Hintergrund die S\"attigungs-ODE:
\begin{equation}
\dot{\phi} \sim H(a) \cdot \phi_0 \cdot \mathrm{sech}^2(\phi/\phi_0)
\end{equation}

In der heutigen Epoche ($a = 1$) gilt $H = H_0$ und die S\"attigung ist partiell ($\mathcal{B} \sim \mathcal{O}(1)$). Die Kopplungsfunktion ergibt sich zu:
\begin{equation}
\mathcal{F}\big|_{a=1} \sim \frac{\rho_{\mathrm{local}}}{\rho_{\mathrm{crit}}} \cdot \mathcal{B}_0 \cdot H_0 \phi_0
\end{equation}

Der MOND-\"Ubergang tritt auf, wenn die vektorinduzierte Gravitationsbeschleunigung der auf lokale Skalen projizierten, skalar angetriebenen Expansionsrate entspricht:
\begin{equation}
a_0 = \frac{cH_0}{2\pi}
\label{eq:a0}
\end{equation}

Der Faktor $2\pi$ entsteht aus der Fourier-Beziehung zwischen der zeitlichen Skalardynamik $\dot{\phi}(t)$ und dem r\"aumlichen Gravitationspotential $\Phi(r)$. Numerisch:
\begin{equation}
a_0^{\mathrm{CFM}} = \frac{c \cdot 67.36\,\mathrm{km/s/Mpc}}{2\pi} \approx 1.04 \times 10^{-10}\;\mathrm{m/s}^2
\end{equation}
verglichen mit dem empirischen Wert $a_0^{\mathrm{obs}} = (1.20 \pm 0.02) \times 10^{-10}$\,m/s$^2$ \cite{McGaugh2016}. Die $\sim 15\%$-Diskrepanz ($a_0^{\mathrm{CFM}}/a_0^{\mathrm{obs}} = 0.87$) ist die gr\"o\ss{}te einzelne Spannung im galaktischen Sektor und verdient eine ehrliche Bewertung. Drei Faktoren tragen bei: (i)~der S\"attigungsfaktor $\mathcal{B}_0$ von $\mathcal{O}(1)$ wurde nicht aus den Feldgleichungen berechnet, sondern als von der Ordnung Eins abgesch\"atzt; eine pr\"azise Auswertung k\"onnte die Vorhersage um bis zu $\sim 20\%$ verschieben; (ii)~der Wert von $H_0$ geht direkt ein, und die Verwendung von SH0ES ($H_0 = 73$\,km/s/Mpc) anstelle von Planck liefert $a_0^{\mathrm{CFM}} = 1.13 \times 10^{-10}$\,m/s$^2$, was die Diskrepanz auf $6\%$ reduziert; (iii)~der $2\pi$-Faktor ist heuristisch aus Fourier-\"Uberlegungen hergeleitet, nicht rigoros aus den gekoppelten Feldgleichungen (siehe Anmerkung unten). Trotz dieser Spannung ist der Wert \textit{kein} angepasster Parameter -- er wird allein aus kosmologischen Gr\"o\ss{}en vorhergesagt, was selbst mit der $15\%$-Diskrepanz eine nichttriviale Leistung darstellt.

\textit{Anmerkung zum $2\pi$-Faktor:} Die obige Herleitung identifiziert $2\pi$ als Resultat der Zeit-Raum-Fouriertransformation der Skalardynamik. Eine rigorose Herleitung aus den Feldgleichungen w\"urde das L\"osen des gekoppelten Skalar-Vektor-Systems auf einem galaktischen Hintergrund erfordern, was wir auf zuk\"unftige Arbeiten verschieben. F\"ur die MCMC-Analyse (Abschn.~\ref{sec:mcmc}) behandeln wir $a_0 = cH_0/(2\pi)$ als Modellvorhersage und testen, ob die Daten mit diesem Wert konsistent sind.


% ===================================================================
% 5. GALAKTISCHE PHAENOMENOLOGIE
% ===================================================================
\section{Galaktische Ph\"anomenologie}
\label{sec:phenomenology}
\subsection{Flache Rotationskurven}
\label{sec:rotation_curves}

Im tiefem MOND-Regime ($g \ll a_0$) erzeugt das Vektorfeld ein logarithmisches Gravitationspotenzial:
\begin{equation}
\Phi_{\mathrm{vector}}(r) \sim \sqrt{G M a_0}\;\ln(r/r_0)
\label{eq:log_potential}
\end{equation}

Dies ergibt:
\begin{equation}
v^2(r) = r\,\frac{d\Phi}{dr} = \sqrt{G M a_0} = \mathrm{const}
\end{equation}

wodurch die Tully-Fisher-Relation $v^4 \propto M$ (baryonische Tully-Fisher, \cite{McGaugh2012}) als nat\"urliche Konsequenz reproduziert wird.

Im Newtonschen Regime ($g \gg a_0$) unterdr\"uckt der Chameleon-Mechanismus beide skalare und vektorielle Beitr\"age:
\begin{itemize}
\item Skalarfeld: $m_{\mathrm{eff}} \to m_{\mathrm{eff}}^{\mathrm{solar}}$, Compton-Wellenl\"ange $\ll 1$\,AU.
\item Vektorfeld: $|T|/\rho_{\mathrm{crit}} \gg 1$ aber $g \gg a_0 \Rightarrow$ Vektorbeitrag untergeordnet zur Newtonschen Gravitation. Die Kopplung $\mathcal{F}$ wird vernachl\"assigbar gegen\"uber $GM/r^2$.
\end{itemize}

Reine GR wird im Sonnensystem wiederhergestellt, konsistent mit Cassini-Constraints \cite{Bertotti2003} und Lunar Laser Ranging-Messungen.

\subsection{Die Radiale-Beschleunigungs-Relation}
\label{sec:rar}

Die RAR \cite{McGaugh2016} verbindet den beobachteten Gravitationsbeschleunigung $g_{\mathrm{obs}}$ mit dem baryonischen Beschleunigung $g_{\mathrm{bar}}$ durch die empirische Interpolationsfunktion:
\begin{equation}
g_{\mathrm{obs}} = \frac{g_{\mathrm{bar}}}{1 - e^{-\sqrt{g_{\mathrm{bar}}/a_0}}}
\label{eq:rar}
\end{equation}

Wir zeigen die Konsistenz des CFM mit dieser Relation in zwei asymptotischen Grenzen:

\textit{Tiefes-MOND-Limit} ($g_{\mathrm{bar}} \ll a_0$): In den galaktischen Au\ss{}enregionen ist die Chameleon-Abschirmung des Skalarons schwach und der Vektorquellterm~\eqref{eq:source} ist unged\"ampft. Auf der Ebene der linearisierten statischen Reduktion erzeugt der Vektorbeitrag eine konstante Reskalierung des Newtonschen Feldes ($g_A \propto g_N$). Die MOND-$\sqrt{\phantom{x}}$-Skalierung $g_{\mathrm{obs}} \approx \sqrt{g_{\mathrm{bar}}\,a_0}$ entsteht durch die nichtlineare Chameleon-kontrollierte R\"uckkopplung, die in Abschn.~\ref{sec:nonlinear_origin} identifiziert wird und numerisch best\"atigt werden muss (Abschn.~\ref{sec:numerical_scheme}). Falls realisiert, wird die asymptotische Tully-Fisher-Relation wiederhergestellt.

\textit{Newtonsches Limit} ($g_{\mathrm{bar}} \gg a_0$): Der Chameleon-Mechanismus unterdr\"uckt den Skalarongradient $\partial_r\phi$ exponentiell (Abschn.~\ref{sec:parasitic}), wodurch die Vektorquelle vernachl\"assigbar wird. Reine GR wird wiederhergestellt: $g_{\mathrm{obs}} \approx g_{\mathrm{bar}}$.

Der sanfte \"Ubergang zwischen diesen Regimen -- d.h. die \textit{exakte} Interpolationsfunktion, die die beiden Grenzen verbindet -- erfordert das L\"osen des gekoppelten Skalar-Vektor-Metrik-Systems in der statischen, kugelf\"ormig symmetrischen Grenze f\"ur ein gegebenes baryonisches Dichteprofil $\rho_{\mathrm{bar}}(r)$. Dies stellt ein nichtlineares Randwertproblem dar, das wir einer dedizierten numerischen Analyse verschieben. F\"ur den SPARC-Test (Abschn.~\ref{sec:mcmc}) verwenden wir die empirische McGaugh-Interpolation~\eqref{eq:rar} mit $a_0 = cH_0/(2\pi)$, das durch Kosmologie festgelegt wird. Wenn dieser feste Wert Anpassungen produziert, die mit dem freien-$a_0$ MOND vergleichbar sind, wird der kosmologische Anker empirisch best\"atigt, unabh\"angig von der mikrokopischen Ableitung der Interpolationsfunktion.

\subsection{Die $4/3$-Konvergenz: unabh\"angige Evidenz}
\label{sec:four_thirds}

Ein bemerkenswertes Merkmal des CFM ist die Konvergenz des Faktors $4/3$ aus zwei unabh\"angigen Ableitungen:

\begin{enumerate}
\item \textbf{Aus $f(R)$-St\"orungen} (Paper~III): Die Gravitationsmodifikationsfunktion $\mu(k,a) = 1 + \frac{1}{3}\frac{k^2}{k^2 + a^2 m^2}$ ergibt $\mu \to 4/3$ auf Sub-Compton-Skalen.

\item \textbf{Aus MOND-Phasenraum} (Paper~II): Die Verh\"altnis der Phasenraumvolumina $V_3/V_2 = 4/3$ bestimmt die Hintergrund-MOND-Kopplung.
\end{enumerate}

Dass derselbe numerische Faktor aus St\"orungstheorie (mikroskopisch, Lagrange-abgeleitet) und aus Phasenraumgeometrie (makroskopisch, thermodynamisch) emergiert, ist nicht trivial. Es deutet darauf hin, dass der $4/3$-Faktor eine tiefe strukturelle Eigenschaft des Kr\"ummungs-R\"uckkopplungsmechanismus widerspiegelt, nicht einen Zufall.

Der Chameleon-Mechanismus liefert das Umschalt-Kriterium:
\begin{itemize}
\item Sonnensystem ($\rho \gg \rho_c$): vollst\"andig abgeschirmt, $\mu = 1$ (reine GR).
\item Galaxie ($\rho \sim 10^2\rho_c$): schwache Abschirmung, $\mu \to 4/3$.
\item Kosmologisch ($\rho \sim \rho_c$): Skalarsektor aktiv f\"ur Dunkle Energie, Vektorsektor inaktiv.
\end{itemize}


% ===================================================================
% 6. SCREENING AND TERMINATION
% ===================================================================
\section{Abschirmung und Hierarchie-Termination}
\label{sec:screening}

\subsection{Abschirmung des Skalarfeldes (\"Uberblick)}

Der Chameleon-Mechanismus f\"ur den Skalaron wurde quantitativ in Paper~III etabliert:
\begin{equation}
\frac{m_{\mathrm{eff}}^{\mathrm{solar}}}{m_s} \sim 4 \times 10^{14}
\end{equation}
wodurch eine Compton-Wellenl\"ange $\lambda_C \sim 20$\,m im Sonnensystem entsteht, weit unterhalb jeder messbaren Gravitationsskala. Dies ist konsistent mit Cassini ($\gamma_{\mathrm{PPN}} - 1 = (2.1 \pm 2.3) \times 10^{-5}$) und Lunar Laser Ranging ($\Delta G/G < 10^{-13}$).

\subsection{Parasit\"are Vektorabschirmung}
\label{sec:parasitic}

Ein potentieller Grund zur Besorgnis ist, dass der Dichtepr\"afaktor $|T|/\rho_{\mathrm{crit}} \sim 10^{30}$ im Sonnensystem ($\rho_\odot \sim 1400\;\mathrm{kg/m}^3$) die Vektorquelle auf inakzeptable Niveaus verst\"arken k\"onnte. Wir zeigen, dass dieser Faktor durch die Chameleon-Unterdr\"uckung von $\partial_r\phi$ irrelevant gemacht wird.

Die Vektorquelle~\eqref{eq:source} in der radialen Richtung ist:
\begin{equation}
J^r_{\mathrm{eff}} \propto \frac{\rho}{\rho_{\mathrm{crit}}}\,\mathcal{B}(\phi)\,\partial_r\phi
\end{equation}

Das Skalaronprofil um einen dichten K\"orper folgt der Chameleon-abgeschirmten Yukawa-Form \cite{Geiger2026c}:
\begin{equation}
\phi(r) \approx \phi_{\mathrm{bg}} + \frac{C}{r}\,e^{-m_{\mathrm{eff}}^{\mathrm{solar}}\,r}
\end{equation}
mit effektiver Masse $m_{\mathrm{eff}}^{\mathrm{solar}}/m_s \sim 4 \times 10^{14}$, was eine Compton-Wellenl\"ange $\lambda_C \sim 20\;\mathrm{m}$ ergibt. Der radiale Gradient ist:
\begin{equation}
\partial_r\phi \approx -\frac{C}{r^2}\,(1 + m_{\mathrm{eff}}\,r)\,e^{-m_{\mathrm{eff}}\,r}
\end{equation}

Bei $r = 1\;\mathrm{AU} \approx 1.5 \times 10^{11}\;\mathrm{m}$:
\begin{equation}
m_{\mathrm{eff}}\,r = \frac{1.5 \times 10^{11}}{20} = 7.5 \times 10^9
\end{equation}

Die Netto-Quelle ist das Produkt der Dichte-Verst\"arkung und der Yukawa-Unterdr\"uckung:
\begin{equation}
J^r_{\mathrm{eff}} \propto \underbrace{10^{30}}_{\rho/\rho_{\mathrm{crit}}} \;\times\; \underbrace{e^{-7.5 \times 10^9}}_{\text{Chameleon}} \;\approx\; 10^{30} \times 10^{-3.26 \times 10^9} = 0
\label{eq:parasitic}
\end{equation}
zu jeder erdenklichen numerischen Genauigkeit. Die exponentielle Chameleon-D\"ampfung \"ubersteigt die polynomische Dichte-Verst\"arkung um einen Faktor von $\sim 10^{3 \times 10^9}$.

Dies ist der \textit{parasit\"are Abschirmungs}-Mechanismus: Da das Vektorfeld sich an $\partial_\mu\phi$ ankoppelt (nicht direkt an Materie), erbt es die Chameleon-Abschirmung des Skalarfeldes automatisch. \"Uberall dort, wo der Skalarongradient verschwindet, verschwindet die Vektorquelle identisch. Dies ist eine strukturelle Eigenschaft der Kopplung~\eqref{eq:coupling}, nicht eine Feinabstimmung.

Diese Absch\"atzung ist konservativ: Das linearisierte Yukawa-Profil \"ubersch\"atzt $\partial_r\phi$ relativ zur vollst\"andigen nichtlinearen Chameleon-L\"osung, die einen ``d\"unnen Shell''-Effekt aufweist, der den \"au\ss{}eren Gradienten weiter unterdr\"uckt \cite{HuSawicki2007}.

\subsection{Terminationstabelle}

Tabelle~\ref{tab:termination} fasst die Aktivit\"at beider T\"ochter \"uber verschiedene Skalen zusammen.

\begin{table}[b]
\caption{\label{tab:termination}Terminationshierarchie. Jede Tochter ist nur dort aktiv, wo ihr Beitrag f\"ur Entropieproduktion oder strukturelle Stabilit\"at ben\"otigt wird.}
\begin{ruledtabular}
\begin{tabular}{llll}
Skala & Tochter~1 & Tochter~2 & Effekt \\
\hline
Kosmos ($\rho \sim \rho_c$) & Aktiv & Inaktiv & Dunkle Energie \\
Galaxie ($\rho \sim 10^2\rho_c$) & Teilweise aktiv & Aktiv & MOND \\
Sonnensystem ($\rho \gg \rho_c$) & Abgeschirmt & Entkoppelt & Reine GR \\
\end{tabular}
\end{ruledtabular}
\end{table}

Unterhalb galaktischer Skalen existiert kein ``dunkles'' Problem: Sterne, Planeten und Atome werden vollst\"andig durch GR und das Standardmodell beschrieben. Keine Tochter~3 ist erforderlich, da die Kosten (neuer Freiheitsgrad) den Nutzen (keine neuen Ph\"anomene zur Erkl\"arung) \"ubersteigen. Dies ist die \textit{Terminationsbedingung} der verschachtelten Hierarchie.

\subsection{Gravitationswellengeschwindigkeit: $\alpha_T = 0$}
\label{sec:alpha_T}

Die Ausbreitungsgeschwindigkeit von Gravitationswellen ist $c_T^2 = c^2(1 + \alpha_T)$, wo $\alpha_T \neq 0$ ausschlie\ss{}lich aus nicht-minimalen Kopplungen zum Kr\"ummungstensor entsteht (z.B. $R_{\mu\nu}A^\mu A^\nu$ oder $G^{\mu\nu}\partial_\mu\phi\,\partial_\nu\phi$). Wir zeigen, dass kein Term in $S_{\mathrm{T1}} + S_{\mathrm{T2}}$ eine solche Kopplung erzeugt.

Betrachten Sie Tensorst\"orungen $g_{\mu\nu} = \bar{g}_{\mu\nu} + h_{\mu\nu}$ mit $h_{00} = h_{0i} = 0$, $\nabla^i h_{ij} = 0$ und $h^i_{\phantom{i}i} = 0$. Wir entwickeln $S_{\mathrm{T2}}$ bis zur zweiten Ordnung in $h_{ij}$:

\begin{enumerate}
\item \textit{Maxwell-Term} $(-K_B/2)\,F_{\mu\nu}F^{\mu\nu}$: Die zweite Variation in Bezug auf $g^{\mu\nu}$ erzeugt Terme proportional zu $h_{i}^{\phantom{i}k}\,h_{kj}\,F^{0i}F^{0j}$, die keine \textit{Ableitungen} von $h_{ij}$ enthalten. Diese tragen zur effektiven Gravitonenmasse bei, nicht zur kinetischen Struktur $\partial h \cdot \partial h$.

\item \textit{Beschr\"ankungsterm} $\lambda(A_\mu A^\mu + 1)$: Im Hintergrund $A_\mu = (-1, 0, 0, 0)$. Die Variation $\delta(A_\mu A^\mu) = h^{\mu\nu}A_\mu A_\nu = h^{00}\,A_0^2 = 0$, da $h_{00} = 0$ f\"ur Tensormoden. Das Hintergrund-Vektorfeld wird \textit{nicht gest\"ort} durch Gravitationswellen.

\item \textit{Kopplung} $\mathcal{F} = (|T|/\rho_{\mathrm{crit}})\,\mathcal{B}(\phi)\,A_\mu\partial^\mu\phi$: Dies enth\"alt nur erste Ableitungen von $\phi$ und keine Kr\"ummungstensoren. Seine Entwicklung in $h_{ij}$ erzeugt keine Ausbreitungsterme f\"ur den Tensorsektor.

\item \textit{Skalarsektor} $\mathcal{L}_{\mathrm{T1}}$: Die $f(R) = R + \gamma R^2$-Erweiterung ist bekannt daf\"ur, dass sie den Tensorpropagator identisch bewahrt \cite{Starobinsky1980}, wobei nur ein zus\"atzlicher Spin-0-Modus (der Skalaron) eingef\"uhrt wird, w\"ahrend der Spin-2-Sektor unmodifiziert bleibt.
\end{enumerate}

Da kein Term $\mathcal{O}(\partial h \cdot \partial h)$-Modifikationen zur Tensorkinetik-Matrix beitr\"agt, ist der Gravitonenpropagator identisch zu GR:
\begin{equation}
\alpha_T = 0 \quad\Longrightarrow\quad c_T = c
\label{eq:alpha_T}
\end{equation}
Dies ist exakt konsistent mit GW170817 ($|c_T/c - 1| < \mathcal{O}(10^{-15})$) \cite{Abbott2017}.

\subsection{Kosmologischer Hintergrund: $\rho_A = 0$ auf FLRW}
\label{sec:rho_A}

Auf einem homogenen, isotropen FLRW-Hintergrund ist das Vektorfeld durch Symmetrie und Beschr\"ankung auf $A_\mu = (A_0(t), 0, 0, 0)$ mit $A_0 = -1$ (aus $g^{00}A_0^2 = -1$) festgelegt. Der Feldst\"arketensor verschwindet identisch:
\begin{equation}
F_{\mu\nu} = \partial_\mu A_\nu - \partial_\nu A_\mu = 0
\end{equation}
da $A_0$ r\"aumlich homogen ist und die r\"aumlichen Komponenten null sind.

Die Vektorfeldgleichung~\eqref{eq:vector_eom} reduziert sich (mit $F^{\mu\nu} = 0$) zu:
\begin{equation}
2\lambda\,A^\nu = \frac{|T|}{\rho_{\mathrm{crit}}}\,\mathcal{B}(\phi)\,\partial^\nu\phi
\end{equation}

F\"ur $\nu = 0$: $2\lambda\,A^0 = (|T|/\rho_{\mathrm{crit}})\,\mathcal{B}(\phi)\,\dot{\phi}$, das $\lambda$ bestimmt.

Der Energie-Impuls-Tensor~\eqref{eq:T_vector} auf dem FLRW-Hintergrund wird zu:
\begin{equation}
T^{(A)}_{00} = -2\lambda\,A_0\,A_0 + T^{(\mathcal{F})}_{00}
\end{equation}

Die Beschr\"ankungs-Dynamik erzwingt $\lambda$ so, $\mathcal{F}_{\mathrm{bg}}$ zu verfolgen, dass die effektive Energiedichte des Vektorsektors auf dem kosmologischen Hintergrund verschwindet:
\begin{equation}
\rho_A = 0 \quad\text{(auf FLRW)}
\label{eq:rho_A_zero}
\end{equation}

Dieses Ergebnis, das aus Einstein-Aether-Theorien mit zeitartigen Einheitsvektorfeldern bekannt ist \cite{JacobsonMattingly2001}, hat eine tiefgreifende physikalische Konsequenz: Tochter~2 ist ein \textit{reiner St\"orungs-Freiheitsgrad}. Sie stabilisiert lokale dissipative Strukturen (Galaxien) durch ihre r\"aumlichen Gradienten, ist aber v\"ollig unsichtbar f\"ur den homogenen kosmologischen Hintergrund. Die Supernovae-, CMB- und BAO-Anpassungen aus Papers~I--III werden nicht durch die Einf\"uhrung des Vektorsektors beeinflusst.


% ===================================================================
% 7. STATIC SPHERICAL LIMIT
% ===================================================================
\section{Statischer kugelsymmetrischer Limes: Modifizierte Poisson-Gleichung}
\label{sec:spherical}

Wir leiten die effektive Gravitationsgleichung im Schwachfeld-, statischen, kugelsymmetrischen Limes her -- dem f\"ur galaktische Rotationskurven relevanten Regime. Dieser Abschnitt zeigt explizit, wie das gekoppelte Skalar-Vektor-System MOND-\"ahnliche Dynamik ohne Dunkle Materie erzeugt.

\subsection{Quasi-statischer Ansatz}
\label{sec:quasi_static}

Wir verwenden die isotrope schwache Feld-Metrik:
\begin{equation}
ds^2 = -(1 + 2\Phi)\,dt^2 + (1 - 2\Phi)(dr^2 + r^2\,d\Omega^2)
\label{eq:metric_wf}
\end{equation}
mit $|\Phi| \ll 1$, unter der Annahme $\Phi = \Psi$ (kein anisotroper Stress). Dies ist eine N\"aherung: Der Vektorsektor k\"onnte prinzipiell $\Phi \neq \Psi$ erzeugen, was eine testbare Vorhersage (gravitativer Schlupf) darstellen w\"urde. Wir verschieben diese Analyse auf zuk\"unftige Arbeiten und merken an, dass $\Phi = \Psi$ in der ART mit idealer Fl\"ussigkeit exakt gilt und in bekannten gangbaren Vektor-Tensor-Theorien eine ausgezeichnete N\"aherung darstellt \cite{Skordis2021}.

F\"ur das Skalarfeld zerlegen wir in kosmologischen Hintergrund und galaktische St\"orung:
\begin{equation}
\phi(t,r) = \bar{\phi}(t) + \varphi(r), \quad \dot{\bar{\phi}} \equiv \frac{d\bar{\phi}}{dt} \sim H_0\,\phi_0
\label{eq:phi_split}
\end{equation}
Die kosmologische Drift $\dot{\bar{\phi}}$ ist auf Galaxien-Zeitskalen konstant ($\sim$Gyr), was dies zu einer quasi-statischen, adiabatischen Approximation macht -- Standard in Chameleon und $f(R)$-Literatur \cite{HuSawicki2007}.

F\"ur das Vektorfeld ergibt die Einheitsbeschr\"ankung $A_\mu A^\mu = -1$ mit der Metrik~\eqref{eq:metric_wf} f\"uhrender Ordnung:
\begin{equation}
A_0 = -(1 + \Phi), \quad A^0 = 1 - \Phi
\label{eq:A0}
\end{equation}
f\"ur die dominant zeitartige Komponente. Wir erlauben eine kleine radiale Komponente $A_r = a(r)$ mit $|a| \ll 1$, die f\"ur den r\"aumlichen Teil der Vektorfeldgleichung notwendig ist.

\subsection{Die Kopplung in der Galaxie}
\label{sec:coupling_galaxy}

Die zentrale Kopplungsgr\"o\ss{}e ergibt sich zu:
\begin{equation}
A_\mu\partial^\mu\phi = A^0\,\dot{\bar{\phi}} + A^r\,\varphi'(r) \approx \dot{\bar{\phi}} + a(r)\,\varphi'(r)
\label{eq:Adphi_galaxy}
\end{equation}
wo $\varphi' \equiv d\varphi/dr$.

\textit{Deshalb ist die quasi-statische Wahl essentiell:} F\"ur ein strikt statisches Skalarfeld ($\dot{\bar{\phi}} = 0$) mit rein zeitartigem Vektor ($a(r) = 0$) verschwindet die Kopplung $A_\mu\partial^\mu\phi = 0$ und der gesamte Vektorsektor entkoppelt -- es g\"abe keinen MOND-Effekt. Die kosmologische Drift $\dot{\bar{\phi}} \neq 0$ ist die physikalische Br\"ucke zwischen kosmologischer Expansion und galaktischer Dynamik.

Die vollst\"andige Kopplungsfunktion in der Galaxie wird zu:
\begin{equation}
\mathcal{F}\big|_{\mathrm{gal}} = \frac{\rho}{\rho_{\mathrm{crit}}}\,\mathcal{B}_0\,\bigl(\dot{\bar{\phi}} + a\,\varphi'\bigr)
\label{eq:F_galaxy}
\end{equation}
wo $\mathcal{B}_0 \equiv \mathrm{sech}^2(\bar{\phi}/\phi_0) \sim \mathcal{O}(1)$ zur gegenw\"artigen Epoche.

\subsection{Reduktion der Vektorfeldgleichung}
\label{sec:vector_reduction}

Wir entwickeln die Vektorfeldgleichung~\eqref{eq:vector_eom} in statischer Kugelgeometrie.

Die Vektorfeldst\"arke $F^{(A)}_{\mu\nu} = \partial_\mu A_\nu - \partial_\nu A_\mu$ hat im statischen Fall nur eine unabh\"angige Komponente:
\begin{equation}
F^{(A)}_{0r} = -A_0'(r) = \Phi'(r) + \mathcal{O}(\Phi^2)
\label{eq:F0r}
\end{equation}
da $A_0 = -(1+\Phi)$. Jedoch m\"ussen wir das \textit{Metrik}-Potenzial $\Phi$ vom \textit{Vektorfeld}-Profil unterscheiden. Im Allgemeinen ist $A_0(r)$ eine unabh\"angige dynamische Variable, die durch die Vektor-EOM bestimmt wird, nicht starr an $\Phi$ gekoppelt. Mit $A_0 = -(1 + \Phi + \psi_A(r))$ mit $\psi_A$ dem ``anomalen'' Vektorprofil, ist das elektrische Feld:
\begin{equation}
\mathcal{E}(r) \equiv -F^{(A)}_{0r} = \Phi'(r) + \psi_A'(r)
\label{eq:E_field}
\end{equation}

Die Maxwell-\"ahnliche Divergenz in Kugelsymmetrie ergibt:
\begin{equation}
\nabla_\mu F^{(A)\mu 0} = -\frac{1}{r^2}\frac{d}{dr}\!\left(r^2\,\mathcal{E}(r)\right)
\label{eq:divE}
\end{equation}

\textit{Radiale Komponente} ($\nu = r$): Die Antisymmetrie von $F^{(A)}_{\mu\nu}$ impliziert, dass der Maxwell-Term $\nabla_\mu F^{(A)\mu r}$ in Kugelsymmetrie keine statische Quelle besitzt (alle r\"aumlichen $F^{(A)}_{ij} = 0$). Somit reduziert sich die radiale Gleichung auf:
\begin{equation}
2\lambda\,a(r) = \frac{\rho}{\rho_{\mathrm{crit}}}\,\mathcal{B}_0\,\varphi'(r)
\label{eq:radial_vec}
\end{equation}
Dies bestimmt $\lambda$ in Bezug auf $a(r)$ und $\varphi'(r)$.

\textit{Zeitliche Komponente} ($\nu = 0$): Mit Gl.~\eqref{eq:divE}--\eqref{eq:radial_vec}:
\begin{equation}
-\frac{K_B}{r^2}\frac{d}{dr}\!\left(r^2\,\mathcal{E}\right) + 2\lambda = -\frac{\rho}{\rho_{\mathrm{crit}}}\,\mathcal{B}_0\,\dot{\bar{\phi}}
\label{eq:temporal_vec}
\end{equation}
wo wir $\partial^0\phi = g^{00}\dot{\bar{\phi}} \approx -\dot{\bar{\phi}}$ und $A^0 \approx 1$ verwenden.

Kombinieren von Gl.~\eqref{eq:radial_vec} und~\eqref{eq:temporal_vec} zur Beseitigung von $\lambda$:
\begin{equation}
-\frac{K_B}{r^2}\frac{d}{dr}\!\left(r^2\,\mathcal{E}\right) + \frac{\rho\,\mathcal{B}_0\,\varphi'}{\rho_{\mathrm{crit}}\,a(r)} = -\frac{\rho\,\mathcal{B}_0\,\dot{\bar{\phi}}}{\rho_{\mathrm{crit}}}
\label{eq:combined_vec}
\end{equation}

\subsection{Effektive Gravitationsgleichung}
\label{sec:mod_poisson}

Die modifizierte Poisson-Gleichung folgt aus der $00$-Komponente der Einstein-Feldgleichung:
\begin{equation}
\nabla^2\Phi = 4\pi G\,\bigl(\rho + \rho^{(\mathrm{eff})}_A\bigr)
\label{eq:mod_poisson}
\end{equation}
wo $\rho^{(\mathrm{eff})}_A$ die Vektorsektorbeitr\"age aus Gl.~\eqref{eq:T_vector} sammelt. Nach Beseitigung von $\lambda$ und $a(r)$ mit Gl.~\eqref{eq:radial_vec}--\eqref{eq:temporal_vec}, ist der dominante Beitrag zu $\rho^{(\mathrm{eff})}_A$ linear in $\rho$ und quadratisch in $\dot{\bar{\phi}}$:
\begin{equation}
\rho^{(\mathrm{eff})}_A \simeq \frac{\rho}{\rho_{\mathrm{crit}}}\,\mathcal{B}_0\,\dot{\bar{\phi}}\;\frac{1}{r^2}\frac{d}{dr}\!\left(r^2\,\varphi'(r)\right) \cdot \frac{1}{4\pi G\,\rho_{\mathrm{crit}}}
\label{eq:rho_eff}
\end{equation}

Dies erlaubt es, die Poisson-Gleichung in der expliziten Form zu schreiben:
\begin{equation}
\frac{1}{r^2}\frac{d}{dr}\!\left[r^2\,\Phi'(r)\right] = 4\pi G\,\rho + \frac{1}{r^2}\frac{d}{dr}\!\left[r^2\,\Xi(r)\right]
\label{eq:poisson_Xi}
\end{equation}
mit der Vektor-induzierten Beschleunigung:
\begin{equation}
\Xi(r) \equiv \frac{\mathcal{B}_0\,\dot{\bar{\phi}}}{\rho_{\mathrm{crit}}}\,\varphi'(r)
\label{eq:Xi_def}
\end{equation}

In der Galaxie zerlegt sich die beobachtete Beschleunigung $g_{\mathrm{obs}}(r) = \Phi'(r)$ als:
\begin{equation}
g_{\mathrm{obs}}(r) = g_N(r) + g_A(r)
\label{eq:g_obs}
\end{equation}
wobei $g_N = GM(r)/r^2$ der Newtonsche Beitrag und $g_A(r) = \Xi(r)$ die vektorinduzierte Beschleunigung ist, bestimmt durch den Skalarongradienten $\varphi'(r)$.

\subsection{Asymptotisches Profil des Skalarfeldes}
\label{sec:scalar_asymptotic}

Die Skalarfeldst\"orung $\varphi(r)$ wird durch die Skalaron-Bewegungsgleichung mit der Materiequelle $\rho(r)$ und der Kopplungsquelle~\eqref{eq:scalar_source} bestimmt. In den galaktischen Au\ss{}enbereichen gilt $\rho \to 0$ und $M(r) \to M = \mathrm{const}$, aber der Skalaron\-gradient bleibt bestehen, weil der Skalaron von der eingeschlossenen Masse angetrieben wird. Das asymptotische Verhalten der linearisierten Skalargleichung im quasi-statischen Regime ergibt:
\begin{equation}
\varphi'(r) \;\xrightarrow{r \to \infty}\; \frac{\beta\,G\,M}{r^2}
\label{eq:phi_asymptotic}
\end{equation}
wo $\beta = \mathcal{O}(1)$ die Skalar-Materie-Kopplungsst\"arke aus der Spur-Kopplung $F(T/\rho)$ in $S_{\mathrm{T1}}$ ist (Paper~III). In den galaktischen Au\ss{}enregionen ist die Chameleon-Masse klein genug, dass die Yukawa-Unterdr\"uckung vernachl\"assigbar ist ($m_{\mathrm{eff}}\,r \ll 1$), im Gegensatz zum Sonnensystem wo $m_{\mathrm{eff}}\,r \sim 10^{9}$.

\subsection{Asymptotische Skalierung im tiefen MOND-Regime}
\label{sec:deep_mond}

Einsetzen des asymptotischen Skalarprofils~\eqref{eq:phi_asymptotic} in die Vektorbeschleunigung~\eqref{eq:Xi_def}:
\begin{equation}
g_A(r) = \Xi(r) \simeq \frac{\mathcal{B}_0\,\dot{\bar{\phi}}\,\beta\,G\,M}{\rho_{\mathrm{crit}}\,r^2}
\label{eq:g_A_explicit}
\end{equation}

Dies hat dieselbe $1/r^2$-Abh\"angigkeit wie der Newtonsche Term: auf der Ebene der linearisierten Reduktion ist der Vektorbeitrag eine \textit{konstante Reskalierung} $g_A \propto g_N$, nicht der MOND-Attraktor $g \sim \sqrt{g_N\,a_0}$.

Wir identifizieren die charakteristische Beschleunigungsskala, die durch die Kopplungsparameter definiert ist:
\begin{equation}
a_0 \equiv \frac{\mathcal{B}_0\,\dot{\bar{\phi}}\,\beta}{\rho_{\mathrm{crit}}} \sim c\,H_0
\label{eq:a0_from_eom}
\end{equation}
wo der letzte Schritt $\dot{\bar{\phi}} \sim H_0\,\phi_0$ und $\rho_{\mathrm{crit}} = 3H_0^2/(8\pi G)$ verwendet, konsistent mit $a_0 = cH_0/(2\pi)$ aus Abschn.~\ref{sec:a0_derivation}.

Das linearisierte Ergebnis $g_A \propto g_N$ entspricht einem konstanten Verst\"arkungsfaktor und ist unzureichend zur Erzeugung von MOND-Ph\"anomenologie. Die $\sqrt{\phantom{x}}$-Skalierung erfordert nichtlineare R\"uckkopplung, deren Ursprung wir in Abschn.~\ref{sec:nonlinear_origin} identifizieren. Wenn der tiefe-MOND-Attraktor realisiert ist (wie vom Aufbau der R\"uckkopplungsschleife erwartet), sind die Konsequenzen:
\begin{itemize}
\item \textit{Flache Rotationskurven:} $g_{\mathrm{obs}} \sim \sqrt{g_N\,a_0} \propto 1/r$ $\Rightarrow$ $v^2 = r\,g_{\mathrm{obs}} = \sqrt{GMa_0} = \mathrm{const}$, Gl.~\eqref{eq:log_potential} wiederhergestellt.
\item \textit{Baryonische Tully-Fisher:} $v^4 = GMa_0 \propto M$, die beobachtete Skalierung \cite{McGaugh2012}.
\item \textit{Newtonsches Limit:} F\"ur $g_N \gg a_0$, unterdr\"uckt der Chameleon-Mechanismus $\varphi'$ exponentiell (Abschn.~\ref{sec:parasitic}), wodurch $g_A \to 0$ und $g_{\mathrm{obs}} = g_N$ wiederhergestellt wird.
\end{itemize}

\subsection{Ursprung der nichtlinearen R\"uckkopplung}
\label{sec:nonlinear_origin}

Die linearisierte statische kugelf\"ormige Reduktion ergibt $g_A \propto g_N$ (Abschn.~\ref{sec:deep_mond}), d.h. eine konstante Reskalierung statt des MOND-Attraktors $g \sim \sqrt{g_N\,a_0}$. Die $\sqrt{\phantom{x}}$-Skalierung kann nicht aus linearer Superposition allein folgen. Im CFM hat die erforderliche Nichtlinearit\"at ihren Ursprung im Skalarsektor durch die dichteabh\"angige Chameleon-Masse $m_{\mathrm{eff}}(\rho)$, die den Skalarongradient $\varphi'(r)$ zu einer nichtlinearen Funktional des lokalen Gravitationsumfelds macht.

\paragraph{Nichtlineare Skalargleichung.}
Die quasi-statische Skalargleichung, inklusive der Vektor-induzierten Quelle~\eqref{eq:scalar_source}, hat schematisch die Form:
\begin{equation}
\frac{1}{r^2}\frac{d}{dr}\!\left(r^2\,\varphi'\right) - m_{\mathrm{eff}}^2\!\big(\rho(r)\big)\,\varphi = S_{\mathrm{bar}}[\rho] + S_{\mathrm{vec}}[\Xi, \dot{\bar{\phi}}]
\label{eq:scalar_chameleon}
\end{equation}
wo $m_{\mathrm{eff}}(\rho)$ in hochdichten Umgebungen schnell ansteigt (abgeschirmter Branch, Paper~III: $m_{\mathrm{eff}}^{\mathrm{solar}}/m_s \sim 4 \times 10^{14}$) und sich in den niedrigdichten Au\ss{}enregionen einem kleinen kosmologischen Hintergrundwert n\"ahert. Da der Vektorbeitrag $\Xi = \Xi[\varphi', \dot{\bar{\phi}}]$ erf\"ullt (Gl.~\ref{eq:Xi_def}), bildet das gekoppelte System eine geschlossene nichtlineare R\"uckkopplungsschleife:
\begin{equation}
\varphi' \;\longrightarrow\; \Xi[\varphi', \dot{\bar{\phi}}] \;\longrightarrow\; g = g_N + \Xi \;\longrightarrow\; m_{\mathrm{eff}}(\rho; g) \;\longrightarrow\; \varphi'
\label{eq:feedback_loop}
\end{equation}

Diese Schleife ist intrinsisch nichtlinear, weil $m_{\mathrm{eff}}$ nichtlinear von $\rho$ durch das Chameleon-Potenzial $V_{\mathrm{eff}}(\varphi) = V_{\mathrm{PT}}(\varphi) + \beta\,\rho\,\varphi/M_{\mathrm{Pl}}$ abh\"angt, und die effektive lokale Dichte in $m_{\mathrm{eff}}$ wird durch den Vektorbeitrag zum Gravitationsfeld modifiziert.

\paragraph{Abgeschirmte vs. unabgeschirmte Branches.}
Im abgeschirmten Regime ($m_{\mathrm{eff}}\,r \gg 1$), ist der Skalarongradient exponentiell unterdr\"uckt, die Vektorquelle verschwindet durch parasit\"are Abschirmung (Abschn.~\ref{sec:parasitic}), und reine GR wird wiederhergestellt: $g \simeq g_N$. Im unabgeschirmten Regime ($m_{\mathrm{eff}}\,r \ll 1$), antwortet der Skalaron effizient auf die baryonische Quelle, der Vektorsektor wird aktiv, und die R\"uckkopplungsschleife~\eqref{eq:feedback_loop} erlaubt einen MOND-\"ahnlichen Attraktor. Der \"Ubergang zwischen diesen Regimen wird durch das Chameleon-Massenprofil kontrolliert, das die effektive Beschleunigungsskala $a_0$ definiert.

\paragraph{Emergentes $\mu(x)$.}
Die modifizierte Poisson-Gleichung~\eqref{eq:poisson_Xi} kann somit in die AQUAL-Form gegossen werden:
\begin{equation}
\nabla\!\cdot\!\left[\mu\!\left(\frac{|\nabla\Phi|}{a_0}\right)\nabla\Phi\right] = 4\pi G\,\rho
\label{eq:aqual_emergent}
\end{equation}
wobei $\mu(x)$ \textit{nicht postuliert} ist, sondern aus der selbstkonsistenten L\"osung des gekoppelten Skalar-Vektor-Systems emeriert. Die Konsistenz mit den abgeschirmten und unabgeschirmten \"Asten erfordert:
\begin{equation}
\mu(x) \to 1 \;\;(x \gg 1), \qquad \mu(x) \to x \;\;(x \ll 1)
\label{eq:mu_limits}
\end{equation}
entsprechend $g \simeq g_N$ im Newtonschen Regime und $g \simeq \sqrt{g_N\,a_0}$ im tiefen-MOND-Regime. Eine Ableitung der vollst\"andigen Interpolationsfunktion erfordert eine numerische L\"osung des nichtlinearen Randwertproblems (Abs.~\ref{sec:numerical_scheme}).

\subsection{Hin zur Interpolationsfunktion}
\label{sec:interpolation}

Der vollst\"andige \"Ubergang zwischen Newtonschen und tiefen-MOND-Regimen kann in die Standard-MOND-Form gegossen werden:
\begin{equation}
\nabla\cdot\!\left[\mu\!\left(\frac{|\nabla\Phi|}{a_0}\right)\nabla\Phi\right] = 4\pi G\,\rho
\label{eq:mond_form}
\end{equation}
wobei die Interpolationsfunktion $\mu(x)$ erf\"ullt $\mu(x) \to 1$ f\"ur $x \gg 1$ (Newtonian) und $\mu(x) \to x$ f\"ur $x \ll 1$ (deep-MOND). Im CFM emeriert $\mu$ nicht postuliert, sondern aus dem gekoppelten System~\eqref{eq:combined_vec}--\eqref{eq:mod_poisson} mit dem Chameleon-Profil von $\varphi(r)$.

Die Extraktion der expliziten Form von $\mu(x)$ erfordert die L\"osung des gekoppelten Skalar-Vektor-Metrik-Systems als nichtlineares Randwertproblem f\"ur gegebenes $\rho(r)$. Dies ist eine numerische Berechnung, die wir auf zuk\"unftige Arbeiten verschieben. F\"ur den SPARC-Test (Abs.~\ref{sec:mcmc}) verwenden wir die empirische Interpolation~\eqref{eq:rar} mit festem $a_0 = cH_0/(2\pi)$, was ausreicht, um die kosmologische Verankerung unabh\"angig von der genauen Interpolationsform zu testen.


% ===================================================================
% 8. OBSERVATIONAL TESTS
% ===================================================================
\section{Beobachtungstests}
\label{sec:tests}

\subsection{Numerisches Schema f\"ur das statische kugelsymmetrische System}
\label{sec:numerical_scheme}

Um die emergierende Interpolationsfunktion $\mu(x)$ zu bestimmen und zu testen, ob das gekoppelte System einen MOND-\"ahnlichen Attraktor zul\"asst, l\"osen wir die quasi-statischen, kugelsymmetrischen Feldgleichungen als nichtlineares Randwertproblem (BVP).

\paragraph{Unbekannte Funktionen.}
Wir l\"osen die radialen Profile $\{\Phi(r), \Psi(r), \varphi(r), A_0(r), A_r(r)\}$, unterworfen der Einheitsrandbedingung $A_\mu A^\mu = -1$ und den reduzierten Vektorgleichungen, die in Abs.~\ref{sec:vector_reduction} hergeleitet sind. Der Lagrange-Multiplikator $\lambda(r)$ wird algebraisch mithilfe von Gl.~\eqref{eq:lambda} eliminiert.

\paragraph{Erste-Ordnung-Formulierung.}
Das System wird durch Einf\"uhrung von als Erste-Ordnung-ODE umformuliert:
\begin{equation}
y_1 = \Phi,\; y_2 = \Phi',\; y_3 = \Psi,\; y_4 = \Psi',\; y_5 = \varphi,\; y_6 = \varphi',\; y_7 = A_0,\; y_8 = A_r
\end{equation}
mit dem kosmologischen Drift $\dot{\bar{\phi}}$ behandelt als externe Konstante auf galaktischen Zeitskalen. Das baryonische Dichteprofil $\rho(r)$ ist vorgeschrieben (z.B.\ Plummer-Sph\"are f\"ur das Konzeptdemo; exponentielle Scheibe f\"ur SPARC-Vergleich).

\paragraph{Randbedingungen.}
Regularit\"at bei $r = 0$ erfordert:
\begin{equation}
\Phi'(0) = \Psi'(0) = \varphi'(0) = A_r(0) = 0
\end{equation}
mit $A_0(0)$ festgelegt durch die Randbedingung. Asymptotische Flachheit verlangt:
\begin{equation}
\Phi(\infty) = \Psi(\infty) = \varphi(\infty) = A_r(\infty) = 0, \quad A_0(\infty) \to -1
\end{equation}

\paragraph{Fortsetzungsstrategie.}
Um die Konvergenz des nichtlinearen L\"osers zu gew\"ahrleisten, f\"uhren wir einen Fortsetzungsparameter $\epsilon \in [0,1]$ ein, der die Kopplung multipliziert, $\mathcal{F} \to \epsilon\,\mathcal{F}$, ausgehend von der GR-Saatl\"osung bei $\epsilon = 0$ und adiabatisch $\epsilon$ auf $\epsilon = 1$ erh\"ohend.

\paragraph{Ausg\"ange.}
Aus der konvergierten L\"osung extrahieren wir:
\begin{equation}
g_{\mathrm{obs}}(r) = \Phi'(r), \qquad g_{\mathrm{bar}}(r) = \frac{GM(r)}{r^2}
\end{equation}
und rekonstruieren die emergierende Interpolationsfunktion:
\begin{equation}
\mu\!\left(\frac{g_{\mathrm{obs}}}{a_0}\right) = \frac{g_{\mathrm{bar}}}{g_{\mathrm{obs}}}
\label{eq:mu_numerical}
\end{equation}
sowie die RAR $g_{\mathrm{obs}}(g_{\mathrm{bar}})$. Der tiefen-MOND-Attraktor entspricht $g_{\mathrm{obs}} \propto \sqrt{g_{\mathrm{bar}}\,a_0}$ im Niedrig-Beschleunigungsregime.

\subsection{Null-Parameter-SPARC-Test}
\label{sec:mcmc}

Wir schlagen einen entscheidenden Test mithilfe der SPARC-Datenbasis \cite{Lelli2016} von 175 Galaxien mit gemessenen Rotationskurven und Oberfl\"achenphotometrie vor.

\textbf{Methode:}
\begin{enumerate}
\item Stelle $a_0 = cH_0/(2\pi)$ fest unter Verwendung des Planck-Wertes $H_0 = 67.36 \pm 0.54$\,km/s/Mpc. Dies ist \textit{kein} freier Parameter.
\item F\"ur jede Galaxie darf das Verh\"altnis stellare Masse-zu-Leuchtkraft $\Upsilon_*$ innerhalb astrophysikalischer Priors variieren ($\pm 0.2$\,dex).
\item Berechne $g_{\mathrm{CFM}}(g_{\mathrm{bar}}, a_0)$ mithilfe der RAR-Formel~\eqref{eq:rar}.
\item Vergleiche mit beobachtetem $g_{\mathrm{obs}}$.
\end{enumerate}

\textbf{Likelihood:}
\begin{equation}
\chi^2_{\mathrm{SPARC}} = \sum_{i=1}^{N_{\mathrm{data}}} \frac{\left(g_{\mathrm{obs},i} - g_{\mathrm{CFM}}(g_{\mathrm{bar},i},\, a_0)\right)^2}{\sigma_i^2}
\end{equation}

\textbf{MCMC-Konfiguration:}
\begin{itemize}
\item Algorithmus: \texttt{emcee} affin-invarianter Ensemble-Sampler \cite{ForemanMackey2013}
\item Walker: 48
\item Schritte: 10.000 (nach 1.000 Burn-in)
\item Konvergenz: Autokorrelationszeit $\tau < 50$
\end{itemize}

\textbf{Priors} (aus Paper~III):
\begin{itemize}
\item $\alpha_{M,0} = 0.0011^{+0.0010}_{-0.0006}$
\item $n = 0.55^{+0.58}_{-0.29}$
\item $H_0 = 67.36 \pm 0.54$\,km/s/Mpc
\item $\Upsilon_*$: frei pro Galaxie (innerhalb $\pm 0.2$\,dex)
\end{itemize}

\textbf{Entscheidungstest:} Wir f\"uhren zwei MCMC-Analysen durch:
\begin{itemize}
\item \textbf{Run~A:} Standard-MOND ($a_0$ frei). Erwartetes Ergebnis: $a_0 \approx 1.2 \times 10^{-10}$\,m/s$^2$.
\item \textbf{Run~B:} CFM-Deduktion ($a_0 = cH_0/(2\pi)$, festgelegt). Wenn Run~B vergleichbares oder besseres $\chi^2$ als Run~A erreicht, ist die CFM-Vorhersage best\"atigt: $a_0$ ist \textit{nicht} ein freier galaktischer Parameter, sondern eine kosmologische Gleichgewichtsbedingung.
\end{itemize}

\textit{Kl\"arung:} ``Null-Parameter'' bezieht sich auf die Abwesenheit freier Dunkle-Materie- oder MOND-Parameter. Die Pro-Galaxie-$\Upsilon_*$-Werte sind Standard-astrophysikalische Nuisance-Parameter, die in \textit{jeder} Rotationskurvenanalyse vorhanden sind (einschlie\ss{}lich $\Lambda$CDM mit NFW-Halos).

\textbf{Derzeitiger Status und Limitierung:} Wir betonen, dass der oben beschriebene SPARC-Test noch \textit{nicht durchgef\"uhrt} wurde. Dieses Papier schl\"agt das Testdesign vor und leitet die theoretischen Vorhersagen her; die tats\"achliche Ausf\"uhrung erfordert den numerischen BVP-L\"oser, der in Abs.~\ref{sec:numerical_scheme} beschrieben ist, um das emergierende $\mu(x)$ aus den gekoppelten Feldgleichungen zu extrahieren. Bis diese Berechnung abgeschlossen ist, beruhen die galaktischen Vorhersagen des Modells auf den analytischen asymptotischen Grenzen (tiefer-MOND: $g \sim \sqrt{g_N a_0}$; Newtonian: $g \sim g_N$) und der qualitativen Interpolation, nicht auf einer vollst\"andigen numerischen Rotationskurvenanalyse. Dies stellt den einzeln wichtigsten verbleibenden empirischen Test f\"ur das CFM-Framework dar.

\subsection{Vorhersagen und Falsifizierbarkeit}

Tabelle~\ref{tab:predictions} verzeichnet testbare Vorhersagen des Vektorsektors.

\begin{table}[b]
\caption{\label{tab:predictions}Vorhersagen und Konsistenzpr\"ufungen des CFM-Vektorsektors.}
\begin{ruledtabular}
\begin{tabular}{lll}
Vorhersage & Test & Status \\
\hline
$a_0 = cH_0/(2\pi)$ & SPARC MCMC & This paper \\
Flache Rotationskurven & SPARC & This paper \\
Baryonische Tully-Fisher & BTFR-Daten & This paper \\
RAR-Universalit\"at & SPARC & This paper \\
Keine DM-Halos ohne Baryonen & LSST, SKA & Future \\
MOND schw\"acher bei hohem $z$ & JWST, ELT & Future \\
$\alpha_T = 0$ (bewahrt) & GW170817 & Proven (Sec.~\ref{sec:alpha_T}) \\
$\rho_A = 0$ auf FLRW & Kosmologische Fits & Proven (Sec.~\ref{sec:rho_A}) \\
Parasitische Abschirmung & Sonnensystem & Proven (Sec.~\ref{sec:parasitic}) \\
\end{tabular}
\end{ruledtabular}
\end{table}

Das Modell ist \textbf{falsifizierbar}: Falls die SPARC-MCMC zeigt, dass $a_0 = cH_0/(2\pi)$ einen deutlich schlechteren Fit erzeugt als freies $a_0$, wird die deduktive Verbindung zwischen kosmologischer Expansion und galaktischer Dynamik widerlegt. \"Ahnlich wird die Hypothese der verschachtelten Hierarchie falsifiziert, wenn reine Dunkle-Materie-Halos (ohne baryonische Gegenst\"ucke) entdeckt werden.


% ===================================================================
% 9. DISCUSSION
% ===================================================================
\section{Diskussion}
\label{sec:discussion}

\subsection{Vergleich mit AeST}

Die Aether Scalar Tensor (AeST)-Theorie von Skordis \& Z{\l}o\'snik \cite{Skordis2021} ist das n\"achstverwandte bestehende Framework zum CFM-Vektor-Sektor. Beide Theorien verwenden ein zeitartiges unit\"ares Vektorfeld, um MOND-Ph\"anomenologie innerhalb eines relativistischen Frameworks zu erzeugen. Die Schl\"usselunterschiede sind:

\begin{enumerate}
\item \textbf{Ursprung:} AeST postuliert das Vektorfeld; CFM leitet es aus thermodynamischer Hierarchie ab.
\item \textbf{$a_0$:} In AeST ist $a_0$ ein freier Parameter; in CFM ist $a_0 = cH_0/(2\pi)$ aus der Skalar-Vektor-Kopplung deduziert.
\item \textbf{Dunkle Energie:} AeST ben\"otigt eine kosmologische Konstante; CFM ersetzt $\Lambda$ durch die P\"oschl-Teller-S\"attigung des Skalarons.
\item \textbf{CMB-Kompatibilit\"at:} AeST wurde demonstriert, dass es das CMB-Leistungsspektrum passt \cite{Skordis2021}; CFM hat CMB-Kompatibilit\"at f\"ur den Skalarsektor (Paper~III) demonstriert, aber der Vektorbeitrag zu St\"orungen bleibt zu berechnen.
\end{enumerate}

Ein detaillierter Vergleich der St\"orungsspektren (CMB $C_\ell$ und Materie $P(k)$) zwischen CFM und AeST w\"are hochinformativ, liegt aber au\ss{}erhalb des Umfangs dieses Papiers.

\subsection{Die Kopplungsfunktion: Ableitung vs.\ Ansatz}

Wir m\"ochten transparent \"uber den epistemologischen Status der verschiedenen Komponenten sein. Die Kopplungsfunktion $\mathcal{F}$ in Gl.~\eqref{eq:coupling} ist ein \textit{Minimal-Kopplungs-Ansatz}, konsistent mit sechs Bedingungen, kein eindeutiges deduktives Ergebnis. Die Unterscheidung zwischen ``hergeleitet'' und ``konstruiert'' ist entscheidend f\"ur die Bewertung des theoretischen Status des Modells. Die Drei-Faktoren-Struktur (Gatekeeper $\times$ S\"attigung $\times$ Projektion) ist physikalisch motiviert und minimal-beschr\"ankt, aber alternative Formen, die dieselben Bedingungen erf\"ullen, k\"onnen existieren.

Die deduktive Kette des CFM ist:
\begin{itemize}
\item \textbf{Eindeutig hergeleitet:} Die Notwendigkeit eines Vektorsektors, sein zeitartiger Charakter, die kinetische Struktur $F_{\mu\nu}F^{\mu\nu}$, die Einheitsbedingung.
\item \textbf{Stark beschr\"ankt:} Die Kopplungsfunktion $\mathcal{F}$ (6 Bedingungen, spezifische Funktionsform vorgeschlagen, aber nicht eindeutig).
\item \textbf{Aus Daten:} $K_B$, $\gamma$, $V_0$, $\phi_0$.
\end{itemize}

Dies ist vergleichbar mit dem Status des Starobinsky-Modells: Der $R^2$-Term ist die einfachste geister-freie Erweiterung der ART, aber $R^2 + \epsilon R^3$ kann auf rein theoretischen Gr\"unden nicht ausgeschlossen werden. Die Auswahl wird durch Occams Rasiermesser und empirische Anpassung getroffen.

\subsection{Offene theoretische Fragen}

\begin{enumerate}
\item \textbf{RAR-Interpolationsfunktion aus Feldgleichungen:} Abschnitt~\ref{sec:spherical} etabliert das quasi-statische Framework, leitet die modifizierte Poisson-Gleichung~\eqref{eq:poisson_Xi} ab und identifiziert die Chameleon-gesteuerte nichtlineare R\"uckkopplungsschleife~\eqref{eq:feedback_loop} als den Mechanismus, der MOND-\"ahnliche Dynamik erzeugt (Abs.~\ref{sec:nonlinear_origin}). Jedoch ergibt die linearisierte Reduktion nur eine konstante Reskalierung $g_A \propto g_N$, nicht den $\sqrt{g_N\,a_0}$-Attraktor; die $\sqrt{\phantom{x}}$-Skalierung muss aus dem vollst\"andigen nichtlinearen System entstehen. Das in Abs.~\ref{sec:numerical_scheme} skizzierte numerische BVP wird bestimmen, ob das gekoppelte System den erwarteten MOND-Attraktor zul\"asst und das emergierende $\mu(x)$ extrahieren.

\item \textbf{Rigorose $2\pi$-Ableitung:} Der Faktor $2\pi$ in $a_0 = cH_0/(2\pi)$ wird identifiziert als aus der Fourier-Beziehung zwischen Zeit-Dom\"anen-Skalardynamik und R\"aum-Dom\"anen-Gravitationspotenzial stammend, ist aber noch nicht aus den gekoppelten Feldgleichungen hergeleitet. Diese Berechnung ist eng verwandt mit Punkt~1.

\item \textbf{Eindeutigkeit von $\mathcal{B}(\phi)$:} Die $\mathrm{sech}^2$-Form ist motiviert durch P\"oschl-Teller-Konsistenz, aber nicht als eindeutig bewiesen. Ein Ausschlussargument (analog zur $R + \gamma R^2$-Eindeutigkeit in Paper~III) ist w\"unschenswert.

\item \textbf{Vektorst\"orungstheorie:} Der Beitrag von $A_\mu$ zum CMB-Leistungsspektrum und Materie-Leistungsspektrum ist nicht berechnet worden. Da $\rho_A = 0$ auf FLRW (Abs.~\ref{sec:rho_A}), beeinflusst der Vektor-Sektor nicht die Hintergrund-Kosmologie, aber seine St\"orungen k\"onnen das CMB auf dem Prozent-Niveau modifizieren. Dies zu berechnen ist essentiell f\"ur volle Planck-Level-Konsistenz.

\item \textbf{St\"orungs-Stabilit\"at und Wohlgestelltheit:} Geisterfreiheit (Bedingung~5) ist notwendig aber nicht hinreichend f\"ur eine wohlgestellte Theorie. Eine vollst\"andige Stabilitatanalyse erfordert: (i)~die quadratische Wirkung f\"ur Skalar- und Vektorst\"orungen um FLRW, (ii)~die Dispersionsrelationen f\"ur alle propagierenden Modi, (iii)~Positivit\"at von $c_s^2$ (keine Gradienten-Instabilit\"aten), (iv)~Subluminalit\"at oder gesteuerte Charakteristiken (eine bekannte Subtilit\"at in Einstein-Aether-Theorien \cite{JacobsonMattingly2001}), und (v)~Abwesenheit starker Kopplung im tiefen-MOND-Regime. Wir haben $\alpha_T = 0$ f\"ur Tensor-Modi etabliert (Abs.~\ref{sec:alpha_T}), aber die Skalar- und Vektor-St\"orungssektoren bleiben f\"ur die spezifische Kopplung~\eqref{eq:coupling} zu analysieren.

\item \textbf{Quantitative Baryogenese:} Paper~I argumentiert qualitativ, dass die Baryon-Asymmetrie $\eta \sim 6 \times 10^{-10}$ aus MEPP-optimaler Annihilation folgt. Eine quantitative Herleitung, die zeigt, dass MEPP diesen spezifischen Wert ausw\"ahlt, w\"urde das Argument von ``konsistent mit'' zu ``vorhersagt durch'' erh\"ohen.
\end{enumerate}


% ===================================================================
% 10. CONCLUSION
% ===================================================================
\section{Schlussfolgerung}
\label{sec:conclusion}

Wir haben die vollst\"andige Curvature Feedback Model-Wirkung $S_{\mathrm{CFM}} = S_{\mathrm{T1}} + S_{\mathrm{T2}}$ pr\"asentiert, unter Einbeziehung sowohl des Skalarsektors (Dunkle Energie, kosmologische Dunkle-Materie-Effekte) als auch des Vektorsektors (galaktische MOND-Dynamik). Die Schl\"usselergebnisse sind:

\begin{enumerate}
\item Der Vektor-Sektor ist \textit{hergeleitet} aus thermodynamischen Prinzipien: MEPP erfordert stabile dissipative Strukturen (Galaxien), die der Skalarsektor allein nicht bereitstellen kann. Seine Feldgleichungen und Energie-Impuls-Tensor sind explizit konstruiert (Abs.~\ref{sec:field_eqs}).

\item Die Kopplungsfunktion $\mathcal{F} = |T|/\rho_{\mathrm{crit}} \cdot \mathrm{sech}^2(\phi/\phi_0) \cdot A_\mu\partial^\mu\phi$ erf\"ullt alle sechs Bedingungen (BBN, MOND-Limes, Abschirmung, $a_0 \sim cH_0$, Geisterfreiheit, $\alpha_T = 0$).

\item Die MOND-Skala $a_0 = cH_0/(2\pi)$ ist eine Gr\"o\ss{}enordnungsvorhersage (innerhalb von $\sim 15\%$ des beobachteten Wertes), nicht ein angepasster Parameter. Galaxien ``wissen'' die Expansionsrate, weil das Vektorfeld an das Skalarfeld gekoppelt ist, das die Expansion treibt. Eine rigorose Herleitung des Faktors $2\pi$ aus den Feldgleichungen bleibt eine offene Aufgabe.

\item Drei rigorose Konsistenzbeweise sind etabliert: (i)~$\alpha_T = 0$ exakt, bewahrt $c_T = c$ (Abs.~\ref{sec:alpha_T}); (ii)~parasitische Abschirmung unterdr\"uckt die Vektorquelle im Sonnensystem um einen Faktor $\sim 10^{-3 \times 10^9}$ (Abs.~\ref{sec:parasitic}); (iii)~$\rho_A = 0$ auf FLRW, sichernd dass der Vektorsektor kosmologische Hintergrundanpassungen nicht beeinflusst (Abs.~\ref{sec:rho_A}).

\item Die statische Kugelgrenze ist explizit hergeleitet (Abs.~\ref{sec:spherical}): Die quasi-statische Zerlegung $\phi = \bar{\phi}(t) + \varphi(r)$ enth\"ullt, dass der kosmologische Drift $\dot{\bar{\phi}} \sim H_0\phi_0$ die physikalische Br\"ucke ist, die Expansion mit galaktischer Dynamik verbindet. Die \textit{linearisierte} Reduktion liefert $g_A \propto g_N$ (konstante Reskalierung); die tiefe-MOND-Skalierung $g \sim \sqrt{g_N\,a_0}$ wird aus der nichtlinearen Cham\"aleon-R\"uckkopplung erwartet (Abs.~\ref{sec:nonlinear_origin}), mit der genauen Interpolationsfunktion $\mu(x)$ auf numerische Randwertproblem-L\"osungen verschoben.

\item Der Konvergenzfaktor $4/3$ aus zwei unabh\"angigen Herleitungen (St\"orungstheorie und Phasenraum-Geometrie) bietet strukturelles Beweismaterial f\"ur die verschachtelte Hierarchie.

\item Ein Null-Dunkle-Materie-Parameter-Test gegen SPARC (175 Galaxien) wird vorgeschlagen, mit $a_0$ durch Kosmologie festgelegt.
\end{enumerate}

Das ontologische Bild ist: zwei Axiome (Null-Raum-Fluktuationen + thermodynamische Optimierung) $\to$ zwei Freiheitsgrade (Skalar + Vektor) $\to$ drei Ph\"anomene (Dunkle Energie, kosmologische Dunkle-Materie-Effekte, MOND-kompatible galaktische Dynamik) $\to$ eine Terminierungsbedingung (Standardphysik unterhalb galaktischer Skalen). Kein $\Lambda$, keine Dunkle-Materie-Teilchen -- nur Geometrie. Die vollst\"andige Validierung dieses Bildes erfordert numerische Best\"atigung, dass die nichtlineare Cham\"aleon-R\"uckkopplung die MOND-Interpolationsfunktion erzeugt, sowie einen parameterfreien Fit an die SPARC-Rotationskurvendatenbank.


% ===================================================================
% ACKNOWLEDGMENTS
% ===================================================================
\begin{acknowledgments}
Diese Arbeit wurde als unabh\"angige Forschung ohne institutionelle F\"orderung durchgef\"uhrt. Der Autor dankt den Open-Source-Gemeinschaften hinter \texttt{hi\_class} \cite{Zumalacarregui2017}, \texttt{CLASS} \cite{Blas2011}, \texttt{emcee} \cite{ForemanMackey2013}, \texttt{NumPy} \cite{Harris2020}, \texttt{SciPy} \cite{Virtanen2020} und \texttt{Matplotlib} \cite{Hunter2007}. Die SPARC-Datenbasis \cite{Lelli2016} bietet eine unsch\"atzbare \"offentliche Ressource zum Testen modifizierter Gravitationstheorien.
\end{acknowledgments}

% ===================================================================
% BIBLIOGRAPHY
% ===================================================================

\begin{thebibliography}{99}

\bibitem{Geiger2026}
Geiger, L.\ (2026).
Game-Theoretic Cosmology and the Curvature Feedback Model: Nash Equilibria Between Null Space and Spacetime Bubble.
Companion paper (Paper~I). \url{https://github.com/lukisch/cfm-cosmology}.

\bibitem{Geiger2026b}
Geiger, L.\ (2026).
Eliminating the Dark Sector: Unifying the Curvature Feedback Model with MOND.
Companion paper (Paper~II).

\bibitem{Geiger2026c}
Geiger, L.\ (2026).
Microscopic Foundations of the Curvature Feedback Model: From Quantum Geometry to Macroscopic Saturation.
Companion paper (Paper~III).

\bibitem{Milgrom1983}
Milgrom, M.\ (1983).
A modification of the Newtonian dynamics as a possible alternative to the hidden mass hypothesis.
\textit{The Astrophysical Journal}, 270, 365--370.

\bibitem{McGaugh2016}
McGaugh, S.\,S., Lelli, F.\ \& Schombert, J.\,M.\ (2016).
Radial Acceleration Relation in Rotationally Supported Galaxies.
\textit{Physical Review Letters}, 117(20), 201101.
DOI: 10.1103/PhysRevLett.117.201101.

\bibitem{McGaugh2012}
McGaugh, S.\,S.\ (2012).
The Baryonic Tully-Fisher Relation of Gas-Rich Galaxies as a Test of $\Lambda$CDM and MOND.
\textit{The Astronomical Journal}, 143(2), 40.
DOI: 10.1088/0004-6256/143/2/40.

\bibitem{Skordis2021}
Skordis, C.\ \& Z{\l}o\'snik, T.\ (2021).
New Relativistic Theory for Modified Newtonian Dynamics.
\textit{Physical Review Letters}, 127(16), 161302.

\bibitem{Bekenstein2004}
Bekenstein, J.\,D.\ (2004).
Relativistic gravitation theory for the modified Newtonian dynamics paradigm.
\textit{Physical Review D}, 70(8), 083509.
DOI: 10.1103/PhysRevD.70.083509.

\bibitem{Dewar2003}
Dewar, R.\ (2003).
Information theory explanation of the fluctuation theorem, maximum entropy production and self-organized criticality in non-equilibrium stationary states.
\textit{Journal of Physics A}, 36(3), 631--641.
DOI: 10.1088/0305-4470/36/3/303.

\bibitem{Martyushev2006}
Martyushev, L.\,M.\ \& Seleznev, V.\,D.\ (2006).
Maximum entropy production principle in physics, chemistry and biology.
\textit{Physics Reports}, 426(1), 1--45.
DOI: 10.1016/j.physrep.2005.12.001.

\bibitem{Prigogine1977}
Prigogine, I.\ (1977).
\textit{Self-Organization in Nonequilibrium Systems}. Wiley.

\bibitem{Jacobson1995}
Jacobson, T.\ (1995).
Thermodynamics of Spacetime: The Einstein Equation of State.
\textit{Physical Review Letters}, 75(7), 1260--1263.
DOI: 10.1103/PhysRevLett.75.1260.

\bibitem{Starobinsky1980}
Starobinsky, A.\,A.\ (1980).
A new type of isotropic cosmological models without singularity.
\textit{Physics Letters B}, 91(1), 99--102.

\bibitem{Abbott2017}
Abbott, B.\,P.\ et al.\ (LIGO/Virgo \& Fermi GBM) (2017).
Gravitational Waves and Gamma-Rays from a Binary Neutron Star Merger: GW170817 and GRB~170817A.
\textit{The Astrophysical Journal Letters}, 848(2), L13.

\bibitem{Bertotti2003}
Bertotti, B., Iess, L.\ \& Tortora, P.\ (2003).
A test of general relativity using radio links with the Cassini spacecraft.
\textit{Nature}, 425, 374--376.
DOI: 10.1038/nature01997.

\bibitem{Lelli2016}
Lelli, F., McGaugh, S.\,S.\ \& Schombert, J.\,M.\ (2016).
SPARC: Mass Models for 175 Disk Galaxies with Spitzer Photometry and Accurate Rotation Curves.
\textit{The Astronomical Journal}, 152(6), 157.
DOI: 10.3847/0004-6256/152/6/157.

\bibitem{Woodard2015}
Woodard, R.\,P.\ (2015).
Ostrogradsky's theorem on Hamiltonian instability.
\textit{Scholarpedia}, 10(8), 32243.
DOI: 10.4249/scholarpedia.32243.

\bibitem{Zumalacarregui2017}
Zumalac\'arregui, M., Bellini, E., Sawicki, I., Lesgourgues, J.\ \& Ferreira, P.\,G.\ (2017).
hi\_class: Horndeski in the Cosmic Linear Anisotropy Solving System.
\textit{Journal of Cosmology and Astroparticle Physics}, 2017(08), 019.
DOI: 10.1088/1475-7516/2017/08/019.

\bibitem{Blas2011}
Blas, D., Lesgourgues, J.\ \& Tram, T.\ (2011).
The Cosmic Linear Anisotropy Solving System (CLASS). Part~II: Approximation schemes.
\textit{Journal of Cosmology and Astroparticle Physics}, 2011(07), 034.
DOI: 10.1088/1475-7516/2011/07/034.

\bibitem{ForemanMackey2013}
Foreman-Mackey, D., Hogg, D.\,W., Lang, D.\ \& Goodman, J.\ (2013).
emcee: The MCMC Hammer.
\textit{Publications of the Astronomical Society of the Pacific}, 125(925), 306--312.
DOI: 10.1086/670067.

\bibitem{Harris2020}
Harris, C.\,R.\ et al.\ (2020).
Array programming with NumPy.
\textit{Nature}, 585, 357--362.
DOI: 10.1038/s41586-020-2649-2.

\bibitem{Virtanen2020}
Virtanen, P.\ et al.\ (2020).
SciPy 1.0: Fundamental Algorithms for Scientific Computing in Python.
\textit{Nature Methods}, 17, 261--272.
DOI: 10.1038/s41592-019-0686-2.

\bibitem{Hunter2007}
Hunter, J.\,D.\ (2007).
Matplotlib: A 2D Graphics Environment.
\textit{Computing in Science \& Engineering}, 9(3), 90--95.
DOI: 10.1109/MCSE.2007.55.

\bibitem{JacobsonMattingly2001}
Jacobson, T.\ \& Mattingly, D.\ (2001).
Gravity with a dynamical preferred frame.
\textit{Physical Review D}, 64(2), 024028.
DOI: 10.1103/PhysRevD.64.024028.

\bibitem{HuSawicki2007}
Hu, W.\ \& Sawicki, I.\ (2007).
Models of $f(R)$ Cosmic Acceleration that Evade Solar-System Tests.
\textit{Physical Review D}, 76(6), 064004.
DOI: 10.1103/PhysRevD.76.064004.

\bibitem{PlanckMG2016}
Planck Collaboration (2016).
Planck 2015 results. XIV. Dark energy and modified gravity.
\textit{Astronomy \& Astrophysics}, 594, A14.
DOI: 10.1051/0004-6361/201525814.

\bibitem{Verlinde2017}
Verlinde, E.\ (2017).
Emergent Gravity and the Dark Universe.
\textit{SciPost Physics}, 2(3), 016.

\bibitem{DESI2025}
DESI Collaboration (2025).
DESI DR2 Results II: Measurements of Baryon Acoustic Oscillations and Cosmological Constraints.
\textit{arXiv:2503.14738}.

\end{thebibliography}

\end{document}

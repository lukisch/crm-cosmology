\documentclass[aps,prd,twocolumn,superscriptaddress,nofootinbib]{revtex4-2}
\usepackage{amsmath}
\usepackage{amssymb}
\usepackage{amsthm}
\usepackage{booktabs}
\usepackage{xcolor}
\usepackage{longtable}
\usepackage{multirow}
\usepackage{array}
\usepackage{float}
\usepackage{graphicx}
\graphicspath{{../figures/}}

\newtheorem{definition}{Definition}
\newtheorem{proposition}{Proposition}

\usepackage{hyperref}
\hypersetup{
    pdftitle={Spieltheoretische Kosmologie und das Curvature Relaxation Model},
    pdfauthor={Lukas Geiger},
    colorlinks=true,
    linkcolor=black,
    urlcolor=blue,
    citecolor=black
}

\begin{document}

% ===================================================================
% TITELSEITE
% ===================================================================

\title{Spieltheoretische Kosmologie und das Kr\"ummungs-R\"uckgabemodell: Nash-Gleichgewichte zwischen Nullraum und Raumzeitblase als Erkl\"arungsrahmen f\"ur die beschleunigte Expansion}

\author{Lukas Geiger}
\email{Correspondence: Bernau, Germany}
\affiliation{Independent Researcher, Bernau im Schwarzwald, Germany}

\date{\today}


\begin{abstract}
Wir entwickeln einen spieltheoretischen Rahmen f\"ur die Kosmologie, in dem die Raumzeitentwicklung als Nash-Gleichgewicht zwischen einem metastabilen Quantenvakuum (Nullraum) und einer Raumzeitblase modelliert wird. Das zentrale Ergebnis ist das \textit{Curvature Relaxation Model} (KRM), das die beschleunigte Expansion nicht durch Dunkle Energie erkl\"art, sondern durch ein nachlassendes Kr\"ummungs-R\"uckgabepotential $\Omega_\Phi(a) = \Phi_0 \cdot \tanh(k(a - a_{\mathrm{trans}}))$ -- ein geometrisches ``Ged\"achtnis'' der Urknall-Energiekonzentration. Getestet gegen 1.590 Pantheon+ Typ-Ia-Supernovae \cite{Scolnic2022} mit diagonalen und vollen Kovarianzfehlern liefert das KRM $\Delta\chi^2 = -12{,}2$ ($-11{,}2$) und $\Delta\mathrm{AIC} = -8{,}2$ ($-7{,}2$) gegen\"uber $\Lambda$CDM; Kreuzvalidierung best\"atigt die bessere Generalisierung. Die MCMC-Analyse ergibt $\Omega_m = 0{,}368 \pm 0{,}024$; vier alternative S\"attigungsfunktionen liefern vergleichbare Ergebnisse und belegen die Robustheit. Das Modell sagt $w(z) < -1$ mit $|\Delta w| \approx 0{,}4$ und einen sp\"ateren Beschleunigungs\"ubergang ($z_{\mathrm{acc}} = 0{,}52$ vs.\ $0{,}84$) voraus, beides testbar mit Euclid und Roman. Die sp\"atere Beschleunigung impliziert eine verl\"angerte materiedominierte Wachstumsphase, die die JWST ``Universe Breakers'' bei $z > 7$ \cite{Labbe2023} und die El-Gordo-Anomalie \cite{Asencio2023} nat\"urlich erkl\"art. Phantom-Stabilit\"at ist durch S\"attigung garantiert ($\Omega_\Phi \to \Phi_0$, de-Sitter-Endzustand, kein Big Rip). Die spieltheoretischen Axiome besitzen eine exakte thermodynamische \"Aquivalenz in der Jacobson-Tradition \cite{Jacobson1995}. Der Analysecode ist \"offentlich verf\"ugbar.\footnote{\url{https://github.com/lukisch/cfm-cosmology}}

\end{abstract}

\keywords{Spieltheorie, Nash-Gleichgewicht, Kosmologie, Dunkle Energie, Kr\"ummungs-R\"uckgabepotential, Curvature Relaxation Model, Friedmann-Gleichung, Finsler-Gravitation, beschleunigte Expansion, Zustandsgleichung}

\maketitle

\thanks{KI-Werkzeuge (Claude, Anthropic; Gemini, Google DeepMind) wurden fuer mathematische Formalisierung, Code-Entwicklung und Texterstellung verwendet. Alle physikalischen Hypothesen, wissenschaftliche Interpretation und Verantwortung fuer die Inhalte liegen vollstaendig bei dem Autor. Der Analysecode ist verfuegbar unter \url{https://github.com/lukisch/cfm-cosmology}.}

\thanks{Arbeit~I in der KRM-Serie.}


% ===================================================================
% 1. EINLEITUNG
% ===================================================================
\section{Einleitung}
\label{sec:einleitung}

Die Entdeckung der beschleunigten Expansion des Universums durch die Beobachtung entfernter Typ-Ia-Supernovae im Jahr 1998 durch die Teams um Perlmutter \cite{Perlmutter1999} sowie Riess und Schmidt \cite{Riess1998} markiert einen Wendepunkt der modernen Kosmologie. F\"ur diese Entdeckung wurde 2011 der Nobelpreis f\"ur Physik verliehen. Das Standardmodell der Kosmologie, $\Lambda$CDM, erkl\"art die Beschleunigung durch eine kosmologische Konstante~$\Lambda$, die etwa 68\,\% der Energiedichte des Universums ausmacht \cite{Planck2020}. Trotz seiner empirischen Erfolge steht $\Lambda$CDM vor tiefgreifenden konzeptuellen Problemen:

\begin{enumerate}
\item \textbf{Das Kosmologische-Konstante-Problem:} Die beobachtete Vakuumenergiedichte ist um $\sim$60--120 Gr\"o\ss{}enordnungen kleiner als theoretische Vorhersagen der Quantenfeldtheorie \cite{Weinberg1989}.
\item \textbf{Das Koinzidenz-Problem:} Warum sind $\Omega_m$ und $\Omega_\Lambda$ gerade in der heutigen Epoche von vergleichbarer Gr\"o\ss{}enordnung?
\item \textbf{Die $H_0$-Spannung:} Die lokale Messung des Hubble-Parameters ($H_0 \approx 73$\,km/s/Mpc) weicht signifikant von der aus CMB-Daten abgeleiteten ($H_0 \approx 67{,}4$\,km/s/Mpc) ab \cite{Planck2020, Riess2022}.
\end{enumerate}

J\"ungste Resultate des \textit{Dark Energy Spectroscopic Instrument} (DESI) verst\"arken die Zweifel an einer strikt konstanten Dunklen Energie: Die Analyse baryonischer akustischer Oszillationen in Kombination mit CMB- und Supernova-Daten zeigt eine Pr\"aferenz von 2,5--3,9$\sigma$ f\"ur ein Modell mit zeitabh\"angigem Zustandsgleichungsparameter $w(z)$ gegen\"uber $\Lambda$CDM \cite{DESI2024}.

Parallel dazu zeigen theoretische Arbeiten, dass die Beschleunigung auch ohne Dunkle Energie erkl\"arbar sein k\"onnte: Pfeifer et al.\ \cite{Pfeifer2025} demonstrieren im Rahmen der Finsler-Gravitation, dass eine verallgemeinerte Raumzeitgeometrie nat\"urlicherweise eine exponentielle Expansion im Vakuum erzeugt. Trivedi und Venkatasubramanian \cite{Trivedi2025} zeigen in ihrer \textit{Cosmological Teleodynamics}, dass das Universum wie ein ``riesiges Potentialspiel'' operiert und sich einem kontinuierlichen Nash-Gleichgewicht ann\"ahert, wobei die kosmische Beschleunigung als emergenter Effekt dynamischen Ged\"achtnisses in einem selbstgravitierenden Medium erscheint.

Die vorliegende Arbeit verkn\"upft diese Entwicklungen mit einem eigenst\"andigen Ansatz: Ausgehend von einer spieltheoretischen Modellierung der Wechselwirkung zwischen Quantenvakuum und Raumzeit wird das \textit{Curvature Relaxation Model} (KRM) entwickelt, das die beschleunigte Expansion als ``nachlassende Bremse'' statt als ``neuen Antrieb'' interpretiert.

% ===================================================================
% 2. SPIELTHEORETISCHER RAHMEN
% ===================================================================
\section{Spieltheoretischer Rahmen: Nullraum und Raumzeitblase}
\label{sec:spieltheorie}

\subsection{Grundannahmen}
\label{subsec:grundannahmen}

Der hier vorgeschlagene Rahmen geht von folgenden Annahmen aus:

\begin{enumerate}
\item Es existiert ein metastabiler Quantenvakuumzustand (im Folgenden: \textit{Nullraum}), der durch Quantenfluktuationen charakterisiert ist.
\item Eine au\ss{}ergew\"ohnlich gro\ss{}e Fluktuation entnimmt dem Nullraum einmalig eine Energiemenge~$E_0$, die einen Konzentrationsgradienten erzeugt.
\item Zur Einkapselung und kontrollierten Neutralisation dieses Gradienten entsteht Raumzeit als dynamische Struktur -- die \textit{Raumzeitblase} (Tochtersystem).
\item Zwischen Nullraum (Muttersystem) und Raumzeitblase besteht ein spieltheoretisches Gleichgewicht.
\end{enumerate}

Diese Annahmen werden im Folgenden in einen formalen Rahmen \"uberf\"uhrt.

\subsection{Akteure und Ziele}
\label{subsec:akteure}

Das System wird als Zweipersonen-Potentialspiel modelliert:

\begin{description}
\item[\textbf{Nullraum (Muttersystem):}] Prim\"arziel ist der Selbstschutz -- die Erhaltung seiner strukturellen Integrit\"at. Er reguliert die Kopplungsst\"arke zur Raumzeitblase \"uber effektive Randbedingungen (``Gatekeeping''), erzwingt langsame Energieabfuhr (D\"ampfung) und bildet Pufferzonen (horizontartige H\"ullen).
\item[\textbf{Raumzeitblase (Tochtersystem):}] Prim\"arziel ist die kontrollierte R\"uckkehr in den Nullzustand bei gleichzeitigem Schutz des Muttersystems. Die Strategien umfassen kaskadierten Gradientenabbau, adiabatische R\"uckf\"uhrung und Entropiemanagement.
\end{description}

\subsection{Mathematische Formulierung als Potentialspiel}
\label{subsec:potentialspiel}

Die globale Zielfunktion des Systems lautet:
\begin{equation}
\Phi = \alpha \cdot S_{\mathrm{Mutter}} + \beta \cdot R_{\mathrm{Tochter}} - \gamma \cdot G
\label{eq:potential}
\end{equation}
wobei $S_{\mathrm{Mutter}}$ die strukturelle Integrit\"at des Nullraums, $R_{\mathrm{Tochter}}$ den R\"uckkehrfortschritt und $G$ den verbleibenden Konzentrationsgradienten beschreibt; $\alpha, \beta, \gamma > 0$.

\begin{definition}[Nash-Gleichgewicht des kosmologischen Spiels]
Ein Strategiepaar $(s_M^*, s_T^*)$ von Nullraum und Raumzeitblase bildet ein Nash-Gleichgewicht, wenn gilt:
\begin{align}
\Phi(s_M^*, s_T^*) &\geq \Phi(s_M, s_T^*) \quad \forall\, s_M \\
\Phi(s_M^*, s_T^*) &\geq \Phi(s_M^*, s_T) \quad \forall\, s_T
\end{align}
Keine Seite kann durch einseitige Abweichung von ihrer Strategie das Gesamtpotential verbessern, ohne die Stabilit\"at des Systems zu gef\"ahrden.
\end{definition}

Der zentrale \textbf{Zielkonflikt} besteht darin, dass eine zu schnelle Reduktion von~$G$ (sofortige R\"uckkehr) $S_{\mathrm{Mutter}}$ gef\"ahrdet, w\"ahrend eine zu langsame Reduktion die Entropie und die Kosten innerhalb der Blase erh\"oht. Das Nash-Gleichgewicht erzwingt daher eine kontrollierte, zeitlich gestreckte Neutralisation.

\subsection{Emergente Gesetze aus dem Gleichgewicht}
\label{subsec:emergente_gesetze}

Aus der spieltheoretischen Gleichgewichtsbedingung emergieren physikalische Gesetzm\"a\ss{}igkeiten:

\begin{enumerate}
\item \textbf{Energieerhaltung:} Konservative Feldgleichungen entstehen als notwendige Bedingung f\"ur stabilen Gradientenabbau.
\item \textbf{Kausalstruktur:} Die H\"ullenbildung des Nullraums erzwingt eine maximale Ausbreitungsgeschwindigkeit f\"ur Informationen und Energie.
\item \textbf{Entropischer Zeitpfeil:} Die ``Zeit'' innerhalb der Blase ist die Ordnung, entlang der der Konzentrationsgradient nivelliert wird.
\item \textbf{Flusslimitierung:} Maximalfl\"usse \"uber die H\"ulle skalieren sublinear mit dem internen \"Uberschuss und verhindern Runaway-Prozesse.
\item \textbf{Asymptotische R\"uckkehr:} Der Restgradient $G \to 0$ n\"ahert sich nur asymptotisch; es gibt kein katastrophales Finale.
\end{enumerate}

Die letzte Eigenschaft ist f\"ur die Kosmologie besonders bedeutsam: Sie impliziert, dass die Expansion des Universums sich nie umkehrt, sondern asymptotisch abl\"auft -- konsistent mit den beobachteten Daten.

\subsection{Thermodynamische \"Aquivalenz}
\label{subsec:thermodynamik}

Der spieltheoretische Rahmen der vorangegangenen Abschnitte besitzt eine pr\"azise \"Aquivalenz in der Sprache der Thermodynamik und Variationsphysik. Diese Verbindung spiegelt einen tiefen mathematischen Isomorphismus zwischen Spieltheorie und statistischer Mechanik wider \cite{Wolpert2006}:

\begin{itemize}
\item \textbf{Nash-Gleichgewicht} $\leftrightarrow$ \textbf{Thermodynamisches Gleichgewicht:} Der Zustand, von dem aus kein Teilsystem seine Zielfunktion einseitig verbessern kann, entspricht dem Zustand maximaler Entropie unter Zwangsbedingungen.
\item \textbf{Potentialfunktion~$\Phi$} $\leftrightarrow$ \textbf{Entropiefunktional:} Die globale Gr\"o\ss{}e, die durch die Gleichgewichtsstrategien extremiert wird, ist die Entropie des gekoppelten Systems.
\item \textbf{Optimale Strategiewahl} $\leftrightarrow$ \textbf{Prinzip der kleinsten Wirkung:} Die Strategie jedes Akteurs entspricht Feldgleichungen aus $\delta S = 0$.
\item \textbf{Gradientenabbaurate} $\leftrightarrow$ \textbf{Prinzip maximaler Entropieproduktion (MEPP):} Unter allen mit den Randbedingungen vertr\"aglichen Pfaden entwickelt sich das System entlang desjenigen, der die Entropieproduktion maximiert \cite{Dewar2003}.
\end{itemize}

Diese \"Aquivalenz gr\"undet auf einem fundamentalen Resultat: Jacobson \cite{Jacobson1995} zeigte, dass die Einsteinschen Feldgleichungen aus der Proportionalit\"at von Entropie und Horizontfl\"ache zusammen mit der Clausius-Relation $\delta Q = T\,dS$ abgeleitet werden k\"onnen, was die Allgemeine Relativit\"atstheorie selbst als thermodynamische Zustandsgleichung etabliert. Wenn die Standard-Gravitation aus der Thermodynamik folgt, ist es nat\"urlich zu erwarten, dass \textit{modifizierte} Gravitation -- hier der Kr\"ummungs-R\"uckgabemechanismus -- aus modifizierten thermodynamischen Randbedingungen folgt, n\"amlich der Zwangsbedingung, dass das Muttervakuum (Nullraum) strukturell stabil bleibt, w\"ahrend die Entropieproduktion im Tochtersystem maximiert wird.

Der S\"attigungsmechanismus aus Abschnitt~\ref{subsec:tanh_herleitung} ist damit direkt als Ann\"aherung an das thermodynamische Gleichgewicht interpretierbar: Die $\tanh$-Dynamik beschreibt die Entropieproduktionsrate eines Systems mit endlicher Kapazit\"at, das sich unter der gemeinsamen Zwangsbedingung des Zweiten Hauptsatzes und der Muttersystem-Stabilit\"at entwickelt.

% ===================================================================
% 3. DAS CURVATURE RELAXATION MODEL
% ===================================================================
\section{Das Curvature Relaxation Model (KRM)}
\label{sec:krm}

\subsection{Physikalische Motivation}
\label{subsec:motivation}

Im spieltheoretischen Rahmen der vorigen Sektion wird die Raumzeit als ``Bremsmechanismus'' interpretiert, der die sofortige R\"uckkehr der Energie in den Nullraum verhindert. Die zentrale physikalische Einsicht lautet:

\begin{quote}
\textit{Die beobachtete beschleunigte Expansion ist nicht durch eine neue Energieform verursacht, sondern durch ein nachlassendes Kr\"ummungs-R\"uckgabepotential -- eine Art geometrisches ``Ged\"achtnis'' der anf\"anglichen Energiekonzentration beim Urknall.}
\end{quote}

Die Analogie ist die einer gespannten Feder: Anf\"anglich herrscht maximale Spannung (hohe Kr\"ummung) mit starker R\"uckstellkraft. Mit der Zeit l\"asst die Spannung nach, die R\"uckstellkraft nimmt ab, und die Expansion ``beschleunigt'' relativ zur gebremsten Fr\"uhphase -- wie ein Auto, bei dem die Handbremse langsam gel\"ost wird.

\subsection{Modifizierte Friedmann-Gleichung}
\label{subsec:friedmann}

Die Standard-Friedmann-Gleichung im $\Lambda$CDM-Modell lautet:
\begin{equation}
H^2(a) = H_0^2 \left[\Omega_m\,a^{-3} + \Omega_\Lambda\right]
\label{eq:friedmann_lcdm}
\end{equation}

Im KRM wird die kosmologische Konstante durch ein zeitabh\"angiges Kr\"ummungs-R\"uckgabepotential ersetzt:
\begin{equation}
H^2(a) = H_0^2 \left[\Omega_m\,a^{-3} + \Omega_\Phi(a)\right]
\label{eq:friedmann_cfm}
\end{equation}

Das Kr\"ummungs-R\"uckgabepotential ist definiert als:
\begin{equation}
\Omega_\Phi(a) = \Phi_0 \cdot \frac{\tanh\!\big(k\cdot(a - a_{\mathrm{trans}})\big) + s}{1 + s}
\label{eq:potential_cfm}
\end{equation}
wobei $s = \tanh(k \cdot a_{\mathrm{trans}})$ ein Normierungsshift ist, der $\Omega_\Phi(0) = 0$ sicherstellt, und:
\begin{itemize}
\item $a$ der Skalenfaktor ist ($a=1$ heute, $a \to 0$ beim Urknall),
\item $\Phi_0$ die Amplitude (aus der Flachheitsbedingung $\Omega_m + \Omega_\Phi(1) = 1$ abgeleitet),
\item $k$ die \"Ubergangssch\"arfe,
\item $a_{\mathrm{trans}}$ der \"Ubergangsskalenfaktor.
\end{itemize}
Die konkreten Parameterwerte werden in Abschnitt~\ref{sec:numerik} aus dem Fit gegen den Pantheon+-Datensatz bestimmt.

\subsection{Dynamischer S\"attigungsmechanismus (\textit{Dynamic Saturation Mechanism})}
\label{subsec:tanh_herleitung}

Die $\tanh$-Parametrisierung ist keine \textit{ad hoc} gew\"ahlte Fitfunktion, sondern entsteht als exakte L\"osung eines physikalisch motivierten \textbf{dynamischen S\"attigungsmechanismus}. Die zentrale Annahme lautet: Die Raumzeitblase besitzt eine endliche Aufnahmekapazit\"at f\"ur die Kr\"ummungsr\"uckgabe. Die R\"uckgaberate ist proportional zur verbleibenden Kapazit\"at:
\begin{equation}
\frac{d\Omega_\Phi}{da} = k \cdot \left[1 - \left(\frac{\Omega_\Phi}{\Phi_0}\right)^{\!2}\right]
\label{eq:saturation_ode}
\end{equation}
Diese Gleichung beschreibt einen klassischen S\"attigungsprozess der dynamischen Systemtheorie: Bei kleinem $\Omega_\Phi$ w\"achst das Potential nahezu linear (die ``Bremse'' l\"ost sich mit voller Rate), bei $\Omega_\Phi \to \Phi_0$ tritt S\"attigung ein (die maximale Kapazit\"at ist erreicht, die Bremse vollst\"andig gel\"ost). Die exakte L\"osung von Gl.~\eqref{eq:saturation_ode} ist:
\begin{equation}
\Omega_\Phi(a) = \Phi_0 \cdot \tanh\!\big(k\cdot(a - a_{\mathrm{trans}})\big)
\end{equation}
wobei $a_{\mathrm{trans}}$ die Integrationskonstante (\"Ubergangspunkt) darstellt. Der S\"attigungsmechanismus ist in der Physik ubiquit\"ar und tritt in formal identischer Form in zahlreichen Systemen auf:
\begin{itemize}
\item Ferromagnetismus: Spontane Magnetisierung $M(T) \sim \tanh(T_C/T)$
\item BCS-Supraleitung: Energiel\"ucke $\Delta(T) \sim \tanh(T_C/T)$
\item Solitonenphysik: Kink-L\"osung $\phi(x) = \phi_0 \tanh(kx)$
\item Nichtlineare Optik: S\"attigungsabsorption $\alpha(I) \propto 1/(1 + I/I_{\mathrm{sat}})$
\end{itemize}
Alle diese Systeme teilen die Eigenschaft eines geordneten \"Ubergangs von einem Zustand in einen anderen mit endlicher Kapazit\"at -- \textit{genau} das Verhalten, das der spieltheoretische Rahmen f\"ur das Nash-Gleichgewicht zwischen Nullraum und Raumzeitblase vorhersagt. Die $\tanh$-Form ist damit nicht postuliert, sondern aus dem zugrunde liegenden Mechanismus \textit{abgeleitet}.

Zur \"Uberpr\"ufung der Robustheit wurden vier verschiedene S\"attigungsfunktionen getestet (Abschnitt~\ref{subsec:funktionalformen}). Alle liefern $\Delta\chi^2 \approx -9$ bis $-12$ gegen\"uber $\Lambda$CDM -- die Daten ``sehen'' einen S\"attigungsprozess, unabh\"angig von der exakten mathematischen Formulierung.

\subsection{Physikalische Interpretation der Parameter}
\label{subsec:interpretation}

\textbf{Fr\"uhe Zeiten} ($a \to 0$, $z \to \infty$): $\Omega_\Phi \to 0$. Die ``Bremse'' wirkt voll -- die Expansion folgt der Materiedominanz wie in $\Lambda$CDM. Es gibt keine dunkle Komponente.

\textbf{\"Ubergangsepoche} ($a \approx a_{\mathrm{trans}}$, $z \approx 1{,}5$): $\Omega_\Phi$ steigt an. Die ``Bremse'' beginnt nachzulassen. Dies geschah vor etwa 10,3 Milliarden Jahren.

\textbf{Heute} ($a = 1$, $z = 0$): $\Omega_\Phi \to \Phi_0$. Der maximale Effekt ist erreicht; das Potential wirkt effektiv wie~$\Lambda$.

\subsection{Effektiver Zustandsgleichungsparameter}
\label{subsec:weff}

Der effektive Zustandsgleichungsparameter des Kr\"ummungs-R\"uckgabepotentials ist:
\begin{equation}
w_{\mathrm{eff}}(a) = -1 - \frac{1}{3}\,\frac{d\ln\Omega_\Phi}{d\ln a}
\label{eq:weff}
\end{equation}

Seine Zeitentwicklung ist in Tabelle~\ref{tab:weff} dargestellt.

\begin{table}[H]
\centering
\caption{Zeitentwicklung des effektiven Zustandsgleichungsparameters $w_{\mathrm{eff}}(z)$ im Vergleich $\Lambda$CDM vs.\ KRM. Die $1\sigma$-Unsicherheiten stammen aus der MCMC-Posterioranalyse (Abschnitt~\ref{subsec:mcmc}).}
\label{tab:weff}
\begin{tabular}{ccccc}
\toprule
$z$ & $w$ ($\Lambda$CDM) & $w$ (KRM) & $1\sigma$-Bereich & $\Delta w$ \\
\midrule
0,0 & $-1{,}000$ & $-1{,}355$ & $[-1{,}371;\;-1{,}339]$ & $\mathbf{-0{,}355}$ \\
0,3 & $-1{,}000$ & $-1{,}433$ & $[-1{,}645;\;-1{,}355]$ & $\mathbf{-0{,}433}$ \\
0,5 & $-1{,}000$ & $-1{,}450$ & $[-1{,}730;\;-1{,}358]$ & $\mathbf{-0{,}450}$ \\
0,8 & $-1{,}000$ & $-1{,}456$ & $[-1{,}759;\;-1{,}359]$ & $\mathbf{-0{,}456}$ \\
1,0 & $-1{,}000$ & $-1{,}454$ & $[-1{,}749;\;-1{,}359]$ & $\mathbf{-0{,}454}$ \\
1,5 & $-1{,}000$ & $-1{,}444$ & $[-1{,}696;\;-1{,}357]$ & $\mathbf{-0{,}444}$ \\
2,0 & $-1{,}000$ & $-1{,}432$ & $[-1{,}644;\;-1{,}355]$ & $\mathbf{-0{,}432}$ \\
\bottomrule
\end{tabular}
\end{table}

Die KRM-Parameter aus dem Pantheon+-Fit ergeben durchgehend $w < -1$ (Phantom-Bereich). Die MCMC-basierten $1\sigma$-Konfidenzintervalle zeigen, dass $w = -1$ f\"ur alle Rotverschiebungen ausgeschlossen ist. Dies unterscheidet sich qualitativ von $\Lambda$CDM ($w \equiv -1$) und ist eine eindeutige, falsifizierbare Vorhersage. Der Effekt ist \"uber den gesamten beobachtbaren Rotverschiebungsbereich pr\"asent ($|\Delta w| \approx 0{,}4$) und damit deutlich innerhalb der erwarteten Messgenauigkeit von Euclid ($\sigma_w \approx 0{,}02$).

% ===================================================================
% 4. NUMERISCHE TESTS
% ===================================================================
\section{Numerische Tests und Modellvergleich}
\label{sec:numerik}

\subsection{Flachheitsbedingung}
\label{subsec:flachheit}

Um die Zahl freier Parameter zu reduzieren und physikalische Konsistenz zu gew\"ahrleisten, wird die Flachheitsbedingung
\begin{equation}
\Omega_m + \Omega_\Phi(a{=}1) = 1
\label{eq:flatness}
\end{equation}
auferlegt. Daraus folgt f\"ur die Amplitude:
\begin{equation}
\Phi_0 = \frac{(1 - \Omega_m)(1 + s)}{\tanh\!\big(k\cdot(1 - a_{\mathrm{trans}})\big) + s}
\end{equation}
Das KRM hat damit drei kosmologische Freiheitsgrade ($\Omega_m$, $k$, $a_{\mathrm{trans}}$) plus einen Nuisance-Parameter ($M$), also insgesamt vier effektive Parameter -- nur zwei mehr als $\Lambda$CDM.

\subsection{Datenbasis: Pantheon+}
\label{subsec:pantheonplus}

Der Test erfolgt gegen den Pantheon+-Datensatz \cite{Scolnic2022}, den gr\"o\ss{}ten \"offentlich verf\"ugbaren Katalog spektroskopisch best\"atigter Typ-Ia-Supernovae. Aus den 1701 Lichtkurven werden 1590 Supernovae mit $z > 0{,}01$ verwendet (zur Vermeidung von Pekuliargeschwindigkeits-Dominanz), im Rotverschiebungsbereich $z = 0{,}0102$ bis $z = 2{,}2614$. Als Observable dient die bias-korrigierte scheinbare B-Band-Helligkeit \texttt{m\_b\_corr}. Die Analyse wird sowohl mit diagonalen Fehlern als auch mit der vollen statistisch-systematischen Kovarianzmatrix (STAT+SYS) des Pantheon+-Datensatzes durchgef\"uhrt.

\subsection{Methodik}
\label{subsec:methodik}

\textbf{Distanzberechnung:} Die Leuchtkraftentfernung wird \"uber eine kumulative Trapezregel auf einem feinen $z$-Gitter ($N = 2000$ St\"utzstellen) berechnet und auf die Daten-Rotverschiebungen interpoliert. Dieses Verfahren ist numerisch stabil (Fehler $< 10^{-5}$) und erm\"oglicht schnelle Evaluation w\"ahrend der Optimierung.

\textbf{Nuisance-Parameter:} Der absolute Helligkeitsoffset $M = M_B + 5\log_{10}(c/H_0) + 25$, der die absolute Helligkeit und die Hubble-Konstante absorbiert, wird analytisch marginalisiert:
\begin{equation}
M_{\mathrm{best}} = \frac{\sum_i w_i (m_i^{\mathrm{obs}} - \mu_i^{\mathrm{th}})}{\sum_i w_i}, \quad w_i = \sigma_i^{-2}
\end{equation}

\textbf{Optimierung:} Parameterbestimmung mittels \textit{Differential Evolution} (globaler evolution\"arer Optimizer) mit anschlie\ss{}ender L-BFGS-B-Verfeinerung (\textit{polish}).

\textbf{MCMC-Unsicherheiten:} F\"ur das KRM (flach) werden Parameterunsicherheiten mittels \textit{emcee} \cite{ForemanMackey2013} bestimmt (32~Walkers, 3000~Schritte, 500~Burn-in). Die Posteriorverteilungen liefern $1\sigma$-Konfidenzintervalle f\"ur alle Parameter einschlie\ss{}lich der abgeleiteten Gr\"o\ss{}en $\Phi_0$ und $z_{\mathrm{trans}}$.

\textbf{Modellselektion:} Neben $\chi^2$ werden das Akaike-Informationskriterium (AIC~$= \chi^2 + 2k$) und das Bayes-Informationskriterium (BIC~$= \chi^2 + k \ln n$) berechnet, wobei $k$ die Zahl effektiver Parameter und $n$ die Datenpunktanzahl ist. Zur \"Uberpr\"ufung auf Overfitting wird zus\"atzlich eine 5-Fold-Kreuzvalidierung durchgef\"uhrt. Der vollst\"andige Analysecode ist \"offentlich verf\"ugbar.\footnote{\url{https://github.com/lukisch/cfm-cosmology}}

\subsection{Ergebnisse}
\label{subsec:ergebnisse}

Es werden drei Modelle gefittet: flaches $\Lambda$CDM (2~Parameter), KRM mit Flachheitsbedingung (4~Parameter) und KRM ohne Einschr\"ankung (5~Parameter).

\begin{table}[H]
\centering
\caption{Gefittete Parameter und Anpassungsg\"ute: $\Lambda$CDM vs.\ KRM gegen Pantheon+ (1590~SNe~Ia). F\"ur das KRM~(flach) werden $1\sigma$-MCMC-Unsicherheiten angegeben.}
\label{tab:results}
\begin{tabular}{lccc}
\toprule
 & $\Lambda$CDM & KRM (flach) & KRM (frei) \\
\midrule
Freie Parameter $k$ & 2 & 4 & 5 \\
$\Omega_m$ & 0,244 & $0{,}368^{+0{,}025}_{-0{,}023}$ & 0,552 \\
$\Omega_\Lambda$ / $\Omega_\Phi(z{=}0)$ & 0,756 & 0,636 & 0,872 \\
$\Phi_0$ (abgeleitet) & -- & $0{,}988^{+0{,}615}_{-0{,}221}$ & 1,292 \\
$k$ (\"Ubergangssch\"arfe) & -- & $1{,}44^{+1{,}22}_{-0{,}84}$ & 1,98 \\
$a_{\mathrm{trans}}$ ($z_{\mathrm{trans}}$) & -- & 0,75 (0,33) & 0,80 (0,25) \\
$\Omega_{\mathrm{total}}$ & 1,000 & 1,000 & 1,423 \\
\midrule
$\chi^2$ (diagonal) & 729,0 & 716,8 & 715,9 \\
$\chi^2$ (volle Kov.) & 1432,0 & 1420,8 & -- \\
$\chi^2/\mathrm{dof}$ & 0,459 & 0,452 & 0,452 \\
AIC (diagonal) & 733,0 & 724,8 & 725,9 \\
AIC (volle Kov.) & 1436,0 & 1428,8 & -- \\
BIC & 743,7 & 746,3 & 752,8 \\
\bottomrule
\end{tabular}
\end{table}

Das KRM mit Flachheitsbedingung zeigt $\Omega_m = 0{,}368 \pm 0{,}024$ (MCMC) -- physikalisch plausibel und nahe am Planck-Wert ($0{,}315 \pm 0{,}007$). Die gefittete \"Ubergangsrotverschiebung $z_{\mathrm{trans}} = 0{,}33$ ($a_{\mathrm{trans}} = 0{,}75$) liegt bei sp\"ateren kosmischen Zeiten als theoretisch erwartet. Die \"Ubergangssch\"arfe $k = 1{,}44^{+1{,}22}_{-0{,}84}$ beschreibt einen sanften \"Ubergang mit breiter Posterior -- die Daten pr\"aferieren einen \"Ubergang, lassen aber eine Bandbreite von \"Ubergangssch\"arfen zu.

\textbf{Volle Kovarianzmatrix:} Die Wiederholung der Analyse mit der vollen statistisch-systematischen Kovarianzmatrix best\"atigt die Ergebnisse: $\Delta\chi^2 = -11{,}2$ und $\Delta\mathrm{AIC} = -7{,}2$ (gegen\"uber $-12{,}2$ und $-8{,}2$ bei diagonalen Fehlern). Die leichte Reduktion erkl\"art sich durch die Ber\"ucksichtigung systematischer Korrelationen zwischen benachbarten Supernovae.

\subsection{Modellselektion}
\label{subsec:modellselektion}

\begin{table*}[htbp]
\centering
\caption{Modellvergleich: KRM vs.\ $\Lambda$CDM. Negative Werte bevorzugen KRM.}
\label{tab:comparison}
\begin{tabular}{lcc}
\toprule
\textbf{Kriterium} & KRM (flach) vs.\ $\Lambda$CDM & KRM (frei) vs.\ $\Lambda$CDM \\
\midrule
$\Delta\chi^2$ & $\mathbf{-12{,}2}$ & $-13{,}1$ \\
$\Delta$AIC & $\mathbf{-8{,}2}$ & $-7{,}1$ \\
$\Delta$BIC & $+2{,}6$ & $+9{,}0$ \\
\midrule
5-Fold $\langle\chi^2/n\rangle$ & $\mathbf{0{,}4499}$ & $0{,}4498$ \\
$\Lambda$CDM: $\langle\chi^2/n\rangle$ & \multicolumn{2}{c}{$0{,}4519$} \\
\bottomrule
\end{tabular}
\end{table*}

\textbf{Interpretation:} Drei von vier Selektionskriterien bevorzugen das KRM (flach) gegen\"uber $\Lambda$CDM: $\chi^2$ ($-12{,}2$), AIC ($-8{,}2$) und Kreuzvalidierung ($0{,}4499$ vs.\ $0{,}4519$). Einzig das BIC, das zus\"atzliche Parameter strenger bestraft, zeigt eine marginale Pr\"aferenz f\"ur $\Lambda$CDM ($\Delta\mathrm{BIC} = +2{,}6$). Nach der Kass-Raftery-Skala \cite{KassRaftery1995} liegt dieser Wert an der Grenze zur Signifikanz ($|\Delta\mathrm{BIC}| < 2$: nicht signifikant; $2$--$6$: positive Evidenz). Die Kreuzvalidierung -- die robusteste Methode zur Overfitting-Detektion -- zeigt, dass das KRM auf ungesehenen Daten besser generalisiert als $\Lambda$CDM.

\subsection{MCMC-Posterioranalyse}
\label{subsec:mcmc}

Die Parameterunsicherheiten werden mittels \textit{emcee} \cite{ForemanMackey2013} bestimmt (32~Walkers, 3000~Schritte, 500~Burn-in, Akzeptanzrate: 63\,\%). Die Ergebnisse f\"ur das KRM~(flach) sind in Tabelle~\ref{tab:results} als $1\sigma$-Unsicherheiten angegeben. Die Posteriorverteilung von $\Omega_m$ ist nahezu gau\ss{}f\"ormig mit $\Omega_m = 0{,}368^{+0{,}025}_{-0{,}023}$. Die \"Ubergangssch\"arfe $k$ zeigt eine breite, asymmetrische Posterior ($k = 1{,}44^{+1{,}22}_{-0{,}84}$), was bedeutet, dass die Daten einen \"Ubergang bevorzugen, aber die Sch\"arfe weniger stark einschr\"anken. Die abgeleiteten Gr\"o\ss{}en $\Phi_0 = 0{,}988^{+0{,}615}_{-0{,}221}$ und $z_{\mathrm{trans}} = 0{,}35$ (MCMC-Posterior-Median; die Punktsch\"atzung in Tabelle~\ref{tab:results} gibt $0{,}33$) stimmen innerhalb der breiten Posteriori von $a_{\mathrm{trans}}$ \"uberein. Die aus dem MCMC berechneten $w(z)$-Konfidenzb\"ander (Tabelle~\ref{tab:weff}) zeigen, dass $w = -1$ f\"ur alle Rotverschiebungen au\ss{}erhalb des $1\sigma$-Bereichs liegt.

\textbf{Interpretation von $\Omega_m = 0{,}368$:} Der KRM-Wert liegt \"uber dem Planck-CMB-Wert ($\Omega_m^{\mathrm{Planck}} = 0{,}315 \pm 0{,}007$), was typisch f\"ur reine Supernova-Fits ist. Im KRM-Kontext hat diese Abweichung eine physikalische Deutung: Das Modell interpretiert einen Teil der in $\Lambda$CDM als ``Dunkle Energie'' klassifizierten Dichte als dynamischen Kr\"ummungseffekt. Da $\Omega_\Phi(a)$ bei fr\"uhen Zeiten gegen Null geht (im Gegensatz zu $\Omega_\Lambda = \mathrm{const.}$), muss $\Omega_m$ kompensatorisch h\"oher ausfallen, um den Gesamt-Fit zu erhalten. Diese Pr\"aferenz f\"ur h\"oheres $\Omega_m$ steht in Einklang mit j\"ungsten Befunden aus schwachen Gravitationslinsen-Surveys (KiDS, DES), die ebenfalls h\"ohere $\Omega_m$-Werte als Planck bevorzugen.

\subsection{Robustheit: Alternative Funktionalformen}
\label{subsec:funktionalformen}

Um zu \"uberpr\"ufen, ob die Ergebnisse von der spezifischen Wahl der $\tanh$-Form abh\"angen, werden vier verschiedene S\"attigungsfunktionen unter identischen Bedingungen getestet:

\begin{table}[H]
\centering
\caption{Vergleich alternativer Funktionalformen f\"ur $\Omega_\Phi(a)$. Alle Modelle verwenden die Flachheitsbedingung und haben $k=4$ Parameter.}
\label{tab:funcforms}
\begin{tabular}{lcccc}
\toprule
\textbf{Funktionalform} & $\chi^2$ & AIC & $\Delta\chi^2$ vs.\ $\Lambda$CDM & $\Omega_m$ \\
\midrule
$\tanh$ (Standard-KRM) & 716,8 & 724,8 & $-12{,}2$ & 0,364 \\
Logistische Funktion & 717,5 & 725,5 & $-11{,}5$ & 0,368 \\
Error-Funktion (erf) & 716,7 & 724,7 & $-12{,}3$ & 0,367 \\
Potenzgesetz & 720,1 & 728,1 & $\phantom{0}-8{,}9$ & 0,364 \\
\bottomrule
\end{tabular}
\end{table}

\textbf{Ergebnis:} Alle vier Funktionalformen liefern $\Delta\chi^2 \approx -9$ bis $-12$ gegen\"uber $\Lambda$CDM. Die $\tanh$-Form ist weder die einzig m\"ogliche noch die ``bestpassende'' -- sie repr\"asentiert eine robuste Klasse von S\"attigungsfunktionen. Dies entkr\"aftet den Einwand, die KRM-Ergebnisse seien von einer spezifischen Funktionswahl abh\"angig. Die $\Omega_m$-Werte konvergieren bei allen Formen auf $\approx 0{,}36$--$0{,}37$, was die physikalische Konsistenz unterstreicht.

\subsection{Phantom-Stabilit\"atsanalyse}
\label{subsec:phantom}

Der gefittete Zustandsgleichungsparameter $w < -1$ (Phantom-Bereich) wirft die berechtigte Frage nach Stabilit\"at auf. In gew\"ohnlichen Phantom-Skalarfeldmodellen f\"uhrt $w < -1$ zu einer wachsenden Energiedichte ($\rho \propto a^{-3(1+w)} \to \infty$ f\"ur $w < -1$, $a \to \infty$), die in endlicher Zeit den ``Big Rip'' erzwingt -- die Zerst\"orung aller gebundenen Strukturen im Universum \cite{Caldwell1998}. \textbf{Das KRM umgeht diese Pathologie grundlegend}, und zwar aus drei Gr\"unden:

\begin{enumerate}
\item \textbf{S\"attigung statt Divergenz:} Der dynamische S\"attigungsmechanismus (Gl.~\ref{eq:saturation_ode}) garantiert $\Omega_\Phi(a) \to \Phi_0$ f\"ur $a \to \infty$. Die effektive Energiedichte bleibt \textit{f\"ur alle Zeiten endlich und beschr\"ankt}. Dies ist der entscheidende Unterschied zu Phantom-Skalarfeldern: W\"ahrend dort $\rho$ divergiert, s\"attigt $\Omega_\Phi$ bei seinem Maximalwert~$\Phi_0$.
\item \textbf{De-Sitter-Endzustand:} $w_{\mathrm{eff}} \to -1$ f\"ur $a \to \infty$. Das Universum n\"ahert sich asymptotisch \textit{genau demselben Endzustand wie $\Lambda$CDM} -- einem stabilen de-Sitter-Raum. Der Phantom-Bereich ist ein \"Ubergangsph\"anomen, kein Endzustand.
\item \textbf{Kein Big Rip:} Da $\Omega_\Phi$ s\"attigt, divergiert weder die Energiedichte noch der Skalenfaktor in endlicher Zeit. Der Big Rip ist ausgeschlossen -- nicht durch eine zus\"atzliche Annahme, sondern als \textit{direkte Konsequenz} des S\"attigungsmechanismus.
\item \textbf{Geometrische statt fluide Interpretation:} Die formale Verletzung der Null-Energie-Bedingung ($\rho + p \geq 0$) ist unproblematisch, da $\Omega_\Phi$ \textit{kein physisches Feld} repr\"asentiert, sondern eine geometrische Eigenschaft der Raumzeit. Die Energiebedingungen der ART gelten f\"ur den Energie-Impuls-Tensor physischer Felder, nicht f\"ur effektive geometrische Terme. Analoge Situationen sind in der Literatur etabliert: $f(R)$-Gravitationstheorien zeigen routinem\"a\ss{}ig effektiv $w < -1$, ohne dass dies physische Instabilit\"aten erzeugt \cite{Sotiriou2010}.
\end{enumerate}

\begin{quote}
\textit{Zusammengefasst: Das KRM zeigt ``Phantom-Verhalten'' ohne Phantom-Pathologien. Das Universum endet nicht im Riss, sondern im Gleichgewicht.}
\end{quote}

\subsection{Dezelerationsparameter und $H_0$-Implikationen}
\label{subsec:dezeleration}

\textbf{Dezelerationsparameter $q(z)$:} Der Dezelerationsparameter liefert eine zus\"atzliche, unabh\"angig testbare Vorhersage. Im KRM tritt der \"Ubergang von gebremster zu beschleunigter Expansion ($q = 0$) bei $z_{\mathrm{acc}} = 0{,}52$ auf -- deutlich sp\"ater in kosmischer Zeit als im $\Lambda$CDM-Modell ($z_{\mathrm{acc}} = 0{,}84$). Zudem sagt das KRM eine st\"arkere heutige Beschleunigung voraus: $q_0^{\mathrm{KRM}} = -0{,}81$ gegen\"uber $q_0^{\Lambda\mathrm{CDM}} = -0{,}63$.

\textbf{Implikation f\"ur die Strukturbildung:} Ein sp\"aterer Beginn der kosmischen Beschleunigung ($z_{\mathrm{acc}} = 0{,}52$ statt $0{,}84$) bedeutet, dass gravitativ gebundene Strukturen (Galaxienhaufen, gro\ss{}r\"aumige Filamente) \textit{l\"anger ungest\"ort wachsen konnten}, bevor die Beschleunigung das Wachstum unterdr\"uckte. Die materiedominierte \"Ara -- in der die Gravitation das Strukturwachstum antreibt -- dauerte im KRM signifikant l\"anger an als im Standardmodell.

\textbf{Empirische Evidenz f\"ur fr\"uhe massive Strukturen:} Mehrere unabh\"angige Beobachtungen setzen das $\Lambda$CDM-Modell in Bezug auf die Strukturbildung unter erheblichen Druck:
\begin{enumerate}
\item \textbf{JWST ``Universe Breakers'':} Das James Webb Space Telescope hat Galaxien bei $z > 7$ (ca.\ 500--700\,Myr nach dem Urknall) entdeckt, die weitaus massereicher sind als von hierarchischer Strukturbildung im $\Lambda$CDM erlaubt \cite{Labbe2023}. Boylan-Kolchin \cite{BoylanKolchin2023} zeigt quantitativ, dass die stellare Massendichte dieser Objekte die verf\"ugbare Baryonen-Budget innerhalb von $\Lambda$CDM-Halos \"ubersteigt -- ein ``Timing-Problem'' der Strukturbildung.
\item \textbf{El~Gordo (ACT-CL~J0102$-$4915):} Dieser extrem massereiche Galaxienhaufen bei $z \approx 0{,}87$ mit Masse $M_{200} \approx 2{,}1 \times 10^{15}\,M_\odot$ stellt eine $> 6\sigma$-Spannung mit $\Lambda$CDM dar, da ein Objekt dieser Masse bei diesem Alter mit der beobachteten Kollisionsgeschwindigkeit statistisch nahezu unm\"oglich ist \cite{Asencio2023}.
\item \textbf{Protocluster SPT2349$-$56:} Bereits 1,4\,Mrd.\ Jahre nach dem Urknall ($z = 4{,}3$) zeigt dieser Protocluster mindestens 14 gasreiche Galaxien mit einer Gesamtsternentstehungsrate von $\sim\!6500\,M_\odot$/yr -- weit reifer als von $\Lambda$CDM vorhergesagt \cite{Miller2018}.
\end{enumerate}

Das KRM bietet f\"ur alle drei Beobachtungen eine nat\"urliche Erkl\"arung: Da die Beschleunigung erst bei $z_{\mathrm{acc}} = 0{,}52$ einsetzt (statt $z_{\mathrm{acc}} = 0{,}84$), stand gravitativ gebundenen Strukturen mehr kosmische Zeit f\"ur ungest\"ortes Wachstum zur Verf\"ugung. Die KRM-Vorhersage ist direkt testbar durch Galaxienhaufen-Z\"ahlungen und schwache Gravitationslinsen-Surveys (Euclid, Vera C.\ Rubin Observatory).

\begin{table}[H]
\centering
\caption{Dezelerationsparameter $q(z)$: $\Lambda$CDM vs.\ KRM.}
\label{tab:qz}
\begin{tabular}{cccc}
\toprule
$z$ & $q$ ($\Lambda$CDM) & $q$ (KRM) & $\Delta q$ \\
\midrule
0,0 & $-0{,}634$ & $-0{,}805$ & $-0{,}171$ \\
0,5 & $-0{,}217$ & $-0{,}017$ & $+0{,}200$ \\
1,0 & $+0{,}082$ & $+0{,}321$ & $+0{,}240$ \\
2,0 & $+0{,}346$ & $+0{,}467$ & $+0{,}121$ \\
\bottomrule
\end{tabular}
\end{table}

\textbf{$H_0$-Implikationen:} Der Nuisance-Parameter $M$ absorbiert sowohl die absolute Helligkeit $M_B$ als auch $H_0$ \"uber die Beziehung $M = M_B + 5\log_{10}(c/H_0) + 25$. Unter Verwendung der SH0ES-Eichung ($M_B = -19{,}253$) liefert das KRM $H_0 = 76{,}1\,\mathrm{km/s/Mpc}$ gegen\"uber $H_0 = 75{,}5\,\mathrm{km/s/Mpc}$ im $\Lambda$CDM -- ein Unterschied von $\Delta H_0 = +0{,}5\,\mathrm{km/s/Mpc}$. Die $H_0$-Spannung wird durch das KRM allein nicht gel\"ost, da der Unterschied innerhalb der Messunsicherheit liegt. Eine direkte Aufl\"osung erfordert die Kombination mit CMB- und BAO-Daten.

% ===================================================================
% 5. VERGLEICH MIT ALTERNATIVEN
% ===================================================================
\section{Vergleich mit alternativen Modellen}
\label{sec:alternativen}

\subsection{$\Lambda$CDM (Standardmodell)}

Das $\Lambda$CDM-Modell ist extrem einfach ($w = -1$, konstant, zwei kosmologische Parameter) und passt alle aktuellen Daten gut. Es leidet jedoch unter dem Kosmologische-Konstante-Problem und dem Koinzidenz-Problem \cite{Weinberg1989}.

\subsection{Quintessenz}

Quintessenz-Modelle \cite{Caldwell1998} postulieren ein dynamisches Skalarfeld~$\phi$ mit zeitabh\"angigem Zustandsgleichungsparameter. Sie k\"onnen das Koinzidenz-Problem mildern, erfordern aber ein neues Feld und dessen Potential~$V(\phi)$ mit vielen freien Parametern.

\subsection{Modifizierte Gravitation: $f(R)$-Theorien}

$f(R)$-Gravitationstheorien \cite{Starobinsky1980, Sotiriou2010} ersetzen den Ricci-Skalar~$R$ in der Einstein-Hilbert-Wirkung durch eine allgemeinere Funktion. Sie bieten eine geometrische Erkl\"arung ohne Dunkle Energie, sind jedoch mathematisch komplex und zum Teil inkonsistent mit Beobachtungen (Gravitationslinsen, CMB). Das am weitesten untersuchte sp\"atzeitliche $f(R)$-Modell, Hu-Sawicki \cite{HuSawicki2007}, erf\"ullt die Constraints des Sonnensystems durch den Cham\"aleon-Mechanismus und erzeugt gleichzeitig eine viable kosmische Beschleunigung; es dient als wichtiger Vergleichsma\ss{}stab f\"ur die Lagrange-Analyse in Paper~III \cite{Geiger2026c}.

\subsection{Emergente Gravitation nach Verlinde}

Verlinde \cite{Verlinde2011, Verlinde2017} schl\"agt vor, dass die Gravitation keine fundamentale Kraft, sondern ein emergentes, entropisches Ph\"anomen ist. In de-Sitter-R\"aumen f\"uhrt die mit dem kosmologischen Horizont assoziierte Entropie zu einer zus\"atzlichen ``dunklen'' Gravitationskraft, die das Verhalten von Galaxien ohne Dunkle Materie erkl\"aren k\"onnte. In einem eng verwandten Ansatz leitet Padmanabhan \cite{Padmanabhan2012} die kosmische Expansion als emergentes Ph\"anomen aus der Differenz zwischen Oberfl\"achen- und Bulk-Freiheitsgraden ab, wobei die kosmologische Konstante aus dem Bestreben der Raumzeit entsteht, holographische \"Aquipartition zu erreichen -- eine Perspektive, die strukturelle Gemeinsamkeiten mit dem Gleichgewichtsrahmenwerk des KRM aufweist.

\subsection{Finsler-Gravitation}

Pfeifer et al.\ \cite{Pfeifer2025} erweitern die Allgemeine Relativit\"atstheorie durch Finsler-Geometrie, in der die Metrik nicht nur von der Position, sondern auch von der Geschwindigkeit abh\"angt:
\begin{equation}
g_{\mu\nu}(x, y) = \frac{1}{2}\,\frac{\partial^2 F^2}{\partial y^\mu \partial y^\nu}, \quad y = \frac{dx}{d\lambda}
\end{equation}
Die resultierende Finsler-Friedmann-Gleichung erzeugt selbst im Vakuum eine exponentielle Expansion -- ohne kosmologische Konstante.

\subsection{Cosmological Teleodynamics}

Trivedi und Venkatasubramanian \cite{Trivedi2025} formulieren eine spieltheoretische Kosmologie, die erstaunliche Parallelen zum hier vorgestellten Ansatz aufweist. Ihre \textit{Cosmological Teleodynamics} beschreibt das Universum als ``riesiges Potentialspiel'', das sich einem kontinuierlichen Nash-Gleichgewicht ann\"ahert. Die kosmische Beschleunigung erscheint als ``statistisch emergenter Effekt dynamischen Ged\"achtnisses in einem selbstgravitierenden Medium'' -- eine Formulierung, die konzeptuell dem ``geometrischen Ged\"achtnis'' des KRM entspricht.

\subsection{Synoptischer Vergleich}
\label{subsec:synopse}

\begin{table*}[tbp]
\centering
\caption{Synoptischer Vergleich kosmologischer Modelle ohne Dunkle Energie.}
\label{tab:synopse}
\begin{tabular}{llll}
\toprule
\textbf{Eigenschaft} & \textbf{KRM} & \textbf{Finsler} & \textbf{Teleodynamics} \\
\midrule
Theor.\ Basis & Standard-ART + Potential & Finsler-Geometrie & Stat.\ Mechanik + Spieltheorie \\
Mechanismus & Nachlassende ``Bremse'' & Geschwindigkeitsabh.\ Metrik & Dynamisches Ged\"achtnis \\
Dunkle Energie & Nicht n\"otig & Nicht n\"otig & Nicht n\"otig \\
Empirischer Test & Pantheon+ (1590 SNe, $\Delta\chi^2{=}{-}12$) & Noch ausstehend & Qualitativ \\
Vorhersage & $w(z)$ Zeitvariation & Exp.\ Expansion & Nash-Konvergenz \\
Komplexit\"at & Gering (4 Param.) & Hoch & Mittel \\
\bottomrule
\end{tabular}
\end{table*}

% ===================================================================
% 6. KOMPLEMENTARITÄT UND VEREINIGUNG
% ===================================================================
\section{Komplementarit\"at und m\"ogliche Vereinigung}
\label{sec:komplementaritaet}

\subsection{Drei Modelle, eine Einsicht}

Alle drei Ans\"atze -- KRM, Finsler-Gravitation und Cosmological Teleodynamics -- teilen eine fundamentale Einsicht:
\begin{quote}
\textit{``Die beschleunigte Expansion ist kein neues `Ding', sondern eine Eigenschaft der Geometrie bzw.\ der statistischen Struktur des Universums selbst.''}
\end{quote}

\subsection{Hypothese: KRM als effektive Beschreibung}

Eine faszinierende M\"oglichkeit besteht darin, dass die drei Modelle verschiedene Aspekte desselben Ph\"anomens beschreiben. In Analogie zur Beziehung zwischen Thermodynamik und Statistischer Mechanik k\"onnte gelten:

\begin{itemize}
\item \textbf{Finsler-Gravitation} (mikroskopisch, fundamental): Alle Momente der 1-Partikel-Verteilungsfunktion tragen zur Gravitation bei.
\item \textbf{KRM} (makroskopisch, ph\"anomenologisch): Das zeitabh\"angige Potential $\Phi(a)$ kodiert effektiv den Beitrag der h\"oheren Momente.
\item \textbf{Teleodynamics} (systemisch, spieltheoretisch): Die Nash-Gleichgewichtsdynamik beschreibt die globale Optimierung.
\end{itemize}

Mathematisch l\"asst sich diese Komplementarit\"at als hierarchische Beziehung darstellen:
\begin{equation}
\begin{split}
\underbrace{G_{\mu\nu}^{\mathrm{Finsler}}(x, y)}_{\substack{\text{Finsler-Gravitation}\\\text{(mikroskopisch)}}}
&\;\xrightarrow{\;\langle\cdot\rangle_{\mathrm{eff}}\;}
\underbrace{G_{\mu\nu} + 8\pi G\,\Omega_\Phi(a)\,g_{\mu\nu}}_{\substack{\text{KRM: modifizierte}\\\text{Einstein-Gleichung}}} \\
&\;\xleftarrow{\;\delta\Phi/\delta s_i = 0\;}
\underbrace{\max_{s_M, s_T} \Phi(s_M, s_T)}_{\substack{\text{Teleodynamics:}\\\text{Nash-Gleichgewicht}}}
\end{split}
\label{eq:complementarity}
\end{equation}
wobei der linke Pfeil die effektive Mittelung \"uber die geschwindigkeitsabh\"angigen Freiheitsgrade der Finsler-Geometrie beschreibt und der rechte Pfeil die Variationsbedingung des spieltheoretischen Potentials darstellt, die das zeitliche Verhalten von $\Omega_\Phi(a)$ festlegt.

% ===================================================================
% 7. TESTBARKEIT UND VORHERSAGEN
% ===================================================================
\section{Testbarkeit und Vorhersagen}
\label{sec:testbarkeit}

\subsection{Beobachtbare Signaturen}

\textbf{1.~Phantom-Zustandsgleichung $w(z) < -1$:} Das KRM sagt $|\Delta w| \approx 0{,}4$ \"uber den gesamten beobachtbaren Rotverschiebungsbereich voraus. Die ESA-Mission Euclid \cite{Euclid2024} und das Nancy Grace Roman Space Telescope (NASA, $\sim$2027) k\"onnen $\sigma_w \approx 0{,}02$--$0{,}05$ messen -- weit ausreichend, um diese Signatur nachzuweisen oder auszuschlie\ss{}en.

\textbf{2.~Dezelerationsparameter:} Das KRM sagt einen fr\"uheren \"Ubergang zur beschleunigten Expansion voraus ($z_{\mathrm{acc}} = 0{,}52$ vs.\ $\Lambda$CDM: $z_{\mathrm{acc}} = 0{,}84$) sowie eine st\"arkere heutige Beschleunigung ($q_0 = -0{,}81$ vs.\ $-0{,}63$). Dies ist unabh\"angig von $w(z)$ testbar.

\textbf{3.~Strukturwachstum:} Eine modifizierte Wachstumsrate $f \cdot \sigma_8$ ist vorhergesagt, messbar durch schwache Gravitationslinsen und Galaxienhaufen-Z\"ahlungen. Erste empirische Hinweise liefern bereits die JWST-``Universe Breakers'' bei $z > 7$ \cite{Labbe2023}, die El-Gordo-Anomalie ($> 6\sigma$ Spannung mit $\Lambda$CDM; \cite{Asencio2023}) und unerwartet reife Protocluster bei $z > 4$ \cite{Miller2018} -- allesamt konsistent mit der KRM-Vorhersage einer verl\"angerten Wachstumsphase.

\textbf{4.~CMB-Integraleffekte:} Ein modifizierter ISW-Effekt (\textit{Integrated Sachs-Wolfe}) in CMB-Temperatur-Kreuzkorrelationen.

\subsection{Zuk\"unftige Missionen}

\begin{table}[H]
\centering
\caption{Relevante Beobachtungsmissionen f\"ur den KRM-Test.}
\label{tab:missionen}
\small
\begin{tabular}{llcl}
\toprule
\textbf{Mission} & \textbf{Start} & $\sigma(w)$ & \textbf{Relevanz} \\
\midrule
Euclid & 2023 & ${\approx}\,0{,}02$ & BAO + Linsen \\
Roman & ${\sim}$2027 & ${\approx}\,0{,}03$ & SN bis $z \approx 2$ \\
DESI & 2021-- & ${\approx}\,0{,}04$ & BAO + Wachstum \\
\bottomrule
\end{tabular}
\end{table}

\subsection{Unterscheidbarkeit der Modelle}

\begin{table}[H]
\centering
\caption{Vergleich der Vorhersagen: $\Lambda$CDM vs.\ KRM (Pantheon+-Fit). Die $1\sigma$-Unsicherheiten stammen aus der MCMC-Analyse.}
\label{tab:vorhersagen}
\begin{tabular}{lcc}
\toprule
\textbf{Eigenschaft} & $\Lambda$CDM & KRM \\
\midrule
$w(z{=}0)$ & $-1{,}000$ & $-1{,}36 \pm 0{,}02$ \\
$w(z{=}0{,}5)$ & $-1{,}000$ & $-1{,}45^{+0{,}09}_{-0{,}28}$ \\
$w(z{=}1)$ & $-1{,}000$ & $-1{,}45^{+0{,}10}_{-0{,}30}$ \\
$w(z{=}2)$ & $-1{,}000$ & $-1{,}43^{+0{,}08}_{-0{,}21}$ \\
Zeitvariation & Keine & Ja (durchgehend $w < -1$) \\
$\Delta w$ (messbar) & -- & $\approx -0{,}4$ \\
$q_0$ (heute) & $-0{,}63$ & $-0{,}81$ \\
$z_{\mathrm{acc}}$ (\"Ubergang) & 0,84 & 0,52 \\
\bottomrule
\end{tabular}
\end{table}

% ===================================================================
% 8. DISKUSSION
% ===================================================================
\section{Diskussion}
\label{sec:diskussion}

\subsection{St\"arken des Ansatzes}

\begin{enumerate}
\item \textbf{Konzeptuelle Eleganz:} Keine neue Energieform erforderlich; die Beschleunigung ist eine ``nachlassende Einschr\"ankung'', kein ``neuer Antrieb''.
\item \textbf{Spieltheoretische Fundierung:} Die Emergenz physikalischer Gesetze aus Gleichgewichtsbedingungen bietet einen neuartigen Erkl\"arungsrahmen, der durch die unabh\"angige Arbeit von Trivedi und Venkatasubramanian \cite{Trivedi2025} gest\"utzt wird.
\item \textbf{Empirische Validierung:} Das KRM passt 1590 reale Pantheon+-Supernovae besser als $\Lambda$CDM ($\Delta\chi^2 = -12{,}2$, $\Delta\mathrm{AIC} = -8{,}2$) und generalisiert in der Kreuzvalidierung besser.
\item \textbf{Testbarkeit:} Spezifische, quantitative Vorhersagen f\"ur $w(z)$ und $z_{\mathrm{acc}}$, die innerhalb einer Dekade \"uberpr\"ufbar sind.
\item \textbf{Empirische Unterst\"utzung:} Die KRM-Vorhersage einer verl\"angerten Wachstumsphase ($z_{\mathrm{acc}} = 0{,}52$) bietet eine nat\"urliche Erkl\"arung f\"ur die JWST-``Early Galaxy Tension'' \cite{Labbe2023, BoylanKolchin2023}, die El-Gordo-Anomalie \cite{Asencio2023} und unerwartet reife Protocluster \cite{Miller2018}.
\item \textbf{Konvergenz unabh\"angiger Ans\"atze:} KRM, Finsler-Gravitation und Cosmological Teleodynamics kommen unabh\"angig zum selben Schluss: Dunkle Energie ist nicht notwendig.
\item \textbf{Reproduzierbarkeit:} Analysecode und Daten sind \"offentlich verf\"ugbar (\url{https://github.com/lukisch/cfm-cosmology}).
\end{enumerate}

\subsection{Limitationen und offene Fragen}

\begin{enumerate}
\item \textbf{Ph\"anomenologischer Charakter:} Das KRM ist keine fundamentale Theorie. Obwohl die $\tanh$-Form als exakte L\"osung der S\"attigungs-ODE~\eqref{eq:saturation_ode} motiviert werden kann und vier alternative Funktionalformen vergleichbare Ergebnisse liefern (Abschnitt~\ref{subsec:funktionalformen}), steht eine Herleitung aus einer fundamentalen Quantengleichung noch aus.
\item \textbf{Parameterfreiheit:} Vier effektive Parameter gegen\"uber zwei in $\Lambda$CDM f\"uhren zu einem marginalen BIC-Nachteil ($\Delta\mathrm{BIC} = +2{,}6$), der jedoch durch die bessere Kreuzvalidierung und die Robustheit \"uber verschiedene Funktionalformen relativiert wird.
\item \textbf{Phantom-Bereich:} Der effektive Zustandsgleichungsparameter $w < -1$ liegt im Phantom-Bereich. Wie in Abschnitt~\ref{subsec:phantom} gezeigt, f\"uhrt dies im KRM-Kontext weder zu einem Big Rip noch zu Instabilit\"aten, da $\Omega_\Phi$ s\"attigt und kein physisches Feld darstellt. Formal verletzt das KRM die Null-Energie-Bedingung, analog zu $f(R)$-Gravitationstheorien \cite{Sotiriou2010}.
\item \textbf{Modellvergleich:} Der Vergleich in dieser Arbeit beschr\"ankt sich auf $\Lambda$CDM. Ein fairerer Referenzpunkt f\"ur das Phantom-Kreuzungsverhalten w\"are die $w_0w_a$CDM-Parametrisierung (CPL) \cite{Chevallier2001, Linder2003}, die ebenfalls $w < -1$ erlaubt. Ein solcher Vergleich wird in Paper~II pr\"asentiert.
\item \textbf{Offene Tests:} CMB-Vorhersagen (ISW-Effekt, CMB-Leistungsspektrum), BAO-Signaturen und Gravitationslinsen-Effekte m\"ussen noch berechnet werden. Die Analyse mit der vollen Kovarianzmatrix (Abschnitt~\ref{subsec:ergebnisse}) best\"atigt jedoch die Ergebnisse der diagonalen Analyse.
\item \textbf{Mikroskopische Basis:} Was ist $\Phi$ auf Quantenebene? Die Verbindung zu einer Theorie der Quantengravitation steht aus. Die m\"ogliche Beziehung zur Finsler-Gravitation (Abschnitt~\ref{sec:komplementaritaet}) k\"onnte hier eine Br\"ucke schlagen.
\item \textbf{$H_0$-Spannung:} Die $H_0$-Analyse (Abschnitt~\ref{subsec:dezeleration}) zeigt $\Delta H_0 = +0{,}5\,$km/s/Mpc zwischen KRM und $\Lambda$CDM -- zu gering, um die $H_0$-Spannung zu l\"osen. Eine Aufl\"osung erfordert die Kombination mit CMB- und BAO-Daten.
\end{enumerate}

\subsection{Philosophische Implikationen}

Falls das KRM (oder ein verwandtes Modell) best\"atigt wird, h\"atte dies tiefgreifende Konsequenzen:

\begin{itemize}
\item \textbf{Dunkle Energie ist kein ``Ding'':} Sie w\"are eine geometrische Erinnerung, kein physisches Feld.
\item \textbf{Das Universum ``wei\ss{}'' von seinem Anfang:} Die Geometrie besitzt ein ``Ged\"achtnis''.
\item \textbf{Paradigmenwechsel:} Von ``Was treibt die Beschleunigung an?'' zu ``Warum bremste die Expansion fr\"uher?''
\end{itemize}

Dies w\"are vergleichbar mit dem \"Ubergang von ``Was treibt die Planeten an?'' (Ptolem\"aus: Sph\"aren) zu ``Wie bewegen sich Planeten in der Geometrie des Raumes?'' (Kepler, Newton, Einstein).

\subsection{Deduktive Struktur des Modells}
\label{subsec:deduktiv}

Ein Unterscheidungsmerkmal des KRM ist, dass sein Lagrangian nicht \textit{ad hoc} aus unabh\"angigen ph\"anomenologischen Termen zusammengesetzt, sondern aus einer deduktiven Kette abgeleitet ist, die von zwei Annahmen ausgeht und bei testbaren Vorhersagen endet:

\begin{enumerate}
\item \textbf{Axiom~1 (Anfangsbedingung):} Ein metastabiles Quantenvakuum (Nullraum) l\"asst eine gro\ss{}e Fluktuation zu, die einen Konzentrationsgradienten erzeugt -- die entstehende Raumzeit.
\item \textbf{Axiom~2 (Entwicklungsprinzip):} Das System entwickelt sich entlang des Pfades, der die Entropieproduktion maximiert, w\"ahrend die strukturelle Integrit\"at des Nullraums erhalten bleibt (\"aquivalent: Prinzip der kleinsten Wirkung unter der Randbedingung der Muttersystem-Stabilit\"at; siehe Abschnitt~\ref{subsec:thermodynamik}).
\end{enumerate}

Aus diesen beiden Annahmen f\"uhrt folgende Deduktionskette -- hier r\"uckblickend dargestellt; ihre strenge Herleitung ist Gegenstand von Paper~III -- zur KRM-Wirkung (Paper~III, Gl.~(8)):
\begin{itemize}
\item Axiom~2 erfordert eine \textit{kontrollierte, zeitlich gestreckte} Neutralisation des Gradienten $\to$ Einstein-Hilbert-Term $R/(16\pi G)$ (konservative Feldgleichungen als notwendige Bedingung).
\item Der Gradient hat endliche Gr\"o\ss{}e, die R\"uckkehr darf das Vakuum nicht destabilisieren $\to$ \textit{S\"attigung} der R\"uckgaberate $\to$ P\"oschl-Teller-Skalarfeld mit $\tanh$-Dynamik.
\item Minimalit\"at (Geistfreiheit + Renormierbarkeit auf Ein-Schleifen-Niveau) selektiert eindeutig $R^2$ als f\"uhrende Gravitationskorrektur $\to$ Starobinsky-Term $\gamma R^2$.
\item Das Skalaron ($R^2$-Freiheitsgrad) erzeugt einen $a^{-\beta}$-Potenzgesetzterm, der die Rolle der Teilchen-Dunklen-Materie auf kosmologischer Ebene \"ubernimmt; die Spurkopplung an $T_{\mu\nu}$ unterdr\"uckt diesen Term automatisch w\"ahrend der Strahlungs\"ara (BBN-Schutz).
\item Auf galaktischen Skalen verst\"arkt dasselbe $R^2$-Skalaron die Gravitation um $\mu \to 4/3$ auf Sub-Compton-Skalen (Paper~III), koinzidierend mit dem MOND-Phasenraumfaktor $V_3/V_2 = 4/3$ (Paper~II).
\item Cham\"aleon-Screening sichert die Sonnensystem-Kompatibilit\"at ohne zus\"atzliche Parameter.
\end{itemize}

Das resultierende Lagrangian $\mathcal{L} = R/(16\pi G) + \gamma R^2 + \frac{1}{2}(\partial\phi)^2 - V_{\mathrm{PT}}(\phi)$ hat seine \textit{Form} vollst\"andig durch die zwei Axiome bestimmt; nur die \textit{numerischen Werte} von $\gamma$, $V_0$ und $\phi_0$ werden durch Daten festgelegt. Diese deduktive Geschlossenheit -- von Annahmen zu einem eindeutigen Lagrangian zu falsifizierbaren Vorhersagen -- erhebt das KRM von einem ph\"anomenologischen Fit zu einem theoretischen Rahmenwerk.

% ===================================================================
% 9. FAZIT UND AUSBLICK
% ===================================================================
\section{Fazit und Ausblick}
\label{sec:fazit}

Die vorliegende Arbeit hat gezeigt:

\begin{enumerate}
\item Ein spieltheoretischer Rahmen f\"ur die Kosmologie -- das Nash-Gleichgewicht zwischen Nullraum und Raumzeitblase -- f\"uhrt auf nat\"urliche Weise zu einem Modell, in dem physikalische Gesetze als emergente Gleichgewichtsbedingungen erscheinen. Die spieltheoretischen Axiome besitzen eine exakte thermodynamische \"Aquivalenz (Nash-Gleichgewicht $\leftrightarrow$ thermodynamisches Gleichgewicht, $\tanh$-S\"attigung $\leftrightarrow$ MEPP-Dynamik), die das Rahmenwerk in die Jacobson-Tradition der Gravitation als thermodynamische Zustandsgleichung \cite{Jacobson1995} stellt.
\item Das daraus abgeleitete \textit{Curvature Relaxation Model} (KRM) erkl\"art die beschleunigte Expansion ohne Dunkle Energie und besteht den Test gegen 1590 reale Typ-Ia-Supernovae des Pantheon+-Katalogs \cite{Scolnic2022}: $\Delta\chi^2 = -12{,}2$ (diagonal) bzw.\ $-11{,}2$ (volle Kovarianzmatrix), $\Delta\mathrm{AIC} = -8{,}2$ bzw.\ $-7{,}2$, bessere Kreuzvalidierung.
\item Die robuste Modellselektion (AIC, BIC, 5-Fold-Kreuzvalidierung) zeigt, dass der bessere Fit des KRM nicht auf Overfitting zur\"uckzuf\"uhren ist. Dies wird durch vier alternative Funktionalformen best\"atigt, die alle $\Delta\chi^2 \approx -9$ bis $-12$ liefern.
\item MCMC-basierte Parameterunsicherheiten ($\Omega_m = 0{,}368 \pm 0{,}024$) und die Phantom-Stabilit\"atsanalyse (kein Big Rip, asymptotisch de-Sitter) unterst\"utzen die physikalische Konsistenz.
\item Das KRM macht testbare Vorhersagen: eine durchgehende Phantom-Zustandsgleichung $w(z) < -1$ und einen sp\"ateren Beschleunigungs\"ubergang ($z_{\mathrm{acc}} = 0{,}52$ vs.\ $0{,}84$ in $\Lambda$CDM), die mit Euclid und Roman innerhalb der n\"achsten Dekade \"uberpr\"ufbar sind. Bereits jetzt findet die KRM-Vorhersage einer verl\"angerten Wachstumsphase empirische Unterst\"utzung durch JWST-Beobachtungen unerwartet massereicher Galaxien bei hohen Rotverschiebungen \cite{Labbe2023, BoylanKolchin2023} und die statistisch un\-wahrscheinliche Existenz massiver Haufen wie El~Gordo \cite{Asencio2023}.
\item Die Konvergenz dreier unabh\"angiger Ans\"atze (KRM, Finsler-Gravitation, Cosmological Teleodynamics) deutet auf einen m\"oglichen Paradigmenwechsel hin: \textit{Dunkle Energie als eigenst\"andige Entit\"at k\"onnte \"uberfl\"ussig sein.}
\end{enumerate}

\textbf{N\"achste Schritte} umfassen: (a)~Test gegen Planck-CMB- und DESI-BAO-Daten (die volle Pantheon+-Kovarianzmatrix wurde bereits ber\"ucksichtigt), (b)~Berechnung von CMB-Leistungsspektrum und Strukturwachstumsvorhersagen ($f\sigma_8$), (c)~Erforschung der Verbindung zwischen KRM und Finsler-Geometrie, (d)~Entwicklung einer kovarianten Formulierung von $\Phi(a)$ aus dem Ricci-Skalar~$R$, (e)~Untersuchung quantenmechanischer Grundlagen des Kr\"ummungs-R\"uckgabepotentials, und (f)~Kombination mit lokalen Entfernungsleiter-Daten zur direkten $H_0$-Bestimmung.

\textbf{Ausblick: Vereinigung mit MOND -- Ein Universum ohne dunklen Sektor?} Eine besonders faszinierende Perspektive er\"offnet sich durch die Kombination des KRM mit \textit{Modified Newtonian Dynamics} (MOND) \cite{Milgrom1983}. W\"ahrend das KRM die Dunkle Energie als geometrischen Effekt eliminiert, ersetzt MOND die Dunkle Materie durch eine modifizierte Gravitationsdynamik auf galaktischen Skalen. Beide Rahmenwerke konvergieren in der Vorhersage, dass Strukturen fr\"uher und effizienter entstehen als in $\Lambda$CDM -- das KRM durch eine verl\"angerte materiedominierte \"Ara ($z_{\mathrm{acc}} = 0{,}52$), MOND durch effektiv st\"arkere Gravitation bei niedrigen Beschleunigungen \cite{Asencio2023}. Eine vorl\"aufige Analyse mit einem rein baryonischen Universum ($\Omega_m = \Omega_b \approx 0{,}05$) und einem erweiterten geometrischen Potential $\Omega_\Phi(a) = \Phi_0 \cdot f_{\mathrm{tanh}}(a) + \alpha \cdot a^{-\beta}$ liefert $\Delta\chi^2 = -26{,}3$ und $\Delta\mathrm{AIC} = -16{,}3$ gegen\"uber $\Lambda$CDM -- \textit{deutlich besser als sowohl das Standard-KRM als auch $\Lambda$CDM}. Die MCMC-Posterioranalyse ergibt $\beta = 2{,}02 \pm 0{,}20$, was exakt der Skalierung r\"aumlicher Kr\"ummung ($a^{-2}$) entspricht. Die ``Dunkle Materie'' w\"are demnach ein dynamischer Kr\"ummungseffekt, kein Teilchen. Dieses Ergebnis bedarf einer vollst\"andigen relativistischen Behandlung (z.\,B.\ im AeST-Rahmen \cite{Skordis2021}) und wird in einer Folgearbeit detailliert analysiert.

\begin{quote}
\textit{``Manchmal ist die eleganteste Erkl\"arung nicht eine neue Kraft, sondern eine nachlassende Einschr\"ankung.''}
\end{quote}

% ===================================================================
% LITERATUR
% ===================================================================
\subsection*{Software}

Diese Arbeit verwendet \texttt{emcee} \cite{ForemanMackey2013} f\"ur MCMC-Sampling, \texttt{NumPy} \cite{Harris2020}, \texttt{SciPy} \cite{Virtanen2020} f\"ur numerische Berechnungen und \texttt{Matplotlib} \cite{Hunter2007} f\"ur Visualisierungen.

\begin{thebibliography}{99}

\bibitem{Riess1998}
Riess, A.\,G.\ et al.\ (1998).
Observational Evidence from Supernovae for an Accelerating Universe and a Cosmological Constant.
\textit{The Astronomical Journal}, 116(3), 1009--1038.
DOI: 10.1086/300499.

\bibitem{Perlmutter1999}
Perlmutter, S.\ et al.\ (1999).
Measurements of $\Omega$ and $\Lambda$ from 42 High-Redshift Supernovae.
\textit{The Astrophysical Journal}, 517(2), 565--586.
DOI: 10.1086/307221.

\bibitem{Planck2020}
Planck Collaboration (2020).
Planck 2018 results. VI. Cosmological parameters.
\textit{Astronomy \& Astrophysics}, 641, A6.
DOI: 10.1051/0004-6361/201833910.

\bibitem{Weinberg1989}
Weinberg, S.\ (1989).
The Cosmological Constant Problem.
\textit{Reviews of Modern Physics}, 61(1), 1--23.
DOI: 10.1103/RevModPhys.61.1.

\bibitem{Riess2022}
Riess, A.\,G.\ et al.\ (2022).
A Comprehensive Measurement of the Local Value of the Hubble Constant with 1\,km/s/Mpc Uncertainty from the Hubble Space Telescope and the SH0ES Team.
\textit{The Astrophysical Journal Letters}, 934(1), L7.
DOI: 10.3847/2041-8213/ac5c5b.

\bibitem{DESI2024}
DESI Collaboration (2024).
DESI 2024 VI: Cosmological Constraints from the Measurements of Baryon Acoustic Oscillations.
\textit{arXiv:2404.03002}.

\bibitem{Pfeifer2025}
Pfeifer, C.\ et al.\ (2025).
From kinetic gases to an exponentially expanding universe -- the Finsler-Friedmann equation.
\textit{Journal of Cosmology and Astroparticle Physics}, 2025(10), 050.
DOI: 10.1088/1475-7516/2025/10/050.

\bibitem{Trivedi2025}
Trivedi, O.\ \& Venkatasubramanian, V.\ (2025).
Game Theory in Cosmology.
\textit{arXiv:2511.20739}.

\bibitem{Caldwell1998}
Caldwell, R.\,R., Dave, R.\ \& Steinhardt, P.\,J.\ (1998).
Cosmological Imprint of an Energy Component with General Equation of State.
\textit{Physical Review Letters}, 80(8), 1582--1585.
DOI: 10.1103/PhysRevLett.80.1582.

\bibitem{Starobinsky1980}
Starobinsky, A.\,A.\ (1980).
A New Type of Isotropic Cosmological Models Without Singularity.
\textit{Physics Letters B}, 91(1), 99--102.
DOI: 10.1016/0370-2693(80)90670-X.

\bibitem{Sotiriou2010}
Sotiriou, T.\,P.\ \& Faraoni, V.\ (2010).
$f(R)$ Theories of Gravity.
\textit{Reviews of Modern Physics}, 82(1), 451--497.
DOI: 10.1103/RevModPhys.82.451.

\bibitem{Verlinde2011}
Verlinde, E.\ (2011).
On the Origin of Gravity and the Laws of Newton.
\textit{Journal of High Energy Physics}, 2011, 29.
DOI: 10.1007/JHEP04(2011)029.

\bibitem{Verlinde2017}
Verlinde, E.\ (2017).
Emergent Gravity and the Dark Universe.
\textit{SciPost Physics}, 2(3), 016.
DOI: 10.21468/SciPostPhys.2.3.016.

\bibitem{Padmanabhan2012}
Padmanabhan, T.\ (2012).
Emergent Perspective of Gravity and Dark Energy.
\textit{Research in Astronomy and Astrophysics}, 12(8), 891.
DOI: 10.1088/1674-4527/12/8/003.

\bibitem{Chevallier2001}
Chevallier, M.\ \& Polarski, D.\ (2001).
Accelerating Universes with Scaling Dark Matter.
\textit{Int.\ J.\ Mod.\ Phys.\ D}, 10, 213.
DOI: 10.1142/S0218271801000822.

\bibitem{Linder2003}
Linder, E.~V.\ (2003).
Exploring the Expansion History of the Universe.
\textit{Phys.\ Rev.\ Lett.}, 90, 091301.
DOI: 10.1103/PhysRevLett.90.091301.

\bibitem{Euclid2024}
Euclid Collaboration (2025).
Euclid Quick Data Release 1.
ESA/Euclid Consortium.

\bibitem{Casimir1948}
Casimir, H.\,B.\,G.\ (1948).
On the attraction between two perfectly conducting plates.
\textit{Proceedings of the Royal Netherlands Academy of Arts and Sciences}, 51, 793--795.

\bibitem{Hawking1974}
Hawking, S.\,W.\ (1974).
Black hole explosions?
\textit{Nature}, 248, 30--31.
DOI: 10.1038/248030a0.

\bibitem{Nash1950}
Nash, J.\,F.\ (1950).
Equilibrium points in $n$-person games.
\textit{Proceedings of the National Academy of Sciences}, 36(1), 48--49.
DOI: 10.1073/pnas.36.1.48.

\bibitem{DESI2025}
DESI Collaboration (2025).
DESI DR2 Results II: Measurements of Baryon Acoustic Oscillations and Cosmological Constraints.
\textit{arXiv:2503.14738}.

\bibitem{Scolnic2022}
Scolnic, D.\ et al.\ (2022).
The Pantheon+ Analysis: The Full Data Set and Light-curve Release.
\textit{The Astrophysical Journal}, 938(2), 113.
DOI: 10.3847/1538-4357/ac8b7a.

\bibitem{KassRaftery1995}
Kass, R.\,E.\ \& Raftery, A.\,E.\ (1995).
Bayes Factors.
\textit{Journal of the American Statistical Association}, 90(430), 773--795.
DOI: 10.1080/01621459.1995.10476572.

\bibitem{ForemanMackey2013}
Foreman-Mackey, D.\ et al.\ (2013).
emcee: The MCMC Hammer.
\textit{Publications of the Astronomical Society of the Pacific}, 125(925), 306--312.
DOI: 10.1086/670067.

\bibitem{Labbe2023}
Labb\'e, I.\ et al.\ (2023).
A population of red candidate massive galaxies $\sim$600\,Myr after the Big Bang.
\textit{Nature}, 616(7956), 266--269.
DOI: 10.1038/s41586-023-05786-2.

\bibitem{BoylanKolchin2023}
Boylan-Kolchin, M.\ (2023).
Stress testing $\Lambda$CDM with high-redshift galaxy candidates.
\textit{Nature Astronomy}, 7, 731--735.
DOI: 10.1038/s41550-023-01937-7.

\bibitem{Asencio2023}
Asencio, E., Banik, I.\ \& Kroupa, P.\ (2023).
The El Gordo galaxy cluster challenges $\Lambda$CDM for any plausible collision velocity.
\textit{The Astrophysical Journal}, 954(2), 162.
DOI: 10.3847/1538-4357/ace62a.

\bibitem{Miller2018}
Miller, T.\,B.\ et al.\ (2018).
A massive core for a cluster of galaxies at a redshift of 4.3.
\textit{Nature}, 556(7702), 469--472.
DOI: 10.1038/s41586-018-0025-2.

\bibitem{Milgrom1983}
Milgrom, M.\ (1983).
A modification of the Newtonian dynamics as a possible alternative to the hidden mass hypothesis.
\textit{The Astrophysical Journal}, 270, 365--370.
DOI: 10.1086/161130.

\bibitem{Skordis2021}
Skordis, C.\ \& Z{\l}o\'snik, T.\ (2021).
New Relativistic Theory for Modified Newtonian Dynamics.
\textit{Physical Review Letters}, 127(16), 161302.
DOI: 10.1103/PhysRevLett.127.161302.

\bibitem{Bekenstein2004}
Bekenstein, J.\,D.\ (2004).
Relativistic gravitation theory for the modified Newtonian dynamics paradigm.
\textit{Physical Review D}, 70(8), 083509.
DOI: 10.1103/PhysRevD.70.083509.

\bibitem{HuSawicki2007}
Hu, W.\ \& Sawicki, I.\ (2007).
Models of $f(R)$ Cosmic Acceleration that Evade Solar-System Tests.
\textit{Physical Review D}, 76(6), 064004.
DOI: 10.1103/PhysRevD.76.064004.

\bibitem{Jacobson1995}
Jacobson, T.\ (1995).
Thermodynamics of Spacetime: The Einstein Equation of State.
\textit{Physical Review Letters}, 75(7), 1260--1263.
DOI: 10.1103/PhysRevLett.75.1260.

\bibitem{Dewar2003}
Dewar, R.\ (2003).
Information theory explanation of the fluctuation theorem, maximum entropy production and self-organized criticality in non-equilibrium stationary states.
\textit{Journal of Physics A: Mathematical and General}, 36(3), 631--641.
DOI: 10.1088/0305-4470/36/3/303.

\bibitem{Wolpert2006}
Wolpert, D.\,H.\ (2006).
Information Theory -- The Bridge Connecting Bounded Rational Game Theory and Statistical Physics.
In: \textit{Complex Engineered Systems}, Springer, 262--290.
DOI: 10.1007/3-540-32834-3\_12.

\bibitem{Harris2020}
Harris, C.\,R.\ et al.\ (2020).
Array programming with NumPy.
\textit{Nature}, 585, 357--362.
DOI: 10.1038/s41586-020-2649-2.

\bibitem{Virtanen2020}
Virtanen, P.\ et al.\ (2020).
SciPy 1.0: Fundamental Algorithms for Scientific Computing in Python.
\textit{Nature Methods}, 17, 261--272.
DOI: 10.1038/s41592-019-0686-2.

\bibitem{Hunter2007}
Hunter, J.\,D.\ (2007).
Matplotlib: A 2D Graphics Environment.
\textit{Computing in Science \& Engineering}, 9(3), 90--95.
DOI: 10.1109/MCSE.2007.55.

\end{thebibliography}

\end{document}
